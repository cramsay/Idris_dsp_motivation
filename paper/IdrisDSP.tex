\documentclass[conference]{IEEEtran}
\IEEEoverridecommandlockouts
% The preceding line is only needed to identify funding in the first footnote.
% If that is unneeded, please comment it out.
\usepackage{cite}
\usepackage{amsmath,amssymb,amsfonts}
\usepackage{algorithmic}
\usepackage{graphicx}
\usepackage{textcomp}
\usepackage{xcolor}
\def\BibTeX{{\rm B\kern-.05em{\sc i\kern-.025em b}\kern-.08em
    T\kern-.1667em\lower.7ex\hbox{E}\kern-.125emX}}

\usepackage[styles]{idrislang}

\newcommand\todo[1]{\textcolor{red}{#1}}
\usepackage{newfloat}
\DeclareFloatingEnvironment[name={Listing}]{codefig}
\usepackage{xcolor}
%\usepackage{caption}
\usepackage{listings}

\definecolor{codegreen}{rgb}{0,0.6,0}
\definecolor{codegray}{rgb}{0.5,0.5,0.5}
\definecolor{codepurple}{rgb}{0.58,0,0.82}
\definecolor{backcolour}{rgb}{0.95,0.95,0.92}

\lstdefinestyle{mystyle}{
    %backgroundcolor=\color{backcolour},   
    commentstyle=\color{codegreen},
    keywordstyle=\color{magenta},
    numberstyle=\tiny\color{codegray},
    stringstyle=\color{codepurple},
    basicstyle=\ttfamily\footnotesize,
    breakatwhitespace=false,         
    breaklines=true,                 
    captionpos=b,                    
    keepspaces=true,                 
    numbers=left,                    
    numbersep=5pt,                  
    showspaces=false,                
    showstringspaces=false,
    showtabs=false,                  
    frame=lines,
    tabsize=2
}

\lstset{style=mystyle}
\usepackage{svg}

\graphicspath{{./img/}}
\begin{document}

\title{On Applications of Dependent Types to Parameterised Digital Signal
  Processing Circuits \\
\thanks{Funded under EPSRC grant no. EP/N509760/1} }

\author{\IEEEauthorblockN{Craig Ramsay}
\IEEEauthorblockA{
\textit{University of Strathclyde}\\
Glasgow, Scotland \\
\href{craig.ramsay.100@strath.ac.uk}}
\and
\IEEEauthorblockN{Louise H. Crockett}
\IEEEauthorblockA{
\textit{University of Strathclyde}\\
Glasgow, Scotland \\
louise.crockett@strath.ac.uk}
\and
\IEEEauthorblockN{Robert W. Stewart}
\IEEEauthorblockA{
\textit{University of Strathclyde}\\
Glasgow, Scotland \\
r.stewart@strath.ac.uk}
}

\maketitle

\begin{abstract}
  We'll introduce Idris for EEE folks and show where there's a dependently typed
  hole in most hardware description languages, and what sort of signal
  processing circuits we can construct if we had such a language feature.
\end{abstract}

\begin{IEEEkeywords}
dependent types, functional programming, Idris, digital signal processing, field programmable gate arrays  
\end{IEEEkeywords}

\section{Introduction}

Issues we see in describing DSP things generically. VHDL no good, so people used
things like perl to generate vhdl in an unsafe way for ages. The essence of this
technique has been continued with more feature rich HDLs, including those
embedded in Haskell. These languages still rely on the execution of the
designer's software program to generate hardware --- meaning structure and
word lengths will be evaluated only when the program is called with a specific
set of arguments. Testing is still a big burden here. It's also hard to gauge
how well tested a 3rd party IP is... What if we could describe our circuits in
such a way so the compiler can check the circuit structure and word lengths, even
when these change in non-trivial ways with changes to the parameters?

Dependent types can do this! What are those? Idris is a good shout.

\section{Modelling synchronous circuits in Idris}

In order to model simple circuit behaviour in Idris, a few choices must be considered. Namely,
\begin{enumerate}
\item \emph{How to best represent numeric types} --- \\A raw collection of bits, or
  native integers? How should binary word lengths be encoded in the types?
\item \emph{How to best model synchronous logic} --- \\Potentially infinite lists of
  discrete-time samples? How do we ensure causality is preserved? Should
  multiple clock domains be supported?
\end{enumerate}

An intuitive solution to choice 1) is to expose only a type representing a
single bit (an equivalent of VHDL's \texttt{std\_logic} type) and let the
designer create their own abstractions on top of this, introducing any number of
different arithmetic semantics and representations (unsigned, signed,
fixed-point, binary-coded decimal, for example). In a slightly more idiomatic
approach, we opt for the use of a type class for these bit-representable types
(much like the \texttt{Rep} type class used in Kansas Lava\cite{gill_13}). This
provides a formal interface that defines what a type must implement in order to
be bit-presentable. An instance of this type class can be written for any
custom type or, importantly, any existing Idris type. The reuse of existing
types is a benefit over simply constructing new data types as collections of
bits directly.

Choice 2) introduces a substantial level of complexity when considering circuit
synthesis --- but much of this can be side-stepped when performing simulation
alone. One long-standing method of modelling synchronous signals is to use
infinite streams where the $k^{th}$ element represents the discrete-time sample
that is stable during the $k^{th}$ clock cycle. This technique can be seen in
various forms in the languages $\mu$FP\cite{ufp}, Kansas Lavas\cite{gill_09},
and C$\lambda$aSH\cite{baaij_15}.

The host language must, generally, have support for lazy evaluation for
definitions of these infinite stream structures. While Idris uses eager
evaluation by default (unlike Haskell, the host language of
\cite{gill_09,baaij_15}), lazy evaluation is still supported when operating on
explicitly ``lazy'' types.

A final nuance is that it is possible to describe a variety of non-synthesisable
circuits using streams directly. For example:

\begin{itemize}
\item Dropping an element from the stream describes a time advance, and is
  non-causal
\begin{lstlisting}[language=idris]
adv : Stream a -> Stream a
adv x :: xs = xs
\end{lstlisting}
      
\item Some recursive uses of streams would infer circuits with infinite memory
  elements\cite{baaij_15}
\begin{lstlisting}[language=idris]
elephant : a -> Stream a -> Stream a
elephant i (x :: xs) = i :: x :: elephant i xs
\end{lstlisting}
\end{itemize}

These non-synthesisable descriptions can be precluded by hiding the Stream
implementation and only exposing safe, hardware-friendly functions that operate
on these streams --- such as \texttt{delay}, and an functor interface.

Before considering how Idris' dependent types enhance DSP circuit models, it is
worth reiterating that, at the time of writing, this environment only simulates
synchronous circuit behaviour and \emph{does not} synthesise circuits as VHDL or
Verilog descriptions. However, circuit synthesis is a promising avenue for
future work given Idris' support for hosting EDSLs, even with their own syntax
overloading\cite{brady_12}.


\section{First steps:\\Minimal Bit Growth for FIR Adder Chains}
\label{sec:fir}

% Talk about direct form FIR filters with diagram
As an introductory example, the consider a direct form Finite Impulse Response
(FIR) filter shown in Figure \ref{fig:fir_direct}. All word lengths have been
annotated, considering the worst-case for each arithmetic operation in
isolation. 

\begin{figure}[htb]
  \centering
  %\includesvg{firworstcase}
  %LaTeX with PSTricks extensions
%%Creator: inkscape 0.92.5
%%Please note this file requires PSTricks extensions
\psset{xunit=.5pt,yunit=.5pt,runit=.5pt}
\begin{pspicture}(396.8503937,120.94488189)
{
\newrgbcolor{curcolor}{0 0 0}
\pscustom[linewidth=0.99999995,linecolor=curcolor]
{
\newpath
\moveto(66.99997068,107.38285095)
\lineto(91.99996948,107.38285095)
\lineto(91.99996948,82.38285215)
\lineto(66.99997068,82.38285215)
\closepath
}
}
{
\newrgbcolor{curcolor}{0 0 0}
\pscustom[linewidth=0.99999995,linecolor=curcolor]
{
\newpath
\moveto(136.99997309,107.38285095)
\lineto(161.99997189,107.38285095)
\lineto(161.99997189,82.38285215)
\lineto(136.99997309,82.38285215)
\closepath
}
}
{
\newrgbcolor{curcolor}{0 0 0}
\pscustom[linewidth=0.99999995,linecolor=curcolor]
{
\newpath
\moveto(206.99996828,107.38285095)
\lineto(231.99996708,107.38285095)
\lineto(231.99996708,82.38285215)
\lineto(206.99996828,82.38285215)
\closepath
}
}
{
\newrgbcolor{curcolor}{0 0 0}
\pscustom[linewidth=0.9508879,linecolor=curcolor]
{
\newpath
\moveto(51.53437096,57.36824096)
\curveto(51.53437096,52.13386254)(47.26308662,47.89056042)(41.99419037,47.89056042)
\curveto(36.72529411,47.89056042)(32.45400977,52.13386254)(32.45400977,57.36824096)
\curveto(32.45400977,62.60261938)(36.72529411,66.84592149)(41.99419037,66.84592149)
\curveto(47.26308662,66.84592149)(51.53437096,62.60261938)(51.53437096,57.36824096)
\closepath
}
}
{
\newrgbcolor{curcolor}{0 0 0}
\pscustom[linewidth=1.00157475,linecolor=curcolor]
{
\newpath
\moveto(66.99997984,57.38286614)
\lineto(51.99997984,57.38286614)
}
}
{
\newrgbcolor{curcolor}{0 0 0}
\pscustom[linestyle=none,fillstyle=solid,fillcolor=curcolor]
{
\newpath
\moveto(56.00627884,57.38286614)
\lineto(58.00942834,59.38601564)
\lineto(50.99840509,57.38286614)
\lineto(58.00942834,55.37971664)
\closepath
}
}
{
\newrgbcolor{curcolor}{0 0 0}
\pscustom[linewidth=0.53417322,linecolor=curcolor]
{
\newpath
\moveto(56.00627884,57.38286614)
\lineto(58.00942834,59.38601564)
\lineto(50.99840509,57.38286614)
\lineto(58.00942834,55.37971664)
\closepath
}
}
{
\newrgbcolor{curcolor}{0.65490198 0.66274512 0.67450982}
\pscustom[linewidth=0.99999995,linecolor=curcolor]
{
\newpath
\moveto(59.9999811,55.38285354)
\lineto(63.99997984,59.38284094)
}
}
{
\newrgbcolor{curcolor}{0 0 0}
\pscustom[linewidth=0.9508879,linecolor=curcolor]
{
\newpath
\moveto(121.53434813,57.36824096)
\curveto(121.53434813,52.13386254)(117.26306379,47.89056042)(111.99416754,47.89056042)
\curveto(106.72527129,47.89056042)(102.45398694,52.13386254)(102.45398694,57.36824096)
\curveto(102.45398694,62.60261938)(106.72527129,66.84592149)(111.99416754,66.84592149)
\curveto(117.26306379,66.84592149)(121.53434813,62.60261938)(121.53434813,57.36824096)
\closepath
}
}
{
\newrgbcolor{curcolor}{0 0 0}
\pscustom[linewidth=1.00157475,linecolor=curcolor]
{
\newpath
\moveto(136.99998236,57.38286614)
\lineto(121.99998236,57.38286614)
}
}
{
\newrgbcolor{curcolor}{0 0 0}
\pscustom[linestyle=none,fillstyle=solid,fillcolor=curcolor]
{
\newpath
\moveto(126.00628136,57.38286614)
\lineto(128.00943086,59.38601564)
\lineto(120.99840761,57.38286614)
\lineto(128.00943086,55.37971664)
\closepath
}
}
{
\newrgbcolor{curcolor}{0 0 0}
\pscustom[linewidth=0.53417322,linecolor=curcolor]
{
\newpath
\moveto(126.00628136,57.38286614)
\lineto(128.00943086,59.38601564)
\lineto(120.99840761,57.38286614)
\lineto(128.00943086,55.37971664)
\closepath
}
}
{
\newrgbcolor{curcolor}{0.65490198 0.66274512 0.67450982}
\pscustom[linewidth=0.99999995,linecolor=curcolor]
{
\newpath
\moveto(129.99998362,55.38285354)
\lineto(133.99998236,59.38284094)
}
}
{
\newrgbcolor{curcolor}{0 0 0}
\pscustom[linewidth=0.9508879,linecolor=curcolor]
{
\newpath
\moveto(191.47440873,57.36824096)
\curveto(191.47440873,52.13386254)(187.20312438,47.89056042)(181.93422813,47.89056042)
\curveto(176.66533188,47.89056042)(172.39404753,52.13386254)(172.39404753,57.36824096)
\curveto(172.39404753,62.60261938)(176.66533188,66.84592149)(181.93422813,66.84592149)
\curveto(187.20312438,66.84592149)(191.47440873,62.60261938)(191.47440873,57.36824096)
\closepath
}
}
{
\newrgbcolor{curcolor}{0 0 0}
\pscustom[linewidth=1.00157475,linecolor=curcolor]
{
\newpath
\moveto(206.9400378,57.38286614)
\lineto(191.9400378,57.38286614)
}
}
{
\newrgbcolor{curcolor}{0 0 0}
\pscustom[linestyle=none,fillstyle=solid,fillcolor=curcolor]
{
\newpath
\moveto(195.94633679,57.38286614)
\lineto(197.94948629,59.38601564)
\lineto(190.93846305,57.38286614)
\lineto(197.94948629,55.37971664)
\closepath
}
}
{
\newrgbcolor{curcolor}{0 0 0}
\pscustom[linewidth=0.53417322,linecolor=curcolor]
{
\newpath
\moveto(195.94633679,57.38286614)
\lineto(197.94948629,59.38601564)
\lineto(190.93846305,57.38286614)
\lineto(197.94948629,55.37971664)
\closepath
}
}
{
\newrgbcolor{curcolor}{0.65490198 0.66274512 0.67450982}
\pscustom[linewidth=0.99999995,linecolor=curcolor]
{
\newpath
\moveto(199.94003906,55.38285354)
\lineto(203.9400378,59.38284094)
}
}
{
\newrgbcolor{curcolor}{0 0 0}
\pscustom[linewidth=0.9508879,linecolor=curcolor]
{
\newpath
\moveto(261.4743895,57.36824096)
\curveto(261.4743895,52.13386254)(257.20310516,47.89056042)(251.93420891,47.89056042)
\curveto(246.66531266,47.89056042)(242.39402831,52.13386254)(242.39402831,57.36824096)
\curveto(242.39402831,62.60261938)(246.66531266,66.84592149)(251.93420891,66.84592149)
\curveto(257.20310516,66.84592149)(261.4743895,62.60261938)(261.4743895,57.36824096)
\closepath
}
}
{
\newrgbcolor{curcolor}{0 0 0}
\pscustom[linewidth=1.00157475,linecolor=curcolor]
{
\newpath
\moveto(276.94003276,57.38286614)
\lineto(261.94003276,57.38286614)
}
}
{
\newrgbcolor{curcolor}{0 0 0}
\pscustom[linestyle=none,fillstyle=solid,fillcolor=curcolor]
{
\newpath
\moveto(265.94633175,57.38286614)
\lineto(267.94948125,59.38601564)
\lineto(260.93845801,57.38286614)
\lineto(267.94948125,55.37971664)
\closepath
}
}
{
\newrgbcolor{curcolor}{0 0 0}
\pscustom[linewidth=0.53417322,linecolor=curcolor]
{
\newpath
\moveto(265.94633175,57.38286614)
\lineto(267.94948125,59.38601564)
\lineto(260.93845801,57.38286614)
\lineto(267.94948125,55.37971664)
\closepath
}
}
{
\newrgbcolor{curcolor}{0.65490198 0.66274512 0.67450982}
\pscustom[linewidth=0.99999995,linecolor=curcolor]
{
\newpath
\moveto(269.94003402,55.38285354)
\lineto(273.94003276,59.38284094)
}
}
{
\newrgbcolor{curcolor}{0.65490198 0.66274512 0.67450982}
\pscustom[linewidth=0.99999995,linecolor=curcolor]
{
\newpath
\moveto(49.99998236,92.38284094)
\lineto(53.9999811,96.38282835)
}
}
{
\newrgbcolor{curcolor}{0.65490198 0.66274512 0.67450982}
\pscustom[linewidth=0.99999995,linecolor=curcolor]
{
\newpath
\moveto(119.99998488,92.38284094)
\lineto(123.99998362,96.38282835)
}
}
{
\newrgbcolor{curcolor}{0.65490198 0.66274512 0.67450982}
\pscustom[linewidth=0.99999995,linecolor=curcolor]
{
\newpath
\moveto(189.99997606,92.38284094)
\lineto(193.9999748,96.38282835)
}
}
{
\newrgbcolor{curcolor}{0.65490198 0.66274512 0.67450982}
\pscustom[linewidth=0.99999995,linecolor=curcolor]
{
\newpath
\moveto(244.99998236,92.38284094)
\lineto(248.9999811,96.38282835)
}
}
{
\newrgbcolor{curcolor}{0 0 0}
\pscustom[linewidth=0.9508879,linecolor=curcolor]
{
\newpath
\moveto(121.53433372,17.37247016)
\curveto(121.53433372,12.13809174)(117.26304937,7.89478963)(111.99415312,7.89478963)
\curveto(106.72525687,7.89478963)(102.45397252,12.13809174)(102.45397252,17.37247016)
\curveto(102.45397252,22.60684859)(106.72525687,26.8501507)(111.99415312,26.8501507)
\curveto(117.26304937,26.8501507)(121.53433372,22.60684859)(121.53433372,17.37247016)
\closepath
}
}
{
\newrgbcolor{curcolor}{0.65490198 0.66274512 0.67450982}
\pscustom[linewidth=0.99999995,linecolor=curcolor]
{
\newpath
\moveto(109.99998236,37.38285354)
\lineto(113.9999811,41.38284094)
}
}
{
\newrgbcolor{curcolor}{0 0 0}
\pscustom[linewidth=0.9508879,linecolor=curcolor]
{
\newpath
\moveto(191.52651449,17.37247016)
\curveto(191.52651449,12.13809174)(187.25523014,7.89478963)(181.98633389,7.89478963)
\curveto(176.71743764,7.89478963)(172.4461533,12.13809174)(172.4461533,17.37247016)
\curveto(172.4461533,22.60684859)(176.71743764,26.8501507)(181.98633389,26.8501507)
\curveto(187.25523014,26.8501507)(191.52651449,22.60684859)(191.52651449,17.37247016)
\closepath
}
}
{
\newrgbcolor{curcolor}{0.65490198 0.66274512 0.67450982}
\pscustom[linewidth=0.99999995,linecolor=curcolor]
{
\newpath
\moveto(179.99216126,37.38285354)
\lineto(183.99216,41.38284094)
}
}
{
\newrgbcolor{curcolor}{0 0 0}
\pscustom[linewidth=0.9508879,linecolor=curcolor]
{
\newpath
\moveto(261.5265241,17.37247016)
\curveto(261.5265241,12.13809174)(257.25523976,7.89478963)(251.9863435,7.89478963)
\curveto(246.71744725,7.89478963)(242.44616291,12.13809174)(242.44616291,17.37247016)
\curveto(242.44616291,22.60684859)(246.71744725,26.8501507)(251.9863435,26.8501507)
\curveto(257.25523976,26.8501507)(261.5265241,22.60684859)(261.5265241,17.37247016)
\closepath
}
}
{
\newrgbcolor{curcolor}{0.65490198 0.66274512 0.67450982}
\pscustom[linewidth=0.99999995,linecolor=curcolor]
{
\newpath
\moveto(249.99216,37.38285354)
\lineto(253.99215874,41.38284094)
}
}
{
\newrgbcolor{curcolor}{0.65490198 0.66274512 0.67450982}
\pscustom[linestyle=none,fillstyle=solid,fillcolor=curcolor]
{
\newpath
\moveto(129.08003236,19.54291256)
\curveto(129.08003236,19.54291256)(129.08003236,19.58291256)(129.04003236,19.68691255)
\lineto(126.3200325,27.20691216)
\curveto(126.2880325,27.29491215)(126.25603251,27.38291215)(126.14403251,27.38291215)
\curveto(126.05603252,27.38291215)(125.98403252,27.31091215)(125.98403252,27.22291216)
\curveto(125.98403252,27.22291216)(125.98403252,27.18291216)(126.02403252,27.07891216)
\lineto(128.74403237,19.55891256)
\curveto(128.77603237,19.47091256)(128.80803237,19.38291257)(128.92003237,19.38291257)
\curveto(129.00803236,19.38291257)(129.08003236,19.45491256)(129.08003236,19.54291256)
\closepath
}
}
{
\newrgbcolor{curcolor}{0.65490198 0.66274512 0.67450982}
\pscustom[linestyle=none,fillstyle=solid,fillcolor=curcolor]
{
\newpath
\moveto(132.41601686,22.40691241)
\curveto(132.41601686,22.83091239)(132.17601688,23.07091237)(132.08001688,23.16691237)
\curveto(131.81601689,23.42291236)(131.50401691,23.48691235)(131.16801693,23.55091235)
\curveto(130.72001695,23.63891234)(130.18401698,23.74291234)(130.18401698,24.20691231)
\curveto(130.18401698,24.4869123)(130.39201697,24.81491228)(131.08001693,24.81491228)
\curveto(131.96001689,24.81491228)(132.00001688,24.09491232)(132.01601688,23.84691233)
\curveto(132.02401688,23.77491234)(132.11201688,23.77491234)(132.11201688,23.77491234)
\curveto(132.21601687,23.77491234)(132.21601687,23.81491234)(132.21601687,23.96691233)
\lineto(132.21601687,24.77491228)
\curveto(132.21601687,24.91091228)(132.21601687,24.96691227)(132.12801688,24.96691227)
\curveto(132.08801688,24.96691227)(132.07201688,24.96691227)(131.96801689,24.87091228)
\curveto(131.94401689,24.83891228)(131.86401689,24.76691229)(131.83201689,24.74291229)
\curveto(131.52801691,24.96691227)(131.20001693,24.96691227)(131.08001693,24.96691227)
\curveto(130.10401698,24.96691227)(129.800017,24.4309123)(129.800017,23.98291233)
\curveto(129.800017,23.70291234)(129.92801699,23.47891235)(130.14401698,23.30291236)
\curveto(130.40001697,23.09491237)(130.62401696,23.04691238)(131.20001693,22.93491238)
\curveto(131.37601692,22.90291238)(132.03201688,22.77491239)(132.03201688,22.19891242)
\curveto(132.03201688,21.79091244)(131.7520169,21.47091246)(131.12801693,21.47091246)
\curveto(130.45601697,21.47091246)(130.16801698,21.92691243)(130.01601699,22.6069124)
\curveto(129.99201699,22.71091239)(129.98401699,22.74291239)(129.90401699,22.74291239)
\curveto(129.800017,22.74291239)(129.800017,22.68691239)(129.800017,22.5429124)
\lineto(129.800017,21.48691246)
\curveto(129.800017,21.35091246)(129.800017,21.29491247)(129.888017,21.29491247)
\curveto(129.92801699,21.29491247)(129.93601699,21.30291247)(130.08801699,21.45491246)
\curveto(130.10401698,21.47091246)(130.10401698,21.48691246)(130.24801698,21.63891245)
\curveto(130.60001696,21.30291247)(130.96001694,21.29491247)(131.12801693,21.29491247)
\curveto(132.04801688,21.29491247)(132.41601686,21.83091244)(132.41601686,22.40691241)
\closepath
}
}
{
\newrgbcolor{curcolor}{0.65490198 0.66274512 0.67450982}
\pscustom[linestyle=none,fillstyle=solid,fillcolor=curcolor]
{
\newpath
\moveto(139.19200926,21.38291246)
\lineto(139.19200926,21.63091245)
\curveto(138.77600928,21.63091245)(138.57600929,21.63091245)(138.56800929,21.87091244)
\lineto(138.56800929,23.39891236)
\curveto(138.56800929,24.08691232)(138.56800929,24.33491231)(138.32000931,24.62291229)
\curveto(138.20800931,24.75891229)(137.94400933,24.91891228)(137.48000935,24.91891228)
\curveto(136.80800939,24.91891228)(136.4560094,24.4389123)(136.32000941,24.13491232)
\curveto(136.20800942,24.83091228)(135.61600945,24.91891228)(135.25600947,24.91891228)
\curveto(134.6720095,24.91891228)(134.29600952,24.5749123)(134.07200953,24.07891232)
\lineto(134.07200953,24.91891228)
\lineto(132.94400959,24.83091228)
\lineto(132.94400959,24.58291229)
\curveto(133.50400956,24.58291229)(133.56800956,24.5269123)(133.56800956,24.13491232)
\lineto(133.56800956,21.99091243)
\curveto(133.56800956,21.63091245)(133.48000956,21.63091245)(132.94400959,21.63091245)
\lineto(132.94400959,21.38291246)
\lineto(133.84800954,21.40691246)
\lineto(134.74400949,21.38291246)
\lineto(134.74400949,21.63091245)
\curveto(134.20800952,21.63091245)(134.12000953,21.63091245)(134.12000953,21.99091243)
\lineto(134.12000953,23.46291235)
\curveto(134.12000953,24.29491231)(134.6880095,24.74291229)(135.20000947,24.74291229)
\curveto(135.70400944,24.74291229)(135.79200944,24.31091231)(135.79200944,23.85491233)
\lineto(135.79200944,21.99091243)
\curveto(135.79200944,21.63091245)(135.70400944,21.63091245)(135.16800947,21.63091245)
\lineto(135.16800947,21.38291246)
\lineto(136.07200942,21.40691246)
\lineto(136.96800938,21.38291246)
\lineto(136.96800938,21.63091245)
\curveto(136.43200941,21.63091245)(136.34400941,21.63091245)(136.34400941,21.99091243)
\lineto(136.34400941,23.46291235)
\curveto(136.34400941,24.29491231)(136.91200938,24.74291229)(137.42400935,24.74291229)
\curveto(137.92800933,24.74291229)(138.01600932,24.31091231)(138.01600932,23.85491233)
\lineto(138.01600932,21.99091243)
\curveto(138.01600932,21.63091245)(137.92800933,21.63091245)(137.39200935,21.63091245)
\lineto(137.39200935,21.38291246)
\lineto(138.29600931,21.40691246)
\closepath
}
}
{
\newrgbcolor{curcolor}{0.65490198 0.66274512 0.67450982}
\pscustom[linestyle=none,fillstyle=solid,fillcolor=curcolor]
{
\newpath
\moveto(143.21599506,22.09491243)
\lineto(143.21599506,22.5429124)
\lineto(143.01599507,22.5429124)
\lineto(143.01599507,22.09491243)
\curveto(143.01599507,21.63091245)(142.81599509,21.58291245)(142.72799509,21.58291245)
\curveto(142.4639951,21.58291245)(142.43199511,21.94291243)(142.43199511,21.98291243)
\lineto(142.43199511,23.58291235)
\curveto(142.43199511,23.91891233)(142.43199511,24.23091231)(142.14399512,24.5269123)
\curveto(141.83199514,24.83891228)(141.43199516,24.96691227)(141.04799518,24.96691227)
\curveto(140.39199521,24.96691227)(139.83999524,24.59091229)(139.83999524,24.06291232)
\curveto(139.83999524,23.82291233)(139.99999523,23.68691234)(140.20799522,23.68691234)
\curveto(140.43199521,23.68691234)(140.5759952,23.84691233)(140.5759952,24.05491232)
\curveto(140.5759952,24.15091232)(140.53599521,24.4149123)(140.16799522,24.4229123)
\curveto(140.38399521,24.70291229)(140.77599519,24.79091228)(141.03199518,24.79091228)
\curveto(141.42399516,24.79091228)(141.87999513,24.4789123)(141.87999513,23.76691234)
\lineto(141.87999513,23.47091235)
\curveto(141.47199516,23.44691235)(140.91199519,23.42291236)(140.40799521,23.18291237)
\curveto(139.80799524,22.91091238)(139.60799525,22.4949124)(139.60799525,22.14291242)
\curveto(139.60799525,21.49491246)(140.38399521,21.29491247)(140.88799519,21.29491247)
\curveto(141.41599516,21.29491247)(141.78399514,21.61491245)(141.93599513,21.99091243)
\curveto(141.96799513,21.67091245)(142.18399512,21.33491247)(142.5599951,21.33491247)
\curveto(142.72799509,21.33491247)(143.21599506,21.44691246)(143.21599506,22.09491243)
\closepath
\moveto(141.87999513,22.5029124)
\curveto(141.87999513,21.74291244)(141.30399516,21.47091246)(140.94399518,21.47091246)
\curveto(140.5519952,21.47091246)(140.22399522,21.75091244)(140.22399522,22.15091242)
\curveto(140.22399522,22.5909124)(140.5599952,23.25491236)(141.87999513,23.30291236)
\closepath
}
}
{
\newrgbcolor{curcolor}{0.65490198 0.66274512 0.67450982}
\pscustom[linestyle=none,fillstyle=solid,fillcolor=curcolor]
{
\newpath
\moveto(145.39198008,21.38291246)
\lineto(145.39198008,21.63091245)
\curveto(144.85598011,21.63091245)(144.76798011,21.63091245)(144.76798011,21.99091243)
\lineto(144.76798011,26.93491217)
\lineto(143.61598017,26.84691218)
\lineto(143.61598017,26.59891219)
\curveto(144.17598015,26.59891219)(144.23998014,26.54291219)(144.23998014,26.15091221)
\lineto(144.23998014,21.99091243)
\curveto(144.23998014,21.63091245)(144.15198015,21.63091245)(143.61598017,21.63091245)
\lineto(143.61598017,21.38291246)
\lineto(144.50398013,21.40691246)
\closepath
}
}
{
\newrgbcolor{curcolor}{0.65490198 0.66274512 0.67450982}
\pscustom[linestyle=none,fillstyle=solid,fillcolor=curcolor]
{
\newpath
\moveto(147.61597802,21.38291246)
\lineto(147.61597802,21.63091245)
\curveto(147.07997805,21.63091245)(146.99197806,21.63091245)(146.99197806,21.99091243)
\lineto(146.99197806,26.93491217)
\lineto(145.83997812,26.84691218)
\lineto(145.83997812,26.59891219)
\curveto(146.39997809,26.59891219)(146.46397808,26.54291219)(146.46397808,26.15091221)
\lineto(146.46397808,21.99091243)
\curveto(146.46397808,21.63091245)(146.37597809,21.63091245)(145.83997812,21.63091245)
\lineto(145.83997812,21.38291246)
\lineto(146.72797807,21.40691246)
\closepath
}
}
{
\newrgbcolor{curcolor}{0.65490198 0.66274512 0.67450982}
\pscustom[linestyle=none,fillstyle=solid,fillcolor=curcolor]
{
\newpath
\moveto(154.00796402,22.94291238)
\curveto(154.00796402,23.60691235)(153.64796403,24.01491232)(153.57596404,24.08691232)
\curveto(153.25596405,24.4389123)(152.96796407,24.5109123)(152.58396409,24.60691229)
\lineto(152.58396409,26.76691218)
\curveto(153.26396405,26.73491218)(153.64796403,26.3829122)(153.76796403,25.93491222)
\curveto(153.73596403,25.94291222)(153.71996403,25.95091222)(153.63996403,25.95091222)
\curveto(153.43196405,25.95091222)(153.27196405,25.80691223)(153.27196405,25.58291224)
\curveto(153.27196405,25.33491226)(153.47196404,25.21491226)(153.63996403,25.21491226)
\curveto(153.66396403,25.21491226)(154.00796402,25.22291226)(154.00796402,25.61491224)
\curveto(154.00796402,26.3909122)(153.47996404,26.97491217)(152.58396409,27.02291217)
\lineto(152.58396409,27.38291215)
\lineto(152.3359641,27.38291215)
\lineto(152.3359641,27.01491217)
\curveto(151.44796415,26.93491217)(150.91196418,26.21491221)(150.91196418,25.48691225)
\curveto(150.91196418,24.93491228)(151.20796416,24.5509123)(151.35196415,24.4149123)
\curveto(151.65596414,24.10291232)(151.92796412,24.03891232)(152.3359641,23.93491233)
\lineto(152.3359641,21.55091245)
\curveto(151.63196414,21.59091245)(151.25596416,21.99891243)(151.15196417,22.4949124)
\curveto(151.18396416,22.48691241)(151.24796416,22.47891241)(151.27996416,22.47891241)
\curveto(151.49596415,22.47891241)(151.64796414,22.6309124)(151.64796414,22.84691239)
\curveto(151.64796414,23.07091237)(151.47996415,23.21491237)(151.27996416,23.21491237)
\curveto(151.23196416,23.21491237)(150.91196418,23.19891237)(150.91196418,22.80691239)
\curveto(150.91196418,22.09491243)(151.33596416,21.35891246)(152.3359641,21.29491247)
\lineto(152.3359641,20.93491249)
\lineto(152.58396409,20.93491249)
\lineto(152.58396409,21.30291247)
\curveto(153.42396405,21.37491246)(154.00796402,22.11891242)(154.00796402,22.94291238)
\closepath
\moveto(153.59996404,22.70291239)
\curveto(153.59996404,22.16691242)(153.21596406,21.63091245)(152.58396409,21.55091245)
\lineto(152.58396409,23.87091233)
\curveto(153.54396404,23.67091234)(153.59996404,22.89491238)(153.59996404,22.70291239)
\closepath
\moveto(152.3359641,24.67091229)
\curveto(151.35196415,24.87891228)(151.31996416,25.56691224)(151.31996416,25.72691223)
\curveto(151.31996416,26.20691221)(151.69596414,26.70291218)(152.3359641,26.76691218)
\closepath
}
}
{
\newrgbcolor{curcolor}{0.65490198 0.66274512 0.67450982}
\pscustom[linestyle=none,fillstyle=solid,fillcolor=curcolor]
{
\newpath
\moveto(157.81594805,21.38291246)
\lineto(157.81594805,21.63091245)
\lineto(157.55994806,21.63091245)
\curveto(156.8399481,21.63091245)(156.8159481,21.71891245)(156.8159481,22.01491243)
\lineto(156.8159481,26.50291219)
\curveto(156.8159481,26.69491218)(156.8159481,26.71091218)(156.63194811,26.71091218)
\curveto(156.13594813,26.19891221)(155.43194817,26.19891221)(155.17594818,26.19891221)
\lineto(155.17594818,25.95091222)
\curveto(155.33594818,25.95091222)(155.80794815,25.95091222)(156.22394813,26.15891221)
\lineto(156.22394813,22.01491243)
\curveto(156.22394813,21.72691245)(156.19994813,21.63091245)(155.47994817,21.63091245)
\lineto(155.22394818,21.63091245)
\lineto(155.22394818,21.38291246)
\curveto(155.50394817,21.40691246)(156.19994813,21.40691246)(156.51994811,21.40691246)
\curveto(156.8399481,21.40691246)(157.53594806,21.40691246)(157.81594805,21.38291246)
\closepath
}
}
{
\newrgbcolor{curcolor}{0.65490198 0.66274512 0.67450982}
\pscustom[linestyle=none,fillstyle=solid,fillcolor=curcolor]
{
\newpath
\moveto(162.23193204,22.70291239)
\lineto(162.23193204,22.95091238)
\lineto(161.43193209,22.95091238)
\lineto(161.43193209,26.59091219)
\curveto(161.43193209,26.75091218)(161.43193209,26.79891218)(161.30393209,26.79891218)
\curveto(161.2319321,26.79891218)(161.2079321,26.79891218)(161.1439321,26.70291218)
\lineto(158.68793223,22.95091238)
\lineto(158.68793223,22.70291239)
\lineto(160.81593212,22.70291239)
\lineto(160.81593212,22.00691243)
\curveto(160.81593212,21.71891245)(160.79993212,21.63091245)(160.20793215,21.63091245)
\lineto(160.03993216,21.63091245)
\lineto(160.03993216,21.38291246)
\curveto(160.36793214,21.40691246)(160.78393212,21.40691246)(161.1199321,21.40691246)
\curveto(161.45593208,21.40691246)(161.87993206,21.40691246)(162.20793205,21.38291246)
\lineto(162.20793205,21.63091245)
\lineto(162.03993205,21.63091245)
\curveto(161.44793209,21.63091245)(161.43193209,21.71891245)(161.43193209,22.00691243)
\lineto(161.43193209,22.70291239)
\closepath
\moveto(160.86393212,22.95091238)
\lineto(158.91193222,22.95091238)
\lineto(160.86393212,25.93491222)
\closepath
}
}
{
\newrgbcolor{curcolor}{0.65490198 0.66274512 0.67450982}
\pscustom[linestyle=none,fillstyle=solid,fillcolor=curcolor]
{
\newpath
\moveto(166.00791788,22.94291238)
\curveto(166.00791788,23.60691235)(165.6479179,24.01491232)(165.5759179,24.08691232)
\curveto(165.25591792,24.4389123)(164.96791793,24.5109123)(164.58391795,24.60691229)
\lineto(164.58391795,26.76691218)
\curveto(165.26391792,26.73491218)(165.6479179,26.3829122)(165.76791789,25.93491222)
\curveto(165.73591789,25.94291222)(165.71991789,25.95091222)(165.6399179,25.95091222)
\curveto(165.43191791,25.95091222)(165.27191792,25.80691223)(165.27191792,25.58291224)
\curveto(165.27191792,25.33491226)(165.47191791,25.21491226)(165.6399179,25.21491226)
\curveto(165.6639179,25.21491226)(166.00791788,25.22291226)(166.00791788,25.61491224)
\curveto(166.00791788,26.3909122)(165.47991791,26.97491217)(164.58391795,27.02291217)
\lineto(164.58391795,27.38291215)
\lineto(164.33591797,27.38291215)
\lineto(164.33591797,27.01491217)
\curveto(163.44791801,26.93491217)(162.91191804,26.21491221)(162.91191804,25.48691225)
\curveto(162.91191804,24.93491228)(163.20791803,24.5509123)(163.35191802,24.4149123)
\curveto(163.655918,24.10291232)(163.92791799,24.03891232)(164.33591797,23.93491233)
\lineto(164.33591797,21.55091245)
\curveto(163.631918,21.59091245)(163.25591802,21.99891243)(163.15191803,22.4949124)
\curveto(163.18391803,22.48691241)(163.24791802,22.47891241)(163.27991802,22.47891241)
\curveto(163.49591801,22.47891241)(163.647918,22.6309124)(163.647918,22.84691239)
\curveto(163.647918,23.07091237)(163.47991801,23.21491237)(163.27991802,23.21491237)
\curveto(163.23191802,23.21491237)(162.91191804,23.19891237)(162.91191804,22.80691239)
\curveto(162.91191804,22.09491243)(163.33591802,21.35891246)(164.33591797,21.29491247)
\lineto(164.33591797,20.93491249)
\lineto(164.58391795,20.93491249)
\lineto(164.58391795,21.30291247)
\curveto(165.42391791,21.37491246)(166.00791788,22.11891242)(166.00791788,22.94291238)
\closepath
\moveto(165.5999179,22.70291239)
\curveto(165.5999179,22.16691242)(165.21591792,21.63091245)(164.58391795,21.55091245)
\lineto(164.58391795,23.87091233)
\curveto(165.5439179,23.67091234)(165.5999179,22.89491238)(165.5999179,22.70291239)
\closepath
\moveto(164.33591797,24.67091229)
\curveto(163.35191802,24.87891228)(163.31991802,25.56691224)(163.31991802,25.72691223)
\curveto(163.31991802,26.20691221)(163.695918,26.70291218)(164.33591797,26.76691218)
\closepath
}
}
{
\newrgbcolor{curcolor}{0.65490198 0.66274512 0.67450982}
\pscustom[linewidth=0.99999995,linecolor=curcolor]
{
\newpath
\moveto(143.99998488,15.38286614)
\lineto(147.99998362,19.38285354)
}
}
{
\newrgbcolor{curcolor}{0.65490198 0.66274512 0.67450982}
\pscustom[linestyle=none,fillstyle=solid,fillcolor=curcolor]
{
\newpath
\moveto(200.08002851,19.54291256)
\curveto(200.08002851,19.54291256)(200.08002851,19.58291256)(200.04002851,19.68691255)
\lineto(197.32002866,27.20691216)
\curveto(197.28802866,27.29491215)(197.25602866,27.38291215)(197.14402867,27.38291215)
\curveto(197.05602867,27.38291215)(196.98402868,27.31091215)(196.98402868,27.22291216)
\curveto(196.98402868,27.22291216)(196.98402868,27.18291216)(197.02402867,27.07891216)
\lineto(199.74402853,19.55891256)
\curveto(199.77602853,19.47091256)(199.80802853,19.38291257)(199.92002852,19.38291257)
\curveto(200.00802852,19.38291257)(200.08002851,19.45491256)(200.08002851,19.54291256)
\closepath
}
}
{
\newrgbcolor{curcolor}{0.65490198 0.66274512 0.67450982}
\pscustom[linestyle=none,fillstyle=solid,fillcolor=curcolor]
{
\newpath
\moveto(203.41601302,22.40691241)
\curveto(203.41601302,22.83091239)(203.17601303,23.07091237)(203.08001304,23.16691237)
\curveto(202.81601305,23.42291236)(202.50401307,23.48691235)(202.16801308,23.55091235)
\curveto(201.72001311,23.63891234)(201.18401314,23.74291234)(201.18401314,24.20691231)
\curveto(201.18401314,24.4869123)(201.39201312,24.81491228)(202.08001309,24.81491228)
\curveto(202.96001304,24.81491228)(203.00001304,24.09491232)(203.01601304,23.84691233)
\curveto(203.02401304,23.77491234)(203.11201303,23.77491234)(203.11201303,23.77491234)
\curveto(203.21601303,23.77491234)(203.21601303,23.81491234)(203.21601303,23.96691233)
\lineto(203.21601303,24.77491228)
\curveto(203.21601303,24.91091228)(203.21601303,24.96691227)(203.12801303,24.96691227)
\curveto(203.08801304,24.96691227)(203.07201304,24.96691227)(202.96801304,24.87091228)
\curveto(202.94401304,24.83891228)(202.86401305,24.76691229)(202.83201305,24.74291229)
\curveto(202.52801306,24.96691227)(202.20001308,24.96691227)(202.08001309,24.96691227)
\curveto(201.10401314,24.96691227)(200.80001316,24.4309123)(200.80001316,23.98291233)
\curveto(200.80001316,23.70291234)(200.92801315,23.47891235)(201.14401314,23.30291236)
\curveto(201.40001312,23.09491237)(201.62401311,23.04691238)(202.20001308,22.93491238)
\curveto(202.37601307,22.90291238)(203.03201304,22.77491239)(203.03201304,22.19891242)
\curveto(203.03201304,21.79091244)(202.75201305,21.47091246)(202.12801309,21.47091246)
\curveto(201.45601312,21.47091246)(201.16801314,21.92691243)(201.01601314,22.6069124)
\curveto(200.99201315,22.71091239)(200.98401315,22.74291239)(200.90401315,22.74291239)
\curveto(200.80001316,22.74291239)(200.80001316,22.68691239)(200.80001316,22.5429124)
\lineto(200.80001316,21.48691246)
\curveto(200.80001316,21.35091246)(200.80001316,21.29491247)(200.88801315,21.29491247)
\curveto(200.92801315,21.29491247)(200.93601315,21.30291247)(201.08801314,21.45491246)
\curveto(201.10401314,21.47091246)(201.10401314,21.48691246)(201.24801313,21.63891245)
\curveto(201.60001311,21.30291247)(201.96001309,21.29491247)(202.12801309,21.29491247)
\curveto(203.04801304,21.29491247)(203.41601302,21.83091244)(203.41601302,22.40691241)
\closepath
}
}
{
\newrgbcolor{curcolor}{0.65490198 0.66274512 0.67450982}
\pscustom[linestyle=none,fillstyle=solid,fillcolor=curcolor]
{
\newpath
\moveto(210.19200542,21.38291246)
\lineto(210.19200542,21.63091245)
\curveto(209.77600544,21.63091245)(209.57600545,21.63091245)(209.56800545,21.87091244)
\lineto(209.56800545,23.39891236)
\curveto(209.56800545,24.08691232)(209.56800545,24.33491231)(209.32000546,24.62291229)
\curveto(209.20800547,24.75891229)(208.94400548,24.91891228)(208.48000551,24.91891228)
\curveto(207.80800554,24.91891228)(207.45600556,24.4389123)(207.32000557,24.13491232)
\curveto(207.20800557,24.83091228)(206.6160056,24.91891228)(206.25600562,24.91891228)
\curveto(205.67200565,24.91891228)(205.29600567,24.5749123)(205.07200568,24.07891232)
\lineto(205.07200568,24.91891228)
\lineto(203.94400574,24.83091228)
\lineto(203.94400574,24.58291229)
\curveto(204.50400571,24.58291229)(204.56800571,24.5269123)(204.56800571,24.13491232)
\lineto(204.56800571,21.99091243)
\curveto(204.56800571,21.63091245)(204.48000572,21.63091245)(203.94400574,21.63091245)
\lineto(203.94400574,21.38291246)
\lineto(204.8480057,21.40691246)
\lineto(205.74400565,21.38291246)
\lineto(205.74400565,21.63091245)
\curveto(205.20800568,21.63091245)(205.12000568,21.63091245)(205.12000568,21.99091243)
\lineto(205.12000568,23.46291235)
\curveto(205.12000568,24.29491231)(205.68800565,24.74291229)(206.20000563,24.74291229)
\curveto(206.7040056,24.74291229)(206.79200559,24.31091231)(206.79200559,23.85491233)
\lineto(206.79200559,21.99091243)
\curveto(206.79200559,21.63091245)(206.7040056,21.63091245)(206.16800563,21.63091245)
\lineto(206.16800563,21.38291246)
\lineto(207.07200558,21.40691246)
\lineto(207.96800553,21.38291246)
\lineto(207.96800553,21.63091245)
\curveto(207.43200556,21.63091245)(207.34400556,21.63091245)(207.34400556,21.99091243)
\lineto(207.34400556,23.46291235)
\curveto(207.34400556,24.29491231)(207.91200554,24.74291229)(208.42400551,24.74291229)
\curveto(208.92800548,24.74291229)(209.01600548,24.31091231)(209.01600548,23.85491233)
\lineto(209.01600548,21.99091243)
\curveto(209.01600548,21.63091245)(208.92800548,21.63091245)(208.39200551,21.63091245)
\lineto(208.39200551,21.38291246)
\lineto(209.29600546,21.40691246)
\closepath
}
}
{
\newrgbcolor{curcolor}{0.65490198 0.66274512 0.67450982}
\pscustom[linestyle=none,fillstyle=solid,fillcolor=curcolor]
{
\newpath
\moveto(214.21599122,22.09491243)
\lineto(214.21599122,22.5429124)
\lineto(214.01599123,22.5429124)
\lineto(214.01599123,22.09491243)
\curveto(214.01599123,21.63091245)(213.81599124,21.58291245)(213.72799125,21.58291245)
\curveto(213.46399126,21.58291245)(213.43199126,21.94291243)(213.43199126,21.98291243)
\lineto(213.43199126,23.58291235)
\curveto(213.43199126,23.91891233)(213.43199126,24.23091231)(213.14399128,24.5269123)
\curveto(212.83199129,24.83891228)(212.43199131,24.96691227)(212.04799133,24.96691227)
\curveto(211.39199137,24.96691227)(210.8399914,24.59091229)(210.8399914,24.06291232)
\curveto(210.8399914,23.82291233)(210.99999139,23.68691234)(211.20799138,23.68691234)
\curveto(211.43199137,23.68691234)(211.57599136,23.84691233)(211.57599136,24.05491232)
\curveto(211.57599136,24.15091232)(211.53599136,24.4149123)(211.16799138,24.4229123)
\curveto(211.38399137,24.70291229)(211.77599135,24.79091228)(212.03199133,24.79091228)
\curveto(212.42399131,24.79091228)(212.87999129,24.4789123)(212.87999129,23.76691234)
\lineto(212.87999129,23.47091235)
\curveto(212.47199131,23.44691235)(211.91199134,23.42291236)(211.40799137,23.18291237)
\curveto(210.8079914,22.91091238)(210.60799141,22.4949124)(210.60799141,22.14291242)
\curveto(210.60799141,21.49491246)(211.38399137,21.29491247)(211.88799134,21.29491247)
\curveto(212.41599131,21.29491247)(212.78399129,21.61491245)(212.93599129,21.99091243)
\curveto(212.96799129,21.67091245)(213.18399127,21.33491247)(213.55999125,21.33491247)
\curveto(213.72799125,21.33491247)(214.21599122,21.44691246)(214.21599122,22.09491243)
\closepath
\moveto(212.87999129,22.5029124)
\curveto(212.87999129,21.74291244)(212.30399132,21.47091246)(211.94399134,21.47091246)
\curveto(211.55199136,21.47091246)(211.22399138,21.75091244)(211.22399138,22.15091242)
\curveto(211.22399138,22.5909124)(211.55999136,23.25491236)(212.87999129,23.30291236)
\closepath
}
}
{
\newrgbcolor{curcolor}{0.65490198 0.66274512 0.67450982}
\pscustom[linestyle=none,fillstyle=solid,fillcolor=curcolor]
{
\newpath
\moveto(216.39197624,21.38291246)
\lineto(216.39197624,21.63091245)
\curveto(215.85597626,21.63091245)(215.76797627,21.63091245)(215.76797627,21.99091243)
\lineto(215.76797627,26.93491217)
\lineto(214.61597633,26.84691218)
\lineto(214.61597633,26.59891219)
\curveto(215.1759763,26.59891219)(215.2399763,26.54291219)(215.2399763,26.15091221)
\lineto(215.2399763,21.99091243)
\curveto(215.2399763,21.63091245)(215.1519763,21.63091245)(214.61597633,21.63091245)
\lineto(214.61597633,21.38291246)
\lineto(215.50397628,21.40691246)
\closepath
}
}
{
\newrgbcolor{curcolor}{0.65490198 0.66274512 0.67450982}
\pscustom[linestyle=none,fillstyle=solid,fillcolor=curcolor]
{
\newpath
\moveto(218.61597418,21.38291246)
\lineto(218.61597418,21.63091245)
\curveto(218.07997421,21.63091245)(217.99197421,21.63091245)(217.99197421,21.99091243)
\lineto(217.99197421,26.93491217)
\lineto(216.83997427,26.84691218)
\lineto(216.83997427,26.59891219)
\curveto(217.39997424,26.59891219)(217.46397424,26.54291219)(217.46397424,26.15091221)
\lineto(217.46397424,21.99091243)
\curveto(217.46397424,21.63091245)(217.37597424,21.63091245)(216.83997427,21.63091245)
\lineto(216.83997427,21.38291246)
\lineto(217.72797423,21.40691246)
\closepath
}
}
{
\newrgbcolor{curcolor}{0.65490198 0.66274512 0.67450982}
\pscustom[linestyle=none,fillstyle=solid,fillcolor=curcolor]
{
\newpath
\moveto(225.00796017,22.94291238)
\curveto(225.00796017,23.60691235)(224.64796019,24.01491232)(224.57596019,24.08691232)
\curveto(224.25596021,24.4389123)(223.96796022,24.5109123)(223.58396025,24.60691229)
\lineto(223.58396025,26.76691218)
\curveto(224.26396021,26.73491218)(224.64796019,26.3829122)(224.76796018,25.93491222)
\curveto(224.73596018,25.94291222)(224.71996019,25.95091222)(224.63996019,25.95091222)
\curveto(224.4319602,25.95091222)(224.27196021,25.80691223)(224.27196021,25.58291224)
\curveto(224.27196021,25.33491226)(224.4719602,25.21491226)(224.63996019,25.21491226)
\curveto(224.66396019,25.21491226)(225.00796017,25.22291226)(225.00796017,25.61491224)
\curveto(225.00796017,26.3909122)(224.4799602,26.97491217)(223.58396025,27.02291217)
\lineto(223.58396025,27.38291215)
\lineto(223.33596026,27.38291215)
\lineto(223.33596026,27.01491217)
\curveto(222.4479603,26.93491217)(221.91196033,26.21491221)(221.91196033,25.48691225)
\curveto(221.91196033,24.93491228)(222.20796032,24.5509123)(222.35196031,24.4149123)
\curveto(222.65596029,24.10291232)(222.92796028,24.03891232)(223.33596026,23.93491233)
\lineto(223.33596026,21.55091245)
\curveto(222.6319603,21.59091245)(222.25596031,21.99891243)(222.15196032,22.4949124)
\curveto(222.18396032,22.48691241)(222.24796032,22.47891241)(222.27996031,22.47891241)
\curveto(222.4959603,22.47891241)(222.64796029,22.6309124)(222.64796029,22.84691239)
\curveto(222.64796029,23.07091237)(222.4799603,23.21491237)(222.27996031,23.21491237)
\curveto(222.23196032,23.21491237)(221.91196033,23.19891237)(221.91196033,22.80691239)
\curveto(221.91196033,22.09491243)(222.33596031,21.35891246)(223.33596026,21.29491247)
\lineto(223.33596026,20.93491249)
\lineto(223.58396025,20.93491249)
\lineto(223.58396025,21.30291247)
\curveto(224.4239602,21.37491246)(225.00796017,22.11891242)(225.00796017,22.94291238)
\closepath
\moveto(224.59996019,22.70291239)
\curveto(224.59996019,22.16691242)(224.21596021,21.63091245)(223.58396025,21.55091245)
\lineto(223.58396025,23.87091233)
\curveto(224.54396019,23.67091234)(224.59996019,22.89491238)(224.59996019,22.70291239)
\closepath
\moveto(223.33596026,24.67091229)
\curveto(222.35196031,24.87891228)(222.31996031,25.56691224)(222.31996031,25.72691223)
\curveto(222.31996031,26.20691221)(222.69596029,26.70291218)(223.33596026,26.76691218)
\closepath
}
}
{
\newrgbcolor{curcolor}{0.65490198 0.66274512 0.67450982}
\pscustom[linestyle=none,fillstyle=solid,fillcolor=curcolor]
{
\newpath
\moveto(228.8159442,21.38291246)
\lineto(228.8159442,21.63091245)
\lineto(228.55994421,21.63091245)
\curveto(227.83994425,21.63091245)(227.81594425,21.71891245)(227.81594425,22.01491243)
\lineto(227.81594425,26.50291219)
\curveto(227.81594425,26.69491218)(227.81594425,26.71091218)(227.63194426,26.71091218)
\curveto(227.13594429,26.19891221)(226.43194433,26.19891221)(226.17594434,26.19891221)
\lineto(226.17594434,25.95091222)
\curveto(226.33594433,25.95091222)(226.80794431,25.95091222)(227.22394428,26.15891221)
\lineto(227.22394428,22.01491243)
\curveto(227.22394428,21.72691245)(227.19994429,21.63091245)(226.47994432,21.63091245)
\lineto(226.22394434,21.63091245)
\lineto(226.22394434,21.38291246)
\curveto(226.50394432,21.40691246)(227.19994429,21.40691246)(227.51994427,21.40691246)
\curveto(227.83994425,21.40691246)(228.53594422,21.40691246)(228.8159442,21.38291246)
\closepath
}
}
{
\newrgbcolor{curcolor}{0.65490198 0.66274512 0.67450982}
\pscustom[linestyle=none,fillstyle=solid,fillcolor=curcolor]
{
\newpath
\moveto(233.05592821,22.99091238)
\curveto(233.05592821,23.94291233)(232.39992824,24.74291229)(231.53592829,24.74291229)
\curveto(231.15192831,24.74291229)(230.80792833,24.61491229)(230.51992834,24.33491231)
\lineto(230.51992834,25.89491223)
\curveto(230.67992833,25.84691223)(230.94392832,25.79091223)(231.19992831,25.79091223)
\curveto(232.18392825,25.79091223)(232.74392822,26.51891219)(232.74392822,26.62291219)
\curveto(232.74392822,26.67091219)(232.71992823,26.71091218)(232.66392823,26.71091218)
\curveto(232.66392823,26.71091218)(232.63992823,26.71091218)(232.59992823,26.68691218)
\curveto(232.43992824,26.61491219)(232.04792826,26.4549122)(231.51192829,26.4549122)
\curveto(231.19192831,26.4549122)(230.82392833,26.51091219)(230.44792835,26.67891218)
\curveto(230.38392835,26.70291218)(230.35192835,26.70291218)(230.35192835,26.70291218)
\curveto(230.27192835,26.70291218)(230.27192835,26.63891219)(230.27192835,26.51091219)
\lineto(230.27192835,24.14291232)
\curveto(230.27192835,23.99891233)(230.27192835,23.93491233)(230.38392835,23.93491233)
\curveto(230.43992835,23.93491233)(230.45592835,23.95891233)(230.48792834,24.00691233)
\curveto(230.57592834,24.13491232)(230.87192832,24.5669123)(231.51992829,24.5669123)
\curveto(231.93592827,24.5669123)(232.13592826,24.19891232)(232.19992825,24.05491232)
\curveto(232.32792825,23.75891234)(232.34392825,23.44691235)(232.34392825,23.04691238)
\curveto(232.34392825,22.76691239)(232.34392825,22.28691242)(232.15192826,21.95091243)
\curveto(231.95992827,21.63891245)(231.66392828,21.43091246)(231.2959283,21.43091246)
\curveto(230.71192833,21.43091246)(230.25592836,21.85491244)(230.11992836,22.32691241)
\curveto(230.14392836,22.31891241)(230.16792836,22.31091241)(230.25592836,22.31091241)
\curveto(230.51992834,22.31091241)(230.65592833,22.5109124)(230.65592833,22.70291239)
\curveto(230.65592833,22.89491238)(230.51992834,23.09491237)(230.25592836,23.09491237)
\curveto(230.14392836,23.09491237)(229.86392838,23.03891238)(229.86392838,22.6709124)
\curveto(229.86392838,21.98291243)(230.41592835,21.20691247)(231.3119283,21.20691247)
\curveto(232.23992825,21.20691247)(233.05592821,21.97491243)(233.05592821,22.99091238)
\closepath
}
}
{
\newrgbcolor{curcolor}{0.65490198 0.66274512 0.67450982}
\pscustom[linestyle=none,fillstyle=solid,fillcolor=curcolor]
{
\newpath
\moveto(237.00791403,22.94291238)
\curveto(237.00791403,23.60691235)(236.64791405,24.01491232)(236.57591406,24.08691232)
\curveto(236.25591407,24.4389123)(235.96791409,24.5109123)(235.58391411,24.60691229)
\lineto(235.58391411,26.76691218)
\curveto(236.26391407,26.73491218)(236.64791405,26.3829122)(236.76791405,25.93491222)
\curveto(236.73591405,25.94291222)(236.71991405,25.95091222)(236.63991405,25.95091222)
\curveto(236.43191406,25.95091222)(236.27191407,25.80691223)(236.27191407,25.58291224)
\curveto(236.27191407,25.33491226)(236.47191406,25.21491226)(236.63991405,25.21491226)
\curveto(236.66391405,25.21491226)(237.00791403,25.22291226)(237.00791403,25.61491224)
\curveto(237.00791403,26.3909122)(236.47991406,26.97491217)(235.58391411,27.02291217)
\lineto(235.58391411,27.38291215)
\lineto(235.33591412,27.38291215)
\lineto(235.33591412,27.01491217)
\curveto(234.44791417,26.93491217)(233.9119142,26.21491221)(233.9119142,25.48691225)
\curveto(233.9119142,24.93491228)(234.20791418,24.5509123)(234.35191417,24.4149123)
\curveto(234.65591416,24.10291232)(234.92791414,24.03891232)(235.33591412,23.93491233)
\lineto(235.33591412,21.55091245)
\curveto(234.63191416,21.59091245)(234.25591418,21.99891243)(234.15191418,22.4949124)
\curveto(234.18391418,22.48691241)(234.24791418,22.47891241)(234.27991418,22.47891241)
\curveto(234.49591417,22.47891241)(234.64791416,22.6309124)(234.64791416,22.84691239)
\curveto(234.64791416,23.07091237)(234.47991417,23.21491237)(234.27991418,23.21491237)
\curveto(234.23191418,23.21491237)(233.9119142,23.19891237)(233.9119142,22.80691239)
\curveto(233.9119142,22.09491243)(234.33591417,21.35891246)(235.33591412,21.29491247)
\lineto(235.33591412,20.93491249)
\lineto(235.58391411,20.93491249)
\lineto(235.58391411,21.30291247)
\curveto(236.42391406,21.37491246)(237.00791403,22.11891242)(237.00791403,22.94291238)
\closepath
\moveto(236.59991405,22.70291239)
\curveto(236.59991405,22.16691242)(236.21591408,21.63091245)(235.58391411,21.55091245)
\lineto(235.58391411,23.87091233)
\curveto(236.54391406,23.67091234)(236.59991405,22.89491238)(236.59991405,22.70291239)
\closepath
\moveto(235.33591412,24.67091229)
\curveto(234.35191417,24.87891228)(234.31991417,25.56691224)(234.31991417,25.72691223)
\curveto(234.31991417,26.20691221)(234.69591416,26.70291218)(235.33591412,26.76691218)
\closepath
}
}
{
\newrgbcolor{curcolor}{0.65490198 0.66274512 0.67450982}
\pscustom[linewidth=0.99999995,linecolor=curcolor]
{
\newpath
\moveto(214.99998236,15.38286614)
\lineto(218.9999811,19.38285354)
}
}
{
\newrgbcolor{curcolor}{0.65490198 0.66274512 0.67450982}
\pscustom[linestyle=none,fillstyle=solid,fillcolor=curcolor]
{
\newpath
\moveto(270.08003812,19.54291256)
\curveto(270.08003812,19.54291256)(270.08003812,19.58291256)(270.04003813,19.68691255)
\lineto(267.32003827,27.20691216)
\curveto(267.28803827,27.29491215)(267.25603827,27.38291215)(267.14403828,27.38291215)
\curveto(267.05603828,27.38291215)(266.98403829,27.31091215)(266.98403829,27.22291216)
\curveto(266.98403829,27.22291216)(266.98403829,27.18291216)(267.02403828,27.07891216)
\lineto(269.74403814,19.55891256)
\curveto(269.77603814,19.47091256)(269.80803814,19.38291257)(269.92003813,19.38291257)
\curveto(270.00803813,19.38291257)(270.08003812,19.45491256)(270.08003812,19.54291256)
\closepath
}
}
{
\newrgbcolor{curcolor}{0.65490198 0.66274512 0.67450982}
\pscustom[linestyle=none,fillstyle=solid,fillcolor=curcolor]
{
\newpath
\moveto(273.41602263,22.40691241)
\curveto(273.41602263,22.83091239)(273.17602264,23.07091237)(273.08002265,23.16691237)
\curveto(272.81602266,23.42291236)(272.50402268,23.48691235)(272.1680227,23.55091235)
\curveto(271.72002272,23.63891234)(271.18402275,23.74291234)(271.18402275,24.20691231)
\curveto(271.18402275,24.4869123)(271.39202274,24.81491228)(272.0800227,24.81491228)
\curveto(272.96002265,24.81491228)(273.00002265,24.09491232)(273.01602265,23.84691233)
\curveto(273.02402265,23.77491234)(273.11202265,23.77491234)(273.11202265,23.77491234)
\curveto(273.21602264,23.77491234)(273.21602264,23.81491234)(273.21602264,23.96691233)
\lineto(273.21602264,24.77491228)
\curveto(273.21602264,24.91091228)(273.21602264,24.96691227)(273.12802265,24.96691227)
\curveto(273.08802265,24.96691227)(273.07202265,24.96691227)(272.96802265,24.87091228)
\curveto(272.94402265,24.83891228)(272.86402266,24.76691229)(272.83202266,24.74291229)
\curveto(272.52802268,24.96691227)(272.20002269,24.96691227)(272.0800227,24.96691227)
\curveto(271.10402275,24.96691227)(270.80002277,24.4309123)(270.80002277,23.98291233)
\curveto(270.80002277,23.70291234)(270.92802276,23.47891235)(271.14402275,23.30291236)
\curveto(271.40002274,23.09491237)(271.62402272,23.04691238)(272.20002269,22.93491238)
\curveto(272.37602268,22.90291238)(273.03202265,22.77491239)(273.03202265,22.19891242)
\curveto(273.03202265,21.79091244)(272.75202266,21.47091246)(272.1280227,21.47091246)
\curveto(271.45602273,21.47091246)(271.16802275,21.92691243)(271.01602276,22.6069124)
\curveto(270.99202276,22.71091239)(270.98402276,22.74291239)(270.90402276,22.74291239)
\curveto(270.80002277,22.74291239)(270.80002277,22.68691239)(270.80002277,22.5429124)
\lineto(270.80002277,21.48691246)
\curveto(270.80002277,21.35091246)(270.80002277,21.29491247)(270.88802276,21.29491247)
\curveto(270.92802276,21.29491247)(270.93602276,21.30291247)(271.08802275,21.45491246)
\curveto(271.10402275,21.47091246)(271.10402275,21.48691246)(271.24802274,21.63891245)
\curveto(271.60002273,21.30291247)(271.96002271,21.29491247)(272.1280227,21.29491247)
\curveto(273.04802265,21.29491247)(273.41602263,21.83091244)(273.41602263,22.40691241)
\closepath
}
}
{
\newrgbcolor{curcolor}{0.65490198 0.66274512 0.67450982}
\pscustom[linestyle=none,fillstyle=solid,fillcolor=curcolor]
{
\newpath
\moveto(280.19201503,21.38291246)
\lineto(280.19201503,21.63091245)
\curveto(279.77601505,21.63091245)(279.57601506,21.63091245)(279.56801506,21.87091244)
\lineto(279.56801506,23.39891236)
\curveto(279.56801506,24.08691232)(279.56801506,24.33491231)(279.32001507,24.62291229)
\curveto(279.20801508,24.75891229)(278.94401509,24.91891228)(278.48001512,24.91891228)
\curveto(277.80801515,24.91891228)(277.45601517,24.4389123)(277.32001518,24.13491232)
\curveto(277.20801518,24.83091228)(276.61601522,24.91891228)(276.25601523,24.91891228)
\curveto(275.67201526,24.91891228)(275.29601528,24.5749123)(275.0720153,24.07891232)
\lineto(275.0720153,24.91891228)
\lineto(273.94401536,24.83091228)
\lineto(273.94401536,24.58291229)
\curveto(274.50401533,24.58291229)(274.56801532,24.5269123)(274.56801532,24.13491232)
\lineto(274.56801532,21.99091243)
\curveto(274.56801532,21.63091245)(274.48001533,21.63091245)(273.94401536,21.63091245)
\lineto(273.94401536,21.38291246)
\lineto(274.84801531,21.40691246)
\lineto(275.74401526,21.38291246)
\lineto(275.74401526,21.63091245)
\curveto(275.20801529,21.63091245)(275.12001529,21.63091245)(275.12001529,21.99091243)
\lineto(275.12001529,23.46291235)
\curveto(275.12001529,24.29491231)(275.68801526,24.74291229)(276.20001524,24.74291229)
\curveto(276.70401521,24.74291229)(276.79201521,24.31091231)(276.79201521,23.85491233)
\lineto(276.79201521,21.99091243)
\curveto(276.79201521,21.63091245)(276.70401521,21.63091245)(276.16801524,21.63091245)
\lineto(276.16801524,21.38291246)
\lineto(277.07201519,21.40691246)
\lineto(277.96801514,21.38291246)
\lineto(277.96801514,21.63091245)
\curveto(277.43201517,21.63091245)(277.34401518,21.63091245)(277.34401518,21.99091243)
\lineto(277.34401518,23.46291235)
\curveto(277.34401518,24.29491231)(277.91201515,24.74291229)(278.42401512,24.74291229)
\curveto(278.92801509,24.74291229)(279.01601509,24.31091231)(279.01601509,23.85491233)
\lineto(279.01601509,21.99091243)
\curveto(279.01601509,21.63091245)(278.92801509,21.63091245)(278.39201512,21.63091245)
\lineto(278.39201512,21.38291246)
\lineto(279.29601507,21.40691246)
\closepath
}
}
{
\newrgbcolor{curcolor}{0.65490198 0.66274512 0.67450982}
\pscustom[linestyle=none,fillstyle=solid,fillcolor=curcolor]
{
\newpath
\moveto(284.21600083,22.09491243)
\lineto(284.21600083,22.5429124)
\lineto(284.01600084,22.5429124)
\lineto(284.01600084,22.09491243)
\curveto(284.01600084,21.63091245)(283.81600085,21.58291245)(283.72800086,21.58291245)
\curveto(283.46400087,21.58291245)(283.43200087,21.94291243)(283.43200087,21.98291243)
\lineto(283.43200087,23.58291235)
\curveto(283.43200087,23.91891233)(283.43200087,24.23091231)(283.14400089,24.5269123)
\curveto(282.8320009,24.83891228)(282.43200093,24.96691227)(282.04800095,24.96691227)
\curveto(281.39200098,24.96691227)(280.84000101,24.59091229)(280.84000101,24.06291232)
\curveto(280.84000101,23.82291233)(281.000001,23.68691234)(281.20800099,23.68691234)
\curveto(281.43200098,23.68691234)(281.57600097,23.84691233)(281.57600097,24.05491232)
\curveto(281.57600097,24.15091232)(281.53600097,24.4149123)(281.16800099,24.4229123)
\curveto(281.38400098,24.70291229)(281.77600096,24.79091228)(282.03200095,24.79091228)
\curveto(282.42400093,24.79091228)(282.8800009,24.4789123)(282.8800009,23.76691234)
\lineto(282.8800009,23.47091235)
\curveto(282.47200092,23.44691235)(281.91200095,23.42291236)(281.40800098,23.18291237)
\curveto(280.80800101,22.91091238)(280.60800102,22.4949124)(280.60800102,22.14291242)
\curveto(280.60800102,21.49491246)(281.38400098,21.29491247)(281.88800095,21.29491247)
\curveto(282.41600093,21.29491247)(282.78400091,21.61491245)(282.9360009,21.99091243)
\curveto(282.9680009,21.67091245)(283.18400089,21.33491247)(283.56000087,21.33491247)
\curveto(283.72800086,21.33491247)(284.21600083,21.44691246)(284.21600083,22.09491243)
\closepath
\moveto(282.8800009,22.5029124)
\curveto(282.8800009,21.74291244)(282.30400093,21.47091246)(281.94400095,21.47091246)
\curveto(281.55200097,21.47091246)(281.22400099,21.75091244)(281.22400099,22.15091242)
\curveto(281.22400099,22.5909124)(281.56000097,23.25491236)(282.8800009,23.30291236)
\closepath
}
}
{
\newrgbcolor{curcolor}{0.65490198 0.66274512 0.67450982}
\pscustom[linestyle=none,fillstyle=solid,fillcolor=curcolor]
{
\newpath
\moveto(286.39198585,21.38291246)
\lineto(286.39198585,21.63091245)
\curveto(285.85598588,21.63091245)(285.76798588,21.63091245)(285.76798588,21.99091243)
\lineto(285.76798588,26.93491217)
\lineto(284.61598594,26.84691218)
\lineto(284.61598594,26.59891219)
\curveto(285.17598591,26.59891219)(285.23998591,26.54291219)(285.23998591,26.15091221)
\lineto(285.23998591,21.99091243)
\curveto(285.23998591,21.63091245)(285.15198591,21.63091245)(284.61598594,21.63091245)
\lineto(284.61598594,21.38291246)
\lineto(285.5039859,21.40691246)
\closepath
}
}
{
\newrgbcolor{curcolor}{0.65490198 0.66274512 0.67450982}
\pscustom[linestyle=none,fillstyle=solid,fillcolor=curcolor]
{
\newpath
\moveto(288.61598379,21.38291246)
\lineto(288.61598379,21.63091245)
\curveto(288.07998382,21.63091245)(287.99198382,21.63091245)(287.99198382,21.99091243)
\lineto(287.99198382,26.93491217)
\lineto(286.83998388,26.84691218)
\lineto(286.83998388,26.59891219)
\curveto(287.39998385,26.59891219)(287.46398385,26.54291219)(287.46398385,26.15091221)
\lineto(287.46398385,21.99091243)
\curveto(287.46398385,21.63091245)(287.37598386,21.63091245)(286.83998388,21.63091245)
\lineto(286.83998388,21.38291246)
\lineto(287.72798384,21.40691246)
\closepath
}
}
{
\newrgbcolor{curcolor}{0.65490198 0.66274512 0.67450982}
\pscustom[linestyle=none,fillstyle=solid,fillcolor=curcolor]
{
\newpath
\moveto(295.00796978,22.94291238)
\curveto(295.00796978,23.60691235)(294.6479698,24.01491232)(294.5759698,24.08691232)
\curveto(294.25596982,24.4389123)(293.96796984,24.5109123)(293.58396986,24.60691229)
\lineto(293.58396986,26.76691218)
\curveto(294.26396982,26.73491218)(294.6479698,26.3829122)(294.76796979,25.93491222)
\curveto(294.7359698,25.94291222)(294.7199698,25.95091222)(294.6399698,25.95091222)
\curveto(294.43196981,25.95091222)(294.27196982,25.80691223)(294.27196982,25.58291224)
\curveto(294.27196982,25.33491226)(294.47196981,25.21491226)(294.6399698,25.21491226)
\curveto(294.6639698,25.21491226)(295.00796978,25.22291226)(295.00796978,25.61491224)
\curveto(295.00796978,26.3909122)(294.47996981,26.97491217)(293.58396986,27.02291217)
\lineto(293.58396986,27.38291215)
\lineto(293.33596987,27.38291215)
\lineto(293.33596987,27.01491217)
\curveto(292.44796992,26.93491217)(291.91196994,26.21491221)(291.91196994,25.48691225)
\curveto(291.91196994,24.93491228)(292.20796993,24.5509123)(292.35196992,24.4149123)
\curveto(292.65596991,24.10291232)(292.92796989,24.03891232)(293.33596987,23.93491233)
\lineto(293.33596987,21.55091245)
\curveto(292.63196991,21.59091245)(292.25596993,21.99891243)(292.15196993,22.4949124)
\curveto(292.18396993,22.48691241)(292.24796993,22.47891241)(292.27996993,22.47891241)
\curveto(292.49596991,22.47891241)(292.64796991,22.6309124)(292.64796991,22.84691239)
\curveto(292.64796991,23.07091237)(292.47996992,23.21491237)(292.27996993,23.21491237)
\curveto(292.23196993,23.21491237)(291.91196994,23.19891237)(291.91196994,22.80691239)
\curveto(291.91196994,22.09491243)(292.33596992,21.35891246)(293.33596987,21.29491247)
\lineto(293.33596987,20.93491249)
\lineto(293.58396986,20.93491249)
\lineto(293.58396986,21.30291247)
\curveto(294.42396981,21.37491246)(295.00796978,22.11891242)(295.00796978,22.94291238)
\closepath
\moveto(294.5999698,22.70291239)
\curveto(294.5999698,22.16691242)(294.21596982,21.63091245)(293.58396986,21.55091245)
\lineto(293.58396986,23.87091233)
\curveto(294.54396981,23.67091234)(294.5999698,22.89491238)(294.5999698,22.70291239)
\closepath
\moveto(293.33596987,24.67091229)
\curveto(292.35196992,24.87891228)(292.31996992,25.56691224)(292.31996992,25.72691223)
\curveto(292.31996992,26.20691221)(292.6959699,26.70291218)(293.33596987,26.76691218)
\closepath
}
}
{
\newrgbcolor{curcolor}{0.65490198 0.66274512 0.67450982}
\pscustom[linestyle=none,fillstyle=solid,fillcolor=curcolor]
{
\newpath
\moveto(298.81595381,21.38291246)
\lineto(298.81595381,21.63091245)
\lineto(298.55995383,21.63091245)
\curveto(297.83995386,21.63091245)(297.81595387,21.71891245)(297.81595387,22.01491243)
\lineto(297.81595387,26.50291219)
\curveto(297.81595387,26.69491218)(297.81595387,26.71091218)(297.63195387,26.71091218)
\curveto(297.1359539,26.19891221)(296.43195394,26.19891221)(296.17595395,26.19891221)
\lineto(296.17595395,25.95091222)
\curveto(296.33595394,25.95091222)(296.80795392,25.95091222)(297.2239539,26.15891221)
\lineto(297.2239539,22.01491243)
\curveto(297.2239539,21.72691245)(297.1999539,21.63091245)(296.47995394,21.63091245)
\lineto(296.22395395,21.63091245)
\lineto(296.22395395,21.38291246)
\curveto(296.50395393,21.40691246)(297.1999539,21.40691246)(297.51995388,21.40691246)
\curveto(297.83995386,21.40691246)(298.53595383,21.40691246)(298.81595381,21.38291246)
\closepath
}
}
{
\newrgbcolor{curcolor}{0.65490198 0.66274512 0.67450982}
\pscustom[linestyle=none,fillstyle=solid,fillcolor=curcolor]
{
\newpath
\moveto(303.11993782,23.01491238)
\curveto(303.11993782,24.03091232)(302.40793785,24.79891228)(301.5199379,24.79891228)
\curveto(300.97593793,24.79891228)(300.67993795,24.3909123)(300.51993795,24.00691233)
\lineto(300.51993795,24.19891232)
\curveto(300.51993795,26.22291221)(301.5119379,26.51091219)(301.91993788,26.51091219)
\curveto(302.11193787,26.51091219)(302.44793785,26.4629122)(302.62393784,26.19091221)
\curveto(302.50393785,26.19091221)(302.18393787,26.19091221)(302.18393787,25.83091223)
\curveto(302.18393787,25.58291224)(302.37593786,25.46291225)(302.55193785,25.46291225)
\curveto(302.67993784,25.46291225)(302.91993783,25.53491224)(302.91993783,25.84691223)
\curveto(302.91993783,26.3269122)(302.56793785,26.71091218)(301.90393788,26.71091218)
\curveto(300.87993793,26.71091218)(299.79993799,25.67891224)(299.79993799,23.91091233)
\curveto(299.79993799,21.77491244)(300.72793794,21.20691247)(301.4719379,21.20691247)
\curveto(302.35993786,21.20691247)(303.11993782,21.95891243)(303.11993782,23.01491238)
\closepath
\moveto(302.39993785,23.02291238)
\curveto(302.39993785,22.6389124)(302.39993785,22.23891242)(302.26393786,21.95091243)
\curveto(302.02393787,21.47091246)(301.65593789,21.43091246)(301.4719379,21.43091246)
\curveto(300.96793793,21.43091246)(300.72793794,21.91091244)(300.67993795,22.03091243)
\curveto(300.53593795,22.40691241)(300.53593795,23.04691238)(300.53593795,23.19091237)
\curveto(300.53593795,23.81491234)(300.79193794,24.61491229)(301.5119379,24.61491229)
\curveto(301.63993789,24.61491229)(302.00793788,24.61491229)(302.25593786,24.11891232)
\curveto(302.39993785,23.82291233)(302.39993785,23.41491236)(302.39993785,23.02291238)
\closepath
}
}
{
\newrgbcolor{curcolor}{0.65490198 0.66274512 0.67450982}
\pscustom[linestyle=none,fillstyle=solid,fillcolor=curcolor]
{
\newpath
\moveto(307.00792365,22.94291238)
\curveto(307.00792365,23.60691235)(306.64792366,24.01491232)(306.57592367,24.08691232)
\curveto(306.25592368,24.4389123)(305.9679237,24.5109123)(305.58392372,24.60691229)
\lineto(305.58392372,26.76691218)
\curveto(306.26392368,26.73491218)(306.64792366,26.3829122)(306.76792366,25.93491222)
\curveto(306.73592366,25.94291222)(306.71992366,25.95091222)(306.63992366,25.95091222)
\curveto(306.43192368,25.95091222)(306.27192368,25.80691223)(306.27192368,25.58291224)
\curveto(306.27192368,25.33491226)(306.47192367,25.21491226)(306.63992366,25.21491226)
\curveto(306.66392366,25.21491226)(307.00792365,25.22291226)(307.00792365,25.61491224)
\curveto(307.00792365,26.3909122)(306.47992367,26.97491217)(305.58392372,27.02291217)
\lineto(305.58392372,27.38291215)
\lineto(305.33592373,27.38291215)
\lineto(305.33592373,27.01491217)
\curveto(304.44792378,26.93491217)(303.91192381,26.21491221)(303.91192381,25.48691225)
\curveto(303.91192381,24.93491228)(304.20792379,24.5509123)(304.35192378,24.4149123)
\curveto(304.65592377,24.10291232)(304.92792375,24.03891232)(305.33592373,23.93491233)
\lineto(305.33592373,21.55091245)
\curveto(304.63192377,21.59091245)(304.25592379,21.99891243)(304.1519238,22.4949124)
\curveto(304.18392379,22.48691241)(304.24792379,22.47891241)(304.27992379,22.47891241)
\curveto(304.49592378,22.47891241)(304.64792377,22.6309124)(304.64792377,22.84691239)
\curveto(304.64792377,23.07091237)(304.47992378,23.21491237)(304.27992379,23.21491237)
\curveto(304.23192379,23.21491237)(303.91192381,23.19891237)(303.91192381,22.80691239)
\curveto(303.91192381,22.09491243)(304.33592379,21.35891246)(305.33592373,21.29491247)
\lineto(305.33592373,20.93491249)
\lineto(305.58392372,20.93491249)
\lineto(305.58392372,21.30291247)
\curveto(306.42392368,21.37491246)(307.00792365,22.11891242)(307.00792365,22.94291238)
\closepath
\moveto(306.59992367,22.70291239)
\curveto(306.59992367,22.16691242)(306.21592369,21.63091245)(305.58392372,21.55091245)
\lineto(305.58392372,23.87091233)
\curveto(306.54392367,23.67091234)(306.59992367,22.89491238)(306.59992367,22.70291239)
\closepath
\moveto(305.33592373,24.67091229)
\curveto(304.35192378,24.87891228)(304.31992379,25.56691224)(304.31992379,25.72691223)
\curveto(304.31992379,26.20691221)(304.69592377,26.70291218)(305.33592373,26.76691218)
\closepath
}
}
{
\newrgbcolor{curcolor}{0.65490198 0.66274512 0.67450982}
\pscustom[linewidth=0.99999995,linecolor=curcolor]
{
\newpath
\moveto(284.99999244,15.38286614)
\lineto(288.99999118,19.38285354)
}
}
{
\newrgbcolor{curcolor}{0 0 0}
\pscustom[linestyle=none,fillstyle=solid,fillcolor=curcolor]
{
\newpath
\moveto(59.86004379,94.14950472)
\curveto(59.86004379,95.25617135)(59.2600438,95.93617133)(59.14004381,96.05617133)
\curveto(58.60671049,96.64283798)(58.12671051,96.76283797)(57.48671052,96.92283797)
\lineto(57.48671052,100.52283786)
\curveto(58.62004382,100.46950453)(59.2600438,99.88283788)(59.4600438,99.13617124)
\curveto(59.40671047,99.14950457)(59.3800438,99.1628379)(59.24671047,99.1628379)
\curveto(58.90004382,99.1628379)(58.63337716,98.92283791)(58.63337716,98.54950459)
\curveto(58.63337716,98.13617127)(58.96671048,97.93617127)(59.24671047,97.93617127)
\curveto(59.28671047,97.93617127)(59.86004379,97.94950461)(59.86004379,98.60283792)
\curveto(59.86004379,99.89617121)(58.98004381,100.86950452)(57.48671052,100.94950452)
\lineto(57.48671052,101.5495045)
\lineto(57.0733772,101.5495045)
\lineto(57.0733772,100.93617118)
\curveto(55.59337725,100.80283785)(54.70004394,99.60283789)(54.70004394,98.38950459)
\curveto(54.70004394,97.46950462)(55.19337726,96.82950464)(55.43337725,96.60283798)
\curveto(55.9400439,96.082838)(56.39337722,95.97617133)(57.0733772,95.802838)
\lineto(57.0733772,91.82950479)
\curveto(55.90004391,91.89617145)(55.27337726,92.57617143)(55.10004393,93.40283808)
\curveto(55.15337726,93.38950474)(55.26004392,93.37617141)(55.31337726,93.37617141)
\curveto(55.67337725,93.37617141)(55.92671057,93.62950474)(55.92671057,93.98950472)
\curveto(55.92671057,94.36283805)(55.64671058,94.60283804)(55.31337726,94.60283804)
\curveto(55.23337726,94.60283804)(54.70004394,94.57617137)(54.70004394,93.92283806)
\curveto(54.70004394,92.73617143)(55.40671059,91.5095048)(57.0733772,91.40283814)
\lineto(57.0733772,90.80283815)
\lineto(57.48671052,90.80283815)
\lineto(57.48671052,91.41617147)
\curveto(58.88671048,91.53617147)(59.86004379,92.77617143)(59.86004379,94.14950472)
\closepath
\moveto(59.18004381,93.74950473)
\curveto(59.18004381,92.85617143)(58.54004383,91.96283812)(57.48671052,91.82950479)
\lineto(57.48671052,95.69617134)
\curveto(59.08671048,95.36283802)(59.18004381,94.06950472)(59.18004381,93.74950473)
\closepath
\moveto(57.0733772,97.02950463)
\curveto(55.43337725,97.37617129)(55.38004392,98.52283792)(55.38004392,98.78950458)
\curveto(55.38004392,99.58950456)(56.00671057,100.4161712)(57.0733772,100.52283786)
\closepath
}
}
{
\newrgbcolor{curcolor}{0 0 0}
\pscustom[linestyle=none,fillstyle=solid,fillcolor=curcolor]
{
\newpath
\moveto(65.96668504,94.04283806)
\lineto(65.63335172,94.04283806)
\curveto(65.51335172,92.46950477)(65.23335173,91.88283812)(63.68668511,91.88283812)
\lineto(62.11335182,91.88283812)
\lineto(65.82001838,96.8961713)
\curveto(65.94001837,97.04283797)(65.94001837,97.06950463)(65.94001837,97.12283796)
\curveto(65.94001837,97.29617129)(65.83335171,97.29617129)(65.59335172,97.29617129)
\lineto(61.32668518,97.29617129)
\lineto(61.18001852,95.14950469)
\lineto(61.51335184,95.14950469)
\curveto(61.59335184,96.50950465)(61.84668516,97.00283797)(63.31335179,97.00283797)
\lineto(64.83335174,97.00283797)
\lineto(61.11335185,91.97617145)
\curveto(60.99335185,91.82950479)(60.99335185,91.80283812)(60.99335185,91.73617146)
\curveto(60.99335185,91.5495048)(61.08668519,91.5495048)(61.34001851,91.5495048)
\lineto(65.74001838,91.5495048)
\closepath
}
}
{
\newrgbcolor{curcolor}{0 0 0}
\pscustom[linestyle=none,fillstyle=solid,fillcolor=curcolor]
{
\newpath
\moveto(74.01999653,99.33617123)
\lineto(70.24666331,101.4695045)
\lineto(66.47333009,99.33617123)
\lineto(66.61999675,99.04283791)
\lineto(70.23332998,100.68283786)
\lineto(73.85999654,99.04283791)
\closepath
}
}
{
\newrgbcolor{curcolor}{0 0 0}
\pscustom[linestyle=none,fillstyle=solid,fillcolor=curcolor]
{
\newpath
\moveto(79.6466594,88.36283823)
\curveto(79.6466594,88.49617156)(79.5666594,88.49617156)(79.43332607,88.50950489)
\curveto(78.37999277,88.57617155)(77.88665945,89.17617154)(77.76665945,89.65617152)
\curveto(77.72665945,89.80283818)(77.72665945,89.82950485)(77.72665945,90.2961715)
\lineto(77.72665945,92.29617144)
\curveto(77.72665945,92.69617143)(77.72665945,93.37617141)(77.69999279,93.50950474)
\curveto(77.52665946,94.38950471)(76.67332615,94.73617137)(76.15332617,94.88283803)
\curveto(77.72665945,95.33617135)(77.72665945,96.28283799)(77.72665945,96.65617131)
\lineto(77.72665945,99.05617124)
\curveto(77.72665945,100.01617121)(77.72665945,100.30950454)(78.04665944,100.64283786)
\curveto(78.28665944,100.88283785)(78.5933261,101.20283784)(79.5266594,101.25617117)
\curveto(79.59332606,101.26950451)(79.6466594,101.32283784)(79.6466594,101.40283784)
\curveto(79.6466594,101.5495045)(79.53999273,101.5495045)(79.37999274,101.5495045)
\curveto(78.04665944,101.5495045)(76.85999281,100.86950452)(76.83332615,99.90950455)
\lineto(76.83332615,97.46950462)
\curveto(76.83332615,96.21617132)(76.83332615,96.002838)(76.48665949,95.62950468)
\curveto(76.29999283,95.44283801)(75.93999284,95.08283803)(75.09999287,95.02950469)
\curveto(75.00665954,95.02950469)(74.91332621,95.01617136)(74.91332621,94.88283803)
\curveto(74.91332621,94.7495047)(74.9933262,94.7495047)(75.12665953,94.73617137)
\curveto(75.69999285,94.69617137)(76.83332615,94.41617138)(76.83332615,93.08283809)
\lineto(76.83332615,90.44283816)
\curveto(76.83332615,89.66950485)(76.83332615,89.21617153)(77.52665946,88.72283822)
\curveto(78.09999278,88.32283823)(78.96665942,88.21617157)(79.37999274,88.21617157)
\curveto(79.53999273,88.21617157)(79.6466594,88.21617157)(79.6466594,88.36283823)
\closepath
}
}
{
\newrgbcolor{curcolor}{0 0 0}
\pscustom[linestyle=none,fillstyle=solid,fillcolor=curcolor]
{
\newpath
\moveto(84.29996736,94.04283806)
\lineto(84.29996736,94.81617137)
\lineto(80.76663413,94.81617137)
\lineto(80.76663413,94.04283806)
\closepath
}
}
{
\newrgbcolor{curcolor}{0 0 0}
\pscustom[linestyle=none,fillstyle=solid,fillcolor=curcolor]
{
\newpath
\moveto(90.64660998,91.5495048)
\lineto(90.64660998,91.96283812)
\lineto(90.21994332,91.96283812)
\curveto(89.01994336,91.96283812)(88.97994336,92.10950478)(88.97994336,92.6028381)
\lineto(88.97994336,100.08283788)
\curveto(88.97994336,100.40283787)(88.97994336,100.42950453)(88.6732767,100.42950453)
\curveto(87.84661006,99.57617122)(86.67327676,99.57617122)(86.24661011,99.57617122)
\lineto(86.24661011,99.1628379)
\curveto(86.51327677,99.1628379)(87.29994341,99.1628379)(87.99327672,99.50950456)
\lineto(87.99327672,92.6028381)
\curveto(87.99327672,92.12283811)(87.95327673,91.96283812)(86.75327676,91.96283812)
\lineto(86.32661011,91.96283812)
\lineto(86.32661011,91.5495048)
\curveto(86.79327676,91.5895048)(87.95327673,91.5895048)(88.48661004,91.5895048)
\curveto(89.01994336,91.5895048)(90.17994332,91.5895048)(90.64660998,91.5495048)
\closepath
}
}
{
\newrgbcolor{curcolor}{0 0 0}
\pscustom[linestyle=none,fillstyle=solid,fillcolor=curcolor]
{
\newpath
\moveto(97.41991968,94.88283803)
\curveto(97.41991968,95.01617136)(97.33991968,95.01617136)(97.20658635,95.02950469)
\curveto(96.63325304,95.06950469)(95.49991974,95.34950468)(95.49991974,96.68283798)
\lineto(95.49991974,99.3228379)
\curveto(95.49991974,100.09617121)(95.49991974,100.54950453)(94.80658642,101.04283785)
\curveto(94.23325311,101.4295045)(93.3799198,101.5495045)(92.95325315,101.5495045)
\curveto(92.81991982,101.5495045)(92.68658649,101.5495045)(92.68658649,101.40283784)
\curveto(92.68658649,101.26950451)(92.76658648,101.26950451)(92.89991981,101.25617117)
\curveto(93.95325312,101.18950451)(94.44658643,100.58950453)(94.56658643,100.10950454)
\curveto(94.60658643,99.96283788)(94.60658643,99.93617121)(94.60658643,99.46950456)
\lineto(94.60658643,97.46950462)
\curveto(94.60658643,97.06950463)(94.60658643,96.38950465)(94.6332531,96.25617132)
\curveto(94.80658642,95.37617135)(95.65991973,95.02950469)(96.17991972,94.88283803)
\curveto(94.60658643,94.42950471)(94.60658643,93.48283807)(94.60658643,93.10950475)
\lineto(94.60658643,90.70950482)
\curveto(94.60658643,89.74950485)(94.60658643,89.45617153)(94.28658644,89.1228382)
\curveto(94.04658645,88.88283821)(93.73991979,88.56283822)(92.80658648,88.50950489)
\curveto(92.73991982,88.49617156)(92.68658649,88.44283822)(92.68658649,88.36283823)
\curveto(92.68658649,88.21617157)(92.81991982,88.21617157)(92.95325315,88.21617157)
\curveto(94.28658644,88.21617157)(95.47325307,88.89617154)(95.49991974,89.85617152)
\lineto(95.49991974,92.29617144)
\curveto(95.49991974,93.54950474)(95.49991974,93.76283807)(95.84658639,94.13617139)
\curveto(96.03325305,94.32283805)(96.39325304,94.68283804)(97.23325302,94.73617137)
\curveto(97.32658635,94.73617137)(97.41991968,94.7495047)(97.41991968,94.88283803)
\closepath
}
}
{
\newrgbcolor{curcolor}{0 0 0}
\pscustom[linestyle=none,fillstyle=solid,fillcolor=curcolor]
{
\newpath
\moveto(104.29989244,94.14950472)
\curveto(104.29989244,95.25617135)(103.69989246,95.93617133)(103.57989246,96.05617133)
\curveto(103.04655914,96.64283798)(102.56655916,96.76283797)(101.92655918,96.92283797)
\lineto(101.92655918,100.52283786)
\curveto(103.05989248,100.46950453)(103.69989246,99.88283788)(103.89989245,99.13617124)
\curveto(103.84655912,99.14950457)(103.81989245,99.1628379)(103.68655912,99.1628379)
\curveto(103.33989247,99.1628379)(103.07322581,98.92283791)(103.07322581,98.54950459)
\curveto(103.07322581,98.13617127)(103.40655913,97.93617127)(103.68655912,97.93617127)
\curveto(103.72655912,97.93617127)(104.29989244,97.94950461)(104.29989244,98.60283792)
\curveto(104.29989244,99.89617121)(103.41989246,100.86950452)(101.92655918,100.94950452)
\lineto(101.92655918,101.5495045)
\lineto(101.51322586,101.5495045)
\lineto(101.51322586,100.93617118)
\curveto(100.0332259,100.80283785)(99.13989259,99.60283789)(99.13989259,98.38950459)
\curveto(99.13989259,97.46950462)(99.63322591,96.82950464)(99.8732259,96.60283798)
\curveto(100.37989256,96.082838)(100.83322588,95.97617133)(101.51322586,95.802838)
\lineto(101.51322586,91.82950479)
\curveto(100.33989256,91.89617145)(99.71322591,92.57617143)(99.53989258,93.40283808)
\curveto(99.59322591,93.38950474)(99.69989258,93.37617141)(99.75322591,93.37617141)
\curveto(100.1132259,93.37617141)(100.36655922,93.62950474)(100.36655922,93.98950472)
\curveto(100.36655922,94.36283805)(100.08655923,94.60283804)(99.75322591,94.60283804)
\curveto(99.67322591,94.60283804)(99.13989259,94.57617137)(99.13989259,93.92283806)
\curveto(99.13989259,92.73617143)(99.84655924,91.5095048)(101.51322586,91.40283814)
\lineto(101.51322586,90.80283815)
\lineto(101.92655918,90.80283815)
\lineto(101.92655918,91.41617147)
\curveto(103.32655913,91.53617147)(104.29989244,92.77617143)(104.29989244,94.14950472)
\closepath
\moveto(103.61989246,93.74950473)
\curveto(103.61989246,92.85617143)(102.97989248,91.96283812)(101.92655918,91.82950479)
\lineto(101.92655918,95.69617134)
\curveto(103.52655913,95.36283802)(103.61989246,94.06950472)(103.61989246,93.74950473)
\closepath
\moveto(101.51322586,97.02950463)
\curveto(99.8732259,97.37617129)(99.81989257,98.52283792)(99.81989257,98.78950458)
\curveto(99.81989257,99.58950456)(100.44655922,100.4161712)(101.51322586,100.52283786)
\closepath
}
}
{
\newrgbcolor{curcolor}{0 0 0}
\pscustom[linestyle=none,fillstyle=solid,fillcolor=curcolor]
{
\newpath
\moveto(129.86004619,94.14950472)
\curveto(129.86004619,95.25617135)(129.26004621,95.93617133)(129.14004621,96.05617133)
\curveto(128.60671289,96.64283798)(128.12671291,96.76283797)(127.48671293,96.92283797)
\lineto(127.48671293,100.52283786)
\curveto(128.62004623,100.46950453)(129.26004621,99.88283788)(129.4600462,99.13617124)
\curveto(129.40671287,99.14950457)(129.3800462,99.1628379)(129.24671287,99.1628379)
\curveto(128.90004622,99.1628379)(128.63337956,98.92283791)(128.63337956,98.54950459)
\curveto(128.63337956,98.13617127)(128.96671288,97.93617127)(129.24671287,97.93617127)
\curveto(129.28671287,97.93617127)(129.86004619,97.94950461)(129.86004619,98.60283792)
\curveto(129.86004619,99.89617121)(128.98004622,100.86950452)(127.48671293,100.94950452)
\lineto(127.48671293,101.5495045)
\lineto(127.07337961,101.5495045)
\lineto(127.07337961,100.93617118)
\curveto(125.59337965,100.80283785)(124.70004634,99.60283789)(124.70004634,98.38950459)
\curveto(124.70004634,97.46950462)(125.19337966,96.82950464)(125.43337966,96.60283798)
\curveto(125.94004631,96.082838)(126.39337963,95.97617133)(127.07337961,95.802838)
\lineto(127.07337961,91.82950479)
\curveto(125.90004631,91.89617145)(125.27337966,92.57617143)(125.10004633,93.40283808)
\curveto(125.15337966,93.38950474)(125.26004633,93.37617141)(125.31337966,93.37617141)
\curveto(125.67337965,93.37617141)(125.92671297,93.62950474)(125.92671297,93.98950472)
\curveto(125.92671297,94.36283805)(125.64671298,94.60283804)(125.31337966,94.60283804)
\curveto(125.23337966,94.60283804)(124.70004634,94.57617137)(124.70004634,93.92283806)
\curveto(124.70004634,92.73617143)(125.40671299,91.5095048)(127.07337961,91.40283814)
\lineto(127.07337961,90.80283815)
\lineto(127.48671293,90.80283815)
\lineto(127.48671293,91.41617147)
\curveto(128.88671289,91.53617147)(129.86004619,92.77617143)(129.86004619,94.14950472)
\closepath
\moveto(129.18004621,93.74950473)
\curveto(129.18004621,92.85617143)(128.54004623,91.96283812)(127.48671293,91.82950479)
\lineto(127.48671293,95.69617134)
\curveto(129.08671288,95.36283802)(129.18004621,94.06950472)(129.18004621,93.74950473)
\closepath
\moveto(127.07337961,97.02950463)
\curveto(125.43337966,97.37617129)(125.38004632,98.52283792)(125.38004632,98.78950458)
\curveto(125.38004632,99.58950456)(126.00671297,100.4161712)(127.07337961,100.52283786)
\closepath
}
}
{
\newrgbcolor{curcolor}{0 0 0}
\pscustom[linestyle=none,fillstyle=solid,fillcolor=curcolor]
{
\newpath
\moveto(135.96668744,94.04283806)
\lineto(135.63335412,94.04283806)
\curveto(135.51335412,92.46950477)(135.23335413,91.88283812)(133.68668751,91.88283812)
\lineto(132.11335422,91.88283812)
\lineto(135.82002078,96.8961713)
\curveto(135.94002078,97.04283797)(135.94002078,97.06950463)(135.94002078,97.12283796)
\curveto(135.94002078,97.29617129)(135.83335411,97.29617129)(135.59335412,97.29617129)
\lineto(131.32668758,97.29617129)
\lineto(131.18002092,95.14950469)
\lineto(131.51335424,95.14950469)
\curveto(131.59335424,96.50950465)(131.84668757,97.00283797)(133.31335419,97.00283797)
\lineto(134.83335414,97.00283797)
\lineto(131.11335425,91.97617145)
\curveto(130.99335426,91.82950479)(130.99335426,91.80283812)(130.99335426,91.73617146)
\curveto(130.99335426,91.5495048)(131.08668759,91.5495048)(131.34002091,91.5495048)
\lineto(135.74002078,91.5495048)
\closepath
}
}
{
\newrgbcolor{curcolor}{0 0 0}
\pscustom[linestyle=none,fillstyle=solid,fillcolor=curcolor]
{
\newpath
\moveto(144.01999894,99.33617123)
\lineto(140.24666572,101.4695045)
\lineto(136.4733325,99.33617123)
\lineto(136.61999916,99.04283791)
\lineto(140.23333238,100.68283786)
\lineto(143.85999894,99.04283791)
\closepath
}
}
{
\newrgbcolor{curcolor}{0 0 0}
\pscustom[linestyle=none,fillstyle=solid,fillcolor=curcolor]
{
\newpath
\moveto(149.6466618,88.36283823)
\curveto(149.6466618,88.49617156)(149.5666618,88.49617156)(149.43332847,88.50950489)
\curveto(148.37999517,88.57617155)(147.88666185,89.17617154)(147.76666186,89.65617152)
\curveto(147.72666186,89.80283818)(147.72666186,89.82950485)(147.72666186,90.2961715)
\lineto(147.72666186,92.29617144)
\curveto(147.72666186,92.69617143)(147.72666186,93.37617141)(147.69999519,93.50950474)
\curveto(147.52666186,94.38950471)(146.67332856,94.73617137)(146.15332857,94.88283803)
\curveto(147.72666186,95.33617135)(147.72666186,96.28283799)(147.72666186,96.65617131)
\lineto(147.72666186,99.05617124)
\curveto(147.72666186,100.01617121)(147.72666186,100.30950454)(148.04666185,100.64283786)
\curveto(148.28666184,100.88283785)(148.5933285,101.20283784)(149.5266618,101.25617117)
\curveto(149.59332847,101.26950451)(149.6466618,101.32283784)(149.6466618,101.40283784)
\curveto(149.6466618,101.5495045)(149.53999514,101.5495045)(149.37999514,101.5495045)
\curveto(148.04666185,101.5495045)(146.85999522,100.86950452)(146.83332855,99.90950455)
\lineto(146.83332855,97.46950462)
\curveto(146.83332855,96.21617132)(146.83332855,96.002838)(146.48666189,95.62950468)
\curveto(146.29999523,95.44283801)(145.93999524,95.08283803)(145.09999527,95.02950469)
\curveto(145.00666194,95.02950469)(144.91332861,95.01617136)(144.91332861,94.88283803)
\curveto(144.91332861,94.7495047)(144.99332861,94.7495047)(145.12666194,94.73617137)
\curveto(145.69999525,94.69617137)(146.83332855,94.41617138)(146.83332855,93.08283809)
\lineto(146.83332855,90.44283816)
\curveto(146.83332855,89.66950485)(146.83332855,89.21617153)(147.52666186,88.72283822)
\curveto(148.09999518,88.32283823)(148.96666182,88.21617157)(149.37999514,88.21617157)
\curveto(149.53999514,88.21617157)(149.6466618,88.21617157)(149.6466618,88.36283823)
\closepath
}
}
{
\newrgbcolor{curcolor}{0 0 0}
\pscustom[linestyle=none,fillstyle=solid,fillcolor=curcolor]
{
\newpath
\moveto(154.29996976,94.04283806)
\lineto(154.29996976,94.81617137)
\lineto(150.76663653,94.81617137)
\lineto(150.76663653,94.04283806)
\closepath
}
}
{
\newrgbcolor{curcolor}{0 0 0}
\pscustom[linestyle=none,fillstyle=solid,fillcolor=curcolor]
{
\newpath
\moveto(160.64661238,91.5495048)
\lineto(160.64661238,91.96283812)
\lineto(160.21994573,91.96283812)
\curveto(159.01994576,91.96283812)(158.97994576,92.10950478)(158.97994576,92.6028381)
\lineto(158.97994576,100.08283788)
\curveto(158.97994576,100.40283787)(158.97994576,100.42950453)(158.67327911,100.42950453)
\curveto(157.84661246,99.57617122)(156.67327917,99.57617122)(156.24661251,99.57617122)
\lineto(156.24661251,99.1628379)
\curveto(156.51327917,99.1628379)(157.29994581,99.1628379)(157.99327913,99.50950456)
\lineto(157.99327913,92.6028381)
\curveto(157.99327913,92.12283811)(157.95327913,91.96283812)(156.75327916,91.96283812)
\lineto(156.32661251,91.96283812)
\lineto(156.32661251,91.5495048)
\curveto(156.79327916,91.5895048)(157.95327913,91.5895048)(158.48661245,91.5895048)
\curveto(159.01994576,91.5895048)(160.17994573,91.5895048)(160.64661238,91.5495048)
\closepath
}
}
{
\newrgbcolor{curcolor}{0 0 0}
\pscustom[linestyle=none,fillstyle=solid,fillcolor=curcolor]
{
\newpath
\moveto(167.41992208,94.88283803)
\curveto(167.41992208,95.01617136)(167.33992208,95.01617136)(167.20658875,95.02950469)
\curveto(166.63325544,95.06950469)(165.49992214,95.34950468)(165.49992214,96.68283798)
\lineto(165.49992214,99.3228379)
\curveto(165.49992214,100.09617121)(165.49992214,100.54950453)(164.80658883,101.04283785)
\curveto(164.23325551,101.4295045)(163.3799222,101.5495045)(162.95325555,101.5495045)
\curveto(162.81992222,101.5495045)(162.68658889,101.5495045)(162.68658889,101.40283784)
\curveto(162.68658889,101.26950451)(162.76658889,101.26950451)(162.89992222,101.25617117)
\curveto(163.95325552,101.18950451)(164.44658884,100.58950453)(164.56658883,100.10950454)
\curveto(164.60658883,99.96283788)(164.60658883,99.93617121)(164.60658883,99.46950456)
\lineto(164.60658883,97.46950462)
\curveto(164.60658883,97.06950463)(164.60658883,96.38950465)(164.6332555,96.25617132)
\curveto(164.80658883,95.37617135)(165.65992213,95.02950469)(166.17992212,94.88283803)
\curveto(164.60658883,94.42950471)(164.60658883,93.48283807)(164.60658883,93.10950475)
\lineto(164.60658883,90.70950482)
\curveto(164.60658883,89.74950485)(164.60658883,89.45617153)(164.28658884,89.1228382)
\curveto(164.04658885,88.88283821)(163.73992219,88.56283822)(162.80658889,88.50950489)
\curveto(162.73992222,88.49617156)(162.68658889,88.44283822)(162.68658889,88.36283823)
\curveto(162.68658889,88.21617157)(162.81992222,88.21617157)(162.95325555,88.21617157)
\curveto(164.28658884,88.21617157)(165.47325547,88.89617154)(165.49992214,89.85617152)
\lineto(165.49992214,92.29617144)
\curveto(165.49992214,93.54950474)(165.49992214,93.76283807)(165.8465888,94.13617139)
\curveto(166.03325546,94.32283805)(166.39325545,94.68283804)(167.23325542,94.73617137)
\curveto(167.32658875,94.73617137)(167.41992208,94.7495047)(167.41992208,94.88283803)
\closepath
}
}
{
\newrgbcolor{curcolor}{0 0 0}
\pscustom[linestyle=none,fillstyle=solid,fillcolor=curcolor]
{
\newpath
\moveto(174.29989484,94.14950472)
\curveto(174.29989484,95.25617135)(173.69989486,95.93617133)(173.57989486,96.05617133)
\curveto(173.04656155,96.64283798)(172.56656156,96.76283797)(171.92656158,96.92283797)
\lineto(171.92656158,100.52283786)
\curveto(173.05989488,100.46950453)(173.69989486,99.88283788)(173.89989485,99.13617124)
\curveto(173.84656152,99.14950457)(173.81989486,99.1628379)(173.68656153,99.1628379)
\curveto(173.33989487,99.1628379)(173.07322821,98.92283791)(173.07322821,98.54950459)
\curveto(173.07322821,98.13617127)(173.40656154,97.93617127)(173.68656153,97.93617127)
\curveto(173.72656153,97.93617127)(174.29989484,97.94950461)(174.29989484,98.60283792)
\curveto(174.29989484,99.89617121)(173.41989487,100.86950452)(171.92656158,100.94950452)
\lineto(171.92656158,101.5495045)
\lineto(171.51322826,101.5495045)
\lineto(171.51322826,100.93617118)
\curveto(170.0332283,100.80283785)(169.139895,99.60283789)(169.139895,98.38950459)
\curveto(169.139895,97.46950462)(169.63322832,96.82950464)(169.87322831,96.60283798)
\curveto(170.37989496,96.082838)(170.83322828,95.97617133)(171.51322826,95.802838)
\lineto(171.51322826,91.82950479)
\curveto(170.33989496,91.89617145)(169.71322831,92.57617143)(169.53989498,93.40283808)
\curveto(169.59322832,93.38950474)(169.69989498,93.37617141)(169.75322831,93.37617141)
\curveto(170.1132283,93.37617141)(170.36656163,93.62950474)(170.36656163,93.98950472)
\curveto(170.36656163,94.36283805)(170.08656163,94.60283804)(169.75322831,94.60283804)
\curveto(169.67322831,94.60283804)(169.139895,94.57617137)(169.139895,93.92283806)
\curveto(169.139895,92.73617143)(169.84656164,91.5095048)(171.51322826,91.40283814)
\lineto(171.51322826,90.80283815)
\lineto(171.92656158,90.80283815)
\lineto(171.92656158,91.41617147)
\curveto(173.32656154,91.53617147)(174.29989484,92.77617143)(174.29989484,94.14950472)
\closepath
\moveto(173.61989486,93.74950473)
\curveto(173.61989486,92.85617143)(172.97989488,91.96283812)(171.92656158,91.82950479)
\lineto(171.92656158,95.69617134)
\curveto(173.52656153,95.36283802)(173.61989486,94.06950472)(173.61989486,93.74950473)
\closepath
\moveto(171.51322826,97.02950463)
\curveto(169.87322831,97.37617129)(169.81989498,98.52283792)(169.81989498,98.78950458)
\curveto(169.81989498,99.58950456)(170.44656162,100.4161712)(171.51322826,100.52283786)
\closepath
}
}
{
\newrgbcolor{curcolor}{0 0 0}
\pscustom[linestyle=none,fillstyle=solid,fillcolor=curcolor]
{
\newpath
\moveto(199.8600558,94.14950472)
\curveto(199.8600558,95.25617135)(199.26005582,95.93617133)(199.14005582,96.05617133)
\curveto(198.60672251,96.64283798)(198.12672252,96.76283797)(197.48672254,96.92283797)
\lineto(197.48672254,100.52283786)
\curveto(198.62005584,100.46950453)(199.26005582,99.88283788)(199.46005581,99.13617124)
\curveto(199.40672248,99.14950457)(199.38005582,99.1628379)(199.24672249,99.1628379)
\curveto(198.90005583,99.1628379)(198.63338917,98.92283791)(198.63338917,98.54950459)
\curveto(198.63338917,98.13617127)(198.96672249,97.93617127)(199.24672249,97.93617127)
\curveto(199.28672249,97.93617127)(199.8600558,97.94950461)(199.8600558,98.60283792)
\curveto(199.8600558,99.89617121)(198.98005583,100.86950452)(197.48672254,100.94950452)
\lineto(197.48672254,101.5495045)
\lineto(197.07338922,101.5495045)
\lineto(197.07338922,100.93617118)
\curveto(195.59338926,100.80283785)(194.70005596,99.60283789)(194.70005596,98.38950459)
\curveto(194.70005596,97.46950462)(195.19338927,96.82950464)(195.43338927,96.60283798)
\curveto(195.94005592,96.082838)(196.39338924,95.97617133)(197.07338922,95.802838)
\lineto(197.07338922,91.82950479)
\curveto(195.90005592,91.89617145)(195.27338927,92.57617143)(195.10005594,93.40283808)
\curveto(195.15338928,93.38950474)(195.26005594,93.37617141)(195.31338927,93.37617141)
\curveto(195.67338926,93.37617141)(195.92672259,93.62950474)(195.92672259,93.98950472)
\curveto(195.92672259,94.36283805)(195.64672259,94.60283804)(195.31338927,94.60283804)
\curveto(195.23338927,94.60283804)(194.70005596,94.57617137)(194.70005596,93.92283806)
\curveto(194.70005596,92.73617143)(195.4067226,91.5095048)(197.07338922,91.40283814)
\lineto(197.07338922,90.80283815)
\lineto(197.48672254,90.80283815)
\lineto(197.48672254,91.41617147)
\curveto(198.8867225,91.53617147)(199.8600558,92.77617143)(199.8600558,94.14950472)
\closepath
\moveto(199.18005582,93.74950473)
\curveto(199.18005582,92.85617143)(198.54005584,91.96283812)(197.48672254,91.82950479)
\lineto(197.48672254,95.69617134)
\curveto(199.08672249,95.36283802)(199.18005582,94.06950472)(199.18005582,93.74950473)
\closepath
\moveto(197.07338922,97.02950463)
\curveto(195.43338927,97.37617129)(195.38005594,98.52283792)(195.38005594,98.78950458)
\curveto(195.38005594,99.58950456)(196.00672258,100.4161712)(197.07338922,100.52283786)
\closepath
}
}
{
\newrgbcolor{curcolor}{0 0 0}
\pscustom[linestyle=none,fillstyle=solid,fillcolor=curcolor]
{
\newpath
\moveto(205.96669705,94.04283806)
\lineto(205.63336373,94.04283806)
\curveto(205.51336373,92.46950477)(205.23336374,91.88283812)(203.68669712,91.88283812)
\lineto(202.11336384,91.88283812)
\lineto(205.82003039,96.8961713)
\curveto(205.94003039,97.04283797)(205.94003039,97.06950463)(205.94003039,97.12283796)
\curveto(205.94003039,97.29617129)(205.83336372,97.29617129)(205.59336373,97.29617129)
\lineto(201.32669719,97.29617129)
\lineto(201.18003053,95.14950469)
\lineto(201.51336385,95.14950469)
\curveto(201.59336385,96.50950465)(201.84669718,97.00283797)(203.3133638,97.00283797)
\lineto(204.83336375,97.00283797)
\lineto(201.11336387,91.97617145)
\curveto(200.99336387,91.82950479)(200.99336387,91.80283812)(200.99336387,91.73617146)
\curveto(200.99336387,91.5495048)(201.0866972,91.5495048)(201.34003053,91.5495048)
\lineto(205.74003039,91.5495048)
\closepath
}
}
{
\newrgbcolor{curcolor}{0 0 0}
\pscustom[linestyle=none,fillstyle=solid,fillcolor=curcolor]
{
\newpath
\moveto(214.02000855,99.33617123)
\lineto(210.24667533,101.4695045)
\lineto(206.47334211,99.33617123)
\lineto(206.62000877,99.04283791)
\lineto(210.23334199,100.68283786)
\lineto(213.86000855,99.04283791)
\closepath
}
}
{
\newrgbcolor{curcolor}{0 0 0}
\pscustom[linestyle=none,fillstyle=solid,fillcolor=curcolor]
{
\newpath
\moveto(219.64667141,88.36283823)
\curveto(219.64667141,88.49617156)(219.56667141,88.49617156)(219.43333808,88.50950489)
\curveto(218.38000478,88.57617155)(217.88667146,89.17617154)(217.76667147,89.65617152)
\curveto(217.72667147,89.80283818)(217.72667147,89.82950485)(217.72667147,90.2961715)
\lineto(217.72667147,92.29617144)
\curveto(217.72667147,92.69617143)(217.72667147,93.37617141)(217.7000048,93.50950474)
\curveto(217.52667148,94.38950471)(216.67333817,94.73617137)(216.15333818,94.88283803)
\curveto(217.72667147,95.33617135)(217.72667147,96.28283799)(217.72667147,96.65617131)
\lineto(217.72667147,99.05617124)
\curveto(217.72667147,100.01617121)(217.72667147,100.30950454)(218.04667146,100.64283786)
\curveto(218.28667145,100.88283785)(218.59333811,101.20283784)(219.52667142,101.25617117)
\curveto(219.59333808,101.26950451)(219.64667141,101.32283784)(219.64667141,101.40283784)
\curveto(219.64667141,101.5495045)(219.54000475,101.5495045)(219.38000475,101.5495045)
\curveto(218.04667146,101.5495045)(216.86000483,100.86950452)(216.83333816,99.90950455)
\lineto(216.83333816,97.46950462)
\curveto(216.83333816,96.21617132)(216.83333816,96.002838)(216.48667151,95.62950468)
\curveto(216.30000485,95.44283801)(215.94000486,95.08283803)(215.10000488,95.02950469)
\curveto(215.00667155,95.02950469)(214.91333822,95.01617136)(214.91333822,94.88283803)
\curveto(214.91333822,94.7495047)(214.99333822,94.7495047)(215.12667155,94.73617137)
\curveto(215.70000486,94.69617137)(216.83333816,94.41617138)(216.83333816,93.08283809)
\lineto(216.83333816,90.44283816)
\curveto(216.83333816,89.66950485)(216.83333816,89.21617153)(217.52667148,88.72283822)
\curveto(218.10000479,88.32283823)(218.96667143,88.21617157)(219.38000475,88.21617157)
\curveto(219.54000475,88.21617157)(219.64667141,88.21617157)(219.64667141,88.36283823)
\closepath
}
}
{
\newrgbcolor{curcolor}{0 0 0}
\pscustom[linestyle=none,fillstyle=solid,fillcolor=curcolor]
{
\newpath
\moveto(224.29997937,94.04283806)
\lineto(224.29997937,94.81617137)
\lineto(220.76664615,94.81617137)
\lineto(220.76664615,94.04283806)
\closepath
}
}
{
\newrgbcolor{curcolor}{0 0 0}
\pscustom[linestyle=none,fillstyle=solid,fillcolor=curcolor]
{
\newpath
\moveto(230.64662199,91.5495048)
\lineto(230.64662199,91.96283812)
\lineto(230.21995534,91.96283812)
\curveto(229.01995537,91.96283812)(228.97995538,92.10950478)(228.97995538,92.6028381)
\lineto(228.97995538,100.08283788)
\curveto(228.97995538,100.40283787)(228.97995538,100.42950453)(228.67328872,100.42950453)
\curveto(227.84662208,99.57617122)(226.67328878,99.57617122)(226.24662212,99.57617122)
\lineto(226.24662212,99.1628379)
\curveto(226.51328878,99.1628379)(227.29995543,99.1628379)(227.99328874,99.50950456)
\lineto(227.99328874,92.6028381)
\curveto(227.99328874,92.12283811)(227.95328874,91.96283812)(226.75328878,91.96283812)
\lineto(226.32662212,91.96283812)
\lineto(226.32662212,91.5495048)
\curveto(226.79328877,91.5895048)(227.95328874,91.5895048)(228.48662206,91.5895048)
\curveto(229.01995537,91.5895048)(230.17995534,91.5895048)(230.64662199,91.5495048)
\closepath
}
}
{
\newrgbcolor{curcolor}{0 0 0}
\pscustom[linestyle=none,fillstyle=solid,fillcolor=curcolor]
{
\newpath
\moveto(237.41993169,94.88283803)
\curveto(237.41993169,95.01617136)(237.3399317,95.01617136)(237.20659837,95.02950469)
\curveto(236.63326505,95.06950469)(235.49993175,95.34950468)(235.49993175,96.68283798)
\lineto(235.49993175,99.3228379)
\curveto(235.49993175,100.09617121)(235.49993175,100.54950453)(234.80659844,101.04283785)
\curveto(234.23326512,101.4295045)(233.37993181,101.5495045)(232.95326516,101.5495045)
\curveto(232.81993183,101.5495045)(232.6865985,101.5495045)(232.6865985,101.40283784)
\curveto(232.6865985,101.26950451)(232.7665985,101.26950451)(232.89993183,101.25617117)
\curveto(233.95326513,101.18950451)(234.44659845,100.58950453)(234.56659845,100.10950454)
\curveto(234.60659844,99.96283788)(234.60659844,99.93617121)(234.60659844,99.46950456)
\lineto(234.60659844,97.46950462)
\curveto(234.60659844,97.06950463)(234.60659844,96.38950465)(234.63326511,96.25617132)
\curveto(234.80659844,95.37617135)(235.65993175,95.02950469)(236.17993173,94.88283803)
\curveto(234.60659844,94.42950471)(234.60659844,93.48283807)(234.60659844,93.10950475)
\lineto(234.60659844,90.70950482)
\curveto(234.60659844,89.74950485)(234.60659844,89.45617153)(234.28659845,89.1228382)
\curveto(234.04659846,88.88283821)(233.7399318,88.56283822)(232.8065985,88.50950489)
\curveto(232.73993183,88.49617156)(232.6865985,88.44283822)(232.6865985,88.36283823)
\curveto(232.6865985,88.21617157)(232.81993183,88.21617157)(232.95326516,88.21617157)
\curveto(234.28659845,88.21617157)(235.47326509,88.89617154)(235.49993175,89.85617152)
\lineto(235.49993175,92.29617144)
\curveto(235.49993175,93.54950474)(235.49993175,93.76283807)(235.84659841,94.13617139)
\curveto(236.03326507,94.32283805)(236.39326506,94.68283804)(237.23326503,94.73617137)
\curveto(237.32659836,94.73617137)(237.41993169,94.7495047)(237.41993169,94.88283803)
\closepath
}
}
{
\newrgbcolor{curcolor}{0 0 0}
\pscustom[linestyle=none,fillstyle=solid,fillcolor=curcolor]
{
\newpath
\moveto(244.29990445,94.14950472)
\curveto(244.29990445,95.25617135)(243.69990447,95.93617133)(243.57990447,96.05617133)
\curveto(243.04657116,96.64283798)(242.56657117,96.76283797)(241.92657119,96.92283797)
\lineto(241.92657119,100.52283786)
\curveto(243.05990449,100.46950453)(243.69990447,99.88283788)(243.89990447,99.13617124)
\curveto(243.84657113,99.14950457)(243.81990447,99.1628379)(243.68657114,99.1628379)
\curveto(243.33990448,99.1628379)(243.07323782,98.92283791)(243.07323782,98.54950459)
\curveto(243.07323782,98.13617127)(243.40657115,97.93617127)(243.68657114,97.93617127)
\curveto(243.72657114,97.93617127)(244.29990445,97.94950461)(244.29990445,98.60283792)
\curveto(244.29990445,99.89617121)(243.41990448,100.86950452)(241.92657119,100.94950452)
\lineto(241.92657119,101.5495045)
\lineto(241.51323787,101.5495045)
\lineto(241.51323787,100.93617118)
\curveto(240.03323791,100.80283785)(239.13990461,99.60283789)(239.13990461,98.38950459)
\curveto(239.13990461,97.46950462)(239.63323793,96.82950464)(239.87323792,96.60283798)
\curveto(240.37990457,96.082838)(240.83323789,95.97617133)(241.51323787,95.802838)
\lineto(241.51323787,91.82950479)
\curveto(240.33990457,91.89617145)(239.71323792,92.57617143)(239.5399046,93.40283808)
\curveto(239.59323793,93.38950474)(239.69990459,93.37617141)(239.75323792,93.37617141)
\curveto(240.11323791,93.37617141)(240.36657124,93.62950474)(240.36657124,93.98950472)
\curveto(240.36657124,94.36283805)(240.08657125,94.60283804)(239.75323792,94.60283804)
\curveto(239.67323793,94.60283804)(239.13990461,94.57617137)(239.13990461,93.92283806)
\curveto(239.13990461,92.73617143)(239.84657125,91.5095048)(241.51323787,91.40283814)
\lineto(241.51323787,90.80283815)
\lineto(241.92657119,90.80283815)
\lineto(241.92657119,91.41617147)
\curveto(243.32657115,91.53617147)(244.29990445,92.77617143)(244.29990445,94.14950472)
\closepath
\moveto(243.61990447,93.74950473)
\curveto(243.61990447,92.85617143)(242.97990449,91.96283812)(241.92657119,91.82950479)
\lineto(241.92657119,95.69617134)
\curveto(243.52657114,95.36283802)(243.61990447,94.06950472)(243.61990447,93.74950473)
\closepath
\moveto(241.51323787,97.02950463)
\curveto(239.87323792,97.37617129)(239.81990459,98.52283792)(239.81990459,98.78950458)
\curveto(239.81990459,99.58950456)(240.44657124,100.4161712)(241.51323787,100.52283786)
\closepath
}
}
{
\newrgbcolor{curcolor}{0 0 0}
\pscustom[linestyle=none,fillstyle=solid,fillcolor=curcolor]
{
\newpath
\moveto(64.15330516,57.88816275)
\curveto(64.15330516,58.99482939)(63.55330518,59.67482937)(63.43330519,59.79482936)
\curveto(62.89997187,60.38149601)(62.41997188,60.50149601)(61.7799719,60.661496)
\lineto(61.7799719,64.2614959)
\curveto(62.9133052,64.20816256)(63.55330518,63.62149592)(63.75330518,62.87482927)
\curveto(63.69997184,62.8881626)(63.67330518,62.90149594)(63.53997185,62.90149594)
\curveto(63.19330519,62.90149594)(62.92663853,62.66149594)(62.92663853,62.28816262)
\curveto(62.92663853,61.8748293)(63.25997186,61.67482931)(63.53997185,61.67482931)
\curveto(63.57997185,61.67482931)(64.15330516,61.68816264)(64.15330516,62.34149595)
\curveto(64.15330516,63.63482925)(63.27330519,64.60816255)(61.7799719,64.68816255)
\lineto(61.7799719,65.28816253)
\lineto(61.36663858,65.28816253)
\lineto(61.36663858,64.67482922)
\curveto(59.88663863,64.54149589)(58.99330532,63.34149592)(58.99330532,62.12816263)
\curveto(58.99330532,61.20816265)(59.48663864,60.56816267)(59.72663863,60.34149601)
\curveto(60.23330528,59.82149603)(60.6866386,59.71482937)(61.36663858,59.54149604)
\lineto(61.36663858,55.56816282)
\curveto(60.19330528,55.63482949)(59.56663864,56.31482947)(59.39330531,57.14149611)
\curveto(59.44663864,57.12816278)(59.5533053,57.11482944)(59.60663863,57.11482944)
\curveto(59.96663862,57.11482944)(60.21997195,57.36816277)(60.21997195,57.72816276)
\curveto(60.21997195,58.10149608)(59.93997196,58.34149607)(59.60663863,58.34149607)
\curveto(59.52663864,58.34149607)(58.99330532,58.31482941)(58.99330532,57.66149609)
\curveto(58.99330532,56.47482946)(59.69997197,55.24816283)(61.36663858,55.14149617)
\lineto(61.36663858,54.54149619)
\lineto(61.7799719,54.54149619)
\lineto(61.7799719,55.1548295)
\curveto(63.17997186,55.2748295)(64.15330516,56.51482946)(64.15330516,57.88816275)
\closepath
\moveto(63.47330519,57.48816277)
\curveto(63.47330519,56.59482946)(62.8333052,55.70149615)(61.7799719,55.56816282)
\lineto(61.7799719,59.43482937)
\curveto(63.37997185,59.10149605)(63.47330519,57.80816276)(63.47330519,57.48816277)
\closepath
\moveto(61.36663858,60.76816267)
\curveto(59.72663863,61.11482932)(59.6733053,62.26149596)(59.6733053,62.52816261)
\curveto(59.6733053,63.32816259)(60.29997195,64.15482923)(61.36663858,64.2614959)
\closepath
}
}
{
\newrgbcolor{curcolor}{0 0 0}
\pscustom[linestyle=none,fillstyle=solid,fillcolor=curcolor]
{
\newpath
\moveto(74.28661296,60.62149601)
\lineto(74.28661296,61.03482933)
\curveto(73.99327964,61.00816266)(73.60661298,60.99482933)(73.31327966,60.99482933)
\lineto(72.0732797,61.03482933)
\lineto(72.0732797,60.62149601)
\curveto(72.55327968,60.60816267)(72.84661301,60.36816268)(72.84661301,59.98149602)
\curveto(72.84661301,59.90149603)(72.84661301,59.87482936)(72.77994634,59.70149603)
\lineto(71.56661304,56.2881628)
\lineto(70.24661308,60.00816269)
\curveto(70.19327975,60.16816269)(70.17994642,60.19482935)(70.17994642,60.26149602)
\curveto(70.17994642,60.62149601)(70.6999464,60.62149601)(70.96661306,60.62149601)
\lineto(70.96661306,61.03482933)
\lineto(69.57994644,60.99482933)
\curveto(69.17994645,60.99482933)(68.79327979,61.00816266)(68.39327981,61.03482933)
\lineto(68.39327981,60.62149601)
\curveto(68.88661312,60.62149601)(69.09994645,60.59482934)(69.23327978,60.42149601)
\curveto(69.29994645,60.34149601)(69.44661311,59.94149603)(69.53994644,59.6881627)
\lineto(68.39327981,56.46149613)
\lineto(67.12661318,60.02149602)
\curveto(67.05994651,60.18149602)(67.05994651,60.20816268)(67.05994651,60.26149602)
\curveto(67.05994651,60.62149601)(67.5799465,60.62149601)(67.84661316,60.62149601)
\lineto(67.84661316,61.03482933)
\lineto(66.39327987,60.99482933)
\lineto(65.1532799,61.03482933)
\lineto(65.1532799,60.62149601)
\curveto(65.81994655,60.62149601)(65.97994655,60.58149601)(66.13994654,60.15482935)
\lineto(67.81994649,55.43482949)
\curveto(67.88661315,55.24816283)(67.92661315,55.14149617)(68.09994648,55.14149617)
\curveto(68.27327981,55.14149617)(68.29994648,55.22149617)(68.36661314,55.40816283)
\lineto(69.71327977,59.18149605)
\lineto(71.07327973,55.3948295)
\curveto(71.12661306,55.24816283)(71.16661306,55.14149617)(71.33994638,55.14149617)
\curveto(71.51327971,55.14149617)(71.55327971,55.26149617)(71.60661304,55.3948295)
\lineto(73.166613,59.7681627)
\curveto(73.40661299,60.43482934)(73.81994631,60.60816267)(74.28661296,60.62149601)
\closepath
}
}
{
\newrgbcolor{curcolor}{0 0 0}
\pscustom[linestyle=none,fillstyle=solid,fillcolor=curcolor]
{
\newpath
\moveto(84.53992167,53.44816289)
\lineto(84.53992167,53.79482954)
\lineto(74.53992197,53.79482954)
\lineto(74.53992197,53.44816289)
\closepath
}
}
{
\newrgbcolor{curcolor}{0 0 0}
\pscustom[linestyle=none,fillstyle=solid,fillcolor=curcolor]
{
\newpath
\moveto(90.67323049,59.55482937)
\curveto(90.67323049,60.62149601)(90.60656383,61.68816264)(90.13989717,62.67482928)
\curveto(89.52656386,63.95482924)(88.43323056,64.16816257)(87.87323058,64.16816257)
\curveto(87.0732306,64.16816257)(86.0998973,63.82149591)(85.55323065,62.58149595)
\curveto(85.12656399,61.66149597)(85.05989733,60.62149601)(85.05989733,59.55482937)
\curveto(85.05989733,58.5548294)(85.11323066,57.35482944)(85.65989731,56.34149613)
\curveto(86.23323062,55.26149617)(87.20656393,54.99482951)(87.85989724,54.99482951)
\curveto(88.57989722,54.99482951)(89.59323052,55.2748295)(90.17989717,56.54149613)
\curveto(90.60656383,57.4614961)(90.67323049,58.50149607)(90.67323049,59.55482937)
\closepath
\moveto(89.56656386,59.71482937)
\curveto(89.56656386,58.7148294)(89.56656386,57.80816276)(89.4198972,56.95482945)
\curveto(89.2198972,55.68816282)(88.45989722,55.28816283)(87.85989724,55.28816283)
\curveto(87.33989726,55.28816283)(86.55323062,55.62149616)(86.31323062,56.90149612)
\curveto(86.16656396,57.70149609)(86.16656396,58.92816272)(86.16656396,59.71482937)
\curveto(86.16656396,60.56816267)(86.16656396,61.44816265)(86.27323062,62.16816263)
\curveto(86.52656395,63.75482924)(87.52656392,63.87482924)(87.85989724,63.87482924)
\curveto(88.29989723,63.87482924)(89.1798972,63.63482925)(89.43323053,62.31482929)
\curveto(89.56656386,61.56816264)(89.56656386,60.55482934)(89.56656386,59.71482937)
\closepath
}
}
{
\newrgbcolor{curcolor}{0 0 0}
\pscustom[linestyle=none,fillstyle=solid,fillcolor=curcolor]
{
\newpath
\moveto(97.11320507,57.88816275)
\curveto(97.11320507,58.99482939)(96.51320509,59.67482937)(96.39320509,59.79482936)
\curveto(95.85987177,60.38149601)(95.37987179,60.50149601)(94.7398718,60.661496)
\lineto(94.7398718,64.2614959)
\curveto(95.8732051,64.20816256)(96.51320509,63.62149592)(96.71320508,62.87482927)
\curveto(96.65987175,62.8881626)(96.63320508,62.90149594)(96.49987175,62.90149594)
\curveto(96.1532051,62.90149594)(95.88653844,62.66149594)(95.88653844,62.28816262)
\curveto(95.88653844,61.8748293)(96.21987176,61.67482931)(96.49987175,61.67482931)
\curveto(96.53987175,61.67482931)(97.11320507,61.68816264)(97.11320507,62.34149595)
\curveto(97.11320507,63.63482925)(96.23320509,64.60816255)(94.7398718,64.68816255)
\lineto(94.7398718,65.28816253)
\lineto(94.32653848,65.28816253)
\lineto(94.32653848,64.67482922)
\curveto(92.84653853,64.54149589)(91.95320522,63.34149592)(91.95320522,62.12816263)
\curveto(91.95320522,61.20816265)(92.44653854,60.56816267)(92.68653853,60.34149601)
\curveto(93.19320518,59.82149603)(93.6465385,59.71482937)(94.32653848,59.54149604)
\lineto(94.32653848,55.56816282)
\curveto(93.15320519,55.63482949)(92.52653854,56.31482947)(92.35320521,57.14149611)
\curveto(92.40653854,57.12816278)(92.51320521,57.11482944)(92.56653854,57.11482944)
\curveto(92.92653853,57.11482944)(93.17987185,57.36816277)(93.17987185,57.72816276)
\curveto(93.17987185,58.10149608)(92.89987186,58.34149607)(92.56653854,58.34149607)
\curveto(92.48653854,58.34149607)(91.95320522,58.31482941)(91.95320522,57.66149609)
\curveto(91.95320522,56.47482946)(92.65987187,55.24816283)(94.32653848,55.14149617)
\lineto(94.32653848,54.54149619)
\lineto(94.7398718,54.54149619)
\lineto(94.7398718,55.1548295)
\curveto(96.13987176,55.2748295)(97.11320507,56.51482946)(97.11320507,57.88816275)
\closepath
\moveto(96.43320509,57.48816277)
\curveto(96.43320509,56.59482946)(95.79320511,55.70149615)(94.7398718,55.56816282)
\lineto(94.7398718,59.43482937)
\curveto(96.33987176,59.10149605)(96.43320509,57.80816276)(96.43320509,57.48816277)
\closepath
\moveto(94.32653848,60.76816267)
\curveto(92.68653853,61.11482932)(92.6332052,62.26149596)(92.6332052,62.52816261)
\curveto(92.6332052,63.32816259)(93.25987185,64.15482923)(94.32653848,64.2614959)
\closepath
}
}
{
\newrgbcolor{curcolor}{0 0 0}
\pscustom[linestyle=none,fillstyle=solid,fillcolor=curcolor]
{
\newpath
\moveto(22.32009089,56.63485628)
\curveto(22.32009089,57.74152292)(21.72009091,58.4215229)(21.60009091,58.54152289)
\curveto(21.0667576,59.12818954)(20.58675761,59.24818954)(19.94675763,59.40818953)
\lineto(19.94675763,63.00818942)
\curveto(21.08009093,62.95485609)(21.72009091,62.36818944)(21.9200909,61.6215228)
\curveto(21.86675757,61.63485613)(21.84009091,61.64818947)(21.70675758,61.64818947)
\curveto(21.36009092,61.64818947)(21.09342426,61.40818947)(21.09342426,61.03485615)
\curveto(21.09342426,60.62152283)(21.42675758,60.42152284)(21.70675758,60.42152284)
\curveto(21.74675758,60.42152284)(22.32009089,60.43485617)(22.32009089,61.08818948)
\curveto(22.32009089,62.38152278)(21.44009092,63.35485608)(19.94675763,63.43485608)
\lineto(19.94675763,64.03485606)
\lineto(19.53342431,64.03485606)
\lineto(19.53342431,63.42152275)
\curveto(18.05342435,63.28818942)(17.16009105,62.08818945)(17.16009105,60.87485616)
\curveto(17.16009105,59.95485618)(17.65342436,59.3148562)(17.89342436,59.08818954)
\curveto(18.40009101,58.56818956)(18.85342433,58.46152289)(19.53342431,58.28818957)
\lineto(19.53342431,54.31485635)
\curveto(18.36009101,54.38152302)(17.73342436,55.061523)(17.56009103,55.88818964)
\curveto(17.61342437,55.87485631)(17.72009103,55.86152297)(17.77342436,55.86152297)
\curveto(18.13342435,55.86152297)(18.38675768,56.1148563)(18.38675768,56.47485629)
\curveto(18.38675768,56.84818961)(18.10675768,57.0881896)(17.77342436,57.0881896)
\curveto(17.69342436,57.0881896)(17.16009105,57.06152294)(17.16009105,56.40818962)
\curveto(17.16009105,55.22152299)(17.86675769,53.99485636)(19.53342431,53.8881897)
\lineto(19.53342431,53.28818972)
\lineto(19.94675763,53.28818972)
\lineto(19.94675763,53.90152303)
\curveto(21.34675759,54.02152303)(22.32009089,55.26152299)(22.32009089,56.63485628)
\closepath
\moveto(21.64009091,56.23485629)
\curveto(21.64009091,55.34152299)(21.00009093,54.44818968)(19.94675763,54.31485635)
\lineto(19.94675763,58.1815229)
\curveto(21.54675758,57.84818958)(21.64009091,56.55485629)(21.64009091,56.23485629)
\closepath
\moveto(19.53342431,59.5148562)
\curveto(17.89342436,59.86152285)(17.84009103,61.00818948)(17.84009103,61.27485614)
\curveto(17.84009103,62.07485612)(18.46675767,62.90152276)(19.53342431,63.00818942)
\closepath
}
}
{
\newrgbcolor{curcolor}{0 0 0}
\pscustom[linestyle=none,fillstyle=solid,fillcolor=curcolor]
{
\newpath
\moveto(28.98673213,50.96818979)
\curveto(28.98673213,50.96818979)(28.98673213,51.03485645)(28.92006546,51.20818978)
\lineto(24.38673226,63.74152274)
\curveto(24.33339893,63.8881894)(24.2800656,64.03485606)(24.09339894,64.03485606)
\curveto(23.94673228,64.03485606)(23.82673228,63.91485606)(23.82673228,63.7681894)
\curveto(23.82673228,63.7681894)(23.82673228,63.70152274)(23.89339895,63.52818941)
\lineto(28.42673214,50.99485645)
\curveto(28.48006548,50.84818979)(28.53339881,50.70152313)(28.72006547,50.70152313)
\curveto(28.86673213,50.70152313)(28.98673213,50.82152312)(28.98673213,50.96818979)
\closepath
}
}
{
\newrgbcolor{curcolor}{0 0 0}
\pscustom[linestyle=none,fillstyle=solid,fillcolor=curcolor]
{
\newpath
\moveto(34.17337341,55.68818964)
\lineto(34.17337341,56.44818962)
\lineto(33.84004008,56.44818962)
\lineto(33.84004008,55.71485631)
\curveto(33.84004008,54.72818967)(33.44004009,54.22152302)(32.94670678,54.22152302)
\curveto(32.05337347,54.22152302)(32.05337347,55.43485632)(32.05337347,55.66152298)
\lineto(32.05337347,59.36818953)
\lineto(33.96004008,59.36818953)
\lineto(33.96004008,59.78152285)
\lineto(32.05337347,59.78152285)
\lineto(32.05337347,62.23485611)
\lineto(31.72004015,62.23485611)
\curveto(31.70670681,61.14152281)(31.30670683,59.71485619)(30.0000402,59.66152286)
\lineto(30.0000402,59.36818953)
\lineto(31.1333735,59.36818953)
\lineto(31.1333735,55.68818964)
\curveto(31.1333735,54.04818969)(32.37337346,53.8881897)(32.85337345,53.8881897)
\curveto(33.80004008,53.8881897)(34.17337341,54.83485634)(34.17337341,55.68818964)
\closepath
}
}
{
\newrgbcolor{curcolor}{0 0 0}
\pscustom[linestyle=none,fillstyle=solid,fillcolor=curcolor]
{
\newpath
\moveto(38.22670616,54.03485636)
\lineto(38.22670616,54.44818968)
\curveto(37.34670619,54.44818968)(37.29337285,54.51485635)(37.29337285,55.03485633)
\lineto(37.29337285,59.92818952)
\lineto(35.42670624,59.78152285)
\lineto(35.42670624,59.36818953)
\curveto(36.29337288,59.36818953)(36.41337288,59.28818954)(36.41337288,58.63485622)
\lineto(36.41337288,55.04818966)
\curveto(36.41337288,54.44818968)(36.26670622,54.44818968)(35.37337291,54.44818968)
\lineto(35.37337291,54.03485636)
\lineto(36.84003954,54.07485636)
\curveto(37.30670619,54.07485636)(37.77337284,54.04818969)(38.22670616,54.03485636)
\closepath
\moveto(37.49337285,62.08818945)
\curveto(37.49337285,62.44818944)(37.18670619,62.7948561)(36.7867062,62.7948561)
\curveto(36.33337288,62.7948561)(36.06670622,62.42152278)(36.06670622,62.08818945)
\curveto(36.06670622,61.72818946)(36.37337288,61.38152281)(36.77337287,61.38152281)
\curveto(37.22670619,61.38152281)(37.49337285,61.75485613)(37.49337285,62.08818945)
\closepath
}
}
{
\newrgbcolor{curcolor}{0 0 0}
\pscustom[linestyle=none,fillstyle=solid,fillcolor=curcolor]
{
\newpath
\moveto(49.48003644,54.03485636)
\lineto(49.48003644,54.44818968)
\curveto(48.78670313,54.44818968)(48.4533698,54.44818968)(48.44003647,54.84818967)
\lineto(48.44003647,57.39485626)
\curveto(48.44003647,58.54152289)(48.44003647,58.95485621)(48.02670315,59.4348562)
\curveto(47.84003649,59.66152286)(47.4000365,59.92818952)(46.62670319,59.92818952)
\curveto(45.50670322,59.92818952)(44.92003657,59.12818954)(44.69336991,58.62152289)
\curveto(44.50670325,59.78152285)(43.52003662,59.92818952)(42.92003663,59.92818952)
\curveto(41.94670333,59.92818952)(41.32003668,59.3548562)(40.94670336,58.52818956)
\lineto(40.94670336,59.92818952)
\lineto(39.06670342,59.78152285)
\lineto(39.06670342,59.36818953)
\curveto(40.00003672,59.36818953)(40.10670339,59.2748562)(40.10670339,58.62152289)
\lineto(40.10670339,55.04818966)
\curveto(40.10670339,54.44818968)(39.96003672,54.44818968)(39.06670342,54.44818968)
\lineto(39.06670342,54.03485636)
\lineto(40.57337004,54.07485636)
\lineto(42.06670333,54.03485636)
\lineto(42.06670333,54.44818968)
\curveto(41.17337002,54.44818968)(41.02670336,54.44818968)(41.02670336,55.04818966)
\lineto(41.02670336,57.50152292)
\curveto(41.02670336,58.88818955)(41.97337,59.63485619)(42.8267033,59.63485619)
\curveto(43.66670328,59.63485619)(43.81336994,58.91485621)(43.81336994,58.15485624)
\lineto(43.81336994,55.04818966)
\curveto(43.81336994,54.44818968)(43.66670328,54.44818968)(42.77336997,54.44818968)
\lineto(42.77336997,54.03485636)
\lineto(44.28003659,54.07485636)
\lineto(45.77336988,54.03485636)
\lineto(45.77336988,54.44818968)
\curveto(44.88003658,54.44818968)(44.73336991,54.44818968)(44.73336991,55.04818966)
\lineto(44.73336991,57.50152292)
\curveto(44.73336991,58.88818955)(45.68003655,59.63485619)(46.53336986,59.63485619)
\curveto(47.37336983,59.63485619)(47.5200365,58.91485621)(47.5200365,58.15485624)
\lineto(47.5200365,55.04818966)
\curveto(47.5200365,54.44818968)(47.37336983,54.44818968)(46.48003653,54.44818968)
\lineto(46.48003653,54.03485636)
\lineto(47.98670315,54.07485636)
\closepath
}
}
{
\newrgbcolor{curcolor}{0 0 0}
\pscustom[linestyle=none,fillstyle=solid,fillcolor=curcolor]
{
\newpath
\moveto(55.28001241,55.62152298)
\curveto(55.28001241,55.75485631)(55.17334574,55.78152297)(55.10667908,55.78152297)
\curveto(54.98667908,55.78152297)(54.96001242,55.70152298)(54.93334575,55.59485631)
\curveto(54.4666791,54.22152302)(53.26667913,54.22152302)(53.1333458,54.22152302)
\curveto(52.46667916,54.22152302)(51.93334584,54.62152301)(51.62667918,55.11485633)
\curveto(51.22667919,55.75485631)(51.22667919,56.63485628)(51.22667919,57.11485627)
\lineto(54.94667908,57.11485627)
\curveto(55.24001241,57.11485627)(55.28001241,57.11485627)(55.28001241,57.39485626)
\curveto(55.28001241,58.71485622)(54.56001243,60.00818951)(52.89334581,60.00818951)
\curveto(51.34667919,60.00818951)(50.12001256,58.63485622)(50.12001256,56.96818961)
\curveto(50.12001256,55.18152299)(51.52001252,53.8881897)(53.05334581,53.8881897)
\curveto(54.68001242,53.8881897)(55.28001241,55.36818965)(55.28001241,55.62152298)
\closepath
\moveto(54.40001243,57.39485626)
\lineto(51.24001253,57.39485626)
\curveto(51.32001252,59.38152287)(52.44001249,59.71485619)(52.89334581,59.71485619)
\curveto(54.2666791,59.71485619)(54.40001243,57.91485624)(54.40001243,57.39485626)
\closepath
}
}
{
\newrgbcolor{curcolor}{0 0 0}
\pscustom[linestyle=none,fillstyle=solid,fillcolor=curcolor]
{
\newpath
\moveto(60.46665912,55.74152298)
\curveto(60.46665912,56.44818962)(60.06665913,56.84818961)(59.90665914,57.0081896)
\curveto(59.46665915,57.43485626)(58.94665917,57.54152292)(58.38665918,57.64818959)
\curveto(57.63999254,57.79485625)(56.74665923,57.96818958)(56.74665923,58.74152289)
\curveto(56.74665923,59.20818954)(57.09332589,59.75485619)(58.23999252,59.75485619)
\curveto(59.70665914,59.75485619)(59.77332581,58.55485622)(59.79999247,58.1415229)
\curveto(59.81332581,58.02152291)(59.95999247,58.02152291)(59.95999247,58.02152291)
\curveto(60.1333258,58.02152291)(60.1333258,58.08818957)(60.1333258,58.3415229)
\lineto(60.1333258,59.68818952)
\curveto(60.1333258,59.91485618)(60.1333258,60.00818951)(59.98665914,60.00818951)
\curveto(59.91999247,60.00818951)(59.8933258,60.00818951)(59.71999248,59.84818952)
\curveto(59.67999248,59.79485619)(59.54665915,59.67485619)(59.49332582,59.63485619)
\curveto(58.98665917,60.00818951)(58.43999252,60.00818951)(58.23999252,60.00818951)
\curveto(56.6133259,60.00818951)(56.10665925,59.11485621)(56.10665925,58.36818956)
\curveto(56.10665925,57.90152291)(56.31999258,57.52818959)(56.67999257,57.23485626)
\curveto(57.10665922,56.88818961)(57.47999254,56.80818961)(58.43999252,56.62152295)
\curveto(58.73332584,56.56818962)(59.82665914,56.35485629)(59.82665914,55.39485632)
\curveto(59.82665914,54.71485634)(59.35999249,54.18152302)(58.31999252,54.18152302)
\curveto(57.19999255,54.18152302)(56.71999257,54.941523)(56.46665924,56.0748563)
\curveto(56.42665924,56.24818963)(56.41332591,56.30152296)(56.27999258,56.30152296)
\curveto(56.10665925,56.30152296)(56.10665925,56.20818963)(56.10665925,55.96818964)
\lineto(56.10665925,54.20818969)
\curveto(56.10665925,53.98152303)(56.10665925,53.8881897)(56.25332591,53.8881897)
\curveto(56.31999258,53.8881897)(56.33332591,53.90152303)(56.58665924,54.15485636)
\curveto(56.6133259,54.18152302)(56.6133259,54.20818969)(56.8533259,54.46152301)
\curveto(57.43999255,53.90152303)(58.03999253,53.8881897)(58.31999252,53.8881897)
\curveto(59.85332581,53.8881897)(60.46665912,54.781523)(60.46665912,55.74152298)
\closepath
}
}
{
\newrgbcolor{curcolor}{0 0 0}
\pscustom[linestyle=none,fillstyle=solid,fillcolor=curcolor]
{
\newpath
\moveto(66.82664688,56.63485628)
\curveto(66.82664688,57.74152292)(66.2266469,58.4215229)(66.10664691,58.54152289)
\curveto(65.57331359,59.12818954)(65.0933136,59.24818954)(64.45331362,59.40818953)
\lineto(64.45331362,63.00818942)
\curveto(65.58664692,62.95485609)(66.2266469,62.36818944)(66.4266469,61.6215228)
\curveto(66.37331356,61.63485613)(66.3466469,61.64818947)(66.21331357,61.64818947)
\curveto(65.86664691,61.64818947)(65.59998025,61.40818947)(65.59998025,61.03485615)
\curveto(65.59998025,60.62152283)(65.93331358,60.42152284)(66.21331357,60.42152284)
\curveto(66.25331357,60.42152284)(66.82664688,60.43485617)(66.82664688,61.08818948)
\curveto(66.82664688,62.38152278)(65.94664691,63.35485608)(64.45331362,63.43485608)
\lineto(64.45331362,64.03485606)
\lineto(64.0399803,64.03485606)
\lineto(64.0399803,63.42152275)
\curveto(62.55998035,63.28818942)(61.66664704,62.08818945)(61.66664704,60.87485616)
\curveto(61.66664704,59.95485618)(62.15998036,59.3148562)(62.39998035,59.08818954)
\curveto(62.906647,58.56818956)(63.35998032,58.46152289)(64.0399803,58.28818957)
\lineto(64.0399803,54.31485635)
\curveto(62.866647,54.38152302)(62.23998036,55.061523)(62.06664703,55.88818964)
\curveto(62.11998036,55.87485631)(62.22664702,55.86152297)(62.27998035,55.86152297)
\curveto(62.63998034,55.86152297)(62.89331367,56.1148563)(62.89331367,56.47485629)
\curveto(62.89331367,56.84818961)(62.61331368,57.0881896)(62.27998035,57.0881896)
\curveto(62.19998036,57.0881896)(61.66664704,57.06152294)(61.66664704,56.40818962)
\curveto(61.66664704,55.22152299)(62.37331368,53.99485636)(64.0399803,53.8881897)
\lineto(64.0399803,53.28818972)
\lineto(64.45331362,53.28818972)
\lineto(64.45331362,53.90152303)
\curveto(65.85331358,54.02152303)(66.82664688,55.26152299)(66.82664688,56.63485628)
\closepath
\moveto(66.1466469,56.23485629)
\curveto(66.1466469,55.34152299)(65.50664692,54.44818968)(64.45331362,54.31485635)
\lineto(64.45331362,58.1815229)
\curveto(66.05331357,57.84818958)(66.1466469,56.55485629)(66.1466469,56.23485629)
\closepath
\moveto(64.0399803,59.5148562)
\curveto(62.39998035,59.86152285)(62.34664702,61.00818948)(62.34664702,61.27485614)
\curveto(62.34664702,62.07485612)(62.97331367,62.90152276)(64.0399803,63.00818942)
\closepath
}
}
{
\newrgbcolor{curcolor}{0.65490198 0.66274512 0.67450982}
\pscustom[linestyle=none,fillstyle=solid,fillcolor=curcolor]
{
\newpath
\moveto(47.08003039,59.54295101)
\curveto(47.08003039,59.54295101)(47.08003039,59.58295101)(47.04003039,59.686951)
\lineto(44.32003053,67.2069506)
\curveto(44.28803054,67.2949506)(44.25603054,67.3829506)(44.14403054,67.3829506)
\curveto(44.05603055,67.3829506)(43.98403055,67.3109506)(43.98403055,67.2229506)
\curveto(43.98403055,67.2229506)(43.98403055,67.18295061)(44.02403055,67.07895061)
\lineto(46.74403041,59.55895101)
\curveto(46.77603041,59.47095101)(46.8080304,59.38295102)(46.9200304,59.38295102)
\curveto(47.00803039,59.38295102)(47.08003039,59.45495101)(47.08003039,59.54295101)
\closepath
}
}
{
\newrgbcolor{curcolor}{0.65490198 0.66274512 0.67450982}
\pscustom[linestyle=none,fillstyle=solid,fillcolor=curcolor]
{
\newpath
\moveto(50.4160149,62.40695086)
\curveto(50.4160149,62.83095083)(50.17601491,63.07095082)(50.08001491,63.16695082)
\curveto(49.81601493,63.4229508)(49.50401494,63.4869508)(49.16801496,63.5509508)
\curveto(48.72001498,63.63895079)(48.18401501,63.74295079)(48.18401501,64.20695076)
\curveto(48.18401501,64.48695075)(48.392015,64.81495073)(49.08001497,64.81495073)
\curveto(49.96001492,64.81495073)(50.00001492,64.09495077)(50.01601492,63.84695078)
\curveto(50.02401492,63.77495078)(50.11201491,63.77495078)(50.11201491,63.77495078)
\curveto(50.21601491,63.77495078)(50.21601491,63.81495078)(50.21601491,63.96695077)
\lineto(50.21601491,64.77495073)
\curveto(50.21601491,64.91095072)(50.21601491,64.96695072)(50.12801491,64.96695072)
\curveto(50.08801491,64.96695072)(50.07201491,64.96695072)(49.96801492,64.87095073)
\curveto(49.94401492,64.83895073)(49.86401492,64.76695073)(49.83201493,64.74295073)
\curveto(49.52801494,64.96695072)(49.20001496,64.96695072)(49.08001497,64.96695072)
\curveto(48.10401502,64.96695072)(47.80001503,64.43095075)(47.80001503,63.98295077)
\curveto(47.80001503,63.70295079)(47.92801503,63.4789508)(48.14401501,63.30295081)
\curveto(48.400015,63.09495082)(48.62401499,63.04695082)(49.20001496,62.93495083)
\curveto(49.37601495,62.90295083)(50.03201492,62.77495084)(50.03201492,62.19895087)
\curveto(50.03201492,61.79095089)(49.75201493,61.47095091)(49.12801496,61.47095091)
\curveto(48.456015,61.47095091)(48.16801501,61.92695088)(48.01601502,62.60695085)
\curveto(47.99201502,62.71095084)(47.98401502,62.74295084)(47.90401503,62.74295084)
\curveto(47.80001503,62.74295084)(47.80001503,62.68695084)(47.80001503,62.54295085)
\lineto(47.80001503,61.4869509)
\curveto(47.80001503,61.35095091)(47.80001503,61.29495092)(47.88801503,61.29495092)
\curveto(47.92801503,61.29495092)(47.93601503,61.30295091)(48.08801502,61.45495091)
\curveto(48.10401502,61.47095091)(48.10401502,61.4869509)(48.24801501,61.6389509)
\curveto(48.60001499,61.30295091)(48.96001497,61.29495092)(49.12801496,61.29495092)
\curveto(50.04801491,61.29495092)(50.4160149,61.83095089)(50.4160149,62.40695086)
\closepath
}
}
{
\newrgbcolor{curcolor}{0.65490198 0.66274512 0.67450982}
\pscustom[linestyle=none,fillstyle=solid,fillcolor=curcolor]
{
\newpath
\moveto(57.19200729,61.38295091)
\lineto(57.19200729,61.6309509)
\curveto(56.77600731,61.6309509)(56.57600732,61.6309509)(56.56800733,61.87095088)
\lineto(56.56800733,63.3989508)
\curveto(56.56800733,64.08695077)(56.56800733,64.33495076)(56.32000734,64.62295074)
\curveto(56.20800734,64.75895073)(55.94400736,64.91895072)(55.48000738,64.91895072)
\curveto(54.80800742,64.91895072)(54.45600744,64.43895075)(54.32000744,64.13495077)
\curveto(54.20800745,64.83095073)(53.61600748,64.91895072)(53.2560075,64.91895072)
\curveto(52.67200753,64.91895072)(52.29600755,64.57495074)(52.07200756,64.07895077)
\lineto(52.07200756,64.91895072)
\lineto(50.94400762,64.83095073)
\lineto(50.94400762,64.58295074)
\curveto(51.50400759,64.58295074)(51.56800759,64.52695075)(51.56800759,64.13495077)
\lineto(51.56800759,61.99095088)
\curveto(51.56800759,61.6309509)(51.48000759,61.6309509)(50.94400762,61.6309509)
\lineto(50.94400762,61.38295091)
\lineto(51.84800757,61.40695091)
\lineto(52.74400753,61.38295091)
\lineto(52.74400753,61.6309509)
\curveto(52.20800755,61.6309509)(52.12000756,61.6309509)(52.12000756,61.99095088)
\lineto(52.12000756,63.4629508)
\curveto(52.12000756,64.29495076)(52.68800753,64.74295073)(53.2000075,64.74295073)
\curveto(53.70400748,64.74295073)(53.79200747,64.31095076)(53.79200747,63.85495078)
\lineto(53.79200747,61.99095088)
\curveto(53.79200747,61.6309509)(53.70400748,61.6309509)(53.1680075,61.6309509)
\lineto(53.1680075,61.38295091)
\lineto(54.07200746,61.40695091)
\lineto(54.96800741,61.38295091)
\lineto(54.96800741,61.6309509)
\curveto(54.43200744,61.6309509)(54.34400744,61.6309509)(54.34400744,61.99095088)
\lineto(54.34400744,63.4629508)
\curveto(54.34400744,64.29495076)(54.91200741,64.74295073)(55.42400739,64.74295073)
\curveto(55.92800736,64.74295073)(56.01600735,64.31095076)(56.01600735,63.85495078)
\lineto(56.01600735,61.99095088)
\curveto(56.01600735,61.6309509)(55.92800736,61.6309509)(55.39200739,61.6309509)
\lineto(55.39200739,61.38295091)
\lineto(56.29600734,61.40695091)
\closepath
}
}
{
\newrgbcolor{curcolor}{0.65490198 0.66274512 0.67450982}
\pscustom[linestyle=none,fillstyle=solid,fillcolor=curcolor]
{
\newpath
\moveto(61.2159931,62.09495087)
\lineto(61.2159931,62.54295085)
\lineto(61.01599311,62.54295085)
\lineto(61.01599311,62.09495087)
\curveto(61.01599311,61.6309509)(60.81599312,61.5829509)(60.72799312,61.5829509)
\curveto(60.46399314,61.5829509)(60.43199314,61.94295088)(60.43199314,61.98295088)
\lineto(60.43199314,63.58295079)
\curveto(60.43199314,63.91895078)(60.43199314,64.23095076)(60.14399315,64.52695075)
\curveto(59.83199317,64.83895073)(59.43199319,64.96695072)(59.04799321,64.96695072)
\curveto(58.39199325,64.96695072)(57.83999327,64.59095074)(57.83999327,64.06295077)
\curveto(57.83999327,63.82295078)(57.99999327,63.68695079)(58.20799325,63.68695079)
\curveto(58.43199324,63.68695079)(58.57599324,63.84695078)(58.57599324,64.05495077)
\curveto(58.57599324,64.15095076)(58.53599324,64.41495075)(58.16799326,64.42295075)
\curveto(58.38399325,64.70295074)(58.77599322,64.79095073)(59.03199321,64.79095073)
\curveto(59.42399319,64.79095073)(59.87999317,64.47895075)(59.87999317,63.76695079)
\lineto(59.87999317,63.4709508)
\curveto(59.47199319,63.4469508)(58.91199322,63.4229508)(58.40799324,63.18295082)
\curveto(57.80799328,62.91095083)(57.60799329,62.49495085)(57.60799329,62.14295087)
\curveto(57.60799329,61.4949509)(58.38399325,61.29495092)(58.88799322,61.29495092)
\curveto(59.41599319,61.29495092)(59.78399317,61.6149509)(59.93599316,61.99095088)
\curveto(59.96799316,61.6709509)(60.18399315,61.33495091)(60.55999313,61.33495091)
\curveto(60.72799312,61.33495091)(61.2159931,61.44695091)(61.2159931,62.09495087)
\closepath
\moveto(59.87999317,62.50295085)
\curveto(59.87999317,61.74295089)(59.3039932,61.47095091)(58.94399322,61.47095091)
\curveto(58.55199324,61.47095091)(58.22399325,61.75095089)(58.22399325,62.15095087)
\curveto(58.22399325,62.59095085)(58.55999324,63.25495081)(59.87999317,63.30295081)
\closepath
}
}
{
\newrgbcolor{curcolor}{0.65490198 0.66274512 0.67450982}
\pscustom[linestyle=none,fillstyle=solid,fillcolor=curcolor]
{
\newpath
\moveto(63.39197811,61.38295091)
\lineto(63.39197811,61.6309509)
\curveto(62.85597814,61.6309509)(62.76797815,61.6309509)(62.76797815,61.99095088)
\lineto(62.76797815,66.93495062)
\lineto(61.61597821,66.84695062)
\lineto(61.61597821,66.59895064)
\curveto(62.17597818,66.59895064)(62.23997817,66.54295064)(62.23997817,66.15095066)
\lineto(62.23997817,61.99095088)
\curveto(62.23997817,61.6309509)(62.15197818,61.6309509)(61.61597821,61.6309509)
\lineto(61.61597821,61.38295091)
\lineto(62.50397816,61.40695091)
\closepath
}
}
{
\newrgbcolor{curcolor}{0.65490198 0.66274512 0.67450982}
\pscustom[linestyle=none,fillstyle=solid,fillcolor=curcolor]
{
\newpath
\moveto(65.61597606,61.38295091)
\lineto(65.61597606,61.6309509)
\curveto(65.07997608,61.6309509)(64.99197609,61.6309509)(64.99197609,61.99095088)
\lineto(64.99197609,66.93495062)
\lineto(63.83997615,66.84695062)
\lineto(63.83997615,66.59895064)
\curveto(64.39997612,66.59895064)(64.46397612,66.54295064)(64.46397612,66.15095066)
\lineto(64.46397612,61.99095088)
\curveto(64.46397612,61.6309509)(64.37597612,61.6309509)(63.83997615,61.6309509)
\lineto(63.83997615,61.38295091)
\lineto(64.7279761,61.40695091)
\closepath
}
}
{
\newrgbcolor{curcolor}{0.65490198 0.66274512 0.67450982}
\pscustom[linestyle=none,fillstyle=solid,fillcolor=curcolor]
{
\newpath
\moveto(72.00796205,62.94295083)
\curveto(72.00796205,63.60695079)(71.64796207,64.01495077)(71.57596207,64.08695077)
\curveto(71.25596209,64.43895075)(70.9679621,64.51095075)(70.58396212,64.60695074)
\lineto(70.58396212,66.76695063)
\curveto(71.26396209,66.73495063)(71.64796207,66.38295065)(71.76796206,65.93495067)
\curveto(71.73596206,65.94295067)(71.71996206,65.95095067)(71.63996207,65.95095067)
\curveto(71.43196208,65.95095067)(71.27196209,65.80695068)(71.27196209,65.58295069)
\curveto(71.27196209,65.3349507)(71.47196208,65.21495071)(71.63996207,65.21495071)
\curveto(71.66396207,65.21495071)(72.00796205,65.22295071)(72.00796205,65.61495069)
\curveto(72.00796205,66.39095065)(71.47996208,66.97495062)(70.58396212,67.02295061)
\lineto(70.58396212,67.3829506)
\lineto(70.33596214,67.3829506)
\lineto(70.33596214,67.01495061)
\curveto(69.44796218,66.93495062)(68.91196221,66.21495066)(68.91196221,65.48695069)
\curveto(68.91196221,64.93495072)(69.20796219,64.55095074)(69.35196219,64.41495075)
\curveto(69.65596217,64.10295077)(69.92796216,64.03895077)(70.33596214,63.93495078)
\lineto(70.33596214,61.5509509)
\curveto(69.63196217,61.5909509)(69.25596219,61.99895088)(69.1519622,62.49495085)
\curveto(69.1839622,62.48695085)(69.24796219,62.47895085)(69.27996219,62.47895085)
\curveto(69.49596218,62.47895085)(69.64796217,62.63095084)(69.64796217,62.84695083)
\curveto(69.64796217,63.07095082)(69.47996218,63.21495081)(69.27996219,63.21495081)
\curveto(69.23196219,63.21495081)(68.91196221,63.19895081)(68.91196221,62.80695084)
\curveto(68.91196221,62.09495087)(69.33596219,61.35895091)(70.33596214,61.29495092)
\lineto(70.33596214,60.93495093)
\lineto(70.58396212,60.93495093)
\lineto(70.58396212,61.30295091)
\curveto(71.42396208,61.37495091)(72.00796205,62.11895087)(72.00796205,62.94295083)
\closepath
\moveto(71.59996207,62.70295084)
\curveto(71.59996207,62.16695087)(71.21596209,61.6309509)(70.58396212,61.5509509)
\lineto(70.58396212,63.87095078)
\curveto(71.54396207,63.67095079)(71.59996207,62.89495083)(71.59996207,62.70295084)
\closepath
\moveto(70.33596214,64.67095074)
\curveto(69.35196219,64.87895073)(69.31996219,65.56695069)(69.31996219,65.72695068)
\curveto(69.31996219,66.20695066)(69.69596217,66.70295063)(70.33596214,66.76695063)
\closepath
}
}
{
\newrgbcolor{curcolor}{0.65490198 0.66274512 0.67450982}
\pscustom[linestyle=none,fillstyle=solid,fillcolor=curcolor]
{
\newpath
\moveto(76.05594607,62.99095083)
\curveto(76.05594607,63.94295078)(75.3999461,64.74295073)(74.53594615,64.74295073)
\curveto(74.15194617,64.74295073)(73.80794618,64.61495074)(73.5199462,64.33495076)
\lineto(73.5199462,65.89495067)
\curveto(73.67994619,65.84695068)(73.94394618,65.79095068)(74.19994616,65.79095068)
\curveto(75.18394611,65.79095068)(75.74394608,66.51895064)(75.74394608,66.62295063)
\curveto(75.74394608,66.67095063)(75.71994608,66.71095063)(75.66394609,66.71095063)
\curveto(75.66394609,66.71095063)(75.63994609,66.71095063)(75.59994609,66.68695063)
\curveto(75.4399461,66.61495064)(75.04794612,66.45495064)(74.51194615,66.45495064)
\curveto(74.19194616,66.45495064)(73.82394618,66.51095064)(73.4479462,66.67895063)
\curveto(73.38394621,66.70295063)(73.35194621,66.70295063)(73.35194621,66.70295063)
\curveto(73.27194621,66.70295063)(73.27194621,66.63895063)(73.27194621,66.51095064)
\lineto(73.27194621,64.14295077)
\curveto(73.27194621,63.99895077)(73.27194621,63.93495078)(73.38394621,63.93495078)
\curveto(73.4399462,63.93495078)(73.4559462,63.95895078)(73.4879462,64.00695077)
\curveto(73.5759462,64.13495077)(73.87194618,64.56695074)(74.51994615,64.56695074)
\curveto(74.93594612,64.56695074)(75.13594611,64.19895076)(75.19994611,64.05495077)
\curveto(75.3279461,63.75895079)(75.3439461,63.4469508)(75.3439461,63.04695082)
\curveto(75.3439461,62.76695084)(75.3439461,62.28695086)(75.15194611,61.95095088)
\curveto(74.95994612,61.6389509)(74.66394614,61.43095091)(74.29594616,61.43095091)
\curveto(73.71194619,61.43095091)(73.25594621,61.85495089)(73.11994622,62.32695086)
\curveto(73.14394622,62.31895086)(73.16794622,62.31095086)(73.25594621,62.31095086)
\curveto(73.5199462,62.31095086)(73.65594619,62.51095085)(73.65594619,62.70295084)
\curveto(73.65594619,62.89495083)(73.5199462,63.09495082)(73.25594621,63.09495082)
\curveto(73.14394622,63.09495082)(72.86394623,63.03895082)(72.86394623,62.67095084)
\curveto(72.86394623,61.98295088)(73.4159462,61.20695092)(74.31194616,61.20695092)
\curveto(75.23994611,61.20695092)(76.05594607,61.97495088)(76.05594607,62.99095083)
\closepath
}
}
{
\newrgbcolor{curcolor}{0.65490198 0.66274512 0.67450982}
\pscustom[linestyle=none,fillstyle=solid,fillcolor=curcolor]
{
\newpath
\moveto(80.00793009,62.94295083)
\curveto(80.00793009,63.60695079)(79.64793011,64.01495077)(79.57593011,64.08695077)
\curveto(79.25593013,64.43895075)(78.96793014,64.51095075)(78.58393016,64.60695074)
\lineto(78.58393016,66.76695063)
\curveto(79.26393013,66.73495063)(79.64793011,66.38295065)(79.7679301,65.93495067)
\curveto(79.7359301,65.94295067)(79.7199301,65.95095067)(79.63993011,65.95095067)
\curveto(79.43193012,65.95095067)(79.27193013,65.80695068)(79.27193013,65.58295069)
\curveto(79.27193013,65.3349507)(79.47193012,65.21495071)(79.63993011,65.21495071)
\curveto(79.66393011,65.21495071)(80.00793009,65.22295071)(80.00793009,65.61495069)
\curveto(80.00793009,66.39095065)(79.47993012,66.97495062)(78.58393016,67.02295061)
\lineto(78.58393016,67.3829506)
\lineto(78.33593018,67.3829506)
\lineto(78.33593018,67.01495061)
\curveto(77.44793022,66.93495062)(76.91193025,66.21495066)(76.91193025,65.48695069)
\curveto(76.91193025,64.93495072)(77.20793024,64.55095074)(77.35193023,64.41495075)
\curveto(77.65593021,64.10295077)(77.9279302,64.03895077)(78.33593018,63.93495078)
\lineto(78.33593018,61.5509509)
\curveto(77.63193021,61.5909509)(77.25593023,61.99895088)(77.15193024,62.49495085)
\curveto(77.18393024,62.48695085)(77.24793023,62.47895085)(77.27993023,62.47895085)
\curveto(77.49593022,62.47895085)(77.64793021,62.63095084)(77.64793021,62.84695083)
\curveto(77.64793021,63.07095082)(77.47993022,63.21495081)(77.27993023,63.21495081)
\curveto(77.23193023,63.21495081)(76.91193025,63.19895081)(76.91193025,62.80695084)
\curveto(76.91193025,62.09495087)(77.33593023,61.35895091)(78.33593018,61.29495092)
\lineto(78.33593018,60.93495093)
\lineto(78.58393016,60.93495093)
\lineto(78.58393016,61.30295091)
\curveto(79.42393012,61.37495091)(80.00793009,62.11895087)(80.00793009,62.94295083)
\closepath
\moveto(79.59993011,62.70295084)
\curveto(79.59993011,62.16695087)(79.21593013,61.6309509)(78.58393016,61.5509509)
\lineto(78.58393016,63.87095078)
\curveto(79.54393011,63.67095079)(79.59993011,62.89495083)(79.59993011,62.70295084)
\closepath
\moveto(78.33593018,64.67095074)
\curveto(77.35193023,64.87895073)(77.31993023,65.56695069)(77.31993023,65.72695068)
\curveto(77.31993023,66.20695066)(77.69593021,66.70295063)(78.33593018,66.76695063)
\closepath
}
}
{
\newrgbcolor{curcolor}{0 0 0}
\pscustom[linestyle=none,fillstyle=solid,fillcolor=curcolor]
{
\newpath
\moveto(134.15331478,57.88816275)
\curveto(134.15331478,58.99482939)(133.55331479,59.67482937)(133.4333148,59.79482936)
\curveto(132.89998148,60.38149601)(132.4199815,60.50149601)(131.77998151,60.661496)
\lineto(131.77998151,64.2614959)
\curveto(132.91331481,64.20816256)(133.55331479,63.62149592)(133.75331479,62.87482927)
\curveto(133.69998146,62.8881626)(133.67331479,62.90149594)(133.53998146,62.90149594)
\curveto(133.19331481,62.90149594)(132.92664815,62.66149594)(132.92664815,62.28816262)
\curveto(132.92664815,61.8748293)(133.25998147,61.67482931)(133.53998146,61.67482931)
\curveto(133.57998146,61.67482931)(134.15331478,61.68816264)(134.15331478,62.34149595)
\curveto(134.15331478,63.63482925)(133.2733148,64.60816255)(131.77998151,64.68816255)
\lineto(131.77998151,65.28816253)
\lineto(131.36664819,65.28816253)
\lineto(131.36664819,64.67482922)
\curveto(129.88664824,64.54149589)(128.99331493,63.34149592)(128.99331493,62.12816263)
\curveto(128.99331493,61.20816265)(129.48664825,60.56816267)(129.72664824,60.34149601)
\curveto(130.23331489,59.82149603)(130.68664821,59.71482937)(131.36664819,59.54149604)
\lineto(131.36664819,55.56816282)
\curveto(130.1933149,55.63482949)(129.56664825,56.31482947)(129.39331492,57.14149611)
\curveto(129.44664825,57.12816278)(129.55331491,57.11482944)(129.60664825,57.11482944)
\curveto(129.96664824,57.11482944)(130.21998156,57.36816277)(130.21998156,57.72816276)
\curveto(130.21998156,58.10149608)(129.93998157,58.34149607)(129.60664825,58.34149607)
\curveto(129.52664825,58.34149607)(128.99331493,58.31482941)(128.99331493,57.66149609)
\curveto(128.99331493,56.47482946)(129.69998158,55.24816283)(131.36664819,55.14149617)
\lineto(131.36664819,54.54149619)
\lineto(131.77998151,54.54149619)
\lineto(131.77998151,55.1548295)
\curveto(133.17998147,55.2748295)(134.15331478,56.51482946)(134.15331478,57.88816275)
\closepath
\moveto(133.4733148,57.48816277)
\curveto(133.4733148,56.59482946)(132.83331482,55.70149615)(131.77998151,55.56816282)
\lineto(131.77998151,59.43482937)
\curveto(133.37998147,59.10149605)(133.4733148,57.80816276)(133.4733148,57.48816277)
\closepath
\moveto(131.36664819,60.76816267)
\curveto(129.72664824,61.11482932)(129.67331491,62.26149596)(129.67331491,62.52816261)
\curveto(129.67331491,63.32816259)(130.29998156,64.15482923)(131.36664819,64.2614959)
\closepath
}
}
{
\newrgbcolor{curcolor}{0 0 0}
\pscustom[linestyle=none,fillstyle=solid,fillcolor=curcolor]
{
\newpath
\moveto(144.28662257,60.62149601)
\lineto(144.28662257,61.03482933)
\curveto(143.99328925,61.00816266)(143.60662259,60.99482933)(143.31328927,60.99482933)
\lineto(142.07328931,61.03482933)
\lineto(142.07328931,60.62149601)
\curveto(142.55328929,60.60816267)(142.84662262,60.36816268)(142.84662262,59.98149602)
\curveto(142.84662262,59.90149603)(142.84662262,59.87482936)(142.77995595,59.70149603)
\lineto(141.56662266,56.2881628)
\lineto(140.2466227,60.00816269)
\curveto(140.19328936,60.16816269)(140.17995603,60.19482935)(140.17995603,60.26149602)
\curveto(140.17995603,60.62149601)(140.69995602,60.62149601)(140.96662267,60.62149601)
\lineto(140.96662267,61.03482933)
\lineto(139.57995605,60.99482933)
\curveto(139.17995606,60.99482933)(138.79328941,61.00816266)(138.39328942,61.03482933)
\lineto(138.39328942,60.62149601)
\curveto(138.88662274,60.62149601)(139.09995606,60.59482934)(139.23328939,60.42149601)
\curveto(139.29995606,60.34149601)(139.44662272,59.94149603)(139.53995605,59.6881627)
\lineto(138.39328942,56.46149613)
\lineto(137.12662279,60.02149602)
\curveto(137.05995612,60.18149602)(137.05995612,60.20816268)(137.05995612,60.26149602)
\curveto(137.05995612,60.62149601)(137.57995611,60.62149601)(137.84662277,60.62149601)
\lineto(137.84662277,61.03482933)
\lineto(136.39328948,60.99482933)
\lineto(135.15328952,61.03482933)
\lineto(135.15328952,60.62149601)
\curveto(135.81995616,60.62149601)(135.97995616,60.58149601)(136.13995615,60.15482935)
\lineto(137.8199561,55.43482949)
\curveto(137.88662277,55.24816283)(137.92662277,55.14149617)(138.09995609,55.14149617)
\curveto(138.27328942,55.14149617)(138.29995609,55.22149617)(138.36662275,55.40816283)
\lineto(139.71328938,59.18149605)
\lineto(141.07328934,55.3948295)
\curveto(141.12662267,55.24816283)(141.16662267,55.14149617)(141.339956,55.14149617)
\curveto(141.51328932,55.14149617)(141.55328932,55.26149617)(141.60662265,55.3948295)
\lineto(143.16662261,59.7681627)
\curveto(143.4066226,60.43482934)(143.81995592,60.60816267)(144.28662257,60.62149601)
\closepath
}
}
{
\newrgbcolor{curcolor}{0 0 0}
\pscustom[linestyle=none,fillstyle=solid,fillcolor=curcolor]
{
\newpath
\moveto(154.53993128,53.44816289)
\lineto(154.53993128,53.79482954)
\lineto(144.53993158,53.79482954)
\lineto(144.53993158,53.44816289)
\closepath
}
}
{
\newrgbcolor{curcolor}{0 0 0}
\pscustom[linestyle=none,fillstyle=solid,fillcolor=curcolor]
{
\newpath
\moveto(160.12657345,55.28816283)
\lineto(160.12657345,55.70149615)
\lineto(159.6999068,55.70149615)
\curveto(158.49990684,55.70149615)(158.45990684,55.84816282)(158.45990684,56.34149613)
\lineto(158.45990684,63.82149591)
\curveto(158.45990684,64.1414959)(158.45990684,64.16816257)(158.15324018,64.16816257)
\curveto(157.32657354,63.31482926)(156.15324024,63.31482926)(155.72657359,63.31482926)
\lineto(155.72657359,62.90149594)
\curveto(155.99324024,62.90149594)(156.77990689,62.90149594)(157.4732402,63.24816259)
\lineto(157.4732402,56.34149613)
\curveto(157.4732402,55.86149615)(157.4332402,55.70149615)(156.23324024,55.70149615)
\lineto(155.80657358,55.70149615)
\lineto(155.80657358,55.28816283)
\curveto(156.27324024,55.32816283)(157.4332402,55.32816283)(157.96657352,55.32816283)
\curveto(158.49990684,55.32816283)(159.6599068,55.32816283)(160.12657345,55.28816283)
\closepath
}
}
{
\newrgbcolor{curcolor}{0 0 0}
\pscustom[linestyle=none,fillstyle=solid,fillcolor=curcolor]
{
\newpath
\moveto(167.11321468,57.88816275)
\curveto(167.11321468,58.99482939)(166.5132147,59.67482937)(166.3932147,59.79482936)
\curveto(165.85988138,60.38149601)(165.3798814,60.50149601)(164.73988142,60.661496)
\lineto(164.73988142,64.2614959)
\curveto(165.87321472,64.20816256)(166.5132147,63.62149592)(166.71321469,62.87482927)
\curveto(166.65988136,62.8881626)(166.63321469,62.90149594)(166.49988136,62.90149594)
\curveto(166.15321471,62.90149594)(165.88654805,62.66149594)(165.88654805,62.28816262)
\curveto(165.88654805,61.8748293)(166.21988137,61.67482931)(166.49988136,61.67482931)
\curveto(166.53988136,61.67482931)(167.11321468,61.68816264)(167.11321468,62.34149595)
\curveto(167.11321468,63.63482925)(166.23321471,64.60816255)(164.73988142,64.68816255)
\lineto(164.73988142,65.28816253)
\lineto(164.3265481,65.28816253)
\lineto(164.3265481,64.67482922)
\curveto(162.84654814,64.54149589)(161.95321483,63.34149592)(161.95321483,62.12816263)
\curveto(161.95321483,61.20816265)(162.44654815,60.56816267)(162.68654815,60.34149601)
\curveto(163.1932148,59.82149603)(163.64654812,59.71482937)(164.3265481,59.54149604)
\lineto(164.3265481,55.56816282)
\curveto(163.1532148,55.63482949)(162.52654815,56.31482947)(162.35321482,57.14149611)
\curveto(162.40654815,57.12816278)(162.51321482,57.11482944)(162.56654815,57.11482944)
\curveto(162.92654814,57.11482944)(163.17988146,57.36816277)(163.17988146,57.72816276)
\curveto(163.17988146,58.10149608)(162.89988147,58.34149607)(162.56654815,58.34149607)
\curveto(162.48654815,58.34149607)(161.95321483,58.31482941)(161.95321483,57.66149609)
\curveto(161.95321483,56.47482946)(162.65988148,55.24816283)(164.3265481,55.14149617)
\lineto(164.3265481,54.54149619)
\lineto(164.73988142,54.54149619)
\lineto(164.73988142,55.1548295)
\curveto(166.13988137,55.2748295)(167.11321468,56.51482946)(167.11321468,57.88816275)
\closepath
\moveto(166.4332147,57.48816277)
\curveto(166.4332147,56.59482946)(165.79321472,55.70149615)(164.73988142,55.56816282)
\lineto(164.73988142,59.43482937)
\curveto(166.33988137,59.10149605)(166.4332147,57.80816276)(166.4332147,57.48816277)
\closepath
\moveto(164.3265481,60.76816267)
\curveto(162.68654815,61.11482932)(162.63321481,62.26149596)(162.63321481,62.52816261)
\curveto(162.63321481,63.32816259)(163.25988146,64.15482923)(164.3265481,64.2614959)
\closepath
}
}
{
\newrgbcolor{curcolor}{0 0 0}
\pscustom[linestyle=none,fillstyle=solid,fillcolor=curcolor]
{
\newpath
\moveto(92.32006085,56.63485628)
\curveto(92.32006085,57.74152292)(91.72006087,58.4215229)(91.60006088,58.54152289)
\curveto(91.06672756,59.12818954)(90.58672757,59.24818954)(89.94672759,59.40818953)
\lineto(89.94672759,63.00818942)
\curveto(91.08006089,62.95485609)(91.72006087,62.36818944)(91.92006087,61.6215228)
\curveto(91.86672753,61.63485613)(91.84006087,61.64818947)(91.70672754,61.64818947)
\curveto(91.36006088,61.64818947)(91.09339422,61.40818947)(91.09339422,61.03485615)
\curveto(91.09339422,60.62152283)(91.42672755,60.42152284)(91.70672754,60.42152284)
\curveto(91.74672754,60.42152284)(92.32006085,60.43485617)(92.32006085,61.08818948)
\curveto(92.32006085,62.38152278)(91.44006088,63.35485608)(89.94672759,63.43485608)
\lineto(89.94672759,64.03485606)
\lineto(89.53339427,64.03485606)
\lineto(89.53339427,63.42152275)
\curveto(88.05339432,63.28818942)(87.16006101,62.08818945)(87.16006101,60.87485616)
\curveto(87.16006101,59.95485618)(87.65339433,59.3148562)(87.89339432,59.08818954)
\curveto(88.40006097,58.56818956)(88.85339429,58.46152289)(89.53339427,58.28818957)
\lineto(89.53339427,54.31485635)
\curveto(88.36006097,54.38152302)(87.73339433,55.061523)(87.560061,55.88818964)
\curveto(87.61339433,55.87485631)(87.72006099,55.86152297)(87.77339432,55.86152297)
\curveto(88.13339431,55.86152297)(88.38672764,56.1148563)(88.38672764,56.47485629)
\curveto(88.38672764,56.84818961)(88.10672765,57.0881896)(87.77339432,57.0881896)
\curveto(87.69339433,57.0881896)(87.16006101,57.06152294)(87.16006101,56.40818962)
\curveto(87.16006101,55.22152299)(87.86672765,53.99485636)(89.53339427,53.8881897)
\lineto(89.53339427,53.28818972)
\lineto(89.94672759,53.28818972)
\lineto(89.94672759,53.90152303)
\curveto(91.34672755,54.02152303)(92.32006085,55.26152299)(92.32006085,56.63485628)
\closepath
\moveto(91.64006087,56.23485629)
\curveto(91.64006087,55.34152299)(91.00006089,54.44818968)(89.94672759,54.31485635)
\lineto(89.94672759,58.1815229)
\curveto(91.54672754,57.84818958)(91.64006087,56.55485629)(91.64006087,56.23485629)
\closepath
\moveto(89.53339427,59.5148562)
\curveto(87.89339432,59.86152285)(87.84006099,61.00818948)(87.84006099,61.27485614)
\curveto(87.84006099,62.07485612)(88.46672764,62.90152276)(89.53339427,63.00818942)
\closepath
}
}
{
\newrgbcolor{curcolor}{0 0 0}
\pscustom[linestyle=none,fillstyle=solid,fillcolor=curcolor]
{
\newpath
\moveto(98.98670209,50.96818979)
\curveto(98.98670209,50.96818979)(98.98670209,51.03485645)(98.92003542,51.20818978)
\lineto(94.38670223,63.74152274)
\curveto(94.3333689,63.8881894)(94.28003556,64.03485606)(94.0933689,64.03485606)
\curveto(93.94670224,64.03485606)(93.82670224,63.91485606)(93.82670224,63.7681894)
\curveto(93.82670224,63.7681894)(93.82670224,63.70152274)(93.89336891,63.52818941)
\lineto(98.42670211,50.99485645)
\curveto(98.48003544,50.84818979)(98.53336877,50.70152313)(98.72003543,50.70152313)
\curveto(98.86670209,50.70152313)(98.98670209,50.82152312)(98.98670209,50.96818979)
\closepath
}
}
{
\newrgbcolor{curcolor}{0 0 0}
\pscustom[linestyle=none,fillstyle=solid,fillcolor=curcolor]
{
\newpath
\moveto(104.17334337,55.68818964)
\lineto(104.17334337,56.44818962)
\lineto(103.84001005,56.44818962)
\lineto(103.84001005,55.71485631)
\curveto(103.84001005,54.72818967)(103.44001006,54.22152302)(102.94667674,54.22152302)
\curveto(102.05334343,54.22152302)(102.05334343,55.43485632)(102.05334343,55.66152298)
\lineto(102.05334343,59.36818953)
\lineto(103.96001004,59.36818953)
\lineto(103.96001004,59.78152285)
\lineto(102.05334343,59.78152285)
\lineto(102.05334343,62.23485611)
\lineto(101.72001011,62.23485611)
\curveto(101.70667678,61.14152281)(101.30667679,59.71485619)(100.00001016,59.66152286)
\lineto(100.00001016,59.36818953)
\lineto(101.13334346,59.36818953)
\lineto(101.13334346,55.68818964)
\curveto(101.13334346,54.04818969)(102.37334342,53.8881897)(102.85334341,53.8881897)
\curveto(103.80001005,53.8881897)(104.17334337,54.83485634)(104.17334337,55.68818964)
\closepath
}
}
{
\newrgbcolor{curcolor}{0 0 0}
\pscustom[linestyle=none,fillstyle=solid,fillcolor=curcolor]
{
\newpath
\moveto(108.22667612,54.03485636)
\lineto(108.22667612,54.44818968)
\curveto(107.34667615,54.44818968)(107.29334282,54.51485635)(107.29334282,55.03485633)
\lineto(107.29334282,59.92818952)
\lineto(105.42667621,59.78152285)
\lineto(105.42667621,59.36818953)
\curveto(106.29334285,59.36818953)(106.41334284,59.28818954)(106.41334284,58.63485622)
\lineto(106.41334284,55.04818966)
\curveto(106.41334284,54.44818968)(106.26667618,54.44818968)(105.37334288,54.44818968)
\lineto(105.37334288,54.03485636)
\lineto(106.8400095,54.07485636)
\curveto(107.30667615,54.07485636)(107.7733428,54.04818969)(108.22667612,54.03485636)
\closepath
\moveto(107.49334281,62.08818945)
\curveto(107.49334281,62.44818944)(107.18667615,62.7948561)(106.78667617,62.7948561)
\curveto(106.33334285,62.7948561)(106.06667619,62.42152278)(106.06667619,62.08818945)
\curveto(106.06667619,61.72818946)(106.37334285,61.38152281)(106.77334283,61.38152281)
\curveto(107.22667615,61.38152281)(107.49334281,61.75485613)(107.49334281,62.08818945)
\closepath
}
}
{
\newrgbcolor{curcolor}{0 0 0}
\pscustom[linestyle=none,fillstyle=solid,fillcolor=curcolor]
{
\newpath
\moveto(119.4800064,54.03485636)
\lineto(119.4800064,54.44818968)
\curveto(118.78667309,54.44818968)(118.45333976,54.44818968)(118.44000643,54.84818967)
\lineto(118.44000643,57.39485626)
\curveto(118.44000643,58.54152289)(118.44000643,58.95485621)(118.02667311,59.4348562)
\curveto(117.84000645,59.66152286)(117.40000646,59.92818952)(116.62667315,59.92818952)
\curveto(115.50667319,59.92818952)(114.92000654,59.12818954)(114.69333988,58.62152289)
\curveto(114.50667322,59.78152285)(113.52000658,59.92818952)(112.9200066,59.92818952)
\curveto(111.94667329,59.92818952)(111.32000665,59.3548562)(110.94667332,58.52818956)
\lineto(110.94667332,59.92818952)
\lineto(109.06667338,59.78152285)
\lineto(109.06667338,59.36818953)
\curveto(110.00000669,59.36818953)(110.10667335,59.2748562)(110.10667335,58.62152289)
\lineto(110.10667335,55.04818966)
\curveto(110.10667335,54.44818968)(109.96000669,54.44818968)(109.06667338,54.44818968)
\lineto(109.06667338,54.03485636)
\lineto(110.57334,54.07485636)
\lineto(112.06667329,54.03485636)
\lineto(112.06667329,54.44818968)
\curveto(111.17333998,54.44818968)(111.02667332,54.44818968)(111.02667332,55.04818966)
\lineto(111.02667332,57.50152292)
\curveto(111.02667332,58.88818955)(111.97333996,59.63485619)(112.82667327,59.63485619)
\curveto(113.66667324,59.63485619)(113.8133399,58.91485621)(113.8133399,58.15485624)
\lineto(113.8133399,55.04818966)
\curveto(113.8133399,54.44818968)(113.66667324,54.44818968)(112.77333994,54.44818968)
\lineto(112.77333994,54.03485636)
\lineto(114.28000656,54.07485636)
\lineto(115.77333985,54.03485636)
\lineto(115.77333985,54.44818968)
\curveto(114.88000654,54.44818968)(114.73333988,54.44818968)(114.73333988,55.04818966)
\lineto(114.73333988,57.50152292)
\curveto(114.73333988,58.88818955)(115.68000651,59.63485619)(116.53333982,59.63485619)
\curveto(117.3733398,59.63485619)(117.52000646,58.91485621)(117.52000646,58.15485624)
\lineto(117.52000646,55.04818966)
\curveto(117.52000646,54.44818968)(117.3733398,54.44818968)(116.48000649,54.44818968)
\lineto(116.48000649,54.03485636)
\lineto(117.98667311,54.07485636)
\closepath
}
}
{
\newrgbcolor{curcolor}{0 0 0}
\pscustom[linestyle=none,fillstyle=solid,fillcolor=curcolor]
{
\newpath
\moveto(125.27998237,55.62152298)
\curveto(125.27998237,55.75485631)(125.17331571,55.78152297)(125.10664904,55.78152297)
\curveto(124.98664904,55.78152297)(124.95998238,55.70152298)(124.93331571,55.59485631)
\curveto(124.46664906,54.22152302)(123.2666491,54.22152302)(123.13331577,54.22152302)
\curveto(122.46664912,54.22152302)(121.9333158,54.62152301)(121.62664915,55.11485633)
\curveto(121.22664916,55.75485631)(121.22664916,56.63485628)(121.22664916,57.11485627)
\lineto(124.94664905,57.11485627)
\curveto(125.23998237,57.11485627)(125.27998237,57.11485627)(125.27998237,57.39485626)
\curveto(125.27998237,58.71485622)(124.55998239,60.00818951)(122.89331577,60.00818951)
\curveto(121.34664915,60.00818951)(120.11998252,58.63485622)(120.11998252,56.96818961)
\curveto(120.11998252,55.18152299)(121.51998248,53.8881897)(123.05331577,53.8881897)
\curveto(124.67998239,53.8881897)(125.27998237,55.36818965)(125.27998237,55.62152298)
\closepath
\moveto(124.3999824,57.39485626)
\lineto(121.23998249,57.39485626)
\curveto(121.31998249,59.38152287)(122.43998245,59.71485619)(122.89331577,59.71485619)
\curveto(124.26664907,59.71485619)(124.3999824,57.91485624)(124.3999824,57.39485626)
\closepath
}
}
{
\newrgbcolor{curcolor}{0 0 0}
\pscustom[linestyle=none,fillstyle=solid,fillcolor=curcolor]
{
\newpath
\moveto(130.46662908,55.74152298)
\curveto(130.46662908,56.44818962)(130.0666291,56.84818961)(129.9066291,57.0081896)
\curveto(129.46662911,57.43485626)(128.94662913,57.54152292)(128.38662915,57.64818959)
\curveto(127.6399625,57.79485625)(126.7466292,57.96818958)(126.7466292,58.74152289)
\curveto(126.7466292,59.20818954)(127.09329585,59.75485619)(128.23996248,59.75485619)
\curveto(129.70662911,59.75485619)(129.77329577,58.55485622)(129.79996244,58.1415229)
\curveto(129.81329577,58.02152291)(129.95996243,58.02152291)(129.95996243,58.02152291)
\curveto(130.13329576,58.02152291)(130.13329576,58.08818957)(130.13329576,58.3415229)
\lineto(130.13329576,59.68818952)
\curveto(130.13329576,59.91485618)(130.13329576,60.00818951)(129.9866291,60.00818951)
\curveto(129.91996243,60.00818951)(129.89329577,60.00818951)(129.71996244,59.84818952)
\curveto(129.67996244,59.79485619)(129.54662911,59.67485619)(129.49329578,59.63485619)
\curveto(128.98662913,60.00818951)(128.43996248,60.00818951)(128.23996248,60.00818951)
\curveto(126.61329587,60.00818951)(126.10662921,59.11485621)(126.10662921,58.36818956)
\curveto(126.10662921,57.90152291)(126.31996254,57.52818959)(126.67996253,57.23485626)
\curveto(127.10662918,56.88818961)(127.47996251,56.80818961)(128.43996248,56.62152295)
\curveto(128.7332958,56.56818962)(129.8266291,56.35485629)(129.8266291,55.39485632)
\curveto(129.8266291,54.71485634)(129.35996245,54.18152302)(128.31996248,54.18152302)
\curveto(127.19996252,54.18152302)(126.71996253,54.941523)(126.4666292,56.0748563)
\curveto(126.42662921,56.24818963)(126.41329587,56.30152296)(126.27996254,56.30152296)
\curveto(126.10662921,56.30152296)(126.10662921,56.20818963)(126.10662921,55.96818964)
\lineto(126.10662921,54.20818969)
\curveto(126.10662921,53.98152303)(126.10662921,53.8881897)(126.25329588,53.8881897)
\curveto(126.31996254,53.8881897)(126.33329587,53.90152303)(126.5866292,54.15485636)
\curveto(126.61329587,54.18152302)(126.61329587,54.20818969)(126.85329586,54.46152301)
\curveto(127.43996251,53.90152303)(128.03996249,53.8881897)(128.31996248,53.8881897)
\curveto(129.85329577,53.8881897)(130.46662908,54.781523)(130.46662908,55.74152298)
\closepath
}
}
{
\newrgbcolor{curcolor}{0 0 0}
\pscustom[linestyle=none,fillstyle=solid,fillcolor=curcolor]
{
\newpath
\moveto(136.82661685,56.63485628)
\curveto(136.82661685,57.74152292)(136.22661686,58.4215229)(136.10661687,58.54152289)
\curveto(135.57328355,59.12818954)(135.09328357,59.24818954)(134.45328358,59.40818953)
\lineto(134.45328358,63.00818942)
\curveto(135.58661688,62.95485609)(136.22661686,62.36818944)(136.42661686,61.6215228)
\curveto(136.37328353,61.63485613)(136.34661686,61.64818947)(136.21328353,61.64818947)
\curveto(135.86661688,61.64818947)(135.59995022,61.40818947)(135.59995022,61.03485615)
\curveto(135.59995022,60.62152283)(135.93328354,60.42152284)(136.21328353,60.42152284)
\curveto(136.25328353,60.42152284)(136.82661685,60.43485617)(136.82661685,61.08818948)
\curveto(136.82661685,62.38152278)(135.94661687,63.35485608)(134.45328358,63.43485608)
\lineto(134.45328358,64.03485606)
\lineto(134.03995026,64.03485606)
\lineto(134.03995026,63.42152275)
\curveto(132.55995031,63.28818942)(131.666617,62.08818945)(131.666617,60.87485616)
\curveto(131.666617,59.95485618)(132.15995032,59.3148562)(132.39995031,59.08818954)
\curveto(132.90661696,58.56818956)(133.35995028,58.46152289)(134.03995026,58.28818957)
\lineto(134.03995026,54.31485635)
\curveto(132.86661697,54.38152302)(132.23995032,55.061523)(132.06661699,55.88818964)
\curveto(132.11995032,55.87485631)(132.22661699,55.86152297)(132.27995032,55.86152297)
\curveto(132.63995031,55.86152297)(132.89328363,56.1148563)(132.89328363,56.47485629)
\curveto(132.89328363,56.84818961)(132.61328364,57.0881896)(132.27995032,57.0881896)
\curveto(132.19995032,57.0881896)(131.666617,57.06152294)(131.666617,56.40818962)
\curveto(131.666617,55.22152299)(132.37328365,53.99485636)(134.03995026,53.8881897)
\lineto(134.03995026,53.28818972)
\lineto(134.45328358,53.28818972)
\lineto(134.45328358,53.90152303)
\curveto(135.85328354,54.02152303)(136.82661685,55.26152299)(136.82661685,56.63485628)
\closepath
\moveto(136.14661687,56.23485629)
\curveto(136.14661687,55.34152299)(135.50661689,54.44818968)(134.45328358,54.31485635)
\lineto(134.45328358,58.1815229)
\curveto(136.05328354,57.84818958)(136.14661687,56.55485629)(136.14661687,56.23485629)
\closepath
\moveto(134.03995026,59.5148562)
\curveto(132.39995031,59.86152285)(132.34661698,61.00818948)(132.34661698,61.27485614)
\curveto(132.34661698,62.07485612)(132.97328363,62.90152276)(134.03995026,63.00818942)
\closepath
}
}
{
\newrgbcolor{curcolor}{0.65490198 0.66274512 0.67450982}
\pscustom[linestyle=none,fillstyle=solid,fillcolor=curcolor]
{
\newpath
\moveto(117.08004,59.54295101)
\curveto(117.08004,59.54295101)(117.08004,59.58295101)(117.04004,59.686951)
\lineto(114.32004015,67.2069506)
\curveto(114.28804015,67.2949506)(114.25604015,67.3829506)(114.14404016,67.3829506)
\curveto(114.05604016,67.3829506)(113.98404016,67.3109506)(113.98404016,67.2229506)
\curveto(113.98404016,67.2229506)(113.98404016,67.18295061)(114.02404016,67.07895061)
\lineto(116.74404002,59.55895101)
\curveto(116.77604002,59.47095101)(116.80804002,59.38295102)(116.92004001,59.38295102)
\curveto(117.00804001,59.38295102)(117.08004,59.45495101)(117.08004,59.54295101)
\closepath
}
}
{
\newrgbcolor{curcolor}{0.65490198 0.66274512 0.67450982}
\pscustom[linestyle=none,fillstyle=solid,fillcolor=curcolor]
{
\newpath
\moveto(120.41602451,62.40695086)
\curveto(120.41602451,62.83095083)(120.17602452,63.07095082)(120.08002452,63.16695082)
\curveto(119.81602454,63.4229508)(119.50402456,63.4869508)(119.16802457,63.5509508)
\curveto(118.7200246,63.63895079)(118.18402462,63.74295079)(118.18402462,64.20695076)
\curveto(118.18402462,64.48695075)(118.39202461,64.81495073)(119.08002458,64.81495073)
\curveto(119.96002453,64.81495073)(120.00002453,64.09495077)(120.01602453,63.84695078)
\curveto(120.02402453,63.77495078)(120.11202452,63.77495078)(120.11202452,63.77495078)
\curveto(120.21602452,63.77495078)(120.21602452,63.81495078)(120.21602452,63.96695077)
\lineto(120.21602452,64.77495073)
\curveto(120.21602452,64.91095072)(120.21602452,64.96695072)(120.12802452,64.96695072)
\curveto(120.08802452,64.96695072)(120.07202453,64.96695072)(119.96802453,64.87095073)
\curveto(119.94402453,64.83895073)(119.86402454,64.76695073)(119.83202454,64.74295073)
\curveto(119.52802455,64.96695072)(119.20002457,64.96695072)(119.08002458,64.96695072)
\curveto(118.10402463,64.96695072)(117.80002464,64.43095075)(117.80002464,63.98295077)
\curveto(117.80002464,63.70295079)(117.92802464,63.4789508)(118.14402463,63.30295081)
\curveto(118.40002461,63.09495082)(118.6240246,63.04695082)(119.20002457,62.93495083)
\curveto(119.37602456,62.90295083)(120.03202453,62.77495084)(120.03202453,62.19895087)
\curveto(120.03202453,61.79095089)(119.75202454,61.47095091)(119.12802457,61.47095091)
\curveto(118.45602461,61.47095091)(118.16802463,61.92695088)(118.01602463,62.60695085)
\curveto(117.99202463,62.71095084)(117.98402464,62.74295084)(117.90402464,62.74295084)
\curveto(117.80002464,62.74295084)(117.80002464,62.68695084)(117.80002464,62.54295085)
\lineto(117.80002464,61.4869509)
\curveto(117.80002464,61.35095091)(117.80002464,61.29495092)(117.88802464,61.29495092)
\curveto(117.92802464,61.29495092)(117.93602464,61.30295091)(118.08802463,61.45495091)
\curveto(118.10402463,61.47095091)(118.10402463,61.4869509)(118.24802462,61.6389509)
\curveto(118.6000246,61.30295091)(118.96002458,61.29495092)(119.12802457,61.29495092)
\curveto(120.04802453,61.29495092)(120.41602451,61.83095089)(120.41602451,62.40695086)
\closepath
}
}
{
\newrgbcolor{curcolor}{0.65490198 0.66274512 0.67450982}
\pscustom[linestyle=none,fillstyle=solid,fillcolor=curcolor]
{
\newpath
\moveto(127.1920169,61.38295091)
\lineto(127.1920169,61.6309509)
\curveto(126.77601693,61.6309509)(126.57601694,61.6309509)(126.56801694,61.87095088)
\lineto(126.56801694,63.3989508)
\curveto(126.56801694,64.08695077)(126.56801694,64.33495076)(126.32001695,64.62295074)
\curveto(126.20801696,64.75895073)(125.94401697,64.91895072)(125.48001699,64.91895072)
\curveto(124.80801703,64.91895072)(124.45601705,64.43895075)(124.32001706,64.13495077)
\curveto(124.20801706,64.83095073)(123.61601709,64.91895072)(123.25601711,64.91895072)
\curveto(122.67201714,64.91895072)(122.29601716,64.57495074)(122.07201717,64.07895077)
\lineto(122.07201717,64.91895072)
\lineto(120.94401723,64.83095073)
\lineto(120.94401723,64.58295074)
\curveto(121.5040172,64.58295074)(121.5680172,64.52695075)(121.5680172,64.13495077)
\lineto(121.5680172,61.99095088)
\curveto(121.5680172,61.6309509)(121.4800172,61.6309509)(120.94401723,61.6309509)
\lineto(120.94401723,61.38295091)
\lineto(121.84801719,61.40695091)
\lineto(122.74401714,61.38295091)
\lineto(122.74401714,61.6309509)
\curveto(122.20801717,61.6309509)(122.12001717,61.6309509)(122.12001717,61.99095088)
\lineto(122.12001717,63.4629508)
\curveto(122.12001717,64.29495076)(122.68801714,64.74295073)(123.20001711,64.74295073)
\curveto(123.70401709,64.74295073)(123.79201708,64.31095076)(123.79201708,63.85495078)
\lineto(123.79201708,61.99095088)
\curveto(123.79201708,61.6309509)(123.70401709,61.6309509)(123.16801712,61.6309509)
\lineto(123.16801712,61.38295091)
\lineto(124.07201707,61.40695091)
\lineto(124.96801702,61.38295091)
\lineto(124.96801702,61.6309509)
\curveto(124.43201705,61.6309509)(124.34401705,61.6309509)(124.34401705,61.99095088)
\lineto(124.34401705,63.4629508)
\curveto(124.34401705,64.29495076)(124.91201702,64.74295073)(125.424017,64.74295073)
\curveto(125.92801697,64.74295073)(126.01601697,64.31095076)(126.01601697,63.85495078)
\lineto(126.01601697,61.99095088)
\curveto(126.01601697,61.6309509)(125.92801697,61.6309509)(125.392017,61.6309509)
\lineto(125.392017,61.38295091)
\lineto(126.29601695,61.40695091)
\closepath
}
}
{
\newrgbcolor{curcolor}{0.65490198 0.66274512 0.67450982}
\pscustom[linestyle=none,fillstyle=solid,fillcolor=curcolor]
{
\newpath
\moveto(131.21600271,62.09495087)
\lineto(131.21600271,62.54295085)
\lineto(131.01600272,62.54295085)
\lineto(131.01600272,62.09495087)
\curveto(131.01600272,61.6309509)(130.81600273,61.5829509)(130.72800273,61.5829509)
\curveto(130.46400275,61.5829509)(130.43200275,61.94295088)(130.43200275,61.98295088)
\lineto(130.43200275,63.58295079)
\curveto(130.43200275,63.91895078)(130.43200275,64.23095076)(130.14400276,64.52695075)
\curveto(129.83200278,64.83895073)(129.4320028,64.96695072)(129.04800282,64.96695072)
\curveto(128.39200286,64.96695072)(127.84000289,64.59095074)(127.84000289,64.06295077)
\curveto(127.84000289,63.82295078)(128.00000288,63.68695079)(128.20800287,63.68695079)
\curveto(128.43200285,63.68695079)(128.57600285,63.84695078)(128.57600285,64.05495077)
\curveto(128.57600285,64.15095076)(128.53600285,64.41495075)(128.16800287,64.42295075)
\curveto(128.38400286,64.70295074)(128.77600284,64.79095073)(129.03200282,64.79095073)
\curveto(129.4240028,64.79095073)(129.88000278,64.47895075)(129.88000278,63.76695079)
\lineto(129.88000278,63.4709508)
\curveto(129.4720028,63.4469508)(128.91200283,63.4229508)(128.40800286,63.18295082)
\curveto(127.80800289,62.91095083)(127.6080029,62.49495085)(127.6080029,62.14295087)
\curveto(127.6080029,61.4949509)(128.38400286,61.29495092)(128.88800283,61.29495092)
\curveto(129.4160028,61.29495092)(129.78400278,61.6149509)(129.93600278,61.99095088)
\curveto(129.96800277,61.6709509)(130.18400276,61.33495091)(130.56000274,61.33495091)
\curveto(130.72800273,61.33495091)(131.21600271,61.44695091)(131.21600271,62.09495087)
\closepath
\moveto(129.88000278,62.50295085)
\curveto(129.88000278,61.74295089)(129.30400281,61.47095091)(128.94400283,61.47095091)
\curveto(128.55200285,61.47095091)(128.22400287,61.75095089)(128.22400287,62.15095087)
\curveto(128.22400287,62.59095085)(128.56000285,63.25495081)(129.88000278,63.30295081)
\closepath
}
}
{
\newrgbcolor{curcolor}{0.65490198 0.66274512 0.67450982}
\pscustom[linestyle=none,fillstyle=solid,fillcolor=curcolor]
{
\newpath
\moveto(133.39198773,61.38295091)
\lineto(133.39198773,61.6309509)
\curveto(132.85598775,61.6309509)(132.76798776,61.6309509)(132.76798776,61.99095088)
\lineto(132.76798776,66.93495062)
\lineto(131.61598782,66.84695062)
\lineto(131.61598782,66.59895064)
\curveto(132.17598779,66.59895064)(132.23998779,66.54295064)(132.23998779,66.15095066)
\lineto(132.23998779,61.99095088)
\curveto(132.23998779,61.6309509)(132.15198779,61.6309509)(131.61598782,61.6309509)
\lineto(131.61598782,61.38295091)
\lineto(132.50398777,61.40695091)
\closepath
}
}
{
\newrgbcolor{curcolor}{0.65490198 0.66274512 0.67450982}
\pscustom[linestyle=none,fillstyle=solid,fillcolor=curcolor]
{
\newpath
\moveto(135.61598567,61.38295091)
\lineto(135.61598567,61.6309509)
\curveto(135.0799857,61.6309509)(134.9919857,61.6309509)(134.9919857,61.99095088)
\lineto(134.9919857,66.93495062)
\lineto(133.83998576,66.84695062)
\lineto(133.83998576,66.59895064)
\curveto(134.39998573,66.59895064)(134.46398573,66.54295064)(134.46398573,66.15095066)
\lineto(134.46398573,61.99095088)
\curveto(134.46398573,61.6309509)(134.37598573,61.6309509)(133.83998576,61.6309509)
\lineto(133.83998576,61.38295091)
\lineto(134.72798571,61.40695091)
\closepath
}
}
{
\newrgbcolor{curcolor}{0.65490198 0.66274512 0.67450982}
\pscustom[linestyle=none,fillstyle=solid,fillcolor=curcolor]
{
\newpath
\moveto(142.00797166,62.94295083)
\curveto(142.00797166,63.60695079)(141.64797168,64.01495077)(141.57597168,64.08695077)
\curveto(141.2559717,64.43895075)(140.96797171,64.51095075)(140.58397173,64.60695074)
\lineto(140.58397173,66.76695063)
\curveto(141.2639717,66.73495063)(141.64797168,66.38295065)(141.76797167,65.93495067)
\curveto(141.73597167,65.94295067)(141.71997167,65.95095067)(141.63997168,65.95095067)
\curveto(141.43197169,65.95095067)(141.2719717,65.80695068)(141.2719717,65.58295069)
\curveto(141.2719717,65.3349507)(141.47197169,65.21495071)(141.63997168,65.21495071)
\curveto(141.66397168,65.21495071)(142.00797166,65.22295071)(142.00797166,65.61495069)
\curveto(142.00797166,66.39095065)(141.47997169,66.97495062)(140.58397173,67.02295061)
\lineto(140.58397173,67.3829506)
\lineto(140.33597175,67.3829506)
\lineto(140.33597175,67.01495061)
\curveto(139.44797179,66.93495062)(138.91197182,66.21495066)(138.91197182,65.48695069)
\curveto(138.91197182,64.93495072)(139.20797181,64.55095074)(139.3519718,64.41495075)
\curveto(139.65597178,64.10295077)(139.92797177,64.03895077)(140.33597175,63.93495078)
\lineto(140.33597175,61.5509509)
\curveto(139.63197178,61.5909509)(139.2559718,61.99895088)(139.15197181,62.49495085)
\curveto(139.18397181,62.48695085)(139.2479718,62.47895085)(139.2799718,62.47895085)
\curveto(139.49597179,62.47895085)(139.64797178,62.63095084)(139.64797178,62.84695083)
\curveto(139.64797178,63.07095082)(139.47997179,63.21495081)(139.2799718,63.21495081)
\curveto(139.23197181,63.21495081)(138.91197182,63.19895081)(138.91197182,62.80695084)
\curveto(138.91197182,62.09495087)(139.3359718,61.35895091)(140.33597175,61.29495092)
\lineto(140.33597175,60.93495093)
\lineto(140.58397173,60.93495093)
\lineto(140.58397173,61.30295091)
\curveto(141.42397169,61.37495091)(142.00797166,62.11895087)(142.00797166,62.94295083)
\closepath
\moveto(141.59997168,62.70295084)
\curveto(141.59997168,62.16695087)(141.2159717,61.6309509)(140.58397173,61.5509509)
\lineto(140.58397173,63.87095078)
\curveto(141.54397168,63.67095079)(141.59997168,62.89495083)(141.59997168,62.70295084)
\closepath
\moveto(140.33597175,64.67095074)
\curveto(139.3519718,64.87895073)(139.3199718,65.56695069)(139.3199718,65.72695068)
\curveto(139.3199718,66.20695066)(139.69597178,66.70295063)(140.33597175,66.76695063)
\closepath
}
}
{
\newrgbcolor{curcolor}{0.65490198 0.66274512 0.67450982}
\pscustom[linestyle=none,fillstyle=solid,fillcolor=curcolor]
{
\newpath
\moveto(146.05595568,62.99095083)
\curveto(146.05595568,63.94295078)(145.39995571,64.74295073)(144.53595576,64.74295073)
\curveto(144.15195578,64.74295073)(143.8079558,64.61495074)(143.51995581,64.33495076)
\lineto(143.51995581,65.89495067)
\curveto(143.6799558,65.84695068)(143.94395579,65.79095068)(144.19995577,65.79095068)
\curveto(145.18395572,65.79095068)(145.74395569,66.51895064)(145.74395569,66.62295063)
\curveto(145.74395569,66.67095063)(145.71995569,66.71095063)(145.6639557,66.71095063)
\curveto(145.6639557,66.71095063)(145.6399557,66.71095063)(145.5999557,66.68695063)
\curveto(145.43995571,66.61495064)(145.04795573,66.45495064)(144.51195576,66.45495064)
\curveto(144.19195578,66.45495064)(143.82395579,66.51095064)(143.44795581,66.67895063)
\curveto(143.38395582,66.70295063)(143.35195582,66.70295063)(143.35195582,66.70295063)
\curveto(143.27195582,66.70295063)(143.27195582,66.63895063)(143.27195582,66.51095064)
\lineto(143.27195582,64.14295077)
\curveto(143.27195582,63.99895077)(143.27195582,63.93495078)(143.38395582,63.93495078)
\curveto(143.43995581,63.93495078)(143.45595581,63.95895078)(143.48795581,64.00695077)
\curveto(143.57595581,64.13495077)(143.87195579,64.56695074)(144.51995576,64.56695074)
\curveto(144.93595574,64.56695074)(145.13595573,64.19895076)(145.19995572,64.05495077)
\curveto(145.32795572,63.75895079)(145.34395571,63.4469508)(145.34395571,63.04695082)
\curveto(145.34395571,62.76695084)(145.34395571,62.28695086)(145.15195572,61.95095088)
\curveto(144.95995573,61.6389509)(144.66395575,61.43095091)(144.29595577,61.43095091)
\curveto(143.7119558,61.43095091)(143.25595582,61.85495089)(143.11995583,62.32695086)
\curveto(143.14395583,62.31895086)(143.16795583,62.31095086)(143.25595582,62.31095086)
\curveto(143.51995581,62.31095086)(143.6559558,62.51095085)(143.6559558,62.70295084)
\curveto(143.6559558,62.89495083)(143.51995581,63.09495082)(143.25595582,63.09495082)
\curveto(143.14395583,63.09495082)(142.86395585,63.03895082)(142.86395585,62.67095084)
\curveto(142.86395585,61.98295088)(143.41595582,61.20695092)(144.31195577,61.20695092)
\curveto(145.23995572,61.20695092)(146.05595568,61.97495088)(146.05595568,62.99095083)
\closepath
}
}
{
\newrgbcolor{curcolor}{0.65490198 0.66274512 0.67450982}
\pscustom[linestyle=none,fillstyle=solid,fillcolor=curcolor]
{
\newpath
\moveto(150.0079397,62.94295083)
\curveto(150.0079397,63.60695079)(149.64793972,64.01495077)(149.57593972,64.08695077)
\curveto(149.25593974,64.43895075)(148.96793975,64.51095075)(148.58393977,64.60695074)
\lineto(148.58393977,66.76695063)
\curveto(149.26393974,66.73495063)(149.64793972,66.38295065)(149.76793971,65.93495067)
\curveto(149.73593971,65.94295067)(149.71993972,65.95095067)(149.63993972,65.95095067)
\curveto(149.43193973,65.95095067)(149.27193974,65.80695068)(149.27193974,65.58295069)
\curveto(149.27193974,65.3349507)(149.47193973,65.21495071)(149.63993972,65.21495071)
\curveto(149.66393972,65.21495071)(150.0079397,65.22295071)(150.0079397,65.61495069)
\curveto(150.0079397,66.39095065)(149.47993973,66.97495062)(148.58393977,67.02295061)
\lineto(148.58393977,67.3829506)
\lineto(148.33593979,67.3829506)
\lineto(148.33593979,67.01495061)
\curveto(147.44793983,66.93495062)(146.91193986,66.21495066)(146.91193986,65.48695069)
\curveto(146.91193986,64.93495072)(147.20793985,64.55095074)(147.35193984,64.41495075)
\curveto(147.65593982,64.10295077)(147.92793981,64.03895077)(148.33593979,63.93495078)
\lineto(148.33593979,61.5509509)
\curveto(147.63193982,61.5909509)(147.25593984,61.99895088)(147.15193985,62.49495085)
\curveto(147.18393985,62.48695085)(147.24793985,62.47895085)(147.27993984,62.47895085)
\curveto(147.49593983,62.47895085)(147.64793982,62.63095084)(147.64793982,62.84695083)
\curveto(147.64793982,63.07095082)(147.47993983,63.21495081)(147.27993984,63.21495081)
\curveto(147.23193985,63.21495081)(146.91193986,63.19895081)(146.91193986,62.80695084)
\curveto(146.91193986,62.09495087)(147.33593984,61.35895091)(148.33593979,61.29495092)
\lineto(148.33593979,60.93495093)
\lineto(148.58393977,60.93495093)
\lineto(148.58393977,61.30295091)
\curveto(149.42393973,61.37495091)(150.0079397,62.11895087)(150.0079397,62.94295083)
\closepath
\moveto(149.59993972,62.70295084)
\curveto(149.59993972,62.16695087)(149.21593974,61.6309509)(148.58393977,61.5509509)
\lineto(148.58393977,63.87095078)
\curveto(149.54393972,63.67095079)(149.59993972,62.89495083)(149.59993972,62.70295084)
\closepath
\moveto(148.33593979,64.67095074)
\curveto(147.35193984,64.87895073)(147.31993984,65.56695069)(147.31993984,65.72695068)
\curveto(147.31993984,66.20695066)(147.69593982,66.70295063)(148.33593979,66.76695063)
\closepath
}
}
{
\newrgbcolor{curcolor}{0 0 0}
\pscustom[linestyle=none,fillstyle=solid,fillcolor=curcolor]
{
\newpath
\moveto(204.09337537,57.88816275)
\curveto(204.09337537,58.99482939)(203.49337539,59.67482937)(203.37337539,59.79482936)
\curveto(202.84004207,60.38149601)(202.36004209,60.50149601)(201.72004211,60.661496)
\lineto(201.72004211,64.2614959)
\curveto(202.85337541,64.20816256)(203.49337539,63.62149592)(203.69337538,62.87482927)
\curveto(203.64004205,62.8881626)(203.61337538,62.90149594)(203.48004205,62.90149594)
\curveto(203.1333754,62.90149594)(202.86670874,62.66149594)(202.86670874,62.28816262)
\curveto(202.86670874,61.8748293)(203.20004206,61.67482931)(203.48004205,61.67482931)
\curveto(203.52004205,61.67482931)(204.09337537,61.68816264)(204.09337537,62.34149595)
\curveto(204.09337537,63.63482925)(203.2133754,64.60816255)(201.72004211,64.68816255)
\lineto(201.72004211,65.28816253)
\lineto(201.30670879,65.28816253)
\lineto(201.30670879,64.67482922)
\curveto(199.82670883,64.54149589)(198.93337552,63.34149592)(198.93337552,62.12816263)
\curveto(198.93337552,61.20816265)(199.42670884,60.56816267)(199.66670884,60.34149601)
\curveto(200.17337549,59.82149603)(200.62670881,59.71482937)(201.30670879,59.54149604)
\lineto(201.30670879,55.56816282)
\curveto(200.13337549,55.63482949)(199.50670884,56.31482947)(199.33337551,57.14149611)
\curveto(199.38670884,57.12816278)(199.49337551,57.11482944)(199.54670884,57.11482944)
\curveto(199.90670883,57.11482944)(200.16004215,57.36816277)(200.16004215,57.72816276)
\curveto(200.16004215,58.10149608)(199.88004216,58.34149607)(199.54670884,58.34149607)
\curveto(199.46670884,58.34149607)(198.93337552,58.31482941)(198.93337552,57.66149609)
\curveto(198.93337552,56.47482946)(199.64004217,55.24816283)(201.30670879,55.14149617)
\lineto(201.30670879,54.54149619)
\lineto(201.72004211,54.54149619)
\lineto(201.72004211,55.1548295)
\curveto(203.12004207,55.2748295)(204.09337537,56.51482946)(204.09337537,57.88816275)
\closepath
\moveto(203.41337539,57.48816277)
\curveto(203.41337539,56.59482946)(202.77337541,55.70149615)(201.72004211,55.56816282)
\lineto(201.72004211,59.43482937)
\curveto(203.32004206,59.10149605)(203.41337539,57.80816276)(203.41337539,57.48816277)
\closepath
\moveto(201.30670879,60.76816267)
\curveto(199.66670884,61.11482932)(199.6133755,62.26149596)(199.6133755,62.52816261)
\curveto(199.6133755,63.32816259)(200.24004215,64.15482923)(201.30670879,64.2614959)
\closepath
}
}
{
\newrgbcolor{curcolor}{0 0 0}
\pscustom[linestyle=none,fillstyle=solid,fillcolor=curcolor]
{
\newpath
\moveto(214.22668317,60.62149601)
\lineto(214.22668317,61.03482933)
\curveto(213.93334984,61.00816266)(213.54668319,60.99482933)(213.25334986,60.99482933)
\lineto(212.0133499,61.03482933)
\lineto(212.0133499,60.62149601)
\curveto(212.49334989,60.60816267)(212.78668321,60.36816268)(212.78668321,59.98149602)
\curveto(212.78668321,59.90149603)(212.78668321,59.87482936)(212.72001655,59.70149603)
\lineto(211.50668325,56.2881628)
\lineto(210.18668329,60.00816269)
\curveto(210.13334996,60.16816269)(210.12001662,60.19482935)(210.12001662,60.26149602)
\curveto(210.12001662,60.62149601)(210.64001661,60.62149601)(210.90668327,60.62149601)
\lineto(210.90668327,61.03482933)
\lineto(209.52001664,60.99482933)
\curveto(209.12001665,60.99482933)(208.73335,61.00816266)(208.33335001,61.03482933)
\lineto(208.33335001,60.62149601)
\curveto(208.82668333,60.62149601)(209.04001666,60.59482934)(209.17334999,60.42149601)
\curveto(209.24001665,60.34149601)(209.38668331,59.94149603)(209.48001664,59.6881627)
\lineto(208.33335001,56.46149613)
\lineto(207.06668338,60.02149602)
\curveto(207.00001672,60.18149602)(207.00001672,60.20816268)(207.00001672,60.26149602)
\curveto(207.00001672,60.62149601)(207.5200167,60.62149601)(207.78668336,60.62149601)
\lineto(207.78668336,61.03482933)
\lineto(206.33335007,60.99482933)
\lineto(205.09335011,61.03482933)
\lineto(205.09335011,60.62149601)
\curveto(205.76001676,60.62149601)(205.92001675,60.58149601)(206.08001675,60.15482935)
\lineto(207.76001669,55.43482949)
\curveto(207.82668336,55.24816283)(207.86668336,55.14149617)(208.04001669,55.14149617)
\curveto(208.21335001,55.14149617)(208.24001668,55.22149617)(208.30668335,55.40816283)
\lineto(209.65334997,59.18149605)
\lineto(211.01334993,55.3948295)
\curveto(211.06668326,55.24816283)(211.10668326,55.14149617)(211.28001659,55.14149617)
\curveto(211.45334992,55.14149617)(211.49334992,55.26149617)(211.54668325,55.3948295)
\lineto(213.1066832,59.7681627)
\curveto(213.34668319,60.43482934)(213.76001651,60.60816267)(214.22668317,60.62149601)
\closepath
}
}
{
\newrgbcolor{curcolor}{0 0 0}
\pscustom[linestyle=none,fillstyle=solid,fillcolor=curcolor]
{
\newpath
\moveto(224.47999188,53.44816289)
\lineto(224.47999188,53.79482954)
\lineto(214.47999218,53.79482954)
\lineto(214.47999218,53.44816289)
\closepath
}
}
{
\newrgbcolor{curcolor}{0 0 0}
\pscustom[linestyle=none,fillstyle=solid,fillcolor=curcolor]
{
\newpath
\moveto(230.46663403,57.60816276)
\lineto(230.13330071,57.60816276)
\curveto(230.06663405,57.20816277)(229.97330072,56.62149613)(229.83996739,56.42149613)
\curveto(229.74663406,56.31482947)(228.86663408,56.31482947)(228.57330076,56.31482947)
\lineto(226.17330083,56.31482947)
\lineto(227.58663412,57.68816276)
\curveto(229.66663406,59.5281627)(230.46663403,60.24816268)(230.46663403,61.58149598)
\curveto(230.46663403,63.10149593)(229.26663407,64.16816257)(227.63996745,64.16816257)
\curveto(226.13330083,64.16816257)(225.14663419,62.94149594)(225.14663419,61.7548293)
\curveto(225.14663419,61.00816266)(225.81330084,61.00816266)(225.85330084,61.00816266)
\curveto(226.0799675,61.00816266)(226.54663415,61.16816266)(226.54663415,61.71482931)
\curveto(226.54663415,62.06149596)(226.30663416,62.40816262)(225.83996751,62.40816262)
\curveto(225.73330084,62.40816262)(225.70663418,62.40816262)(225.66663418,62.39482929)
\curveto(225.97330084,63.26149593)(226.69330081,63.75482924)(227.46663412,63.75482924)
\curveto(228.67996742,63.75482924)(229.25330074,62.67482928)(229.25330074,61.58149598)
\curveto(229.25330074,60.51482934)(228.58663409,59.46149604)(227.85330078,58.6348294)
\lineto(225.29330086,55.78149615)
\curveto(225.14663419,55.63482949)(225.14663419,55.60816282)(225.14663419,55.28816283)
\lineto(230.09330071,55.28816283)
\closepath
}
}
{
\newrgbcolor{curcolor}{0 0 0}
\pscustom[linestyle=none,fillstyle=solid,fillcolor=curcolor]
{
\newpath
\moveto(237.05327527,57.88816275)
\curveto(237.05327527,58.99482939)(236.45327529,59.67482937)(236.33327529,59.79482936)
\curveto(235.79994198,60.38149601)(235.31994199,60.50149601)(234.67994201,60.661496)
\lineto(234.67994201,64.2614959)
\curveto(235.81327531,64.20816256)(236.45327529,63.62149592)(236.65327528,62.87482927)
\curveto(236.59994195,62.8881626)(236.57327529,62.90149594)(236.43994196,62.90149594)
\curveto(236.0932753,62.90149594)(235.82660864,62.66149594)(235.82660864,62.28816262)
\curveto(235.82660864,61.8748293)(236.15994197,61.67482931)(236.43994196,61.67482931)
\curveto(236.47994196,61.67482931)(237.05327527,61.68816264)(237.05327527,62.34149595)
\curveto(237.05327527,63.63482925)(236.1732753,64.60816255)(234.67994201,64.68816255)
\lineto(234.67994201,65.28816253)
\lineto(234.26660869,65.28816253)
\lineto(234.26660869,64.67482922)
\curveto(232.78660873,64.54149589)(231.89327543,63.34149592)(231.89327543,62.12816263)
\curveto(231.89327543,61.20816265)(232.38660875,60.56816267)(232.62660874,60.34149601)
\curveto(233.13327539,59.82149603)(233.58660871,59.71482937)(234.26660869,59.54149604)
\lineto(234.26660869,55.56816282)
\curveto(233.09327539,55.63482949)(232.46660874,56.31482947)(232.29327541,57.14149611)
\curveto(232.34660875,57.12816278)(232.45327541,57.11482944)(232.50660874,57.11482944)
\curveto(232.86660873,57.11482944)(233.11994206,57.36816277)(233.11994206,57.72816276)
\curveto(233.11994206,58.10149608)(232.83994207,58.34149607)(232.50660874,58.34149607)
\curveto(232.42660874,58.34149607)(231.89327543,58.31482941)(231.89327543,57.66149609)
\curveto(231.89327543,56.47482946)(232.59994207,55.24816283)(234.26660869,55.14149617)
\lineto(234.26660869,54.54149619)
\lineto(234.67994201,54.54149619)
\lineto(234.67994201,55.1548295)
\curveto(236.07994197,55.2748295)(237.05327527,56.51482946)(237.05327527,57.88816275)
\closepath
\moveto(236.37327529,57.48816277)
\curveto(236.37327529,56.59482946)(235.73327531,55.70149615)(234.67994201,55.56816282)
\lineto(234.67994201,59.43482937)
\curveto(236.27994196,59.10149605)(236.37327529,57.80816276)(236.37327529,57.48816277)
\closepath
\moveto(234.26660869,60.76816267)
\curveto(232.62660874,61.11482932)(232.57327541,62.26149596)(232.57327541,62.52816261)
\curveto(232.57327541,63.32816259)(233.19994205,64.15482923)(234.26660869,64.2614959)
\closepath
}
}
{
\newrgbcolor{curcolor}{0 0 0}
\pscustom[linestyle=none,fillstyle=solid,fillcolor=curcolor]
{
\newpath
\moveto(162.26012145,56.63485628)
\curveto(162.26012145,57.74152292)(161.66012147,58.4215229)(161.54012147,58.54152289)
\curveto(161.00678815,59.12818954)(160.52678817,59.24818954)(159.88678819,59.40818953)
\lineto(159.88678819,63.00818942)
\curveto(161.02012148,62.95485609)(161.66012147,62.36818944)(161.86012146,61.6215228)
\curveto(161.80678813,61.63485613)(161.78012146,61.64818947)(161.64678813,61.64818947)
\curveto(161.30012148,61.64818947)(161.03345482,61.40818947)(161.03345482,61.03485615)
\curveto(161.03345482,60.62152283)(161.36678814,60.42152284)(161.64678813,60.42152284)
\curveto(161.68678813,60.42152284)(162.26012145,60.43485617)(162.26012145,61.08818948)
\curveto(162.26012145,62.38152278)(161.38012147,63.35485608)(159.88678819,63.43485608)
\lineto(159.88678819,64.03485606)
\lineto(159.47345486,64.03485606)
\lineto(159.47345486,63.42152275)
\curveto(157.99345491,63.28818942)(157.1001216,62.08818945)(157.1001216,60.87485616)
\curveto(157.1001216,59.95485618)(157.59345492,59.3148562)(157.83345491,59.08818954)
\curveto(158.34012156,58.56818956)(158.79345488,58.46152289)(159.47345486,58.28818957)
\lineto(159.47345486,54.31485635)
\curveto(158.30012157,54.38152302)(157.67345492,55.061523)(157.50012159,55.88818964)
\curveto(157.55345492,55.87485631)(157.66012159,55.86152297)(157.71345492,55.86152297)
\curveto(158.07345491,55.86152297)(158.32678823,56.1148563)(158.32678823,56.47485629)
\curveto(158.32678823,56.84818961)(158.04678824,57.0881896)(157.71345492,57.0881896)
\curveto(157.63345492,57.0881896)(157.1001216,57.06152294)(157.1001216,56.40818962)
\curveto(157.1001216,55.22152299)(157.80678825,53.99485636)(159.47345486,53.8881897)
\lineto(159.47345486,53.28818972)
\lineto(159.88678819,53.28818972)
\lineto(159.88678819,53.90152303)
\curveto(161.28678814,54.02152303)(162.26012145,55.26152299)(162.26012145,56.63485628)
\closepath
\moveto(161.58012147,56.23485629)
\curveto(161.58012147,55.34152299)(160.94012149,54.44818968)(159.88678819,54.31485635)
\lineto(159.88678819,58.1815229)
\curveto(161.48678814,57.84818958)(161.58012147,56.55485629)(161.58012147,56.23485629)
\closepath
\moveto(159.47345486,59.5148562)
\curveto(157.83345491,59.86152285)(157.78012158,61.00818948)(157.78012158,61.27485614)
\curveto(157.78012158,62.07485612)(158.40678823,62.90152276)(159.47345486,63.00818942)
\closepath
}
}
{
\newrgbcolor{curcolor}{0 0 0}
\pscustom[linestyle=none,fillstyle=solid,fillcolor=curcolor]
{
\newpath
\moveto(168.92676268,50.96818979)
\curveto(168.92676268,50.96818979)(168.92676268,51.03485645)(168.86009602,51.20818978)
\lineto(164.32676282,63.74152274)
\curveto(164.27342949,63.8881894)(164.22009616,64.03485606)(164.0334295,64.03485606)
\curveto(163.88676283,64.03485606)(163.76676284,63.91485606)(163.76676284,63.7681894)
\curveto(163.76676284,63.7681894)(163.76676284,63.70152274)(163.8334295,63.52818941)
\lineto(168.3667627,50.99485645)
\curveto(168.42009603,50.84818979)(168.47342936,50.70152313)(168.66009602,50.70152313)
\curveto(168.80676269,50.70152313)(168.92676268,50.82152312)(168.92676268,50.96818979)
\closepath
}
}
{
\newrgbcolor{curcolor}{0 0 0}
\pscustom[linestyle=none,fillstyle=solid,fillcolor=curcolor]
{
\newpath
\moveto(174.11340396,55.68818964)
\lineto(174.11340396,56.44818962)
\lineto(173.78007064,56.44818962)
\lineto(173.78007064,55.71485631)
\curveto(173.78007064,54.72818967)(173.38007065,54.22152302)(172.88673733,54.22152302)
\curveto(171.99340403,54.22152302)(171.99340403,55.43485632)(171.99340403,55.66152298)
\lineto(171.99340403,59.36818953)
\lineto(173.90007064,59.36818953)
\lineto(173.90007064,59.78152285)
\lineto(171.99340403,59.78152285)
\lineto(171.99340403,62.23485611)
\lineto(171.6600707,62.23485611)
\curveto(171.64673737,61.14152281)(171.24673738,59.71485619)(169.94007075,59.66152286)
\lineto(169.94007075,59.36818953)
\lineto(171.07340405,59.36818953)
\lineto(171.07340405,55.68818964)
\curveto(171.07340405,54.04818969)(172.31340402,53.8881897)(172.793404,53.8881897)
\curveto(173.74007064,53.8881897)(174.11340396,54.83485634)(174.11340396,55.68818964)
\closepath
}
}
{
\newrgbcolor{curcolor}{0 0 0}
\pscustom[linestyle=none,fillstyle=solid,fillcolor=curcolor]
{
\newpath
\moveto(178.16673672,54.03485636)
\lineto(178.16673672,54.44818968)
\curveto(177.28673674,54.44818968)(177.23340341,54.51485635)(177.23340341,55.03485633)
\lineto(177.23340341,59.92818952)
\lineto(175.3667368,59.78152285)
\lineto(175.3667368,59.36818953)
\curveto(176.23340344,59.36818953)(176.35340344,59.28818954)(176.35340344,58.63485622)
\lineto(176.35340344,55.04818966)
\curveto(176.35340344,54.44818968)(176.20673677,54.44818968)(175.31340347,54.44818968)
\lineto(175.31340347,54.03485636)
\lineto(176.78007009,54.07485636)
\curveto(177.24673674,54.07485636)(177.7134034,54.04818969)(178.16673672,54.03485636)
\closepath
\moveto(177.4334034,62.08818945)
\curveto(177.4334034,62.44818944)(177.12673675,62.7948561)(176.72673676,62.7948561)
\curveto(176.27340344,62.7948561)(176.00673678,62.42152278)(176.00673678,62.08818945)
\curveto(176.00673678,61.72818946)(176.31340344,61.38152281)(176.71340343,61.38152281)
\curveto(177.16673675,61.38152281)(177.4334034,61.75485613)(177.4334034,62.08818945)
\closepath
}
}
{
\newrgbcolor{curcolor}{0 0 0}
\pscustom[linestyle=none,fillstyle=solid,fillcolor=curcolor]
{
\newpath
\moveto(189.42006699,54.03485636)
\lineto(189.42006699,54.44818968)
\curveto(188.72673368,54.44818968)(188.39340036,54.44818968)(188.38006702,54.84818967)
\lineto(188.38006702,57.39485626)
\curveto(188.38006702,58.54152289)(188.38006702,58.95485621)(187.9667337,59.4348562)
\curveto(187.78006704,59.66152286)(187.34006706,59.92818952)(186.56673375,59.92818952)
\curveto(185.44673378,59.92818952)(184.86006713,59.12818954)(184.63340047,58.62152289)
\curveto(184.44673381,59.78152285)(183.46006717,59.92818952)(182.86006719,59.92818952)
\curveto(181.88673389,59.92818952)(181.26006724,59.3548562)(180.88673392,58.52818956)
\lineto(180.88673392,59.92818952)
\lineto(179.00673397,59.78152285)
\lineto(179.00673397,59.36818953)
\curveto(179.94006728,59.36818953)(180.04673394,59.2748562)(180.04673394,58.62152289)
\lineto(180.04673394,55.04818966)
\curveto(180.04673394,54.44818968)(179.90006728,54.44818968)(179.00673397,54.44818968)
\lineto(179.00673397,54.03485636)
\lineto(180.51340059,54.07485636)
\lineto(182.00673388,54.03485636)
\lineto(182.00673388,54.44818968)
\curveto(181.11340058,54.44818968)(180.96673391,54.44818968)(180.96673391,55.04818966)
\lineto(180.96673391,57.50152292)
\curveto(180.96673391,58.88818955)(181.91340055,59.63485619)(182.76673386,59.63485619)
\curveto(183.60673383,59.63485619)(183.7534005,58.91485621)(183.7534005,58.15485624)
\lineto(183.7534005,55.04818966)
\curveto(183.7534005,54.44818968)(183.60673383,54.44818968)(182.71340053,54.44818968)
\lineto(182.71340053,54.03485636)
\lineto(184.22006715,54.07485636)
\lineto(185.71340044,54.03485636)
\lineto(185.71340044,54.44818968)
\curveto(184.82006713,54.44818968)(184.67340047,54.44818968)(184.67340047,55.04818966)
\lineto(184.67340047,57.50152292)
\curveto(184.67340047,58.88818955)(185.62006711,59.63485619)(186.47340042,59.63485619)
\curveto(187.31340039,59.63485619)(187.46006705,58.91485621)(187.46006705,58.15485624)
\lineto(187.46006705,55.04818966)
\curveto(187.46006705,54.44818968)(187.31340039,54.44818968)(186.42006708,54.44818968)
\lineto(186.42006708,54.03485636)
\lineto(187.92673371,54.07485636)
\closepath
}
}
{
\newrgbcolor{curcolor}{0 0 0}
\pscustom[linestyle=none,fillstyle=solid,fillcolor=curcolor]
{
\newpath
\moveto(195.22004296,55.62152298)
\curveto(195.22004296,55.75485631)(195.1133763,55.78152297)(195.04670963,55.78152297)
\curveto(194.92670964,55.78152297)(194.90004297,55.70152298)(194.87337631,55.59485631)
\curveto(194.40670965,54.22152302)(193.20670969,54.22152302)(193.07337636,54.22152302)
\curveto(192.40670971,54.22152302)(191.8733764,54.62152301)(191.56670974,55.11485633)
\curveto(191.16670975,55.75485631)(191.16670975,56.63485628)(191.16670975,57.11485627)
\lineto(194.88670964,57.11485627)
\curveto(195.18004296,57.11485627)(195.22004296,57.11485627)(195.22004296,57.39485626)
\curveto(195.22004296,58.71485622)(194.50004298,60.00818951)(192.83337637,60.00818951)
\curveto(191.28670975,60.00818951)(190.06004312,58.63485622)(190.06004312,56.96818961)
\curveto(190.06004312,55.18152299)(191.46004307,53.8881897)(192.99337636,53.8881897)
\curveto(194.62004298,53.8881897)(195.22004296,55.36818965)(195.22004296,55.62152298)
\closepath
\moveto(194.34004299,57.39485626)
\lineto(191.18004308,57.39485626)
\curveto(191.26004308,59.38152287)(192.38004305,59.71485619)(192.83337637,59.71485619)
\curveto(194.20670966,59.71485619)(194.34004299,57.91485624)(194.34004299,57.39485626)
\closepath
}
}
{
\newrgbcolor{curcolor}{0 0 0}
\pscustom[linestyle=none,fillstyle=solid,fillcolor=curcolor]
{
\newpath
\moveto(200.40668968,55.74152298)
\curveto(200.40668968,56.44818962)(200.00668969,56.84818961)(199.84668969,57.0081896)
\curveto(199.40668971,57.43485626)(198.88668972,57.54152292)(198.32668974,57.64818959)
\curveto(197.5800231,57.79485625)(196.68668979,57.96818958)(196.68668979,58.74152289)
\curveto(196.68668979,59.20818954)(197.03335644,59.75485619)(198.18002308,59.75485619)
\curveto(199.6466897,59.75485619)(199.71335636,58.55485622)(199.74002303,58.1415229)
\curveto(199.75335636,58.02152291)(199.90002303,58.02152291)(199.90002303,58.02152291)
\curveto(200.07335635,58.02152291)(200.07335635,58.08818957)(200.07335635,58.3415229)
\lineto(200.07335635,59.68818952)
\curveto(200.07335635,59.91485618)(200.07335635,60.00818951)(199.92668969,60.00818951)
\curveto(199.86002303,60.00818951)(199.83335636,60.00818951)(199.66002303,59.84818952)
\curveto(199.62002303,59.79485619)(199.4866897,59.67485619)(199.43335637,59.63485619)
\curveto(198.92668972,60.00818951)(198.38002307,60.00818951)(198.18002308,60.00818951)
\curveto(196.55335646,60.00818951)(196.04668981,59.11485621)(196.04668981,58.36818956)
\curveto(196.04668981,57.90152291)(196.26002313,57.52818959)(196.62002312,57.23485626)
\curveto(197.04668978,56.88818961)(197.4200231,56.80818961)(198.38002307,56.62152295)
\curveto(198.6733564,56.56818962)(199.7666897,56.35485629)(199.7666897,55.39485632)
\curveto(199.7666897,54.71485634)(199.30002304,54.18152302)(198.26002307,54.18152302)
\curveto(197.14002311,54.18152302)(196.66002312,54.941523)(196.4066898,56.0748563)
\curveto(196.3666898,56.24818963)(196.35335647,56.30152296)(196.22002314,56.30152296)
\curveto(196.04668981,56.30152296)(196.04668981,56.20818963)(196.04668981,55.96818964)
\lineto(196.04668981,54.20818969)
\curveto(196.04668981,53.98152303)(196.04668981,53.8881897)(196.19335647,53.8881897)
\curveto(196.26002313,53.8881897)(196.27335647,53.90152303)(196.52668979,54.15485636)
\curveto(196.55335646,54.18152302)(196.55335646,54.20818969)(196.79335645,54.46152301)
\curveto(197.3800231,53.90152303)(197.98002308,53.8881897)(198.26002307,53.8881897)
\curveto(199.79335636,53.8881897)(200.40668968,54.781523)(200.40668968,55.74152298)
\closepath
}
}
{
\newrgbcolor{curcolor}{0 0 0}
\pscustom[linestyle=none,fillstyle=solid,fillcolor=curcolor]
{
\newpath
\moveto(206.76667744,56.63485628)
\curveto(206.76667744,57.74152292)(206.16667746,58.4215229)(206.04667746,58.54152289)
\curveto(205.51334414,59.12818954)(205.03334416,59.24818954)(204.39334418,59.40818953)
\lineto(204.39334418,63.00818942)
\curveto(205.52667748,62.95485609)(206.16667746,62.36818944)(206.36667745,61.6215228)
\curveto(206.31334412,61.63485613)(206.28667745,61.64818947)(206.15334413,61.64818947)
\curveto(205.80667747,61.64818947)(205.54001081,61.40818947)(205.54001081,61.03485615)
\curveto(205.54001081,60.62152283)(205.87334413,60.42152284)(206.15334413,60.42152284)
\curveto(206.19334412,60.42152284)(206.76667744,60.43485617)(206.76667744,61.08818948)
\curveto(206.76667744,62.38152278)(205.88667747,63.35485608)(204.39334418,63.43485608)
\lineto(204.39334418,64.03485606)
\lineto(203.98001086,64.03485606)
\lineto(203.98001086,63.42152275)
\curveto(202.5000109,63.28818942)(201.60667759,62.08818945)(201.60667759,60.87485616)
\curveto(201.60667759,59.95485618)(202.10001091,59.3148562)(202.34001091,59.08818954)
\curveto(202.84667756,58.56818956)(203.30001088,58.46152289)(203.98001086,58.28818957)
\lineto(203.98001086,54.31485635)
\curveto(202.80667756,54.38152302)(202.18001091,55.061523)(202.00667758,55.88818964)
\curveto(202.06001091,55.87485631)(202.16667758,55.86152297)(202.22001091,55.86152297)
\curveto(202.5800109,55.86152297)(202.83334422,56.1148563)(202.83334422,56.47485629)
\curveto(202.83334422,56.84818961)(202.55334423,57.0881896)(202.22001091,57.0881896)
\curveto(202.14001091,57.0881896)(201.60667759,57.06152294)(201.60667759,56.40818962)
\curveto(201.60667759,55.22152299)(202.31334424,53.99485636)(203.98001086,53.8881897)
\lineto(203.98001086,53.28818972)
\lineto(204.39334418,53.28818972)
\lineto(204.39334418,53.90152303)
\curveto(205.79334414,54.02152303)(206.76667744,55.26152299)(206.76667744,56.63485628)
\closepath
\moveto(206.08667746,56.23485629)
\curveto(206.08667746,55.34152299)(205.44667748,54.44818968)(204.39334418,54.31485635)
\lineto(204.39334418,58.1815229)
\curveto(205.99334413,57.84818958)(206.08667746,56.55485629)(206.08667746,56.23485629)
\closepath
\moveto(203.98001086,59.5148562)
\curveto(202.34001091,59.86152285)(202.28667757,61.00818948)(202.28667757,61.27485614)
\curveto(202.28667757,62.07485612)(202.91334422,62.90152276)(203.98001086,63.00818942)
\closepath
}
}
{
\newrgbcolor{curcolor}{0.65490198 0.66274512 0.67450982}
\pscustom[linestyle=none,fillstyle=solid,fillcolor=curcolor]
{
\newpath
\moveto(187.02010059,59.54295101)
\curveto(187.02010059,59.54295101)(187.02010059,59.58295101)(186.9801006,59.686951)
\lineto(184.26010074,67.2069506)
\curveto(184.22810074,67.2949506)(184.19610074,67.3829506)(184.08410075,67.3829506)
\curveto(183.99610075,67.3829506)(183.92410076,67.3109506)(183.92410076,67.2229506)
\curveto(183.92410076,67.2229506)(183.92410076,67.18295061)(183.96410076,67.07895061)
\lineto(186.68410061,59.55895101)
\curveto(186.71610061,59.47095101)(186.74810061,59.38295102)(186.8601006,59.38295102)
\curveto(186.9481006,59.38295102)(187.02010059,59.45495101)(187.02010059,59.54295101)
\closepath
}
}
{
\newrgbcolor{curcolor}{0.65490198 0.66274512 0.67450982}
\pscustom[linestyle=none,fillstyle=solid,fillcolor=curcolor]
{
\newpath
\moveto(190.3560851,62.40695086)
\curveto(190.3560851,62.83095083)(190.11608511,63.07095082)(190.02008512,63.16695082)
\curveto(189.75608513,63.4229508)(189.44408515,63.4869508)(189.10808517,63.5509508)
\curveto(188.66008519,63.63895079)(188.12408522,63.74295079)(188.12408522,64.20695076)
\curveto(188.12408522,64.48695075)(188.33208521,64.81495073)(189.02008517,64.81495073)
\curveto(189.90008512,64.81495073)(189.94008512,64.09495077)(189.95608512,63.84695078)
\curveto(189.96408512,63.77495078)(190.05208512,63.77495078)(190.05208512,63.77495078)
\curveto(190.15608511,63.77495078)(190.15608511,63.81495078)(190.15608511,63.96695077)
\lineto(190.15608511,64.77495073)
\curveto(190.15608511,64.91095072)(190.15608511,64.96695072)(190.06808512,64.96695072)
\curveto(190.02808512,64.96695072)(190.01208512,64.96695072)(189.90808512,64.87095073)
\curveto(189.88408513,64.83895073)(189.80408513,64.76695073)(189.77208513,64.74295073)
\curveto(189.46808515,64.96695072)(189.14008516,64.96695072)(189.02008517,64.96695072)
\curveto(188.04408522,64.96695072)(187.74008524,64.43095075)(187.74008524,63.98295077)
\curveto(187.74008524,63.70295079)(187.86808523,63.4789508)(188.08408522,63.30295081)
\curveto(188.34008521,63.09495082)(188.56408519,63.04695082)(189.14008516,62.93495083)
\curveto(189.31608515,62.90295083)(189.97208512,62.77495084)(189.97208512,62.19895087)
\curveto(189.97208512,61.79095089)(189.69208514,61.47095091)(189.06808517,61.47095091)
\curveto(188.3960852,61.47095091)(188.10808522,61.92695088)(187.95608523,62.60695085)
\curveto(187.93208523,62.71095084)(187.92408523,62.74295084)(187.84408523,62.74295084)
\curveto(187.74008524,62.74295084)(187.74008524,62.68695084)(187.74008524,62.54295085)
\lineto(187.74008524,61.4869509)
\curveto(187.74008524,61.35095091)(187.74008524,61.29495092)(187.82808523,61.29495092)
\curveto(187.86808523,61.29495092)(187.87608523,61.30295091)(188.02808522,61.45495091)
\curveto(188.04408522,61.47095091)(188.04408522,61.4869509)(188.18808521,61.6389509)
\curveto(188.5400852,61.30295091)(188.90008518,61.29495092)(189.06808517,61.29495092)
\curveto(189.98808512,61.29495092)(190.3560851,61.83095089)(190.3560851,62.40695086)
\closepath
}
}
{
\newrgbcolor{curcolor}{0.65490198 0.66274512 0.67450982}
\pscustom[linestyle=none,fillstyle=solid,fillcolor=curcolor]
{
\newpath
\moveto(197.1320775,61.38295091)
\lineto(197.1320775,61.6309509)
\curveto(196.71607752,61.6309509)(196.51607753,61.6309509)(196.50807753,61.87095088)
\lineto(196.50807753,63.3989508)
\curveto(196.50807753,64.08695077)(196.50807753,64.33495076)(196.26007754,64.62295074)
\curveto(196.14807755,64.75895073)(195.88407756,64.91895072)(195.42007759,64.91895072)
\curveto(194.74807762,64.91895072)(194.39607764,64.43895075)(194.26007765,64.13495077)
\curveto(194.14807765,64.83095073)(193.55607769,64.91895072)(193.1960777,64.91895072)
\curveto(192.61207773,64.91895072)(192.23607775,64.57495074)(192.01207777,64.07895077)
\lineto(192.01207777,64.91895072)
\lineto(190.88407783,64.83095073)
\lineto(190.88407783,64.58295074)
\curveto(191.4440778,64.58295074)(191.50807779,64.52695075)(191.50807779,64.13495077)
\lineto(191.50807779,61.99095088)
\curveto(191.50807779,61.6309509)(191.4200778,61.6309509)(190.88407783,61.6309509)
\lineto(190.88407783,61.38295091)
\lineto(191.78807778,61.40695091)
\lineto(192.68407773,61.38295091)
\lineto(192.68407773,61.6309509)
\curveto(192.14807776,61.6309509)(192.06007776,61.6309509)(192.06007776,61.99095088)
\lineto(192.06007776,63.4629508)
\curveto(192.06007776,64.29495076)(192.62807773,64.74295073)(193.14007771,64.74295073)
\curveto(193.64407768,64.74295073)(193.73207768,64.31095076)(193.73207768,63.85495078)
\lineto(193.73207768,61.99095088)
\curveto(193.73207768,61.6309509)(193.64407768,61.6309509)(193.10807771,61.6309509)
\lineto(193.10807771,61.38295091)
\lineto(194.01207766,61.40695091)
\lineto(194.90807761,61.38295091)
\lineto(194.90807761,61.6309509)
\curveto(194.37207764,61.6309509)(194.28407765,61.6309509)(194.28407765,61.99095088)
\lineto(194.28407765,63.4629508)
\curveto(194.28407765,64.29495076)(194.85207762,64.74295073)(195.36407759,64.74295073)
\curveto(195.86807756,64.74295073)(195.95607756,64.31095076)(195.95607756,63.85495078)
\lineto(195.95607756,61.99095088)
\curveto(195.95607756,61.6309509)(195.86807756,61.6309509)(195.33207759,61.6309509)
\lineto(195.33207759,61.38295091)
\lineto(196.23607754,61.40695091)
\closepath
}
}
{
\newrgbcolor{curcolor}{0.65490198 0.66274512 0.67450982}
\pscustom[linestyle=none,fillstyle=solid,fillcolor=curcolor]
{
\newpath
\moveto(201.1560633,62.09495087)
\lineto(201.1560633,62.54295085)
\lineto(200.95606331,62.54295085)
\lineto(200.95606331,62.09495087)
\curveto(200.95606331,61.6309509)(200.75606332,61.5829509)(200.66806333,61.5829509)
\curveto(200.40406334,61.5829509)(200.37206334,61.94295088)(200.37206334,61.98295088)
\lineto(200.37206334,63.58295079)
\curveto(200.37206334,63.91895078)(200.37206334,64.23095076)(200.08406336,64.52695075)
\curveto(199.77206337,64.83895073)(199.3720634,64.96695072)(198.98806342,64.96695072)
\curveto(198.33206345,64.96695072)(197.78006348,64.59095074)(197.78006348,64.06295077)
\curveto(197.78006348,63.82295078)(197.94006347,63.68695079)(198.14806346,63.68695079)
\curveto(198.37206345,63.68695079)(198.51606344,63.84695078)(198.51606344,64.05495077)
\curveto(198.51606344,64.15095076)(198.47606344,64.41495075)(198.10806346,64.42295075)
\curveto(198.32406345,64.70295074)(198.71606343,64.79095073)(198.97206342,64.79095073)
\curveto(199.3640634,64.79095073)(199.82006337,64.47895075)(199.82006337,63.76695079)
\lineto(199.82006337,63.4709508)
\curveto(199.41206339,63.4469508)(198.85206342,63.4229508)(198.34806345,63.18295082)
\curveto(197.74806348,62.91095083)(197.54806349,62.49495085)(197.54806349,62.14295087)
\curveto(197.54806349,61.4949509)(198.32406345,61.29495092)(198.82806342,61.29495092)
\curveto(199.3560634,61.29495092)(199.72406338,61.6149509)(199.87606337,61.99095088)
\curveto(199.90806337,61.6709509)(200.12406336,61.33495091)(200.50006334,61.33495091)
\curveto(200.66806333,61.33495091)(201.1560633,61.44695091)(201.1560633,62.09495087)
\closepath
\moveto(199.82006337,62.50295085)
\curveto(199.82006337,61.74295089)(199.2440634,61.47095091)(198.88406342,61.47095091)
\curveto(198.49206344,61.47095091)(198.16406346,61.75095089)(198.16406346,62.15095087)
\curveto(198.16406346,62.59095085)(198.50006344,63.25495081)(199.82006337,63.30295081)
\closepath
}
}
{
\newrgbcolor{curcolor}{0.65490198 0.66274512 0.67450982}
\pscustom[linestyle=none,fillstyle=solid,fillcolor=curcolor]
{
\newpath
\moveto(203.33204832,61.38295091)
\lineto(203.33204832,61.6309509)
\curveto(202.79604835,61.6309509)(202.70804835,61.6309509)(202.70804835,61.99095088)
\lineto(202.70804835,66.93495062)
\lineto(201.55604841,66.84695062)
\lineto(201.55604841,66.59895064)
\curveto(202.11604838,66.59895064)(202.18004838,66.54295064)(202.18004838,66.15095066)
\lineto(202.18004838,61.99095088)
\curveto(202.18004838,61.6309509)(202.09204838,61.6309509)(201.55604841,61.6309509)
\lineto(201.55604841,61.38295091)
\lineto(202.44404837,61.40695091)
\closepath
}
}
{
\newrgbcolor{curcolor}{0.65490198 0.66274512 0.67450982}
\pscustom[linestyle=none,fillstyle=solid,fillcolor=curcolor]
{
\newpath
\moveto(205.55604626,61.38295091)
\lineto(205.55604626,61.6309509)
\curveto(205.02004629,61.6309509)(204.93204629,61.6309509)(204.93204629,61.99095088)
\lineto(204.93204629,66.93495062)
\lineto(203.78004635,66.84695062)
\lineto(203.78004635,66.59895064)
\curveto(204.34004632,66.59895064)(204.40404632,66.54295064)(204.40404632,66.15095066)
\lineto(204.40404632,61.99095088)
\curveto(204.40404632,61.6309509)(204.31604633,61.6309509)(203.78004635,61.6309509)
\lineto(203.78004635,61.38295091)
\lineto(204.66804631,61.40695091)
\closepath
}
}
{
\newrgbcolor{curcolor}{0.65490198 0.66274512 0.67450982}
\pscustom[linestyle=none,fillstyle=solid,fillcolor=curcolor]
{
\newpath
\moveto(211.94803225,62.94295083)
\curveto(211.94803225,63.60695079)(211.58803227,64.01495077)(211.51603228,64.08695077)
\curveto(211.19603229,64.43895075)(210.90803231,64.51095075)(210.52403233,64.60695074)
\lineto(210.52403233,66.76695063)
\curveto(211.20403229,66.73495063)(211.58803227,66.38295065)(211.70803227,65.93495067)
\curveto(211.67603227,65.94295067)(211.66003227,65.95095067)(211.58003227,65.95095067)
\curveto(211.37203228,65.95095067)(211.21203229,65.80695068)(211.21203229,65.58295069)
\curveto(211.21203229,65.3349507)(211.41203228,65.21495071)(211.58003227,65.21495071)
\curveto(211.60403227,65.21495071)(211.94803225,65.22295071)(211.94803225,65.61495069)
\curveto(211.94803225,66.39095065)(211.42003228,66.97495062)(210.52403233,67.02295061)
\lineto(210.52403233,67.3829506)
\lineto(210.27603234,67.3829506)
\lineto(210.27603234,67.01495061)
\curveto(209.38803239,66.93495062)(208.85203242,66.21495066)(208.85203242,65.48695069)
\curveto(208.85203242,64.93495072)(209.1480324,64.55095074)(209.29203239,64.41495075)
\curveto(209.59603238,64.10295077)(209.86803236,64.03895077)(210.27603234,63.93495078)
\lineto(210.27603234,61.5509509)
\curveto(209.57203238,61.5909509)(209.1960324,61.99895088)(209.0920324,62.49495085)
\curveto(209.1240324,62.48695085)(209.1880324,62.47895085)(209.2200324,62.47895085)
\curveto(209.43603238,62.47895085)(209.58803238,62.63095084)(209.58803238,62.84695083)
\curveto(209.58803238,63.07095082)(209.42003239,63.21495081)(209.2200324,63.21495081)
\curveto(209.1720324,63.21495081)(208.85203242,63.19895081)(208.85203242,62.80695084)
\curveto(208.85203242,62.09495087)(209.27603239,61.35895091)(210.27603234,61.29495092)
\lineto(210.27603234,60.93495093)
\lineto(210.52403233,60.93495093)
\lineto(210.52403233,61.30295091)
\curveto(211.36403228,61.37495091)(211.94803225,62.11895087)(211.94803225,62.94295083)
\closepath
\moveto(211.54003227,62.70295084)
\curveto(211.54003227,62.16695087)(211.15603229,61.6309509)(210.52403233,61.5509509)
\lineto(210.52403233,63.87095078)
\curveto(211.48403228,63.67095079)(211.54003227,62.89495083)(211.54003227,62.70295084)
\closepath
\moveto(210.27603234,64.67095074)
\curveto(209.29203239,64.87895073)(209.26003239,65.56695069)(209.26003239,65.72695068)
\curveto(209.26003239,66.20695066)(209.63603237,66.70295063)(210.27603234,66.76695063)
\closepath
}
}
{
\newrgbcolor{curcolor}{0.65490198 0.66274512 0.67450982}
\pscustom[linestyle=none,fillstyle=solid,fillcolor=curcolor]
{
\newpath
\moveto(215.99601627,62.99095083)
\curveto(215.99601627,63.94295078)(215.3400163,64.74295073)(214.47601635,64.74295073)
\curveto(214.09201637,64.74295073)(213.74801639,64.61495074)(213.4600164,64.33495076)
\lineto(213.4600164,65.89495067)
\curveto(213.6200164,65.84695068)(213.88401638,65.79095068)(214.14001637,65.79095068)
\curveto(215.12401632,65.79095068)(215.68401629,66.51895064)(215.68401629,66.62295063)
\curveto(215.68401629,66.67095063)(215.66001629,66.71095063)(215.60401629,66.71095063)
\curveto(215.60401629,66.71095063)(215.58001629,66.71095063)(215.54001629,66.68695063)
\curveto(215.3800163,66.61495064)(214.98801632,66.45495064)(214.45201635,66.45495064)
\curveto(214.13201637,66.45495064)(213.76401639,66.51095064)(213.38801641,66.67895063)
\curveto(213.32401641,66.70295063)(213.29201641,66.70295063)(213.29201641,66.70295063)
\curveto(213.21201642,66.70295063)(213.21201642,66.63895063)(213.21201642,66.51095064)
\lineto(213.21201642,64.14295077)
\curveto(213.21201642,63.99895077)(213.21201642,63.93495078)(213.32401641,63.93495078)
\curveto(213.38001641,63.93495078)(213.39601641,63.95895078)(213.42801641,64.00695077)
\curveto(213.5160164,64.13495077)(213.81201639,64.56695074)(214.46001635,64.56695074)
\curveto(214.87601633,64.56695074)(215.07601632,64.19895076)(215.14001632,64.05495077)
\curveto(215.26801631,63.75895079)(215.28401631,63.4469508)(215.28401631,63.04695082)
\curveto(215.28401631,62.76695084)(215.28401631,62.28695086)(215.09201632,61.95095088)
\curveto(214.90001633,61.6389509)(214.60401634,61.43095091)(214.23601636,61.43095091)
\curveto(213.65201639,61.43095091)(213.19601642,61.85495089)(213.06001642,62.32695086)
\curveto(213.08401642,62.31895086)(213.10801642,62.31095086)(213.19601642,62.31095086)
\curveto(213.4600164,62.31095086)(213.5960164,62.51095085)(213.5960164,62.70295084)
\curveto(213.5960164,62.89495083)(213.4600164,63.09495082)(213.19601642,63.09495082)
\curveto(213.08401642,63.09495082)(212.80401644,63.03895082)(212.80401644,62.67095084)
\curveto(212.80401644,61.98295088)(213.35601641,61.20695092)(214.25201636,61.20695092)
\curveto(215.18001631,61.20695092)(215.99601627,61.97495088)(215.99601627,62.99095083)
\closepath
}
}
{
\newrgbcolor{curcolor}{0.65490198 0.66274512 0.67450982}
\pscustom[linestyle=none,fillstyle=solid,fillcolor=curcolor]
{
\newpath
\moveto(219.94800029,62.94295083)
\curveto(219.94800029,63.60695079)(219.58800031,64.01495077)(219.51600032,64.08695077)
\curveto(219.19600033,64.43895075)(218.90800035,64.51095075)(218.52400037,64.60695074)
\lineto(218.52400037,66.76695063)
\curveto(219.20400033,66.73495063)(219.58800031,66.38295065)(219.70800031,65.93495067)
\curveto(219.67600031,65.94295067)(219.66000031,65.95095067)(219.58000031,65.95095067)
\curveto(219.37200032,65.95095067)(219.21200033,65.80695068)(219.21200033,65.58295069)
\curveto(219.21200033,65.3349507)(219.41200032,65.21495071)(219.58000031,65.21495071)
\curveto(219.60400031,65.21495071)(219.94800029,65.22295071)(219.94800029,65.61495069)
\curveto(219.94800029,66.39095065)(219.42000032,66.97495062)(218.52400037,67.02295061)
\lineto(218.52400037,67.3829506)
\lineto(218.27600038,67.3829506)
\lineto(218.27600038,67.01495061)
\curveto(217.38800043,66.93495062)(216.85200046,66.21495066)(216.85200046,65.48695069)
\curveto(216.85200046,64.93495072)(217.14800044,64.55095074)(217.29200043,64.41495075)
\curveto(217.59600042,64.10295077)(217.8680004,64.03895077)(218.27600038,63.93495078)
\lineto(218.27600038,61.5509509)
\curveto(217.57200042,61.5909509)(217.19600044,61.99895088)(217.09200044,62.49495085)
\curveto(217.12400044,62.48695085)(217.18800044,62.47895085)(217.22000044,62.47895085)
\curveto(217.43600043,62.47895085)(217.58800042,62.63095084)(217.58800042,62.84695083)
\curveto(217.58800042,63.07095082)(217.42000043,63.21495081)(217.22000044,63.21495081)
\curveto(217.17200044,63.21495081)(216.85200046,63.19895081)(216.85200046,62.80695084)
\curveto(216.85200046,62.09495087)(217.27600043,61.35895091)(218.27600038,61.29495092)
\lineto(218.27600038,60.93495093)
\lineto(218.52400037,60.93495093)
\lineto(218.52400037,61.30295091)
\curveto(219.36400032,61.37495091)(219.94800029,62.11895087)(219.94800029,62.94295083)
\closepath
\moveto(219.54000031,62.70295084)
\curveto(219.54000031,62.16695087)(219.15600033,61.6309509)(218.52400037,61.5509509)
\lineto(218.52400037,63.87095078)
\curveto(219.48400032,63.67095079)(219.54000031,62.89495083)(219.54000031,62.70295084)
\closepath
\moveto(218.27600038,64.67095074)
\curveto(217.29200043,64.87895073)(217.26000043,65.56695069)(217.26000043,65.72695068)
\curveto(217.26000043,66.20695066)(217.63600041,66.70295063)(218.27600038,66.76695063)
\closepath
}
}
{
\newrgbcolor{curcolor}{0 0 0}
\pscustom[linestyle=none,fillstyle=solid,fillcolor=curcolor]
{
\newpath
\moveto(274.09335615,57.88816275)
\curveto(274.09335615,58.99482939)(273.49335616,59.67482937)(273.37335617,59.79482936)
\curveto(272.84002285,60.38149601)(272.36002286,60.50149601)(271.72002288,60.661496)
\lineto(271.72002288,64.2614959)
\curveto(272.85335618,64.20816256)(273.49335616,63.62149592)(273.69335616,62.87482927)
\curveto(273.64002283,62.8881626)(273.61335616,62.90149594)(273.48002283,62.90149594)
\curveto(273.13335617,62.90149594)(272.86668952,62.66149594)(272.86668952,62.28816262)
\curveto(272.86668952,61.8748293)(273.20002284,61.67482931)(273.48002283,61.67482931)
\curveto(273.52002283,61.67482931)(274.09335615,61.68816264)(274.09335615,62.34149595)
\curveto(274.09335615,63.63482925)(273.21335617,64.60816255)(271.72002288,64.68816255)
\lineto(271.72002288,65.28816253)
\lineto(271.30668956,65.28816253)
\lineto(271.30668956,64.67482922)
\curveto(269.82668961,64.54149589)(268.9333563,63.34149592)(268.9333563,62.12816263)
\curveto(268.9333563,61.20816265)(269.42668962,60.56816267)(269.66668961,60.34149601)
\curveto(270.17335626,59.82149603)(270.62668958,59.71482937)(271.30668956,59.54149604)
\lineto(271.30668956,55.56816282)
\curveto(270.13335626,55.63482949)(269.50668962,56.31482947)(269.33335629,57.14149611)
\curveto(269.38668962,57.12816278)(269.49335628,57.11482944)(269.54668962,57.11482944)
\curveto(269.9066896,57.11482944)(270.16002293,57.36816277)(270.16002293,57.72816276)
\curveto(270.16002293,58.10149608)(269.88002294,58.34149607)(269.54668962,58.34149607)
\curveto(269.46668962,58.34149607)(268.9333563,58.31482941)(268.9333563,57.66149609)
\curveto(268.9333563,56.47482946)(269.64002295,55.24816283)(271.30668956,55.14149617)
\lineto(271.30668956,54.54149619)
\lineto(271.72002288,54.54149619)
\lineto(271.72002288,55.1548295)
\curveto(273.12002284,55.2748295)(274.09335615,56.51482946)(274.09335615,57.88816275)
\closepath
\moveto(273.41335617,57.48816277)
\curveto(273.41335617,56.59482946)(272.77335619,55.70149615)(271.72002288,55.56816282)
\lineto(271.72002288,59.43482937)
\curveto(273.32002284,59.10149605)(273.41335617,57.80816276)(273.41335617,57.48816277)
\closepath
\moveto(271.30668956,60.76816267)
\curveto(269.66668961,61.11482932)(269.61335628,62.26149596)(269.61335628,62.52816261)
\curveto(269.61335628,63.32816259)(270.24002293,64.15482923)(271.30668956,64.2614959)
\closepath
}
}
{
\newrgbcolor{curcolor}{0 0 0}
\pscustom[linestyle=none,fillstyle=solid,fillcolor=curcolor]
{
\newpath
\moveto(284.22666394,60.62149601)
\lineto(284.22666394,61.03482933)
\curveto(283.93333062,61.00816266)(283.54666396,60.99482933)(283.25333064,60.99482933)
\lineto(282.01333068,61.03482933)
\lineto(282.01333068,60.62149601)
\curveto(282.49333066,60.60816267)(282.78666399,60.36816268)(282.78666399,59.98149602)
\curveto(282.78666399,59.90149603)(282.78666399,59.87482936)(282.71999732,59.70149603)
\lineto(281.50666403,56.2881628)
\lineto(280.18666407,60.00816269)
\curveto(280.13333073,60.16816269)(280.1199974,60.19482935)(280.1199974,60.26149602)
\curveto(280.1199974,60.62149601)(280.63999738,60.62149601)(280.90666404,60.62149601)
\lineto(280.90666404,61.03482933)
\lineto(279.51999742,60.99482933)
\curveto(279.11999743,60.99482933)(278.73333078,61.00816266)(278.33333079,61.03482933)
\lineto(278.33333079,60.62149601)
\curveto(278.82666411,60.62149601)(279.03999743,60.59482934)(279.17333076,60.42149601)
\curveto(279.23999743,60.34149601)(279.38666409,59.94149603)(279.47999742,59.6881627)
\lineto(278.33333079,56.46149613)
\lineto(277.06666416,60.02149602)
\curveto(276.99999749,60.18149602)(276.99999749,60.20816268)(276.99999749,60.26149602)
\curveto(276.99999749,60.62149601)(277.51999748,60.62149601)(277.78666414,60.62149601)
\lineto(277.78666414,61.03482933)
\lineto(276.33333085,60.99482933)
\lineto(275.09333088,61.03482933)
\lineto(275.09333088,60.62149601)
\curveto(275.75999753,60.62149601)(275.91999753,60.58149601)(276.07999752,60.15482935)
\lineto(277.75999747,55.43482949)
\curveto(277.82666414,55.24816283)(277.86666413,55.14149617)(278.03999746,55.14149617)
\curveto(278.21333079,55.14149617)(278.23999746,55.22149617)(278.30666412,55.40816283)
\lineto(279.65333075,59.18149605)
\lineto(281.01333071,55.3948295)
\curveto(281.06666404,55.24816283)(281.10666404,55.14149617)(281.27999737,55.14149617)
\curveto(281.45333069,55.14149617)(281.49333069,55.26149617)(281.54666402,55.3948295)
\lineto(283.10666398,59.7681627)
\curveto(283.34666397,60.43482934)(283.75999729,60.60816267)(284.22666394,60.62149601)
\closepath
}
}
{
\newrgbcolor{curcolor}{0 0 0}
\pscustom[linestyle=none,fillstyle=solid,fillcolor=curcolor]
{
\newpath
\moveto(294.47997265,53.44816289)
\lineto(294.47997265,53.79482954)
\lineto(284.47997295,53.79482954)
\lineto(284.47997295,53.44816289)
\closepath
}
}
{
\newrgbcolor{curcolor}{0 0 0}
\pscustom[linestyle=none,fillstyle=solid,fillcolor=curcolor]
{
\newpath
\moveto(300.57328147,57.56816276)
\curveto(300.57328147,58.66149606)(299.7332815,59.70149603)(298.34661487,59.98149602)
\curveto(299.43994817,60.34149601)(300.21328148,61.27482932)(300.21328148,62.32816262)
\curveto(300.21328148,63.42149592)(299.03994819,64.16816257)(297.75994823,64.16816257)
\curveto(296.4132816,64.16816257)(295.3999483,63.36816259)(295.3999483,62.35482929)
\curveto(295.3999483,61.9148293)(295.69328162,61.66149597)(296.07994828,61.66149597)
\curveto(296.4932816,61.66149597)(296.75994826,61.9548293)(296.75994826,62.34149595)
\curveto(296.75994826,63.0081626)(296.13328161,63.0081626)(295.93328161,63.0081626)
\curveto(296.34661493,63.66149591)(297.22661491,63.83482924)(297.70661489,63.83482924)
\curveto(298.25328154,63.83482924)(298.98661485,63.54149592)(298.98661485,62.34149595)
\curveto(298.98661485,62.18149596)(298.95994819,61.40816265)(298.61328153,60.821496)
\curveto(298.21328154,60.18149602)(297.75994823,60.14149602)(297.4266149,60.12816269)
\curveto(297.31994824,60.11482935)(296.99994825,60.08816269)(296.90661492,60.08816269)
\curveto(296.79994825,60.07482936)(296.70661492,60.06149602)(296.70661492,59.92816269)
\curveto(296.70661492,59.78149603)(296.79994825,59.78149603)(297.02661491,59.78149603)
\lineto(297.61328156,59.78149603)
\curveto(298.70661486,59.78149603)(299.19994818,58.87482939)(299.19994818,57.56816276)
\curveto(299.19994818,55.75482948)(298.27994821,55.36816283)(297.69328156,55.36816283)
\curveto(297.11994824,55.36816283)(296.11994827,55.59482949)(295.65328162,56.38149613)
\curveto(296.11994827,56.31482947)(296.5332816,56.60816279)(296.5332816,57.11482944)
\curveto(296.5332816,57.59482943)(296.17328161,57.86149609)(295.78661495,57.86149609)
\curveto(295.46661496,57.86149609)(295.03994831,57.67482943)(295.03994831,57.08816278)
\curveto(295.03994831,55.87482948)(296.27994827,54.99482951)(297.73328156,54.99482951)
\curveto(299.35994818,54.99482951)(300.57328147,56.2081628)(300.57328147,57.56816276)
\closepath
}
}
{
\newrgbcolor{curcolor}{0 0 0}
\pscustom[linestyle=none,fillstyle=solid,fillcolor=curcolor]
{
\newpath
\moveto(307.05325605,57.88816275)
\curveto(307.05325605,58.99482939)(306.45325607,59.67482937)(306.33325607,59.79482936)
\curveto(305.79992275,60.38149601)(305.31992277,60.50149601)(304.67992279,60.661496)
\lineto(304.67992279,64.2614959)
\curveto(305.81325609,64.20816256)(306.45325607,63.62149592)(306.65325606,62.87482927)
\curveto(306.59992273,62.8881626)(306.57325606,62.90149594)(306.43992273,62.90149594)
\curveto(306.09325608,62.90149594)(305.82658942,62.66149594)(305.82658942,62.28816262)
\curveto(305.82658942,61.8748293)(306.15992274,61.67482931)(306.43992273,61.67482931)
\curveto(306.47992273,61.67482931)(307.05325605,61.68816264)(307.05325605,62.34149595)
\curveto(307.05325605,63.63482925)(306.17325607,64.60816255)(304.67992279,64.68816255)
\lineto(304.67992279,65.28816253)
\lineto(304.26658947,65.28816253)
\lineto(304.26658947,64.67482922)
\curveto(302.78658951,64.54149589)(301.8932562,63.34149592)(301.8932562,62.12816263)
\curveto(301.8932562,61.20816265)(302.38658952,60.56816267)(302.62658951,60.34149601)
\curveto(303.13325617,59.82149603)(303.58658949,59.71482937)(304.26658947,59.54149604)
\lineto(304.26658947,55.56816282)
\curveto(303.09325617,55.63482949)(302.46658952,56.31482947)(302.29325619,57.14149611)
\curveto(302.34658952,57.12816278)(302.45325619,57.11482944)(302.50658952,57.11482944)
\curveto(302.86658951,57.11482944)(303.11992283,57.36816277)(303.11992283,57.72816276)
\curveto(303.11992283,58.10149608)(302.83992284,58.34149607)(302.50658952,58.34149607)
\curveto(302.42658952,58.34149607)(301.8932562,58.31482941)(301.8932562,57.66149609)
\curveto(301.8932562,56.47482946)(302.59992285,55.24816283)(304.26658947,55.14149617)
\lineto(304.26658947,54.54149619)
\lineto(304.67992279,54.54149619)
\lineto(304.67992279,55.1548295)
\curveto(306.07992274,55.2748295)(307.05325605,56.51482946)(307.05325605,57.88816275)
\closepath
\moveto(306.37325607,57.48816277)
\curveto(306.37325607,56.59482946)(305.73325609,55.70149615)(304.67992279,55.56816282)
\lineto(304.67992279,59.43482937)
\curveto(306.27992274,59.10149605)(306.37325607,57.80816276)(306.37325607,57.48816277)
\closepath
\moveto(304.26658947,60.76816267)
\curveto(302.62658951,61.11482932)(302.57325618,62.26149596)(302.57325618,62.52816261)
\curveto(302.57325618,63.32816259)(303.19992283,64.15482923)(304.26658947,64.2614959)
\closepath
}
}
{
\newrgbcolor{curcolor}{0 0 0}
\pscustom[linestyle=none,fillstyle=solid,fillcolor=curcolor]
{
\newpath
\moveto(232.26010222,56.63485628)
\curveto(232.26010222,57.74152292)(231.66010224,58.4215229)(231.54010225,58.54152289)
\curveto(231.00676893,59.12818954)(230.52676894,59.24818954)(229.88676896,59.40818953)
\lineto(229.88676896,63.00818942)
\curveto(231.02010226,62.95485609)(231.66010224,62.36818944)(231.86010224,61.6215228)
\curveto(231.8067689,61.63485613)(231.78010224,61.64818947)(231.64676891,61.64818947)
\curveto(231.30010225,61.64818947)(231.03343559,61.40818947)(231.03343559,61.03485615)
\curveto(231.03343559,60.62152283)(231.36676892,60.42152284)(231.64676891,60.42152284)
\curveto(231.68676891,60.42152284)(232.26010222,60.43485617)(232.26010222,61.08818948)
\curveto(232.26010222,62.38152278)(231.38010225,63.35485608)(229.88676896,63.43485608)
\lineto(229.88676896,64.03485606)
\lineto(229.47343564,64.03485606)
\lineto(229.47343564,63.42152275)
\curveto(227.99343568,63.28818942)(227.10010238,62.08818945)(227.10010238,60.87485616)
\curveto(227.10010238,59.95485618)(227.5934357,59.3148562)(227.83343569,59.08818954)
\curveto(228.34010234,58.56818956)(228.79343566,58.46152289)(229.47343564,58.28818957)
\lineto(229.47343564,54.31485635)
\curveto(228.30010234,54.38152302)(227.67343569,55.061523)(227.50010237,55.88818964)
\curveto(227.5534357,55.87485631)(227.66010236,55.86152297)(227.71343569,55.86152297)
\curveto(228.07343568,55.86152297)(228.32676901,56.1148563)(228.32676901,56.47485629)
\curveto(228.32676901,56.84818961)(228.04676902,57.0881896)(227.71343569,57.0881896)
\curveto(227.6334357,57.0881896)(227.10010238,57.06152294)(227.10010238,56.40818962)
\curveto(227.10010238,55.22152299)(227.80676902,53.99485636)(229.47343564,53.8881897)
\lineto(229.47343564,53.28818972)
\lineto(229.88676896,53.28818972)
\lineto(229.88676896,53.90152303)
\curveto(231.28676892,54.02152303)(232.26010222,55.26152299)(232.26010222,56.63485628)
\closepath
\moveto(231.58010224,56.23485629)
\curveto(231.58010224,55.34152299)(230.94010226,54.44818968)(229.88676896,54.31485635)
\lineto(229.88676896,58.1815229)
\curveto(231.48676891,57.84818958)(231.58010224,56.55485629)(231.58010224,56.23485629)
\closepath
\moveto(229.47343564,59.5148562)
\curveto(227.83343569,59.86152285)(227.78010236,61.00818948)(227.78010236,61.27485614)
\curveto(227.78010236,62.07485612)(228.40676901,62.90152276)(229.47343564,63.00818942)
\closepath
}
}
{
\newrgbcolor{curcolor}{0 0 0}
\pscustom[linestyle=none,fillstyle=solid,fillcolor=curcolor]
{
\newpath
\moveto(238.92674346,50.96818979)
\curveto(238.92674346,50.96818979)(238.92674346,51.03485645)(238.86007679,51.20818978)
\lineto(234.3267436,63.74152274)
\curveto(234.27341027,63.8881894)(234.22007693,64.03485606)(234.03341027,64.03485606)
\curveto(233.88674361,64.03485606)(233.76674361,63.91485606)(233.76674361,63.7681894)
\curveto(233.76674361,63.7681894)(233.76674361,63.70152274)(233.83341028,63.52818941)
\lineto(238.36674348,50.99485645)
\curveto(238.42007681,50.84818979)(238.47341014,50.70152313)(238.6600768,50.70152313)
\curveto(238.80674346,50.70152313)(238.92674346,50.82152312)(238.92674346,50.96818979)
\closepath
}
}
{
\newrgbcolor{curcolor}{0 0 0}
\pscustom[linestyle=none,fillstyle=solid,fillcolor=curcolor]
{
\newpath
\moveto(244.11338474,55.68818964)
\lineto(244.11338474,56.44818962)
\lineto(243.78005142,56.44818962)
\lineto(243.78005142,55.71485631)
\curveto(243.78005142,54.72818967)(243.38005143,54.22152302)(242.88671811,54.22152302)
\curveto(241.9933848,54.22152302)(241.9933848,55.43485632)(241.9933848,55.66152298)
\lineto(241.9933848,59.36818953)
\lineto(243.90005141,59.36818953)
\lineto(243.90005141,59.78152285)
\lineto(241.9933848,59.78152285)
\lineto(241.9933848,62.23485611)
\lineto(241.66005148,62.23485611)
\curveto(241.64671815,61.14152281)(241.24671816,59.71485619)(239.94005153,59.66152286)
\lineto(239.94005153,59.36818953)
\lineto(241.07338483,59.36818953)
\lineto(241.07338483,55.68818964)
\curveto(241.07338483,54.04818969)(242.31338479,53.8881897)(242.79338478,53.8881897)
\curveto(243.74005142,53.8881897)(244.11338474,54.83485634)(244.11338474,55.68818964)
\closepath
}
}
{
\newrgbcolor{curcolor}{0 0 0}
\pscustom[linestyle=none,fillstyle=solid,fillcolor=curcolor]
{
\newpath
\moveto(248.16671749,54.03485636)
\lineto(248.16671749,54.44818968)
\curveto(247.28671752,54.44818968)(247.23338419,54.51485635)(247.23338419,55.03485633)
\lineto(247.23338419,59.92818952)
\lineto(245.36671758,59.78152285)
\lineto(245.36671758,59.36818953)
\curveto(246.23338422,59.36818953)(246.35338421,59.28818954)(246.35338421,58.63485622)
\lineto(246.35338421,55.04818966)
\curveto(246.35338421,54.44818968)(246.20671755,54.44818968)(245.31338424,54.44818968)
\lineto(245.31338424,54.03485636)
\lineto(246.78005087,54.07485636)
\curveto(247.24671752,54.07485636)(247.71338417,54.04818969)(248.16671749,54.03485636)
\closepath
\moveto(247.43338418,62.08818945)
\curveto(247.43338418,62.44818944)(247.12671752,62.7948561)(246.72671754,62.7948561)
\curveto(246.27338422,62.7948561)(246.00671756,62.42152278)(246.00671756,62.08818945)
\curveto(246.00671756,61.72818946)(246.31338421,61.38152281)(246.7133842,61.38152281)
\curveto(247.16671752,61.38152281)(247.43338418,61.75485613)(247.43338418,62.08818945)
\closepath
}
}
{
\newrgbcolor{curcolor}{0 0 0}
\pscustom[linestyle=none,fillstyle=solid,fillcolor=curcolor]
{
\newpath
\moveto(259.42004777,54.03485636)
\lineto(259.42004777,54.44818968)
\curveto(258.72671446,54.44818968)(258.39338113,54.44818968)(258.3800478,54.84818967)
\lineto(258.3800478,57.39485626)
\curveto(258.3800478,58.54152289)(258.3800478,58.95485621)(257.96671448,59.4348562)
\curveto(257.78004782,59.66152286)(257.34004783,59.92818952)(256.56671452,59.92818952)
\curveto(255.44671456,59.92818952)(254.86004791,59.12818954)(254.63338125,58.62152289)
\curveto(254.44671459,59.78152285)(253.46004795,59.92818952)(252.86004797,59.92818952)
\curveto(251.88671466,59.92818952)(251.26004802,59.3548562)(250.88671469,58.52818956)
\lineto(250.88671469,59.92818952)
\lineto(249.00671475,59.78152285)
\lineto(249.00671475,59.36818953)
\curveto(249.94004805,59.36818953)(250.04671472,59.2748562)(250.04671472,58.62152289)
\lineto(250.04671472,55.04818966)
\curveto(250.04671472,54.44818968)(249.90004806,54.44818968)(249.00671475,54.44818968)
\lineto(249.00671475,54.03485636)
\lineto(250.51338137,54.07485636)
\lineto(252.00671466,54.03485636)
\lineto(252.00671466,54.44818968)
\curveto(251.11338135,54.44818968)(250.96671469,54.44818968)(250.96671469,55.04818966)
\lineto(250.96671469,57.50152292)
\curveto(250.96671469,58.88818955)(251.91338133,59.63485619)(252.76671464,59.63485619)
\curveto(253.60671461,59.63485619)(253.75338127,58.91485621)(253.75338127,58.15485624)
\lineto(253.75338127,55.04818966)
\curveto(253.75338127,54.44818968)(253.60671461,54.44818968)(252.7133813,54.44818968)
\lineto(252.7133813,54.03485636)
\lineto(254.22004793,54.07485636)
\lineto(255.71338121,54.03485636)
\lineto(255.71338121,54.44818968)
\curveto(254.82004791,54.44818968)(254.67338125,54.44818968)(254.67338125,55.04818966)
\lineto(254.67338125,57.50152292)
\curveto(254.67338125,58.88818955)(255.62004788,59.63485619)(256.47338119,59.63485619)
\curveto(257.31338117,59.63485619)(257.46004783,58.91485621)(257.46004783,58.15485624)
\lineto(257.46004783,55.04818966)
\curveto(257.46004783,54.44818968)(257.31338117,54.44818968)(256.42004786,54.44818968)
\lineto(256.42004786,54.03485636)
\lineto(257.92671448,54.07485636)
\closepath
}
}
{
\newrgbcolor{curcolor}{0 0 0}
\pscustom[linestyle=none,fillstyle=solid,fillcolor=curcolor]
{
\newpath
\moveto(265.22002374,55.62152298)
\curveto(265.22002374,55.75485631)(265.11335707,55.78152297)(265.04669041,55.78152297)
\curveto(264.92669041,55.78152297)(264.90002375,55.70152298)(264.87335708,55.59485631)
\curveto(264.40669043,54.22152302)(263.20669047,54.22152302)(263.07335714,54.22152302)
\curveto(262.40669049,54.22152302)(261.87335717,54.62152301)(261.56669051,55.11485633)
\curveto(261.16669053,55.75485631)(261.16669053,56.63485628)(261.16669053,57.11485627)
\lineto(264.88669041,57.11485627)
\curveto(265.18002374,57.11485627)(265.22002374,57.11485627)(265.22002374,57.39485626)
\curveto(265.22002374,58.71485622)(264.50002376,60.00818951)(262.83335714,60.00818951)
\curveto(261.28669052,60.00818951)(260.06002389,58.63485622)(260.06002389,56.96818961)
\curveto(260.06002389,55.18152299)(261.46002385,53.8881897)(262.99335714,53.8881897)
\curveto(264.62002376,53.8881897)(265.22002374,55.36818965)(265.22002374,55.62152298)
\closepath
\moveto(264.34002376,57.39485626)
\lineto(261.18002386,57.39485626)
\curveto(261.26002386,59.38152287)(262.38002382,59.71485619)(262.83335714,59.71485619)
\curveto(264.20669044,59.71485619)(264.34002376,57.91485624)(264.34002376,57.39485626)
\closepath
}
}
{
\newrgbcolor{curcolor}{0 0 0}
\pscustom[linestyle=none,fillstyle=solid,fillcolor=curcolor]
{
\newpath
\moveto(270.40667045,55.74152298)
\curveto(270.40667045,56.44818962)(270.00667047,56.84818961)(269.84667047,57.0081896)
\curveto(269.40667048,57.43485626)(268.8866705,57.54152292)(268.32667052,57.64818959)
\curveto(267.58000387,57.79485625)(266.68667057,57.96818958)(266.68667057,58.74152289)
\curveto(266.68667057,59.20818954)(267.03333722,59.75485619)(268.18000385,59.75485619)
\curveto(269.64667048,59.75485619)(269.71333714,58.55485622)(269.74000381,58.1415229)
\curveto(269.75333714,58.02152291)(269.9000038,58.02152291)(269.9000038,58.02152291)
\curveto(270.07333713,58.02152291)(270.07333713,58.08818957)(270.07333713,58.3415229)
\lineto(270.07333713,59.68818952)
\curveto(270.07333713,59.91485618)(270.07333713,60.00818951)(269.92667047,60.00818951)
\curveto(269.8600038,60.00818951)(269.83333714,60.00818951)(269.66000381,59.84818952)
\curveto(269.62000381,59.79485619)(269.48667048,59.67485619)(269.43333715,59.63485619)
\curveto(268.9266705,60.00818951)(268.38000385,60.00818951)(268.18000385,60.00818951)
\curveto(266.55333724,60.00818951)(266.04667058,59.11485621)(266.04667058,58.36818956)
\curveto(266.04667058,57.90152291)(266.26000391,57.52818959)(266.6200039,57.23485626)
\curveto(267.04667055,56.88818961)(267.42000388,56.80818961)(268.38000385,56.62152295)
\curveto(268.67333717,56.56818962)(269.76667047,56.35485629)(269.76667047,55.39485632)
\curveto(269.76667047,54.71485634)(269.30000382,54.18152302)(268.26000385,54.18152302)
\curveto(267.14000388,54.18152302)(266.6600039,54.941523)(266.40667057,56.0748563)
\curveto(266.36667057,56.24818963)(266.35333724,56.30152296)(266.22000391,56.30152296)
\curveto(266.04667058,56.30152296)(266.04667058,56.20818963)(266.04667058,55.96818964)
\lineto(266.04667058,54.20818969)
\curveto(266.04667058,53.98152303)(266.04667058,53.8881897)(266.19333725,53.8881897)
\curveto(266.26000391,53.8881897)(266.27333724,53.90152303)(266.52667057,54.15485636)
\curveto(266.55333724,54.18152302)(266.55333724,54.20818969)(266.79333723,54.46152301)
\curveto(267.38000388,53.90152303)(267.98000386,53.8881897)(268.26000385,53.8881897)
\curveto(269.79333714,53.8881897)(270.40667045,54.781523)(270.40667045,55.74152298)
\closepath
}
}
{
\newrgbcolor{curcolor}{0 0 0}
\pscustom[linestyle=none,fillstyle=solid,fillcolor=curcolor]
{
\newpath
\moveto(276.76665822,56.63485628)
\curveto(276.76665822,57.74152292)(276.16665823,58.4215229)(276.04665824,58.54152289)
\curveto(275.51332492,59.12818954)(275.03332493,59.24818954)(274.39332495,59.40818953)
\lineto(274.39332495,63.00818942)
\curveto(275.52665825,62.95485609)(276.16665823,62.36818944)(276.36665823,61.6215228)
\curveto(276.3133249,61.63485613)(276.28665823,61.64818947)(276.1533249,61.64818947)
\curveto(275.80665825,61.64818947)(275.53999159,61.40818947)(275.53999159,61.03485615)
\curveto(275.53999159,60.62152283)(275.87332491,60.42152284)(276.1533249,60.42152284)
\curveto(276.1933249,60.42152284)(276.76665822,60.43485617)(276.76665822,61.08818948)
\curveto(276.76665822,62.38152278)(275.88665824,63.35485608)(274.39332495,63.43485608)
\lineto(274.39332495,64.03485606)
\lineto(273.97999163,64.03485606)
\lineto(273.97999163,63.42152275)
\curveto(272.49999168,63.28818942)(271.60665837,62.08818945)(271.60665837,60.87485616)
\curveto(271.60665837,59.95485618)(272.09999169,59.3148562)(272.33999168,59.08818954)
\curveto(272.84665833,58.56818956)(273.29999165,58.46152289)(273.97999163,58.28818957)
\lineto(273.97999163,54.31485635)
\curveto(272.80665834,54.38152302)(272.17999169,55.061523)(272.00665836,55.88818964)
\curveto(272.05999169,55.87485631)(272.16665835,55.86152297)(272.21999169,55.86152297)
\curveto(272.57999168,55.86152297)(272.833325,56.1148563)(272.833325,56.47485629)
\curveto(272.833325,56.84818961)(272.55332501,57.0881896)(272.21999169,57.0881896)
\curveto(272.13999169,57.0881896)(271.60665837,57.06152294)(271.60665837,56.40818962)
\curveto(271.60665837,55.22152299)(272.31332502,53.99485636)(273.97999163,53.8881897)
\lineto(273.97999163,53.28818972)
\lineto(274.39332495,53.28818972)
\lineto(274.39332495,53.90152303)
\curveto(275.79332491,54.02152303)(276.76665822,55.26152299)(276.76665822,56.63485628)
\closepath
\moveto(276.08665824,56.23485629)
\curveto(276.08665824,55.34152299)(275.44665826,54.44818968)(274.39332495,54.31485635)
\lineto(274.39332495,58.1815229)
\curveto(275.99332491,57.84818958)(276.08665824,56.55485629)(276.08665824,56.23485629)
\closepath
\moveto(273.97999163,59.5148562)
\curveto(272.33999168,59.86152285)(272.28665835,61.00818948)(272.28665835,61.27485614)
\curveto(272.28665835,62.07485612)(272.913325,62.90152276)(273.97999163,63.00818942)
\closepath
}
}
{
\newrgbcolor{curcolor}{0.65490198 0.66274512 0.67450982}
\pscustom[linestyle=none,fillstyle=solid,fillcolor=curcolor]
{
\newpath
\moveto(257.02008137,59.54295101)
\curveto(257.02008137,59.54295101)(257.02008137,59.58295101)(256.98008137,59.686951)
\lineto(254.26008152,67.2069506)
\curveto(254.22808152,67.2949506)(254.19608152,67.3829506)(254.08408153,67.3829506)
\curveto(253.99608153,67.3829506)(253.92408153,67.3109506)(253.92408153,67.2229506)
\curveto(253.92408153,67.2229506)(253.92408153,67.18295061)(253.96408153,67.07895061)
\lineto(256.68408139,59.55895101)
\curveto(256.71608139,59.47095101)(256.74808139,59.38295102)(256.86008138,59.38295102)
\curveto(256.94808137,59.38295102)(257.02008137,59.45495101)(257.02008137,59.54295101)
\closepath
}
}
{
\newrgbcolor{curcolor}{0.65490198 0.66274512 0.67450982}
\pscustom[linestyle=none,fillstyle=solid,fillcolor=curcolor]
{
\newpath
\moveto(260.35606588,62.40695086)
\curveto(260.35606588,62.83095083)(260.11606589,63.07095082)(260.02006589,63.16695082)
\curveto(259.75606591,63.4229508)(259.44406592,63.4869508)(259.10806594,63.5509508)
\curveto(258.66006597,63.63895079)(258.12406599,63.74295079)(258.12406599,64.20695076)
\curveto(258.12406599,64.48695075)(258.33206598,64.81495073)(259.02006595,64.81495073)
\curveto(259.9000659,64.81495073)(259.9400659,64.09495077)(259.9560659,63.84695078)
\curveto(259.9640659,63.77495078)(260.05206589,63.77495078)(260.05206589,63.77495078)
\curveto(260.15606589,63.77495078)(260.15606589,63.81495078)(260.15606589,63.96695077)
\lineto(260.15606589,64.77495073)
\curveto(260.15606589,64.91095072)(260.15606589,64.96695072)(260.06806589,64.96695072)
\curveto(260.02806589,64.96695072)(260.01206589,64.96695072)(259.9080659,64.87095073)
\curveto(259.8840659,64.83895073)(259.80406591,64.76695073)(259.77206591,64.74295073)
\curveto(259.46806592,64.96695072)(259.14006594,64.96695072)(259.02006595,64.96695072)
\curveto(258.044066,64.96695072)(257.74006601,64.43095075)(257.74006601,63.98295077)
\curveto(257.74006601,63.70295079)(257.86806601,63.4789508)(258.084066,63.30295081)
\curveto(258.34006598,63.09495082)(258.56406597,63.04695082)(259.14006594,62.93495083)
\curveto(259.31606593,62.90295083)(259.9720659,62.77495084)(259.9720659,62.19895087)
\curveto(259.9720659,61.79095089)(259.69206591,61.47095091)(259.06806594,61.47095091)
\curveto(258.39606598,61.47095091)(258.10806599,61.92695088)(257.956066,62.60695085)
\curveto(257.932066,62.71095084)(257.924066,62.74295084)(257.84406601,62.74295084)
\curveto(257.74006601,62.74295084)(257.74006601,62.68695084)(257.74006601,62.54295085)
\lineto(257.74006601,61.4869509)
\curveto(257.74006601,61.35095091)(257.74006601,61.29495092)(257.82806601,61.29495092)
\curveto(257.86806601,61.29495092)(257.87606601,61.30295091)(258.028066,61.45495091)
\curveto(258.044066,61.47095091)(258.044066,61.4869509)(258.18806599,61.6389509)
\curveto(258.54006597,61.30295091)(258.90006595,61.29495092)(259.06806594,61.29495092)
\curveto(259.9880659,61.29495092)(260.35606588,61.83095089)(260.35606588,62.40695086)
\closepath
}
}
{
\newrgbcolor{curcolor}{0.65490198 0.66274512 0.67450982}
\pscustom[linestyle=none,fillstyle=solid,fillcolor=curcolor]
{
\newpath
\moveto(267.13205827,61.38295091)
\lineto(267.13205827,61.6309509)
\curveto(266.7160583,61.6309509)(266.51605831,61.6309509)(266.50805831,61.87095088)
\lineto(266.50805831,63.3989508)
\curveto(266.50805831,64.08695077)(266.50805831,64.33495076)(266.26005832,64.62295074)
\curveto(266.14805833,64.75895073)(265.88405834,64.91895072)(265.42005836,64.91895072)
\curveto(264.7480584,64.91895072)(264.39605842,64.43895075)(264.26005842,64.13495077)
\curveto(264.14805843,64.83095073)(263.55605846,64.91895072)(263.19605848,64.91895072)
\curveto(262.61205851,64.91895072)(262.23605853,64.57495074)(262.01205854,64.07895077)
\lineto(262.01205854,64.91895072)
\lineto(260.8840586,64.83095073)
\lineto(260.8840586,64.58295074)
\curveto(261.44405857,64.58295074)(261.50805857,64.52695075)(261.50805857,64.13495077)
\lineto(261.50805857,61.99095088)
\curveto(261.50805857,61.6309509)(261.42005857,61.6309509)(260.8840586,61.6309509)
\lineto(260.8840586,61.38295091)
\lineto(261.78805855,61.40695091)
\lineto(262.68405851,61.38295091)
\lineto(262.68405851,61.6309509)
\curveto(262.14805854,61.6309509)(262.06005854,61.6309509)(262.06005854,61.99095088)
\lineto(262.06005854,63.4629508)
\curveto(262.06005854,64.29495076)(262.62805851,64.74295073)(263.14005848,64.74295073)
\curveto(263.64405846,64.74295073)(263.73205845,64.31095076)(263.73205845,63.85495078)
\lineto(263.73205845,61.99095088)
\curveto(263.73205845,61.6309509)(263.64405846,61.6309509)(263.10805849,61.6309509)
\lineto(263.10805849,61.38295091)
\lineto(264.01205844,61.40695091)
\lineto(264.90805839,61.38295091)
\lineto(264.90805839,61.6309509)
\curveto(264.37205842,61.6309509)(264.28405842,61.6309509)(264.28405842,61.99095088)
\lineto(264.28405842,63.4629508)
\curveto(264.28405842,64.29495076)(264.85205839,64.74295073)(265.36405837,64.74295073)
\curveto(265.86805834,64.74295073)(265.95605834,64.31095076)(265.95605834,63.85495078)
\lineto(265.95605834,61.99095088)
\curveto(265.95605834,61.6309509)(265.86805834,61.6309509)(265.33205837,61.6309509)
\lineto(265.33205837,61.38295091)
\lineto(266.23605832,61.40695091)
\closepath
}
}
{
\newrgbcolor{curcolor}{0.65490198 0.66274512 0.67450982}
\pscustom[linestyle=none,fillstyle=solid,fillcolor=curcolor]
{
\newpath
\moveto(271.15604408,62.09495087)
\lineto(271.15604408,62.54295085)
\lineto(270.95604409,62.54295085)
\lineto(270.95604409,62.09495087)
\curveto(270.95604409,61.6309509)(270.7560441,61.5829509)(270.6680441,61.5829509)
\curveto(270.40404412,61.5829509)(270.37204412,61.94295088)(270.37204412,61.98295088)
\lineto(270.37204412,63.58295079)
\curveto(270.37204412,63.91895078)(270.37204412,64.23095076)(270.08404413,64.52695075)
\curveto(269.77204415,64.83895073)(269.37204417,64.96695072)(268.98804419,64.96695072)
\curveto(268.33204423,64.96695072)(267.78004426,64.59095074)(267.78004426,64.06295077)
\curveto(267.78004426,63.82295078)(267.94004425,63.68695079)(268.14804424,63.68695079)
\curveto(268.37204422,63.68695079)(268.51604422,63.84695078)(268.51604422,64.05495077)
\curveto(268.51604422,64.15095076)(268.47604422,64.41495075)(268.10804424,64.42295075)
\curveto(268.32404423,64.70295074)(268.71604421,64.79095073)(268.97204419,64.79095073)
\curveto(269.36404417,64.79095073)(269.82004415,64.47895075)(269.82004415,63.76695079)
\lineto(269.82004415,63.4709508)
\curveto(269.41204417,63.4469508)(268.8520442,63.4229508)(268.34804423,63.18295082)
\curveto(267.74804426,62.91095083)(267.54804427,62.49495085)(267.54804427,62.14295087)
\curveto(267.54804427,61.4949509)(268.32404423,61.29495092)(268.8280442,61.29495092)
\curveto(269.35604417,61.29495092)(269.72404415,61.6149509)(269.87604415,61.99095088)
\curveto(269.90804414,61.6709509)(270.12404413,61.33495091)(270.50004411,61.33495091)
\curveto(270.6680441,61.33495091)(271.15604408,61.44695091)(271.15604408,62.09495087)
\closepath
\moveto(269.82004415,62.50295085)
\curveto(269.82004415,61.74295089)(269.24404418,61.47095091)(268.8840442,61.47095091)
\curveto(268.49204422,61.47095091)(268.16404424,61.75095089)(268.16404424,62.15095087)
\curveto(268.16404424,62.59095085)(268.50004422,63.25495081)(269.82004415,63.30295081)
\closepath
}
}
{
\newrgbcolor{curcolor}{0.65490198 0.66274512 0.67450982}
\pscustom[linestyle=none,fillstyle=solid,fillcolor=curcolor]
{
\newpath
\moveto(273.3320291,61.38295091)
\lineto(273.3320291,61.6309509)
\curveto(272.79602912,61.6309509)(272.70802913,61.6309509)(272.70802913,61.99095088)
\lineto(272.70802913,66.93495062)
\lineto(271.55602919,66.84695062)
\lineto(271.55602919,66.59895064)
\curveto(272.11602916,66.59895064)(272.18002916,66.54295064)(272.18002916,66.15095066)
\lineto(272.18002916,61.99095088)
\curveto(272.18002916,61.6309509)(272.09202916,61.6309509)(271.55602919,61.6309509)
\lineto(271.55602919,61.38295091)
\lineto(272.44402914,61.40695091)
\closepath
}
}
{
\newrgbcolor{curcolor}{0.65490198 0.66274512 0.67450982}
\pscustom[linestyle=none,fillstyle=solid,fillcolor=curcolor]
{
\newpath
\moveto(275.55602704,61.38295091)
\lineto(275.55602704,61.6309509)
\curveto(275.02002707,61.6309509)(274.93202707,61.6309509)(274.93202707,61.99095088)
\lineto(274.93202707,66.93495062)
\lineto(273.78002713,66.84695062)
\lineto(273.78002713,66.59895064)
\curveto(274.3400271,66.59895064)(274.4040271,66.54295064)(274.4040271,66.15095066)
\lineto(274.4040271,61.99095088)
\curveto(274.4040271,61.6309509)(274.3160271,61.6309509)(273.78002713,61.6309509)
\lineto(273.78002713,61.38295091)
\lineto(274.66802708,61.40695091)
\closepath
}
}
{
\newrgbcolor{curcolor}{0.65490198 0.66274512 0.67450982}
\pscustom[linestyle=none,fillstyle=solid,fillcolor=curcolor]
{
\newpath
\moveto(281.94801303,62.94295083)
\curveto(281.94801303,63.60695079)(281.58801305,64.01495077)(281.51601305,64.08695077)
\curveto(281.19601307,64.43895075)(280.90801308,64.51095075)(280.5240131,64.60695074)
\lineto(280.5240131,66.76695063)
\curveto(281.20401307,66.73495063)(281.58801305,66.38295065)(281.70801304,65.93495067)
\curveto(281.67601304,65.94295067)(281.66001304,65.95095067)(281.58001305,65.95095067)
\curveto(281.37201306,65.95095067)(281.21201307,65.80695068)(281.21201307,65.58295069)
\curveto(281.21201307,65.3349507)(281.41201306,65.21495071)(281.58001305,65.21495071)
\curveto(281.60401305,65.21495071)(281.94801303,65.22295071)(281.94801303,65.61495069)
\curveto(281.94801303,66.39095065)(281.42001306,66.97495062)(280.5240131,67.02295061)
\lineto(280.5240131,67.3829506)
\lineto(280.27601312,67.3829506)
\lineto(280.27601312,67.01495061)
\curveto(279.38801316,66.93495062)(278.85201319,66.21495066)(278.85201319,65.48695069)
\curveto(278.85201319,64.93495072)(279.14801318,64.55095074)(279.29201317,64.41495075)
\curveto(279.59601315,64.10295077)(279.86801314,64.03895077)(280.27601312,63.93495078)
\lineto(280.27601312,61.5509509)
\curveto(279.57201315,61.5909509)(279.19601317,61.99895088)(279.09201318,62.49495085)
\curveto(279.12401318,62.48695085)(279.18801317,62.47895085)(279.22001317,62.47895085)
\curveto(279.43601316,62.47895085)(279.58801315,62.63095084)(279.58801315,62.84695083)
\curveto(279.58801315,63.07095082)(279.42001316,63.21495081)(279.22001317,63.21495081)
\curveto(279.17201317,63.21495081)(278.85201319,63.19895081)(278.85201319,62.80695084)
\curveto(278.85201319,62.09495087)(279.27601317,61.35895091)(280.27601312,61.29495092)
\lineto(280.27601312,60.93495093)
\lineto(280.5240131,60.93495093)
\lineto(280.5240131,61.30295091)
\curveto(281.36401306,61.37495091)(281.94801303,62.11895087)(281.94801303,62.94295083)
\closepath
\moveto(281.54001305,62.70295084)
\curveto(281.54001305,62.16695087)(281.15601307,61.6309509)(280.5240131,61.5509509)
\lineto(280.5240131,63.87095078)
\curveto(281.48401305,63.67095079)(281.54001305,62.89495083)(281.54001305,62.70295084)
\closepath
\moveto(280.27601312,64.67095074)
\curveto(279.29201317,64.87895073)(279.26001317,65.56695069)(279.26001317,65.72695068)
\curveto(279.26001317,66.20695066)(279.63601315,66.70295063)(280.27601312,66.76695063)
\closepath
}
}
{
\newrgbcolor{curcolor}{0.65490198 0.66274512 0.67450982}
\pscustom[linestyle=none,fillstyle=solid,fillcolor=curcolor]
{
\newpath
\moveto(285.99599705,62.99095083)
\curveto(285.99599705,63.94295078)(285.33999708,64.74295073)(284.47599713,64.74295073)
\curveto(284.09199715,64.74295073)(283.74799716,64.61495074)(283.45999718,64.33495076)
\lineto(283.45999718,65.89495067)
\curveto(283.61999717,65.84695068)(283.88399716,65.79095068)(284.13999714,65.79095068)
\curveto(285.12399709,65.79095068)(285.68399706,66.51895064)(285.68399706,66.62295063)
\curveto(285.68399706,66.67095063)(285.65999706,66.71095063)(285.60399707,66.71095063)
\curveto(285.60399707,66.71095063)(285.57999707,66.71095063)(285.53999707,66.68695063)
\curveto(285.37999708,66.61495064)(284.9879971,66.45495064)(284.45199713,66.45495064)
\curveto(284.13199714,66.45495064)(283.76399716,66.51095064)(283.38799718,66.67895063)
\curveto(283.32399719,66.70295063)(283.29199719,66.70295063)(283.29199719,66.70295063)
\curveto(283.21199719,66.70295063)(283.21199719,66.63895063)(283.21199719,66.51095064)
\lineto(283.21199719,64.14295077)
\curveto(283.21199719,63.99895077)(283.21199719,63.93495078)(283.32399719,63.93495078)
\curveto(283.37999718,63.93495078)(283.39599718,63.95895078)(283.42799718,64.00695077)
\curveto(283.51599718,64.13495077)(283.81199716,64.56695074)(284.45999713,64.56695074)
\curveto(284.87599711,64.56695074)(285.07599709,64.19895076)(285.13999709,64.05495077)
\curveto(285.26799708,63.75895079)(285.28399708,63.4469508)(285.28399708,63.04695082)
\curveto(285.28399708,62.76695084)(285.28399708,62.28695086)(285.09199709,61.95095088)
\curveto(284.8999971,61.6389509)(284.60399712,61.43095091)(284.23599714,61.43095091)
\curveto(283.65199717,61.43095091)(283.19599719,61.85495089)(283.0599972,62.32695086)
\curveto(283.0839972,62.31895086)(283.1079972,62.31095086)(283.19599719,62.31095086)
\curveto(283.45999718,62.31095086)(283.59599717,62.51095085)(283.59599717,62.70295084)
\curveto(283.59599717,62.89495083)(283.45999718,63.09495082)(283.19599719,63.09495082)
\curveto(283.0839972,63.09495082)(282.80399721,63.03895082)(282.80399721,62.67095084)
\curveto(282.80399721,61.98295088)(283.35599719,61.20695092)(284.25199714,61.20695092)
\curveto(285.17999709,61.20695092)(285.99599705,61.97495088)(285.99599705,62.99095083)
\closepath
}
}
{
\newrgbcolor{curcolor}{0.65490198 0.66274512 0.67450982}
\pscustom[linestyle=none,fillstyle=solid,fillcolor=curcolor]
{
\newpath
\moveto(289.94798107,62.94295083)
\curveto(289.94798107,63.60695079)(289.58798109,64.01495077)(289.51598109,64.08695077)
\curveto(289.19598111,64.43895075)(288.90798112,64.51095075)(288.52398114,64.60695074)
\lineto(288.52398114,66.76695063)
\curveto(289.20398111,66.73495063)(289.58798109,66.38295065)(289.70798108,65.93495067)
\curveto(289.67598108,65.94295067)(289.65998108,65.95095067)(289.57998109,65.95095067)
\curveto(289.3719811,65.95095067)(289.21198111,65.80695068)(289.21198111,65.58295069)
\curveto(289.21198111,65.3349507)(289.4119811,65.21495071)(289.57998109,65.21495071)
\curveto(289.60398109,65.21495071)(289.94798107,65.22295071)(289.94798107,65.61495069)
\curveto(289.94798107,66.39095065)(289.4199811,66.97495062)(288.52398114,67.02295061)
\lineto(288.52398114,67.3829506)
\lineto(288.27598116,67.3829506)
\lineto(288.27598116,67.01495061)
\curveto(287.3879812,66.93495062)(286.85198123,66.21495066)(286.85198123,65.48695069)
\curveto(286.85198123,64.93495072)(287.14798122,64.55095074)(287.29198121,64.41495075)
\curveto(287.59598119,64.10295077)(287.86798118,64.03895077)(288.27598116,63.93495078)
\lineto(288.27598116,61.5509509)
\curveto(287.57198119,61.5909509)(287.19598121,61.99895088)(287.09198122,62.49495085)
\curveto(287.12398122,62.48695085)(287.18798121,62.47895085)(287.21998121,62.47895085)
\curveto(287.4359812,62.47895085)(287.58798119,62.63095084)(287.58798119,62.84695083)
\curveto(287.58798119,63.07095082)(287.4199812,63.21495081)(287.21998121,63.21495081)
\curveto(287.17198122,63.21495081)(286.85198123,63.19895081)(286.85198123,62.80695084)
\curveto(286.85198123,62.09495087)(287.27598121,61.35895091)(288.27598116,61.29495092)
\lineto(288.27598116,60.93495093)
\lineto(288.52398114,60.93495093)
\lineto(288.52398114,61.30295091)
\curveto(289.3639811,61.37495091)(289.94798107,62.11895087)(289.94798107,62.94295083)
\closepath
\moveto(289.53998109,62.70295084)
\curveto(289.53998109,62.16695087)(289.15598111,61.6309509)(288.52398114,61.5509509)
\lineto(288.52398114,63.87095078)
\curveto(289.48398109,63.67095079)(289.53998109,62.89495083)(289.53998109,62.70295084)
\closepath
\moveto(288.27598116,64.67095074)
\curveto(287.29198121,64.87895073)(287.25998121,65.56695069)(287.25998121,65.72695068)
\curveto(287.25998121,66.20695066)(287.63598119,66.70295063)(288.27598116,66.76695063)
\closepath
}
}
{
\newrgbcolor{curcolor}{0 0 0}
\pscustom[linewidth=0.99999995,linecolor=curcolor]
{
\newpath
\moveto(21.99999269,94.38285354)
\lineto(66.99997984,94.38285354)
}
}
{
\newrgbcolor{curcolor}{0 0 0}
\pscustom[linestyle=none,fillstyle=solid,fillcolor=curcolor]
{
\newpath
\moveto(62.99998005,94.38285354)
\lineto(60.99998016,92.38285365)
\lineto(67.99997979,94.38285354)
\lineto(60.99998016,96.38285344)
\closepath
}
}
{
\newrgbcolor{curcolor}{0 0 0}
\pscustom[linewidth=0.53333332,linecolor=curcolor]
{
\newpath
\moveto(62.99998005,94.38285354)
\lineto(60.99998016,92.38285365)
\lineto(67.99997979,94.38285354)
\lineto(60.99998016,96.38285344)
\closepath
}
}
{
\newrgbcolor{curcolor}{0 0 0}
\pscustom[linewidth=0.99999995,linecolor=curcolor]
{
\newpath
\moveto(91.99998614,94.38285354)
\lineto(136.99998236,94.38285354)
}
}
{
\newrgbcolor{curcolor}{0 0 0}
\pscustom[linestyle=none,fillstyle=solid,fillcolor=curcolor]
{
\newpath
\moveto(132.99998257,94.38285354)
\lineto(130.99998268,92.38285365)
\lineto(137.99998231,94.38285354)
\lineto(130.99998268,96.38285344)
\closepath
}
}
{
\newrgbcolor{curcolor}{0 0 0}
\pscustom[linewidth=0.53333332,linecolor=curcolor]
{
\newpath
\moveto(132.99998257,94.38285354)
\lineto(130.99998268,92.38285365)
\lineto(137.99998231,94.38285354)
\lineto(130.99998268,96.38285344)
\closepath
}
}
{
\newrgbcolor{curcolor}{0 0 0}
\pscustom[linewidth=0.99999995,linecolor=curcolor]
{
\newpath
\moveto(161.99998488,94.38285354)
\lineto(206.9999811,94.38285354)
}
}
{
\newrgbcolor{curcolor}{0 0 0}
\pscustom[linestyle=none,fillstyle=solid,fillcolor=curcolor]
{
\newpath
\moveto(202.99998131,94.38285354)
\lineto(200.99998142,92.38285365)
\lineto(207.99998105,94.38285354)
\lineto(200.99998142,96.38285344)
\closepath
}
}
{
\newrgbcolor{curcolor}{0 0 0}
\pscustom[linewidth=0.53333332,linecolor=curcolor]
{
\newpath
\moveto(202.99998131,94.38285354)
\lineto(200.99998142,92.38285365)
\lineto(207.99998105,94.38285354)
\lineto(200.99998142,96.38285344)
\closepath
}
}
{
\newrgbcolor{curcolor}{0 0 0}
\pscustom[linestyle=none,fillstyle=solid,fillcolor=curcolor]
{
\newpath
\moveto(-13.35993465,96.14953548)
\curveto(-13.35993465,97.25620211)(-13.95993464,97.93620209)(-14.07993463,98.05620209)
\curveto(-14.61326795,98.64286874)(-15.09326794,98.76286873)(-15.73326792,98.92286873)
\lineto(-15.73326792,102.52286862)
\curveto(-14.59993462,102.46953529)(-13.95993464,101.88286864)(-13.75993464,101.13620199)
\curveto(-13.81326797,101.14953533)(-13.83993464,101.16286866)(-13.97326797,101.16286866)
\curveto(-14.31993463,101.16286866)(-14.58660128,100.92286867)(-14.58660128,100.54953535)
\curveto(-14.58660128,100.13620202)(-14.25326796,99.93620203)(-13.97326797,99.93620203)
\curveto(-13.93326797,99.93620203)(-13.35993465,99.94953536)(-13.35993465,100.60286868)
\curveto(-13.35993465,101.89620197)(-14.23993463,102.86953528)(-15.73326792,102.94953527)
\lineto(-15.73326792,103.54953526)
\lineto(-16.14660124,103.54953526)
\lineto(-16.14660124,102.93620194)
\curveto(-17.62660119,102.80286861)(-18.5199345,101.60286865)(-18.5199345,100.38953535)
\curveto(-18.5199345,99.46953538)(-18.02660118,98.8295354)(-17.78660119,98.60286874)
\curveto(-17.27993454,98.08286875)(-16.82660122,97.97620209)(-16.14660124,97.80286876)
\lineto(-16.14660124,93.82953555)
\curveto(-17.31993454,93.89620221)(-17.94660118,94.57620219)(-18.11993451,95.40286883)
\curveto(-18.06660118,95.3895355)(-17.95993452,95.37620217)(-17.90660118,95.37620217)
\curveto(-17.5466012,95.37620217)(-17.29326787,95.62953549)(-17.29326787,95.98953548)
\curveto(-17.29326787,96.3628688)(-17.57326786,96.6028688)(-17.90660118,96.6028688)
\curveto(-17.98660118,96.6028688)(-18.5199345,96.57620213)(-18.5199345,95.92286882)
\curveto(-18.5199345,94.73620219)(-17.81326785,93.50953556)(-16.14660124,93.40286889)
\lineto(-16.14660124,92.80286891)
\lineto(-15.73326792,92.80286891)
\lineto(-15.73326792,93.41620223)
\curveto(-14.33326796,93.53620222)(-13.35993465,94.77620219)(-13.35993465,96.14953548)
\closepath
\moveto(-14.03993463,95.74953549)
\curveto(-14.03993463,94.85620218)(-14.67993462,93.96286888)(-15.73326792,93.82953555)
\lineto(-15.73326792,97.6962021)
\curveto(-14.13326796,97.36286877)(-14.03993463,96.06953548)(-14.03993463,95.74953549)
\closepath
\moveto(-16.14660124,99.02953539)
\curveto(-17.78660119,99.37620205)(-17.83993452,100.52286868)(-17.83993452,100.78953534)
\curveto(-17.83993452,101.58953531)(-17.21326787,102.41620196)(-16.14660124,102.52286862)
\closepath
}
}
{
\newrgbcolor{curcolor}{0 0 0}
\pscustom[linestyle=none,fillstyle=solid,fillcolor=curcolor]
{
\newpath
\moveto(-5.71996012,93.54953556)
\lineto(-5.71996012,93.96286888)
\curveto(-6.43996009,93.96286888)(-6.67996009,93.98953554)(-6.98662674,94.3762022)
\lineto(-8.77329336,96.6828688)
\curveto(-8.37329337,97.18953545)(-7.86662672,97.84286876)(-7.54662673,98.18953542)
\curveto(-7.13329341,98.6695354)(-6.58662676,98.8695354)(-5.95996011,98.88286873)
\lineto(-5.95996011,99.29620205)
\curveto(-6.30662676,99.26953538)(-6.70662675,99.25620205)(-7.05329341,99.25620205)
\curveto(-7.4532934,99.25620205)(-8.15996004,99.28286872)(-8.33329337,99.29620205)
\lineto(-8.33329337,98.88286873)
\curveto(-8.05329338,98.85620206)(-7.94662672,98.68286874)(-7.94662672,98.46953541)
\curveto(-7.94662672,98.25620208)(-8.07996004,98.08286875)(-8.14662671,98.00286876)
\lineto(-8.97329335,96.96286879)
\lineto(-10.01329332,98.30953541)
\curveto(-10.13329332,98.44286874)(-10.13329332,98.46953541)(-10.13329332,98.54953541)
\curveto(-10.13329332,98.7495354)(-9.93329332,98.8695354)(-9.66662666,98.88286873)
\lineto(-9.66662666,99.29620205)
\lineto(-11.10662662,99.25620205)
\curveto(-11.38662661,99.25620205)(-12.01329326,99.26953538)(-12.37329325,99.29620205)
\lineto(-12.37329325,98.88286873)
\curveto(-11.43995994,98.88286873)(-11.42662661,98.8695354)(-10.79995996,98.06953542)
\lineto(-9.47996,96.34953547)
\curveto(-10.10662665,95.5495355)(-10.10662665,95.52286883)(-10.7332933,94.76286885)
\curveto(-11.37329328,93.98953554)(-12.15995992,93.96286888)(-12.43995991,93.96286888)
\lineto(-12.43995991,93.54953556)
\curveto(-12.09329326,93.57620222)(-11.67995994,93.58953555)(-11.33329328,93.58953555)
\lineto(-10.06662665,93.54953556)
\lineto(-10.06662665,93.96286888)
\curveto(-10.35995998,94.00286888)(-10.45329331,94.1762022)(-10.45329331,94.3762022)
\curveto(-10.45329331,94.66953552)(-10.06662665,95.10953551)(-9.25329334,96.06953548)
\lineto(-8.23996004,94.73620219)
\curveto(-8.13329338,94.58953552)(-7.95996005,94.3762022)(-7.95996005,94.2962022)
\curveto(-7.95996005,94.1762022)(-8.07996004,93.97620221)(-8.43996003,93.96286888)
\lineto(-8.43996003,93.54953556)
\lineto(-6.99996008,93.58953555)
\curveto(-6.63996009,93.58953555)(-6.1199601,93.57620222)(-5.71996012,93.54953556)
\closepath
}
}
{
\newrgbcolor{curcolor}{0 0 0}
\pscustom[linestyle=none,fillstyle=solid,fillcolor=curcolor]
{
\newpath
\moveto(4.44003776,91.70953561)
\lineto(4.44003776,92.05620227)
\lineto(-5.55996194,92.05620227)
\lineto(-5.55996194,91.70953561)
\closepath
}
}
{
\newrgbcolor{curcolor}{0 0 0}
\pscustom[linestyle=none,fillstyle=solid,fillcolor=curcolor]
{
\newpath
\moveto(10.1333466,90.36286899)
\curveto(10.1333466,90.49620231)(10.0533466,90.49620231)(9.92001327,90.50953565)
\curveto(8.86667997,90.57620231)(8.37334665,91.17620229)(8.25334665,91.65620228)
\curveto(8.21334665,91.80286894)(8.21334665,91.82953561)(8.21334665,92.29620226)
\lineto(8.21334665,94.2962022)
\curveto(8.21334665,94.69620219)(8.21334665,95.37620217)(8.18667999,95.5095355)
\curveto(8.01334666,96.38953547)(7.16001335,96.73620213)(6.64001337,96.88286879)
\curveto(8.21334665,97.33620211)(8.21334665,98.28286875)(8.21334665,98.65620207)
\lineto(8.21334665,101.056202)
\curveto(8.21334665,102.01620197)(8.21334665,102.30953529)(8.53334664,102.64286862)
\curveto(8.77334664,102.88286861)(9.08001329,103.2028686)(10.0133466,103.25620193)
\curveto(10.08001326,103.26953526)(10.1333466,103.3228686)(10.1333466,103.40286859)
\curveto(10.1333466,103.54953526)(10.02667993,103.54953526)(9.86667994,103.54953526)
\curveto(8.53334664,103.54953526)(7.34668001,102.86953528)(7.32001335,101.9095353)
\lineto(7.32001335,99.46953538)
\curveto(7.32001335,98.21620208)(7.32001335,98.00286876)(6.97334669,97.62953543)
\curveto(6.78668003,97.44286877)(6.42668004,97.08286878)(5.58668007,97.02953545)
\curveto(5.49334673,97.02953545)(5.4000134,97.01620212)(5.4000134,96.88286879)
\curveto(5.4000134,96.74953546)(5.4800134,96.74953546)(5.61334673,96.73620213)
\curveto(6.18668005,96.69620213)(7.32001335,96.41620214)(7.32001335,95.08286884)
\lineto(7.32001335,92.44286892)
\curveto(7.32001335,91.66953561)(7.32001335,91.21620229)(8.01334666,90.72286897)
\curveto(8.58667998,90.32286899)(9.45334662,90.21620232)(9.86667994,90.21620232)
\curveto(10.02667993,90.21620232)(10.1333466,90.21620232)(10.1333466,90.36286899)
\closepath
}
}
{
\newrgbcolor{curcolor}{0 0 0}
\pscustom[linestyle=none,fillstyle=solid,fillcolor=curcolor]
{
\newpath
\moveto(14.50665276,90.21620232)
\lineto(14.50665276,90.74953564)
\lineto(13.21331947,90.74953564)
\lineto(13.21331947,103.01620194)
\lineto(14.50665276,103.01620194)
\lineto(14.50665276,103.54953526)
\lineto(12.67998615,103.54953526)
\lineto(12.67998615,90.21620232)
\closepath
}
}
{
\newrgbcolor{curcolor}{0 0 0}
\pscustom[linestyle=none,fillstyle=solid,fillcolor=curcolor]
{
\newpath
\moveto(21.94665343,93.54953556)
\lineto(21.94665343,93.96286888)
\curveto(21.25332011,93.96286888)(20.91998679,93.96286888)(20.90665346,94.36286887)
\lineto(20.90665346,96.90953546)
\curveto(20.90665346,98.05620209)(20.90665346,98.46953541)(20.49332014,98.94953539)
\curveto(20.30665348,99.17620205)(19.86665349,99.44286871)(19.09332018,99.44286871)
\curveto(18.11998688,99.44286871)(17.49332023,98.8695354)(17.11998691,98.04286875)
\lineto(17.11998691,99.44286871)
\lineto(15.23998696,99.29620205)
\lineto(15.23998696,98.88286873)
\curveto(16.17332027,98.88286873)(16.27998693,98.7895354)(16.27998693,98.13620208)
\lineto(16.27998693,94.56286886)
\curveto(16.27998693,93.96286888)(16.13332027,93.96286888)(15.23998696,93.96286888)
\lineto(15.23998696,93.54953556)
\lineto(16.74665358,93.58953555)
\lineto(18.23998687,93.54953556)
\lineto(18.23998687,93.96286888)
\curveto(17.34665357,93.96286888)(17.1999869,93.96286888)(17.1999869,94.56286886)
\lineto(17.1999869,97.01620212)
\curveto(17.1999869,98.40286874)(18.14665354,99.14953539)(18.99998685,99.14953539)
\curveto(19.83998682,99.14953539)(19.98665349,98.42953541)(19.98665349,97.66953543)
\lineto(19.98665349,94.56286886)
\curveto(19.98665349,93.96286888)(19.83998682,93.96286888)(18.94665352,93.96286888)
\lineto(18.94665352,93.54953556)
\lineto(20.45332014,93.58953555)
\closepath
}
}
{
\newrgbcolor{curcolor}{0 0 0}
\pscustom[linestyle=none,fillstyle=solid,fillcolor=curcolor]
{
\newpath
\moveto(24.34664792,90.21620232)
\lineto(24.34664792,103.54953526)
\lineto(22.51998131,103.54953526)
\lineto(22.51998131,103.01620194)
\lineto(23.8133146,103.01620194)
\lineto(23.8133146,90.74953564)
\lineto(22.51998131,90.74953564)
\lineto(22.51998131,90.21620232)
\closepath
}
}
{
\newrgbcolor{curcolor}{0 0 0}
\pscustom[linestyle=none,fillstyle=solid,fillcolor=curcolor]
{
\newpath
\moveto(31.62664859,96.88286879)
\curveto(31.62664859,97.01620212)(31.54664859,97.01620212)(31.41331526,97.02953545)
\curveto(30.83998194,97.06953545)(29.70664865,97.34953544)(29.70664865,98.68286874)
\lineto(29.70664865,101.32286866)
\curveto(29.70664865,102.09620197)(29.70664865,102.54953529)(29.01331533,103.0428686)
\curveto(28.43998202,103.42953526)(27.58664871,103.54953526)(27.15998206,103.54953526)
\curveto(27.02664873,103.54953526)(26.8933154,103.54953526)(26.8933154,103.40286859)
\curveto(26.8933154,103.26953526)(26.97331539,103.26953526)(27.10664872,103.25620193)
\curveto(28.15998203,103.18953527)(28.65331534,102.58953528)(28.77331534,102.1095353)
\curveto(28.81331534,101.96286864)(28.81331534,101.93620197)(28.81331534,101.46953532)
\lineto(28.81331534,99.46953538)
\curveto(28.81331534,99.06953539)(28.81331534,98.38953541)(28.839982,98.25620208)
\curveto(29.01331533,97.37620211)(29.86664864,97.02953545)(30.38664862,96.88286879)
\curveto(28.81331534,96.42953547)(28.81331534,95.48286883)(28.81331534,95.10953551)
\lineto(28.81331534,92.70953558)
\curveto(28.81331534,91.74953561)(28.81331534,91.45620229)(28.49331535,91.12286896)
\curveto(28.25331536,90.88286897)(27.9466487,90.56286898)(27.01331539,90.50953565)
\curveto(26.94664873,90.49620231)(26.8933154,90.44286898)(26.8933154,90.36286899)
\curveto(26.8933154,90.21620232)(27.02664873,90.21620232)(27.15998206,90.21620232)
\curveto(28.49331535,90.21620232)(29.67998198,90.8962023)(29.70664865,91.85620227)
\lineto(29.70664865,94.2962022)
\curveto(29.70664865,95.5495355)(29.70664865,95.76286882)(30.0533153,96.13620215)
\curveto(30.23998196,96.32286881)(30.59998195,96.6828688)(31.43998193,96.73620213)
\curveto(31.53331526,96.73620213)(31.62664859,96.74953546)(31.62664859,96.88286879)
\closepath
}
}
{
\newrgbcolor{curcolor}{0 0 0}
\pscustom[linestyle=none,fillstyle=solid,fillcolor=curcolor]
{
\newpath
\moveto(38.50662135,96.14953548)
\curveto(38.50662135,97.25620211)(37.90662137,97.93620209)(37.78662137,98.05620209)
\curveto(37.25328805,98.64286874)(36.77328807,98.76286873)(36.13328809,98.92286873)
\lineto(36.13328809,102.52286862)
\curveto(37.26662138,102.46953529)(37.90662137,101.88286864)(38.10662136,101.13620199)
\curveto(38.05328803,101.14953533)(38.02662136,101.16286866)(37.89328803,101.16286866)
\curveto(37.54662138,101.16286866)(37.27995472,100.92286867)(37.27995472,100.54953535)
\curveto(37.27995472,100.13620202)(37.61328804,99.93620203)(37.89328803,99.93620203)
\curveto(37.93328803,99.93620203)(38.50662135,99.94953536)(38.50662135,100.60286868)
\curveto(38.50662135,101.89620197)(37.62662137,102.86953528)(36.13328809,102.94953527)
\lineto(36.13328809,103.54953526)
\lineto(35.71995476,103.54953526)
\lineto(35.71995476,102.93620194)
\curveto(34.23995481,102.80286861)(33.3466215,101.60286865)(33.3466215,100.38953535)
\curveto(33.3466215,99.46953538)(33.83995482,98.8295354)(34.07995481,98.60286874)
\curveto(34.58662147,98.08286875)(35.03995479,97.97620209)(35.71995476,97.80286876)
\lineto(35.71995476,93.82953555)
\curveto(34.54662147,93.89620221)(33.91995482,94.57620219)(33.74662149,95.40286883)
\curveto(33.79995482,95.3895355)(33.90662149,95.37620217)(33.95995482,95.37620217)
\curveto(34.31995481,95.37620217)(34.57328813,95.62953549)(34.57328813,95.98953548)
\curveto(34.57328813,96.3628688)(34.29328814,96.6028688)(33.95995482,96.6028688)
\curveto(33.87995482,96.6028688)(33.3466215,96.57620213)(33.3466215,95.92286882)
\curveto(33.3466215,94.73620219)(34.05328815,93.50953556)(35.71995476,93.40286889)
\lineto(35.71995476,92.80286891)
\lineto(36.13328809,92.80286891)
\lineto(36.13328809,93.41620223)
\curveto(37.53328804,93.53620222)(38.50662135,94.77620219)(38.50662135,96.14953548)
\closepath
\moveto(37.82662137,95.74953549)
\curveto(37.82662137,94.85620218)(37.18662139,93.96286888)(36.13328809,93.82953555)
\lineto(36.13328809,97.6962021)
\curveto(37.73328804,97.36286877)(37.82662137,96.06953548)(37.82662137,95.74953549)
\closepath
\moveto(35.71995476,99.02953539)
\curveto(34.07995481,99.37620205)(34.02662148,100.52286868)(34.02662148,100.78953534)
\curveto(34.02662148,101.58953531)(34.65328813,102.41620196)(35.71995476,102.52286862)
\closepath
}
}
{
\newrgbcolor{curcolor}{0.65490198 0.66274512 0.67450982}
\pscustom[linestyle=none,fillstyle=solid,fillcolor=curcolor]
{
\newpath
\moveto(37.08004601,96.54294332)
\curveto(37.08004601,96.54294332)(37.08004601,96.58294332)(37.04004601,96.68694331)
\lineto(34.32004615,104.20694291)
\curveto(34.28804616,104.29494291)(34.25604616,104.38294291)(34.14404616,104.38294291)
\curveto(34.05604617,104.38294291)(33.98404617,104.31094291)(33.98404617,104.22294291)
\curveto(33.98404617,104.22294291)(33.98404617,104.18294292)(34.02404617,104.07894292)
\lineto(36.74404603,96.55894332)
\curveto(36.77604602,96.47094332)(36.80804602,96.38294333)(36.92004602,96.38294333)
\curveto(37.00804601,96.38294333)(37.08004601,96.45494332)(37.08004601,96.54294332)
\closepath
}
}
{
\newrgbcolor{curcolor}{0.65490198 0.66274512 0.67450982}
\pscustom[linestyle=none,fillstyle=solid,fillcolor=curcolor]
{
\newpath
\moveto(40.41603051,99.40694317)
\curveto(40.41603051,99.83094314)(40.17603053,100.07094313)(40.08003053,100.16694313)
\curveto(39.81603055,100.42294311)(39.50403056,100.48694311)(39.16803058,100.55094311)
\curveto(38.7200306,100.6389431)(38.18403063,100.7429431)(38.18403063,101.20694307)
\curveto(38.18403063,101.48694306)(38.39203062,101.81494304)(39.08003058,101.81494304)
\curveto(39.96003054,101.81494304)(40.00003054,101.09494308)(40.01603054,100.84694309)
\curveto(40.02403054,100.7749431)(40.11203053,100.7749431)(40.11203053,100.7749431)
\curveto(40.21603053,100.7749431)(40.21603053,100.81494309)(40.21603053,100.96694309)
\lineto(40.21603053,101.77494304)
\curveto(40.21603053,101.91094304)(40.21603053,101.96694303)(40.12803053,101.96694303)
\curveto(40.08803053,101.96694303)(40.07203053,101.96694303)(39.96803054,101.87094304)
\curveto(39.94403054,101.83894304)(39.86403054,101.76694304)(39.83203055,101.74294304)
\curveto(39.52803056,101.96694303)(39.20003058,101.96694303)(39.08003058,101.96694303)
\curveto(38.10403064,101.96694303)(37.80003065,101.43094306)(37.80003065,100.98294308)
\curveto(37.80003065,100.7029431)(37.92803065,100.47894311)(38.14403063,100.30294312)
\curveto(38.40003062,100.09494313)(38.62403061,100.04694313)(39.20003058,99.93494314)
\curveto(39.37603057,99.90294314)(40.03203053,99.77494315)(40.03203053,99.19894318)
\curveto(40.03203053,98.7909432)(39.75203055,98.47094322)(39.12803058,98.47094322)
\curveto(38.45603062,98.47094322)(38.16803063,98.92694319)(38.01603064,99.60694316)
\curveto(37.99203064,99.71094315)(37.98403064,99.74294315)(37.90403065,99.74294315)
\curveto(37.80003065,99.74294315)(37.80003065,99.68694315)(37.80003065,99.54294316)
\lineto(37.80003065,98.48694322)
\curveto(37.80003065,98.35094322)(37.80003065,98.29494323)(37.88803065,98.29494323)
\curveto(37.92803065,98.29494323)(37.93603065,98.30294323)(38.08803064,98.45494322)
\curveto(38.10403064,98.47094322)(38.10403064,98.48694322)(38.24803063,98.63894321)
\curveto(38.60003061,98.30294323)(38.96003059,98.29494323)(39.12803058,98.29494323)
\curveto(40.04803053,98.29494323)(40.41603051,98.8309432)(40.41603051,99.40694317)
\closepath
}
}
{
\newrgbcolor{curcolor}{0.65490198 0.66274512 0.67450982}
\pscustom[linestyle=none,fillstyle=solid,fillcolor=curcolor]
{
\newpath
\moveto(47.19202291,98.38294322)
\lineto(47.19202291,98.63094321)
\curveto(46.77602293,98.63094321)(46.57602294,98.63094321)(46.56802294,98.8709432)
\lineto(46.56802294,100.39894311)
\curveto(46.56802294,101.08694308)(46.56802294,101.33494307)(46.32002296,101.62294305)
\curveto(46.20802296,101.75894304)(45.94402298,101.91894304)(45.480023,101.91894304)
\curveto(44.80802304,101.91894304)(44.45602306,101.43894306)(44.32002306,101.13494308)
\curveto(44.20802307,101.83094304)(43.6160231,101.91894304)(43.25602312,101.91894304)
\curveto(42.67202315,101.91894304)(42.29602317,101.57494305)(42.07202318,101.07894308)
\lineto(42.07202318,101.91894304)
\lineto(40.94402324,101.83094304)
\lineto(40.94402324,101.58294305)
\curveto(41.50402321,101.58294305)(41.56802321,101.52694306)(41.56802321,101.13494308)
\lineto(41.56802321,98.99094319)
\curveto(41.56802321,98.63094321)(41.48002321,98.63094321)(40.94402324,98.63094321)
\lineto(40.94402324,98.38294322)
\lineto(41.84802319,98.40694322)
\lineto(42.74402315,98.38294322)
\lineto(42.74402315,98.63094321)
\curveto(42.20802317,98.63094321)(42.12002318,98.63094321)(42.12002318,98.99094319)
\lineto(42.12002318,100.46294311)
\curveto(42.12002318,101.29494307)(42.68802315,101.74294304)(43.20002312,101.74294304)
\curveto(43.7040231,101.74294304)(43.79202309,101.31094307)(43.79202309,100.85494309)
\lineto(43.79202309,98.99094319)
\curveto(43.79202309,98.63094321)(43.7040231,98.63094321)(43.16802312,98.63094321)
\lineto(43.16802312,98.38294322)
\lineto(44.07202308,98.40694322)
\lineto(44.96802303,98.38294322)
\lineto(44.96802303,98.63094321)
\curveto(44.43202306,98.63094321)(44.34402306,98.63094321)(44.34402306,98.99094319)
\lineto(44.34402306,100.46294311)
\curveto(44.34402306,101.29494307)(44.91202303,101.74294304)(45.424023,101.74294304)
\curveto(45.92802298,101.74294304)(46.01602297,101.31094307)(46.01602297,100.85494309)
\lineto(46.01602297,98.99094319)
\curveto(46.01602297,98.63094321)(45.92802298,98.63094321)(45.39202301,98.63094321)
\lineto(45.39202301,98.38294322)
\lineto(46.29602296,98.40694322)
\closepath
}
}
{
\newrgbcolor{curcolor}{0.65490198 0.66274512 0.67450982}
\pscustom[linestyle=none,fillstyle=solid,fillcolor=curcolor]
{
\newpath
\moveto(51.21600872,99.09494318)
\lineto(51.21600872,99.54294316)
\lineto(51.01600873,99.54294316)
\lineto(51.01600873,99.09494318)
\curveto(51.01600873,98.63094321)(50.81600874,98.58294321)(50.72800874,98.58294321)
\curveto(50.46400876,98.58294321)(50.43200876,98.94294319)(50.43200876,98.98294319)
\lineto(50.43200876,100.58294311)
\curveto(50.43200876,100.91894309)(50.43200876,101.23094307)(50.14400877,101.52694306)
\curveto(49.83200879,101.83894304)(49.43200881,101.96694303)(49.04800883,101.96694303)
\curveto(48.39200886,101.96694303)(47.84000889,101.59094305)(47.84000889,101.06294308)
\curveto(47.84000889,100.82294309)(48.00000888,100.6869431)(48.20800887,100.6869431)
\curveto(48.43200886,100.6869431)(48.57600885,100.84694309)(48.57600885,101.05494308)
\curveto(48.57600885,101.15094308)(48.53600886,101.41494306)(48.16800888,101.42294306)
\curveto(48.38400886,101.70294305)(48.77600884,101.79094304)(49.03200883,101.79094304)
\curveto(49.42400881,101.79094304)(49.88000879,101.47894306)(49.88000879,100.7669431)
\lineto(49.88000879,100.47094311)
\curveto(49.47200881,100.44694311)(48.91200884,100.42294311)(48.40800886,100.18294313)
\curveto(47.8080089,99.91094314)(47.60800891,99.49494316)(47.60800891,99.14294318)
\curveto(47.60800891,98.49494322)(48.38400886,98.29494323)(48.88800884,98.29494323)
\curveto(49.41600881,98.29494323)(49.78400879,98.61494321)(49.93600878,98.99094319)
\curveto(49.96800878,98.67094321)(50.18400877,98.33494322)(50.56000875,98.33494322)
\curveto(50.72800874,98.33494322)(51.21600872,98.44694322)(51.21600872,99.09494318)
\closepath
\moveto(49.88000879,99.50294316)
\curveto(49.88000879,98.7429432)(49.30400882,98.47094322)(48.94400884,98.47094322)
\curveto(48.55200886,98.47094322)(48.22400887,98.7509432)(48.22400887,99.15094318)
\curveto(48.22400887,99.59094316)(48.56000886,100.25494312)(49.88000879,100.30294312)
\closepath
}
}
{
\newrgbcolor{curcolor}{0.65490198 0.66274512 0.67450982}
\pscustom[linestyle=none,fillstyle=solid,fillcolor=curcolor]
{
\newpath
\moveto(53.39199373,98.38294322)
\lineto(53.39199373,98.63094321)
\curveto(52.85599376,98.63094321)(52.76799377,98.63094321)(52.76799377,98.99094319)
\lineto(52.76799377,103.93494293)
\lineto(51.61599383,103.84694293)
\lineto(51.61599383,103.59894295)
\curveto(52.1759938,103.59894295)(52.23999379,103.54294295)(52.23999379,103.15094297)
\lineto(52.23999379,98.99094319)
\curveto(52.23999379,98.63094321)(52.1519938,98.63094321)(51.61599383,98.63094321)
\lineto(51.61599383,98.38294322)
\lineto(52.50399378,98.40694322)
\closepath
}
}
{
\newrgbcolor{curcolor}{0.65490198 0.66274512 0.67450982}
\pscustom[linestyle=none,fillstyle=solid,fillcolor=curcolor]
{
\newpath
\moveto(55.61599168,98.38294322)
\lineto(55.61599168,98.63094321)
\curveto(55.0799917,98.63094321)(54.99199171,98.63094321)(54.99199171,98.99094319)
\lineto(54.99199171,103.93494293)
\lineto(53.83999177,103.84694293)
\lineto(53.83999177,103.59894295)
\curveto(54.39999174,103.59894295)(54.46399174,103.54294295)(54.46399174,103.15094297)
\lineto(54.46399174,98.99094319)
\curveto(54.46399174,98.63094321)(54.37599174,98.63094321)(53.83999177,98.63094321)
\lineto(53.83999177,98.38294322)
\lineto(54.72799172,98.40694322)
\closepath
}
}
{
\newrgbcolor{curcolor}{0.65490198 0.66274512 0.67450982}
\pscustom[linestyle=none,fillstyle=solid,fillcolor=curcolor]
{
\newpath
\moveto(62.00797767,99.94294314)
\curveto(62.00797767,100.6069431)(61.64797769,101.01494308)(61.57597769,101.08694308)
\curveto(61.25597771,101.43894306)(60.96797772,101.51094306)(60.58397774,101.60694305)
\lineto(60.58397774,103.76694294)
\curveto(61.26397771,103.73494294)(61.64797769,103.38294296)(61.76797768,102.93494298)
\curveto(61.73597768,102.94294298)(61.71997768,102.95094298)(61.63997769,102.95094298)
\curveto(61.4319777,102.95094298)(61.27197771,102.80694299)(61.27197771,102.582943)
\curveto(61.27197771,102.33494301)(61.4719777,102.21494302)(61.63997769,102.21494302)
\curveto(61.66397768,102.21494302)(62.00797767,102.22294302)(62.00797767,102.614943)
\curveto(62.00797767,103.39094296)(61.47997769,103.97494293)(60.58397774,104.02294292)
\lineto(60.58397774,104.38294291)
\lineto(60.33597775,104.38294291)
\lineto(60.33597775,104.01494292)
\curveto(59.4479778,103.93494293)(58.91197783,103.21494297)(58.91197783,102.48694301)
\curveto(58.91197783,101.93494303)(59.20797781,101.55094305)(59.35197781,101.41494306)
\curveto(59.65597779,101.10294308)(59.92797778,101.03894308)(60.33597775,100.93494309)
\lineto(60.33597775,98.55094321)
\curveto(59.63197779,98.59094321)(59.25597781,98.99894319)(59.15197782,99.49494316)
\curveto(59.18397782,99.48694316)(59.24797781,99.47894316)(59.27997781,99.47894316)
\curveto(59.4959778,99.47894316)(59.64797779,99.63094316)(59.64797779,99.84694314)
\curveto(59.64797779,100.07094313)(59.4799778,100.21494312)(59.27997781,100.21494312)
\curveto(59.23197781,100.21494312)(58.91197783,100.19894313)(58.91197783,99.80694315)
\curveto(58.91197783,99.09494318)(59.33597781,98.35894322)(60.33597775,98.29494323)
\lineto(60.33597775,97.93494324)
\lineto(60.58397774,97.93494324)
\lineto(60.58397774,98.30294323)
\curveto(61.4239777,98.37494322)(62.00797767,99.11894318)(62.00797767,99.94294314)
\closepath
\moveto(61.59997769,99.70294315)
\curveto(61.59997769,99.16694318)(61.21597771,98.63094321)(60.58397774,98.55094321)
\lineto(60.58397774,100.87094309)
\curveto(61.54397769,100.6709431)(61.59997769,99.89494314)(61.59997769,99.70294315)
\closepath
\moveto(60.33597775,101.67094305)
\curveto(59.35197781,101.87894304)(59.31997781,102.566943)(59.31997781,102.72694299)
\curveto(59.31997781,103.20694297)(59.69597779,103.70294294)(60.33597775,103.76694294)
\closepath
}
}
{
\newrgbcolor{curcolor}{0.65490198 0.66274512 0.67450982}
\pscustom[linestyle=none,fillstyle=solid,fillcolor=curcolor]
{
\newpath
\moveto(66.11996168,99.72694315)
\curveto(66.11996168,100.01494314)(66.03196169,100.37494312)(65.7279617,100.7109431)
\curveto(65.57596171,100.87894309)(65.44796172,100.95894309)(64.93596174,101.27894307)
\curveto(65.51196171,101.57494305)(65.90396169,101.99094303)(65.90396169,102.518943)
\curveto(65.90396169,103.25494296)(65.19196173,103.71094294)(64.46396177,103.71094294)
\curveto(63.66396181,103.71094294)(63.01596184,103.11894297)(63.01596184,102.37494301)
\curveto(63.01596184,102.23094302)(63.03196184,101.87094304)(63.36796183,101.49494306)
\curveto(63.45596182,101.39894306)(63.75196181,101.19894307)(63.9519618,101.06294308)
\curveto(63.48796182,100.83094309)(62.79996186,100.38294312)(62.79996186,99.59094316)
\curveto(62.79996186,98.7429432)(63.61596181,98.20694323)(64.45596177,98.20694323)
\curveto(65.35996172,98.20694323)(66.11996168,98.8709432)(66.11996168,99.72694315)
\closepath
\moveto(65.55196171,102.518943)
\curveto(65.55196171,102.06294303)(65.23996173,101.67894305)(64.75996175,101.39894306)
\lineto(63.7679618,102.03894303)
\curveto(63.39996182,102.27894302)(63.36796183,102.550943)(63.36796183,102.68694299)
\curveto(63.36796183,103.17494297)(63.8879618,103.51094295)(64.45596177,103.51094295)
\curveto(65.03996174,103.51094295)(65.55196171,103.09494297)(65.55196171,102.518943)
\closepath
\moveto(65.7199617,99.43894317)
\curveto(65.7199617,98.8469432)(65.11996173,98.43094322)(64.46396177,98.43094322)
\curveto(63.7759618,98.43094322)(63.19996183,98.92694319)(63.19996183,99.59094316)
\curveto(63.19996183,100.05494313)(63.45596182,100.56694311)(64.13596179,100.94294309)
\lineto(65.11996173,100.31894312)
\curveto(65.34396172,100.16694313)(65.7199617,99.92694314)(65.7199617,99.43894317)
\closepath
}
}
{
\newrgbcolor{curcolor}{0.65490198 0.66274512 0.67450982}
\pscustom[linestyle=none,fillstyle=solid,fillcolor=curcolor]
{
\newpath
\moveto(70.00794571,99.94294314)
\curveto(70.00794571,100.6069431)(69.64794573,101.01494308)(69.57594573,101.08694308)
\curveto(69.25594575,101.43894306)(68.96794576,101.51094306)(68.58394578,101.60694305)
\lineto(68.58394578,103.76694294)
\curveto(69.26394575,103.73494294)(69.64794573,103.38294296)(69.76794572,102.93494298)
\curveto(69.73594572,102.94294298)(69.71994572,102.95094298)(69.63994573,102.95094298)
\curveto(69.43194574,102.95094298)(69.27194575,102.80694299)(69.27194575,102.582943)
\curveto(69.27194575,102.33494301)(69.47194574,102.21494302)(69.63994573,102.21494302)
\curveto(69.66394573,102.21494302)(70.00794571,102.22294302)(70.00794571,102.614943)
\curveto(70.00794571,103.39094296)(69.47994574,103.97494293)(68.58394578,104.02294292)
\lineto(68.58394578,104.38294291)
\lineto(68.3359458,104.38294291)
\lineto(68.3359458,104.01494292)
\curveto(67.44794584,103.93494293)(66.91194587,103.21494297)(66.91194587,102.48694301)
\curveto(66.91194587,101.93494303)(67.20794585,101.55094305)(67.35194585,101.41494306)
\curveto(67.65594583,101.10294308)(67.92794582,101.03894308)(68.3359458,100.93494309)
\lineto(68.3359458,98.55094321)
\curveto(67.63194583,98.59094321)(67.25594585,98.99894319)(67.15194586,99.49494316)
\curveto(67.18394586,99.48694316)(67.24794585,99.47894316)(67.27994585,99.47894316)
\curveto(67.49594584,99.47894316)(67.64794583,99.63094316)(67.64794583,99.84694314)
\curveto(67.64794583,100.07094313)(67.47994584,100.21494312)(67.27994585,100.21494312)
\curveto(67.23194585,100.21494312)(66.91194587,100.19894313)(66.91194587,99.80694315)
\curveto(66.91194587,99.09494318)(67.33594585,98.35894322)(68.3359458,98.29494323)
\lineto(68.3359458,97.93494324)
\lineto(68.58394578,97.93494324)
\lineto(68.58394578,98.30294323)
\curveto(69.42394574,98.37494322)(70.00794571,99.11894318)(70.00794571,99.94294314)
\closepath
\moveto(69.59994573,99.70294315)
\curveto(69.59994573,99.16694318)(69.21594575,98.63094321)(68.58394578,98.55094321)
\lineto(68.58394578,100.87094309)
\curveto(69.54394573,100.6709431)(69.59994573,99.89494314)(69.59994573,99.70294315)
\closepath
\moveto(68.3359458,101.67094305)
\curveto(67.35194585,101.87894304)(67.31994585,102.566943)(67.31994585,102.72694299)
\curveto(67.31994585,103.20694297)(67.69594583,103.70294294)(68.3359458,103.76694294)
\closepath
}
}
{
\newrgbcolor{curcolor}{0 0 0}
\pscustom[linewidth=0.99999995,linecolor=curcolor]
{
\newpath
\moveto(41.9999811,94.38285354)
\lineto(41.9999811,66.38286614)
}
}
{
\newrgbcolor{curcolor}{0 0 0}
\pscustom[linestyle=none,fillstyle=solid,fillcolor=curcolor]
{
\newpath
\moveto(41.9999811,93.40285359)
\curveto(41.44798113,93.40285359)(40.99998115,93.85085357)(40.99998115,94.40285354)
\curveto(40.99998115,94.95485351)(41.44798113,95.40285349)(41.9999811,95.40285349)
\curveto(42.55198107,95.40285349)(42.99998105,94.95485351)(42.99998105,94.40285354)
\curveto(42.99998105,93.85085357)(42.55198107,93.40285359)(41.9999811,93.40285359)
\closepath
}
}
{
\newrgbcolor{curcolor}{0 0 0}
\pscustom[linewidth=0.26666666,linecolor=curcolor]
{
\newpath
\moveto(41.9999811,93.40285359)
\curveto(41.44798113,93.40285359)(40.99998115,93.85085357)(40.99998115,94.40285354)
\curveto(40.99998115,94.95485351)(41.44798113,95.40285349)(41.9999811,95.40285349)
\curveto(42.55198107,95.40285349)(42.99998105,94.95485351)(42.99998105,94.40285354)
\curveto(42.99998105,93.85085357)(42.55198107,93.40285359)(41.9999811,93.40285359)
\closepath
}
}
{
\newrgbcolor{curcolor}{0 0 0}
\pscustom[linestyle=none,fillstyle=solid,fillcolor=curcolor]
{
\newpath
\moveto(41.9999811,70.38286593)
\lineto(39.99998121,72.38286583)
\lineto(41.9999811,65.38286619)
\lineto(43.999981,72.38286583)
\closepath
}
}
{
\newrgbcolor{curcolor}{0 0 0}
\pscustom[linewidth=0.53333332,linecolor=curcolor]
{
\newpath
\moveto(41.9999811,70.38286593)
\lineto(39.99998121,72.38286583)
\lineto(41.9999811,65.38286619)
\lineto(43.999981,72.38286583)
\closepath
}
}
{
\newrgbcolor{curcolor}{0 0 0}
\pscustom[linewidth=0.99999995,linecolor=curcolor]
{
\newpath
\moveto(111.99998362,94.38285354)
\lineto(111.99998362,66.38286614)
}
}
{
\newrgbcolor{curcolor}{0 0 0}
\pscustom[linestyle=none,fillstyle=solid,fillcolor=curcolor]
{
\newpath
\moveto(111.99998362,93.40285359)
\curveto(111.44798365,93.40285359)(110.99998367,93.85085357)(110.99998367,94.40285354)
\curveto(110.99998367,94.95485351)(111.44798365,95.40285349)(111.99998362,95.40285349)
\curveto(112.55198359,95.40285349)(112.99998357,94.95485351)(112.99998357,94.40285354)
\curveto(112.99998357,93.85085357)(112.55198359,93.40285359)(111.99998362,93.40285359)
\closepath
}
}
{
\newrgbcolor{curcolor}{0 0 0}
\pscustom[linewidth=0.26666666,linecolor=curcolor]
{
\newpath
\moveto(111.99998362,93.40285359)
\curveto(111.44798365,93.40285359)(110.99998367,93.85085357)(110.99998367,94.40285354)
\curveto(110.99998367,94.95485351)(111.44798365,95.40285349)(111.99998362,95.40285349)
\curveto(112.55198359,95.40285349)(112.99998357,94.95485351)(112.99998357,94.40285354)
\curveto(112.99998357,93.85085357)(112.55198359,93.40285359)(111.99998362,93.40285359)
\closepath
}
}
{
\newrgbcolor{curcolor}{0 0 0}
\pscustom[linestyle=none,fillstyle=solid,fillcolor=curcolor]
{
\newpath
\moveto(111.99998362,70.38286593)
\lineto(109.99998373,72.38286583)
\lineto(111.99998362,65.38286619)
\lineto(113.99998352,72.38286583)
\closepath
}
}
{
\newrgbcolor{curcolor}{0 0 0}
\pscustom[linewidth=0.53333332,linecolor=curcolor]
{
\newpath
\moveto(111.99998362,70.38286593)
\lineto(109.99998373,72.38286583)
\lineto(111.99998362,65.38286619)
\lineto(113.99998352,72.38286583)
\closepath
}
}
{
\newrgbcolor{curcolor}{0 0 0}
\pscustom[linewidth=0.99999995,linecolor=curcolor]
{
\newpath
\moveto(181.99998236,94.38285354)
\lineto(181.99998236,66.38286614)
}
}
{
\newrgbcolor{curcolor}{0 0 0}
\pscustom[linestyle=none,fillstyle=solid,fillcolor=curcolor]
{
\newpath
\moveto(181.99998236,93.40285359)
\curveto(181.44798239,93.40285359)(180.99998241,93.85085357)(180.99998241,94.40285354)
\curveto(180.99998241,94.95485351)(181.44798239,95.40285349)(181.99998236,95.40285349)
\curveto(182.55198233,95.40285349)(182.99998231,94.95485351)(182.99998231,94.40285354)
\curveto(182.99998231,93.85085357)(182.55198233,93.40285359)(181.99998236,93.40285359)
\closepath
}
}
{
\newrgbcolor{curcolor}{0 0 0}
\pscustom[linewidth=0.26666666,linecolor=curcolor]
{
\newpath
\moveto(181.99998236,93.40285359)
\curveto(181.44798239,93.40285359)(180.99998241,93.85085357)(180.99998241,94.40285354)
\curveto(180.99998241,94.95485351)(181.44798239,95.40285349)(181.99998236,95.40285349)
\curveto(182.55198233,95.40285349)(182.99998231,94.95485351)(182.99998231,94.40285354)
\curveto(182.99998231,93.85085357)(182.55198233,93.40285359)(181.99998236,93.40285359)
\closepath
}
}
{
\newrgbcolor{curcolor}{0 0 0}
\pscustom[linestyle=none,fillstyle=solid,fillcolor=curcolor]
{
\newpath
\moveto(181.99998236,70.38286593)
\lineto(179.99998247,72.38286583)
\lineto(181.99998236,65.38286619)
\lineto(183.99998226,72.38286583)
\closepath
}
}
{
\newrgbcolor{curcolor}{0 0 0}
\pscustom[linewidth=0.53333332,linecolor=curcolor]
{
\newpath
\moveto(181.99998236,70.38286593)
\lineto(179.99998247,72.38286583)
\lineto(181.99998236,65.38286619)
\lineto(183.99998226,72.38286583)
\closepath
}
}
{
\newrgbcolor{curcolor}{0 0 0}
\pscustom[linewidth=0.99999995,linecolor=curcolor]
{
\newpath
\moveto(251.9999811,94.38285354)
\lineto(251.9999811,66.38286614)
}
}
{
\newrgbcolor{curcolor}{0 0 0}
\pscustom[linestyle=none,fillstyle=solid,fillcolor=curcolor]
{
\newpath
\moveto(251.9999811,93.40285359)
\curveto(251.44798113,93.40285359)(250.99998115,93.85085357)(250.99998115,94.40285354)
\curveto(250.99998115,94.95485351)(251.44798113,95.40285349)(251.9999811,95.40285349)
\curveto(252.55198107,95.40285349)(252.99998105,94.95485351)(252.99998105,94.40285354)
\curveto(252.99998105,93.85085357)(252.55198107,93.40285359)(251.9999811,93.40285359)
\closepath
}
}
{
\newrgbcolor{curcolor}{0 0 0}
\pscustom[linewidth=0.26666666,linecolor=curcolor]
{
\newpath
\moveto(251.9999811,93.40285359)
\curveto(251.44798113,93.40285359)(250.99998115,93.85085357)(250.99998115,94.40285354)
\curveto(250.99998115,94.95485351)(251.44798113,95.40285349)(251.9999811,95.40285349)
\curveto(252.55198107,95.40285349)(252.99998105,94.95485351)(252.99998105,94.40285354)
\curveto(252.99998105,93.85085357)(252.55198107,93.40285359)(251.9999811,93.40285359)
\closepath
}
}
{
\newrgbcolor{curcolor}{0 0 0}
\pscustom[linestyle=none,fillstyle=solid,fillcolor=curcolor]
{
\newpath
\moveto(251.9999811,70.38286593)
\lineto(249.99998121,72.38286583)
\lineto(251.9999811,65.38286619)
\lineto(253.999981,72.38286583)
\closepath
}
}
{
\newrgbcolor{curcolor}{0 0 0}
\pscustom[linewidth=0.53333332,linecolor=curcolor]
{
\newpath
\moveto(251.9999811,70.38286593)
\lineto(249.99998121,72.38286583)
\lineto(251.9999811,65.38286619)
\lineto(253.999981,72.38286583)
\closepath
}
}
{
\newrgbcolor{curcolor}{0 0 0}
\pscustom[linewidth=0.99999995,linecolor=curcolor]
{
\newpath
\moveto(231.99998362,94.38285354)
\lineto(251.9999811,94.38285354)
}
}
{
\newrgbcolor{curcolor}{0.65490198 0.66274512 0.67450982}
\pscustom[linestyle=none,fillstyle=solid,fillcolor=curcolor]
{
\newpath
\moveto(107.08003039,96.54294332)
\curveto(107.08003039,96.54294332)(107.08003039,96.58294332)(107.04003039,96.68694331)
\lineto(104.32003053,104.20694291)
\curveto(104.28803054,104.29494291)(104.25603054,104.38294291)(104.14403054,104.38294291)
\curveto(104.05603055,104.38294291)(103.98403055,104.31094291)(103.98403055,104.22294291)
\curveto(103.98403055,104.22294291)(103.98403055,104.18294292)(104.02403055,104.07894292)
\lineto(106.74403041,96.55894332)
\curveto(106.77603041,96.47094332)(106.8080304,96.38294333)(106.9200304,96.38294333)
\curveto(107.00803039,96.38294333)(107.08003039,96.45494332)(107.08003039,96.54294332)
\closepath
}
}
{
\newrgbcolor{curcolor}{0.65490198 0.66274512 0.67450982}
\pscustom[linestyle=none,fillstyle=solid,fillcolor=curcolor]
{
\newpath
\moveto(110.4160149,99.40694317)
\curveto(110.4160149,99.83094314)(110.17601491,100.07094313)(110.08001491,100.16694313)
\curveto(109.81601493,100.42294311)(109.50401494,100.48694311)(109.16801496,100.55094311)
\curveto(108.72001498,100.6389431)(108.18401501,100.7429431)(108.18401501,101.20694307)
\curveto(108.18401501,101.48694306)(108.392015,101.81494304)(109.08001497,101.81494304)
\curveto(109.96001492,101.81494304)(110.00001492,101.09494308)(110.01601492,100.84694309)
\curveto(110.02401492,100.7749431)(110.11201491,100.7749431)(110.11201491,100.7749431)
\curveto(110.21601491,100.7749431)(110.21601491,100.81494309)(110.21601491,100.96694309)
\lineto(110.21601491,101.77494304)
\curveto(110.21601491,101.91094304)(110.21601491,101.96694303)(110.12801491,101.96694303)
\curveto(110.08801491,101.96694303)(110.07201491,101.96694303)(109.96801492,101.87094304)
\curveto(109.94401492,101.83894304)(109.86401492,101.76694304)(109.83201493,101.74294304)
\curveto(109.52801494,101.96694303)(109.20001496,101.96694303)(109.08001497,101.96694303)
\curveto(108.10401502,101.96694303)(107.80001503,101.43094306)(107.80001503,100.98294308)
\curveto(107.80001503,100.7029431)(107.92801503,100.47894311)(108.14401501,100.30294312)
\curveto(108.400015,100.09494313)(108.62401499,100.04694313)(109.20001496,99.93494314)
\curveto(109.37601495,99.90294314)(110.03201492,99.77494315)(110.03201492,99.19894318)
\curveto(110.03201492,98.7909432)(109.75201493,98.47094322)(109.12801496,98.47094322)
\curveto(108.456015,98.47094322)(108.16801501,98.92694319)(108.01601502,99.60694316)
\curveto(107.99201502,99.71094315)(107.98401502,99.74294315)(107.90401503,99.74294315)
\curveto(107.80001503,99.74294315)(107.80001503,99.68694315)(107.80001503,99.54294316)
\lineto(107.80001503,98.48694322)
\curveto(107.80001503,98.35094322)(107.80001503,98.29494323)(107.88801503,98.29494323)
\curveto(107.92801503,98.29494323)(107.93601503,98.30294323)(108.08801502,98.45494322)
\curveto(108.10401502,98.47094322)(108.10401502,98.48694322)(108.24801501,98.63894321)
\curveto(108.60001499,98.30294323)(108.96001497,98.29494323)(109.12801496,98.29494323)
\curveto(110.04801491,98.29494323)(110.4160149,98.8309432)(110.4160149,99.40694317)
\closepath
}
}
{
\newrgbcolor{curcolor}{0.65490198 0.66274512 0.67450982}
\pscustom[linestyle=none,fillstyle=solid,fillcolor=curcolor]
{
\newpath
\moveto(117.19200729,98.38294322)
\lineto(117.19200729,98.63094321)
\curveto(116.77600731,98.63094321)(116.57600732,98.63094321)(116.56800733,98.8709432)
\lineto(116.56800733,100.39894311)
\curveto(116.56800733,101.08694308)(116.56800733,101.33494307)(116.32000734,101.62294305)
\curveto(116.20800734,101.75894304)(115.94400736,101.91894304)(115.48000738,101.91894304)
\curveto(114.80800742,101.91894304)(114.45600744,101.43894306)(114.32000744,101.13494308)
\curveto(114.20800745,101.83094304)(113.61600748,101.91894304)(113.2560075,101.91894304)
\curveto(112.67200753,101.91894304)(112.29600755,101.57494305)(112.07200756,101.07894308)
\lineto(112.07200756,101.91894304)
\lineto(110.94400762,101.83094304)
\lineto(110.94400762,101.58294305)
\curveto(111.50400759,101.58294305)(111.56800759,101.52694306)(111.56800759,101.13494308)
\lineto(111.56800759,98.99094319)
\curveto(111.56800759,98.63094321)(111.48000759,98.63094321)(110.94400762,98.63094321)
\lineto(110.94400762,98.38294322)
\lineto(111.84800757,98.40694322)
\lineto(112.74400753,98.38294322)
\lineto(112.74400753,98.63094321)
\curveto(112.20800755,98.63094321)(112.12000756,98.63094321)(112.12000756,98.99094319)
\lineto(112.12000756,100.46294311)
\curveto(112.12000756,101.29494307)(112.68800753,101.74294304)(113.2000075,101.74294304)
\curveto(113.70400748,101.74294304)(113.79200747,101.31094307)(113.79200747,100.85494309)
\lineto(113.79200747,98.99094319)
\curveto(113.79200747,98.63094321)(113.70400748,98.63094321)(113.1680075,98.63094321)
\lineto(113.1680075,98.38294322)
\lineto(114.07200746,98.40694322)
\lineto(114.96800741,98.38294322)
\lineto(114.96800741,98.63094321)
\curveto(114.43200744,98.63094321)(114.34400744,98.63094321)(114.34400744,98.99094319)
\lineto(114.34400744,100.46294311)
\curveto(114.34400744,101.29494307)(114.91200741,101.74294304)(115.42400739,101.74294304)
\curveto(115.92800736,101.74294304)(116.01600735,101.31094307)(116.01600735,100.85494309)
\lineto(116.01600735,98.99094319)
\curveto(116.01600735,98.63094321)(115.92800736,98.63094321)(115.39200739,98.63094321)
\lineto(115.39200739,98.38294322)
\lineto(116.29600734,98.40694322)
\closepath
}
}
{
\newrgbcolor{curcolor}{0.65490198 0.66274512 0.67450982}
\pscustom[linestyle=none,fillstyle=solid,fillcolor=curcolor]
{
\newpath
\moveto(121.2159931,99.09494318)
\lineto(121.2159931,99.54294316)
\lineto(121.01599311,99.54294316)
\lineto(121.01599311,99.09494318)
\curveto(121.01599311,98.63094321)(120.81599312,98.58294321)(120.72799312,98.58294321)
\curveto(120.46399314,98.58294321)(120.43199314,98.94294319)(120.43199314,98.98294319)
\lineto(120.43199314,100.58294311)
\curveto(120.43199314,100.91894309)(120.43199314,101.23094307)(120.14399315,101.52694306)
\curveto(119.83199317,101.83894304)(119.43199319,101.96694303)(119.04799321,101.96694303)
\curveto(118.39199325,101.96694303)(117.83999327,101.59094305)(117.83999327,101.06294308)
\curveto(117.83999327,100.82294309)(117.99999327,100.6869431)(118.20799325,100.6869431)
\curveto(118.43199324,100.6869431)(118.57599324,100.84694309)(118.57599324,101.05494308)
\curveto(118.57599324,101.15094308)(118.53599324,101.41494306)(118.16799326,101.42294306)
\curveto(118.38399325,101.70294305)(118.77599322,101.79094304)(119.03199321,101.79094304)
\curveto(119.42399319,101.79094304)(119.87999317,101.47894306)(119.87999317,100.7669431)
\lineto(119.87999317,100.47094311)
\curveto(119.47199319,100.44694311)(118.91199322,100.42294311)(118.40799324,100.18294313)
\curveto(117.80799328,99.91094314)(117.60799329,99.49494316)(117.60799329,99.14294318)
\curveto(117.60799329,98.49494322)(118.38399325,98.29494323)(118.88799322,98.29494323)
\curveto(119.41599319,98.29494323)(119.78399317,98.61494321)(119.93599316,98.99094319)
\curveto(119.96799316,98.67094321)(120.18399315,98.33494322)(120.55999313,98.33494322)
\curveto(120.72799312,98.33494322)(121.2159931,98.44694322)(121.2159931,99.09494318)
\closepath
\moveto(119.87999317,99.50294316)
\curveto(119.87999317,98.7429432)(119.3039932,98.47094322)(118.94399322,98.47094322)
\curveto(118.55199324,98.47094322)(118.22399325,98.7509432)(118.22399325,99.15094318)
\curveto(118.22399325,99.59094316)(118.55999324,100.25494312)(119.87999317,100.30294312)
\closepath
}
}
{
\newrgbcolor{curcolor}{0.65490198 0.66274512 0.67450982}
\pscustom[linestyle=none,fillstyle=solid,fillcolor=curcolor]
{
\newpath
\moveto(123.39197811,98.38294322)
\lineto(123.39197811,98.63094321)
\curveto(122.85597814,98.63094321)(122.76797815,98.63094321)(122.76797815,98.99094319)
\lineto(122.76797815,103.93494293)
\lineto(121.61597821,103.84694293)
\lineto(121.61597821,103.59894295)
\curveto(122.17597818,103.59894295)(122.23997817,103.54294295)(122.23997817,103.15094297)
\lineto(122.23997817,98.99094319)
\curveto(122.23997817,98.63094321)(122.15197818,98.63094321)(121.61597821,98.63094321)
\lineto(121.61597821,98.38294322)
\lineto(122.50397816,98.40694322)
\closepath
}
}
{
\newrgbcolor{curcolor}{0.65490198 0.66274512 0.67450982}
\pscustom[linestyle=none,fillstyle=solid,fillcolor=curcolor]
{
\newpath
\moveto(125.61597606,98.38294322)
\lineto(125.61597606,98.63094321)
\curveto(125.07997608,98.63094321)(124.99197609,98.63094321)(124.99197609,98.99094319)
\lineto(124.99197609,103.93494293)
\lineto(123.83997615,103.84694293)
\lineto(123.83997615,103.59894295)
\curveto(124.39997612,103.59894295)(124.46397612,103.54294295)(124.46397612,103.15094297)
\lineto(124.46397612,98.99094319)
\curveto(124.46397612,98.63094321)(124.37597612,98.63094321)(123.83997615,98.63094321)
\lineto(123.83997615,98.38294322)
\lineto(124.7279761,98.40694322)
\closepath
}
}
{
\newrgbcolor{curcolor}{0.65490198 0.66274512 0.67450982}
\pscustom[linestyle=none,fillstyle=solid,fillcolor=curcolor]
{
\newpath
\moveto(132.00796205,99.94294314)
\curveto(132.00796205,100.6069431)(131.64796207,101.01494308)(131.57596207,101.08694308)
\curveto(131.25596209,101.43894306)(130.9679621,101.51094306)(130.58396212,101.60694305)
\lineto(130.58396212,103.76694294)
\curveto(131.26396209,103.73494294)(131.64796207,103.38294296)(131.76796206,102.93494298)
\curveto(131.73596206,102.94294298)(131.71996206,102.95094298)(131.63996207,102.95094298)
\curveto(131.43196208,102.95094298)(131.27196209,102.80694299)(131.27196209,102.582943)
\curveto(131.27196209,102.33494301)(131.47196208,102.21494302)(131.63996207,102.21494302)
\curveto(131.66396207,102.21494302)(132.00796205,102.22294302)(132.00796205,102.614943)
\curveto(132.00796205,103.39094296)(131.47996208,103.97494293)(130.58396212,104.02294292)
\lineto(130.58396212,104.38294291)
\lineto(130.33596214,104.38294291)
\lineto(130.33596214,104.01494292)
\curveto(129.44796218,103.93494293)(128.91196221,103.21494297)(128.91196221,102.48694301)
\curveto(128.91196221,101.93494303)(129.20796219,101.55094305)(129.35196219,101.41494306)
\curveto(129.65596217,101.10294308)(129.92796216,101.03894308)(130.33596214,100.93494309)
\lineto(130.33596214,98.55094321)
\curveto(129.63196217,98.59094321)(129.25596219,98.99894319)(129.1519622,99.49494316)
\curveto(129.1839622,99.48694316)(129.24796219,99.47894316)(129.27996219,99.47894316)
\curveto(129.49596218,99.47894316)(129.64796217,99.63094316)(129.64796217,99.84694314)
\curveto(129.64796217,100.07094313)(129.47996218,100.21494312)(129.27996219,100.21494312)
\curveto(129.23196219,100.21494312)(128.91196221,100.19894313)(128.91196221,99.80694315)
\curveto(128.91196221,99.09494318)(129.33596219,98.35894322)(130.33596214,98.29494323)
\lineto(130.33596214,97.93494324)
\lineto(130.58396212,97.93494324)
\lineto(130.58396212,98.30294323)
\curveto(131.42396208,98.37494322)(132.00796205,99.11894318)(132.00796205,99.94294314)
\closepath
\moveto(131.59996207,99.70294315)
\curveto(131.59996207,99.16694318)(131.21596209,98.63094321)(130.58396212,98.55094321)
\lineto(130.58396212,100.87094309)
\curveto(131.54396207,100.6709431)(131.59996207,99.89494314)(131.59996207,99.70294315)
\closepath
\moveto(130.33596214,101.67094305)
\curveto(129.35196219,101.87894304)(129.31996219,102.566943)(129.31996219,102.72694299)
\curveto(129.31996219,103.20694297)(129.69596217,103.70294294)(130.33596214,103.76694294)
\closepath
}
}
{
\newrgbcolor{curcolor}{0.65490198 0.66274512 0.67450982}
\pscustom[linestyle=none,fillstyle=solid,fillcolor=curcolor]
{
\newpath
\moveto(136.11994606,99.72694315)
\curveto(136.11994606,100.01494314)(136.03194607,100.37494312)(135.72794608,100.7109431)
\curveto(135.57594609,100.87894309)(135.4479461,100.95894309)(134.93594612,101.27894307)
\curveto(135.51194609,101.57494305)(135.90394607,101.99094303)(135.90394607,102.518943)
\curveto(135.90394607,103.25494296)(135.19194611,103.71094294)(134.46394615,103.71094294)
\curveto(133.66394619,103.71094294)(133.01594623,103.11894297)(133.01594623,102.37494301)
\curveto(133.01594623,102.23094302)(133.03194622,101.87094304)(133.36794621,101.49494306)
\curveto(133.4559462,101.39894306)(133.75194619,101.19894307)(133.95194618,101.06294308)
\curveto(133.4879462,100.83094309)(132.79994624,100.38294312)(132.79994624,99.59094316)
\curveto(132.79994624,98.7429432)(133.61594619,98.20694323)(134.45594615,98.20694323)
\curveto(135.3599461,98.20694323)(136.11994606,98.8709432)(136.11994606,99.72694315)
\closepath
\moveto(135.55194609,102.518943)
\curveto(135.55194609,102.06294303)(135.23994611,101.67894305)(134.75994613,101.39894306)
\lineto(133.76794619,102.03894303)
\curveto(133.39994621,102.27894302)(133.36794621,102.550943)(133.36794621,102.68694299)
\curveto(133.36794621,103.17494297)(133.88794618,103.51094295)(134.45594615,103.51094295)
\curveto(135.03994612,103.51094295)(135.55194609,103.09494297)(135.55194609,102.518943)
\closepath
\moveto(135.71994608,99.43894317)
\curveto(135.71994608,98.8469432)(135.11994611,98.43094322)(134.46394615,98.43094322)
\curveto(133.77594619,98.43094322)(133.19994622,98.92694319)(133.19994622,99.59094316)
\curveto(133.19994622,100.05494313)(133.4559462,100.56694311)(134.13594617,100.94294309)
\lineto(135.11994611,100.31894312)
\curveto(135.3439461,100.16694313)(135.71994608,99.92694314)(135.71994608,99.43894317)
\closepath
}
}
{
\newrgbcolor{curcolor}{0.65490198 0.66274512 0.67450982}
\pscustom[linestyle=none,fillstyle=solid,fillcolor=curcolor]
{
\newpath
\moveto(140.00793009,99.94294314)
\curveto(140.00793009,100.6069431)(139.64793011,101.01494308)(139.57593011,101.08694308)
\curveto(139.25593013,101.43894306)(138.96793014,101.51094306)(138.58393016,101.60694305)
\lineto(138.58393016,103.76694294)
\curveto(139.26393013,103.73494294)(139.64793011,103.38294296)(139.7679301,102.93494298)
\curveto(139.7359301,102.94294298)(139.7199301,102.95094298)(139.63993011,102.95094298)
\curveto(139.43193012,102.95094298)(139.27193013,102.80694299)(139.27193013,102.582943)
\curveto(139.27193013,102.33494301)(139.47193012,102.21494302)(139.63993011,102.21494302)
\curveto(139.66393011,102.21494302)(140.00793009,102.22294302)(140.00793009,102.614943)
\curveto(140.00793009,103.39094296)(139.47993012,103.97494293)(138.58393016,104.02294292)
\lineto(138.58393016,104.38294291)
\lineto(138.33593018,104.38294291)
\lineto(138.33593018,104.01494292)
\curveto(137.44793022,103.93494293)(136.91193025,103.21494297)(136.91193025,102.48694301)
\curveto(136.91193025,101.93494303)(137.20793024,101.55094305)(137.35193023,101.41494306)
\curveto(137.65593021,101.10294308)(137.9279302,101.03894308)(138.33593018,100.93494309)
\lineto(138.33593018,98.55094321)
\curveto(137.63193021,98.59094321)(137.25593023,98.99894319)(137.15193024,99.49494316)
\curveto(137.18393024,99.48694316)(137.24793023,99.47894316)(137.27993023,99.47894316)
\curveto(137.49593022,99.47894316)(137.64793021,99.63094316)(137.64793021,99.84694314)
\curveto(137.64793021,100.07094313)(137.47993022,100.21494312)(137.27993023,100.21494312)
\curveto(137.23193023,100.21494312)(136.91193025,100.19894313)(136.91193025,99.80694315)
\curveto(136.91193025,99.09494318)(137.33593023,98.35894322)(138.33593018,98.29494323)
\lineto(138.33593018,97.93494324)
\lineto(138.58393016,97.93494324)
\lineto(138.58393016,98.30294323)
\curveto(139.42393012,98.37494322)(140.00793009,99.11894318)(140.00793009,99.94294314)
\closepath
\moveto(139.59993011,99.70294315)
\curveto(139.59993011,99.16694318)(139.21593013,98.63094321)(138.58393016,98.55094321)
\lineto(138.58393016,100.87094309)
\curveto(139.54393011,100.6709431)(139.59993011,99.89494314)(139.59993011,99.70294315)
\closepath
\moveto(138.33593018,101.67094305)
\curveto(137.35193023,101.87894304)(137.31993023,102.566943)(137.31993023,102.72694299)
\curveto(137.31993023,103.20694297)(137.69593021,103.70294294)(138.33593018,103.76694294)
\closepath
}
}
{
\newrgbcolor{curcolor}{0.65490198 0.66274512 0.67450982}
\pscustom[linestyle=none,fillstyle=solid,fillcolor=curcolor]
{
\newpath
\moveto(177.08002558,96.54294332)
\curveto(177.08002558,96.54294332)(177.08002558,96.58294332)(177.04002559,96.68694331)
\lineto(174.32002573,104.20694291)
\curveto(174.28802573,104.29494291)(174.25602573,104.38294291)(174.14402574,104.38294291)
\curveto(174.05602574,104.38294291)(173.98402575,104.31094291)(173.98402575,104.22294291)
\curveto(173.98402575,104.22294291)(173.98402575,104.18294292)(174.02402574,104.07894292)
\lineto(176.7440256,96.55894332)
\curveto(176.7760256,96.47094332)(176.8080256,96.38294333)(176.92002559,96.38294333)
\curveto(177.00802559,96.38294333)(177.08002558,96.45494332)(177.08002558,96.54294332)
\closepath
}
}
{
\newrgbcolor{curcolor}{0.65490198 0.66274512 0.67450982}
\pscustom[linestyle=none,fillstyle=solid,fillcolor=curcolor]
{
\newpath
\moveto(180.41601009,99.40694317)
\curveto(180.41601009,99.83094314)(180.1760101,100.07094313)(180.08001011,100.16694313)
\curveto(179.81601012,100.42294311)(179.50401014,100.48694311)(179.16801016,100.55094311)
\curveto(178.72001018,100.6389431)(178.18401021,100.7429431)(178.18401021,101.20694307)
\curveto(178.18401021,101.48694306)(178.3920102,101.81494304)(179.08001016,101.81494304)
\curveto(179.96001011,101.81494304)(180.00001011,101.09494308)(180.01601011,100.84694309)
\curveto(180.02401011,100.7749431)(180.11201011,100.7749431)(180.11201011,100.7749431)
\curveto(180.2160101,100.7749431)(180.2160101,100.81494309)(180.2160101,100.96694309)
\lineto(180.2160101,101.77494304)
\curveto(180.2160101,101.91094304)(180.2160101,101.96694303)(180.1280101,101.96694303)
\curveto(180.08801011,101.96694303)(180.07201011,101.96694303)(179.96801011,101.87094304)
\curveto(179.94401011,101.83894304)(179.86401012,101.76694304)(179.83201012,101.74294304)
\curveto(179.52801014,101.96694303)(179.20001015,101.96694303)(179.08001016,101.96694303)
\curveto(178.10401021,101.96694303)(177.80001023,101.43094306)(177.80001023,100.98294308)
\curveto(177.80001023,100.7029431)(177.92801022,100.47894311)(178.14401021,100.30294312)
\curveto(178.4000102,100.09494313)(178.62401018,100.04694313)(179.20001015,99.93494314)
\curveto(179.37601014,99.90294314)(180.03201011,99.77494315)(180.03201011,99.19894318)
\curveto(180.03201011,98.7909432)(179.75201012,98.47094322)(179.12801016,98.47094322)
\curveto(178.45601019,98.47094322)(178.16801021,98.92694319)(178.01601022,99.60694316)
\curveto(177.99201022,99.71094315)(177.98401022,99.74294315)(177.90401022,99.74294315)
\curveto(177.80001023,99.74294315)(177.80001023,99.68694315)(177.80001023,99.54294316)
\lineto(177.80001023,98.48694322)
\curveto(177.80001023,98.35094322)(177.80001023,98.29494323)(177.88801022,98.29494323)
\curveto(177.92801022,98.29494323)(177.93601022,98.30294323)(178.08801021,98.45494322)
\curveto(178.10401021,98.47094322)(178.10401021,98.48694322)(178.2480102,98.63894321)
\curveto(178.60001018,98.30294323)(178.96001017,98.29494323)(179.12801016,98.29494323)
\curveto(180.04801011,98.29494323)(180.41601009,98.8309432)(180.41601009,99.40694317)
\closepath
}
}
{
\newrgbcolor{curcolor}{0.65490198 0.66274512 0.67450982}
\pscustom[linestyle=none,fillstyle=solid,fillcolor=curcolor]
{
\newpath
\moveto(187.19200249,98.38294322)
\lineto(187.19200249,98.63094321)
\curveto(186.77600251,98.63094321)(186.57600252,98.63094321)(186.56800252,98.8709432)
\lineto(186.56800252,100.39894311)
\curveto(186.56800252,101.08694308)(186.56800252,101.33494307)(186.32000253,101.62294305)
\curveto(186.20800254,101.75894304)(185.94400255,101.91894304)(185.48000258,101.91894304)
\curveto(184.80800261,101.91894304)(184.45600263,101.43894306)(184.32000264,101.13494308)
\curveto(184.20800264,101.83094304)(183.61600267,101.91894304)(183.25600269,101.91894304)
\curveto(182.67200272,101.91894304)(182.29600274,101.57494305)(182.07200276,101.07894308)
\lineto(182.07200276,101.91894304)
\lineto(180.94400282,101.83094304)
\lineto(180.94400282,101.58294305)
\curveto(181.50400279,101.58294305)(181.56800278,101.52694306)(181.56800278,101.13494308)
\lineto(181.56800278,98.99094319)
\curveto(181.56800278,98.63094321)(181.48000279,98.63094321)(180.94400282,98.63094321)
\lineto(180.94400282,98.38294322)
\lineto(181.84800277,98.40694322)
\lineto(182.74400272,98.38294322)
\lineto(182.74400272,98.63094321)
\curveto(182.20800275,98.63094321)(182.12000275,98.63094321)(182.12000275,98.99094319)
\lineto(182.12000275,100.46294311)
\curveto(182.12000275,101.29494307)(182.68800272,101.74294304)(183.2000027,101.74294304)
\curveto(183.70400267,101.74294304)(183.79200267,101.31094307)(183.79200267,100.85494309)
\lineto(183.79200267,98.99094319)
\curveto(183.79200267,98.63094321)(183.70400267,98.63094321)(183.1680027,98.63094321)
\lineto(183.1680027,98.38294322)
\lineto(184.07200265,98.40694322)
\lineto(184.9680026,98.38294322)
\lineto(184.9680026,98.63094321)
\curveto(184.43200263,98.63094321)(184.34400264,98.63094321)(184.34400264,98.99094319)
\lineto(184.34400264,100.46294311)
\curveto(184.34400264,101.29494307)(184.91200261,101.74294304)(185.42400258,101.74294304)
\curveto(185.92800255,101.74294304)(186.01600255,101.31094307)(186.01600255,100.85494309)
\lineto(186.01600255,98.99094319)
\curveto(186.01600255,98.63094321)(185.92800255,98.63094321)(185.39200258,98.63094321)
\lineto(185.39200258,98.38294322)
\lineto(186.29600253,98.40694322)
\closepath
}
}
{
\newrgbcolor{curcolor}{0.65490198 0.66274512 0.67450982}
\pscustom[linestyle=none,fillstyle=solid,fillcolor=curcolor]
{
\newpath
\moveto(191.21598829,99.09494318)
\lineto(191.21598829,99.54294316)
\lineto(191.0159883,99.54294316)
\lineto(191.0159883,99.09494318)
\curveto(191.0159883,98.63094321)(190.81598831,98.58294321)(190.72798832,98.58294321)
\curveto(190.46398833,98.58294321)(190.43198833,98.94294319)(190.43198833,98.98294319)
\lineto(190.43198833,100.58294311)
\curveto(190.43198833,100.91894309)(190.43198833,101.23094307)(190.14398835,101.52694306)
\curveto(189.83198836,101.83894304)(189.43198838,101.96694303)(189.0479884,101.96694303)
\curveto(188.39198844,101.96694303)(187.83998847,101.59094305)(187.83998847,101.06294308)
\curveto(187.83998847,100.82294309)(187.99998846,100.6869431)(188.20798845,100.6869431)
\curveto(188.43198844,100.6869431)(188.57598843,100.84694309)(188.57598843,101.05494308)
\curveto(188.57598843,101.15094308)(188.53598843,101.41494306)(188.16798845,101.42294306)
\curveto(188.38398844,101.70294305)(188.77598842,101.79094304)(189.03198841,101.79094304)
\curveto(189.42398838,101.79094304)(189.87998836,101.47894306)(189.87998836,100.7669431)
\lineto(189.87998836,100.47094311)
\curveto(189.47198838,100.44694311)(188.91198841,100.42294311)(188.40798844,100.18294313)
\curveto(187.80798847,99.91094314)(187.60798848,99.49494316)(187.60798848,99.14294318)
\curveto(187.60798848,98.49494322)(188.38398844,98.29494323)(188.88798841,98.29494323)
\curveto(189.41598839,98.29494323)(189.78398837,98.61494321)(189.93598836,98.99094319)
\curveto(189.96798836,98.67094321)(190.18398835,98.33494322)(190.55998833,98.33494322)
\curveto(190.72798832,98.33494322)(191.21598829,98.44694322)(191.21598829,99.09494318)
\closepath
\moveto(189.87998836,99.50294316)
\curveto(189.87998836,98.7429432)(189.30398839,98.47094322)(188.94398841,98.47094322)
\curveto(188.55198843,98.47094322)(188.22398845,98.7509432)(188.22398845,99.15094318)
\curveto(188.22398845,99.59094316)(188.55998843,100.25494312)(189.87998836,100.30294312)
\closepath
}
}
{
\newrgbcolor{curcolor}{0.65490198 0.66274512 0.67450982}
\pscustom[linestyle=none,fillstyle=solid,fillcolor=curcolor]
{
\newpath
\moveto(193.39197331,98.38294322)
\lineto(193.39197331,98.63094321)
\curveto(192.85597334,98.63094321)(192.76797334,98.63094321)(192.76797334,98.99094319)
\lineto(192.76797334,103.93494293)
\lineto(191.6159734,103.84694293)
\lineto(191.6159734,103.59894295)
\curveto(192.17597337,103.59894295)(192.23997337,103.54294295)(192.23997337,103.15094297)
\lineto(192.23997337,98.99094319)
\curveto(192.23997337,98.63094321)(192.15197337,98.63094321)(191.6159734,98.63094321)
\lineto(191.6159734,98.38294322)
\lineto(192.50397335,98.40694322)
\closepath
}
}
{
\newrgbcolor{curcolor}{0.65490198 0.66274512 0.67450982}
\pscustom[linestyle=none,fillstyle=solid,fillcolor=curcolor]
{
\newpath
\moveto(195.61597125,98.38294322)
\lineto(195.61597125,98.63094321)
\curveto(195.07997128,98.63094321)(194.99197128,98.63094321)(194.99197128,98.99094319)
\lineto(194.99197128,103.93494293)
\lineto(193.83997134,103.84694293)
\lineto(193.83997134,103.59894295)
\curveto(194.39997131,103.59894295)(194.46397131,103.54294295)(194.46397131,103.15094297)
\lineto(194.46397131,98.99094319)
\curveto(194.46397131,98.63094321)(194.37597132,98.63094321)(193.83997134,98.63094321)
\lineto(193.83997134,98.38294322)
\lineto(194.7279713,98.40694322)
\closepath
}
}
{
\newrgbcolor{curcolor}{0.65490198 0.66274512 0.67450982}
\pscustom[linestyle=none,fillstyle=solid,fillcolor=curcolor]
{
\newpath
\moveto(202.00795724,99.94294314)
\curveto(202.00795724,100.6069431)(201.64795726,101.01494308)(201.57595726,101.08694308)
\curveto(201.25595728,101.43894306)(200.9679573,101.51094306)(200.58395732,101.60694305)
\lineto(200.58395732,103.76694294)
\curveto(201.26395728,103.73494294)(201.64795726,103.38294296)(201.76795725,102.93494298)
\curveto(201.73595726,102.94294298)(201.71995726,102.95094298)(201.63995726,102.95094298)
\curveto(201.43195727,102.95094298)(201.27195728,102.80694299)(201.27195728,102.582943)
\curveto(201.27195728,102.33494301)(201.47195727,102.21494302)(201.63995726,102.21494302)
\curveto(201.66395726,102.21494302)(202.00795724,102.22294302)(202.00795724,102.614943)
\curveto(202.00795724,103.39094296)(201.47995727,103.97494293)(200.58395732,104.02294292)
\lineto(200.58395732,104.38294291)
\lineto(200.33595733,104.38294291)
\lineto(200.33595733,104.01494292)
\curveto(199.44795738,103.93494293)(198.9119574,103.21494297)(198.9119574,102.48694301)
\curveto(198.9119574,101.93494303)(199.20795739,101.55094305)(199.35195738,101.41494306)
\curveto(199.65595737,101.10294308)(199.92795735,101.03894308)(200.33595733,100.93494309)
\lineto(200.33595733,98.55094321)
\curveto(199.63195737,98.59094321)(199.25595739,98.99894319)(199.15195739,99.49494316)
\curveto(199.18395739,99.48694316)(199.24795739,99.47894316)(199.27995739,99.47894316)
\curveto(199.49595737,99.47894316)(199.64795737,99.63094316)(199.64795737,99.84694314)
\curveto(199.64795737,100.07094313)(199.47995737,100.21494312)(199.27995739,100.21494312)
\curveto(199.23195739,100.21494312)(198.9119574,100.19894313)(198.9119574,99.80694315)
\curveto(198.9119574,99.09494318)(199.33595738,98.35894322)(200.33595733,98.29494323)
\lineto(200.33595733,97.93494324)
\lineto(200.58395732,97.93494324)
\lineto(200.58395732,98.30294323)
\curveto(201.42395727,98.37494322)(202.00795724,99.11894318)(202.00795724,99.94294314)
\closepath
\moveto(201.59995726,99.70294315)
\curveto(201.59995726,99.16694318)(201.21595728,98.63094321)(200.58395732,98.55094321)
\lineto(200.58395732,100.87094309)
\curveto(201.54395727,100.6709431)(201.59995726,99.89494314)(201.59995726,99.70294315)
\closepath
\moveto(200.33595733,101.67094305)
\curveto(199.35195738,101.87894304)(199.31995738,102.566943)(199.31995738,102.72694299)
\curveto(199.31995738,103.20694297)(199.69595736,103.70294294)(200.33595733,103.76694294)
\closepath
}
}
{
\newrgbcolor{curcolor}{0.65490198 0.66274512 0.67450982}
\pscustom[linestyle=none,fillstyle=solid,fillcolor=curcolor]
{
\newpath
\moveto(206.11994126,99.72694315)
\curveto(206.11994126,100.01494314)(206.03194126,100.37494312)(205.72794128,100.7109431)
\curveto(205.57594128,100.87894309)(205.44794129,100.95894309)(204.93594132,101.27894307)
\curveto(205.51194129,101.57494305)(205.90394127,101.99094303)(205.90394127,102.518943)
\curveto(205.90394127,103.25494296)(205.1919413,103.71094294)(204.46394134,103.71094294)
\curveto(203.66394139,103.71094294)(203.01594142,103.11894297)(203.01594142,102.37494301)
\curveto(203.01594142,102.23094302)(203.03194142,101.87094304)(203.3679414,101.49494306)
\curveto(203.4559414,101.39894306)(203.75194138,101.19894307)(203.95194137,101.06294308)
\curveto(203.48794139,100.83094309)(202.79994143,100.38294312)(202.79994143,99.59094316)
\curveto(202.79994143,98.7429432)(203.61594139,98.20694323)(204.45594134,98.20694323)
\curveto(205.3599413,98.20694323)(206.11994126,98.8709432)(206.11994126,99.72694315)
\closepath
\moveto(205.55194129,102.518943)
\curveto(205.55194129,102.06294303)(205.2399413,101.67894305)(204.75994133,101.39894306)
\lineto(203.76794138,102.03894303)
\curveto(203.3999414,102.27894302)(203.3679414,102.550943)(203.3679414,102.68694299)
\curveto(203.3679414,103.17494297)(203.88794137,103.51094295)(204.45594134,103.51094295)
\curveto(205.03994131,103.51094295)(205.55194129,103.09494297)(205.55194129,102.518943)
\closepath
\moveto(205.71994128,99.43894317)
\curveto(205.71994128,98.8469432)(205.11994131,98.43094322)(204.46394134,98.43094322)
\curveto(203.77594138,98.43094322)(203.19994141,98.92694319)(203.19994141,99.59094316)
\curveto(203.19994141,100.05494313)(203.4559414,100.56694311)(204.13594136,100.94294309)
\lineto(205.11994131,100.31894312)
\curveto(205.3439413,100.16694313)(205.71994128,99.92694314)(205.71994128,99.43894317)
\closepath
}
}
{
\newrgbcolor{curcolor}{0.65490198 0.66274512 0.67450982}
\pscustom[linestyle=none,fillstyle=solid,fillcolor=curcolor]
{
\newpath
\moveto(210.00792528,99.94294314)
\curveto(210.00792528,100.6069431)(209.6479253,101.01494308)(209.57592531,101.08694308)
\curveto(209.25592532,101.43894306)(208.96792534,101.51094306)(208.58392536,101.60694305)
\lineto(208.58392536,103.76694294)
\curveto(209.26392532,103.73494294)(209.6479253,103.38294296)(209.76792529,102.93494298)
\curveto(209.7359253,102.94294298)(209.7199253,102.95094298)(209.6399253,102.95094298)
\curveto(209.43192531,102.95094298)(209.27192532,102.80694299)(209.27192532,102.582943)
\curveto(209.27192532,102.33494301)(209.47192531,102.21494302)(209.6399253,102.21494302)
\curveto(209.6639253,102.21494302)(210.00792528,102.22294302)(210.00792528,102.614943)
\curveto(210.00792528,103.39094296)(209.47992531,103.97494293)(208.58392536,104.02294292)
\lineto(208.58392536,104.38294291)
\lineto(208.33592537,104.38294291)
\lineto(208.33592537,104.01494292)
\curveto(207.44792542,103.93494293)(206.91192545,103.21494297)(206.91192545,102.48694301)
\curveto(206.91192545,101.93494303)(207.20792543,101.55094305)(207.35192542,101.41494306)
\curveto(207.65592541,101.10294308)(207.92792539,101.03894308)(208.33592537,100.93494309)
\lineto(208.33592537,98.55094321)
\curveto(207.63192541,98.59094321)(207.25592543,98.99894319)(207.15192543,99.49494316)
\curveto(207.18392543,99.48694316)(207.24792543,99.47894316)(207.27992543,99.47894316)
\curveto(207.49592541,99.47894316)(207.64792541,99.63094316)(207.64792541,99.84694314)
\curveto(207.64792541,100.07094313)(207.47992542,100.21494312)(207.27992543,100.21494312)
\curveto(207.23192543,100.21494312)(206.91192545,100.19894313)(206.91192545,99.80694315)
\curveto(206.91192545,99.09494318)(207.33592542,98.35894322)(208.33592537,98.29494323)
\lineto(208.33592537,97.93494324)
\lineto(208.58392536,97.93494324)
\lineto(208.58392536,98.30294323)
\curveto(209.42392531,98.37494322)(210.00792528,99.11894318)(210.00792528,99.94294314)
\closepath
\moveto(209.5999253,99.70294315)
\curveto(209.5999253,99.16694318)(209.21592532,98.63094321)(208.58392536,98.55094321)
\lineto(208.58392536,100.87094309)
\curveto(209.54392531,100.6709431)(209.5999253,99.89494314)(209.5999253,99.70294315)
\closepath
\moveto(208.33592537,101.67094305)
\curveto(207.35192542,101.87894304)(207.31992542,102.566943)(207.31992542,102.72694299)
\curveto(207.31992542,103.20694297)(207.6959254,103.70294294)(208.33592537,103.76694294)
\closepath
}
}
{
\newrgbcolor{curcolor}{0.65490198 0.66274512 0.67450982}
\pscustom[linestyle=none,fillstyle=solid,fillcolor=curcolor]
{
\newpath
\moveto(232.08002078,96.54294332)
\curveto(232.08002078,96.54294332)(232.08002078,96.58294332)(232.04002078,96.68694331)
\lineto(229.32002092,104.20694291)
\curveto(229.28802092,104.29494291)(229.25602093,104.38294291)(229.14402093,104.38294291)
\curveto(229.05602094,104.38294291)(228.98402094,104.31094291)(228.98402094,104.22294291)
\curveto(228.98402094,104.22294291)(228.98402094,104.18294292)(229.02402094,104.07894292)
\lineto(231.7440208,96.55894332)
\curveto(231.77602079,96.47094332)(231.80802079,96.38294333)(231.92002079,96.38294333)
\curveto(232.00802078,96.38294333)(232.08002078,96.45494332)(232.08002078,96.54294332)
\closepath
}
}
{
\newrgbcolor{curcolor}{0.65490198 0.66274512 0.67450982}
\pscustom[linestyle=none,fillstyle=solid,fillcolor=curcolor]
{
\newpath
\moveto(235.41600528,99.40694317)
\curveto(235.41600528,99.83094314)(235.1760053,100.07094313)(235.0800053,100.16694313)
\curveto(234.81600532,100.42294311)(234.50400533,100.48694311)(234.16800535,100.55094311)
\curveto(233.72000537,100.6389431)(233.1840054,100.7429431)(233.1840054,101.20694307)
\curveto(233.1840054,101.48694306)(233.39200539,101.81494304)(234.08000535,101.81494304)
\curveto(234.96000531,101.81494304)(235.00000531,101.09494308)(235.0160053,100.84694309)
\curveto(235.0240053,100.7749431)(235.1120053,100.7749431)(235.1120053,100.7749431)
\curveto(235.21600529,100.7749431)(235.21600529,100.81494309)(235.21600529,100.96694309)
\lineto(235.21600529,101.77494304)
\curveto(235.21600529,101.91094304)(235.21600529,101.96694303)(235.1280053,101.96694303)
\curveto(235.0880053,101.96694303)(235.0720053,101.96694303)(234.96800531,101.87094304)
\curveto(234.94400531,101.83894304)(234.86400531,101.76694304)(234.83200531,101.74294304)
\curveto(234.52800533,101.96694303)(234.20000535,101.96694303)(234.08000535,101.96694303)
\curveto(233.10400541,101.96694303)(232.80000542,101.43094306)(232.80000542,100.98294308)
\curveto(232.80000542,100.7029431)(232.92800541,100.47894311)(233.1440054,100.30294312)
\curveto(233.40000539,100.09494313)(233.62400538,100.04694313)(234.20000535,99.93494314)
\curveto(234.37600534,99.90294314)(235.0320053,99.77494315)(235.0320053,99.19894318)
\curveto(235.0320053,98.7909432)(234.75200532,98.47094322)(234.12800535,98.47094322)
\curveto(233.45600539,98.47094322)(233.1680054,98.92694319)(233.01600541,99.60694316)
\curveto(232.99200541,99.71094315)(232.98400541,99.74294315)(232.90400542,99.74294315)
\curveto(232.80000542,99.74294315)(232.80000542,99.68694315)(232.80000542,99.54294316)
\lineto(232.80000542,98.48694322)
\curveto(232.80000542,98.35094322)(232.80000542,98.29494323)(232.88800542,98.29494323)
\curveto(232.92800541,98.29494323)(232.93600541,98.30294323)(233.08800541,98.45494322)
\curveto(233.10400541,98.47094322)(233.10400541,98.48694322)(233.2480054,98.63894321)
\curveto(233.60000538,98.30294323)(233.96000536,98.29494323)(234.12800535,98.29494323)
\curveto(235.0480053,98.29494323)(235.41600528,98.8309432)(235.41600528,99.40694317)
\closepath
}
}
{
\newrgbcolor{curcolor}{0.65490198 0.66274512 0.67450982}
\pscustom[linestyle=none,fillstyle=solid,fillcolor=curcolor]
{
\newpath
\moveto(242.19199768,98.38294322)
\lineto(242.19199768,98.63094321)
\curveto(241.7759977,98.63094321)(241.57599771,98.63094321)(241.56799771,98.8709432)
\lineto(241.56799771,100.39894311)
\curveto(241.56799771,101.08694308)(241.56799771,101.33494307)(241.31999773,101.62294305)
\curveto(241.20799773,101.75894304)(240.94399775,101.91894304)(240.47999777,101.91894304)
\curveto(239.80799781,101.91894304)(239.45599782,101.43894306)(239.31999783,101.13494308)
\curveto(239.20799784,101.83094304)(238.61599787,101.91894304)(238.25599789,101.91894304)
\curveto(237.67199792,101.91894304)(237.29599794,101.57494305)(237.07199795,101.07894308)
\lineto(237.07199795,101.91894304)
\lineto(235.94399801,101.83094304)
\lineto(235.94399801,101.58294305)
\curveto(236.50399798,101.58294305)(236.56799798,101.52694306)(236.56799798,101.13494308)
\lineto(236.56799798,98.99094319)
\curveto(236.56799798,98.63094321)(236.47999798,98.63094321)(235.94399801,98.63094321)
\lineto(235.94399801,98.38294322)
\lineto(236.84799796,98.40694322)
\lineto(237.74399791,98.38294322)
\lineto(237.74399791,98.63094321)
\curveto(237.20799794,98.63094321)(237.11999795,98.63094321)(237.11999795,98.99094319)
\lineto(237.11999795,100.46294311)
\curveto(237.11999795,101.29494307)(237.68799792,101.74294304)(238.19999789,101.74294304)
\curveto(238.70399786,101.74294304)(238.79199786,101.31094307)(238.79199786,100.85494309)
\lineto(238.79199786,98.99094319)
\curveto(238.79199786,98.63094321)(238.70399786,98.63094321)(238.16799789,98.63094321)
\lineto(238.16799789,98.38294322)
\lineto(239.07199784,98.40694322)
\lineto(239.9679978,98.38294322)
\lineto(239.9679978,98.63094321)
\curveto(239.43199783,98.63094321)(239.34399783,98.63094321)(239.34399783,98.99094319)
\lineto(239.34399783,100.46294311)
\curveto(239.34399783,101.29494307)(239.9119978,101.74294304)(240.42399777,101.74294304)
\curveto(240.92799775,101.74294304)(241.01599774,101.31094307)(241.01599774,100.85494309)
\lineto(241.01599774,98.99094319)
\curveto(241.01599774,98.63094321)(240.92799775,98.63094321)(240.39199778,98.63094321)
\lineto(240.39199778,98.38294322)
\lineto(241.29599773,98.40694322)
\closepath
}
}
{
\newrgbcolor{curcolor}{0.65490198 0.66274512 0.67450982}
\pscustom[linestyle=none,fillstyle=solid,fillcolor=curcolor]
{
\newpath
\moveto(246.21598348,99.09494318)
\lineto(246.21598348,99.54294316)
\lineto(246.0159835,99.54294316)
\lineto(246.0159835,99.09494318)
\curveto(246.0159835,98.63094321)(245.81598351,98.58294321)(245.72798351,98.58294321)
\curveto(245.46398352,98.58294321)(245.43198353,98.94294319)(245.43198353,98.98294319)
\lineto(245.43198353,100.58294311)
\curveto(245.43198353,100.91894309)(245.43198353,101.23094307)(245.14398354,101.52694306)
\curveto(244.83198356,101.83894304)(244.43198358,101.96694303)(244.0479836,101.96694303)
\curveto(243.39198363,101.96694303)(242.83998366,101.59094305)(242.83998366,101.06294308)
\curveto(242.83998366,100.82294309)(242.99998365,100.6869431)(243.20798364,100.6869431)
\curveto(243.43198363,100.6869431)(243.57598362,100.84694309)(243.57598362,101.05494308)
\curveto(243.57598362,101.15094308)(243.53598363,101.41494306)(243.16798365,101.42294306)
\curveto(243.38398363,101.70294305)(243.77598361,101.79094304)(244.0319836,101.79094304)
\curveto(244.42398358,101.79094304)(244.87998356,101.47894306)(244.87998356,100.7669431)
\lineto(244.87998356,100.47094311)
\curveto(244.47198358,100.44694311)(243.91198361,100.42294311)(243.40798363,100.18294313)
\curveto(242.80798366,99.91094314)(242.60798367,99.49494316)(242.60798367,99.14294318)
\curveto(242.60798367,98.49494322)(243.38398363,98.29494323)(243.88798361,98.29494323)
\curveto(244.41598358,98.29494323)(244.78398356,98.61494321)(244.93598355,98.99094319)
\curveto(244.96798355,98.67094321)(245.18398354,98.33494322)(245.55998352,98.33494322)
\curveto(245.72798351,98.33494322)(246.21598348,98.44694322)(246.21598348,99.09494318)
\closepath
\moveto(244.87998356,99.50294316)
\curveto(244.87998356,98.7429432)(244.30398359,98.47094322)(243.9439836,98.47094322)
\curveto(243.55198362,98.47094322)(243.22398364,98.7509432)(243.22398364,99.15094318)
\curveto(243.22398364,99.59094316)(243.55998362,100.25494312)(244.87998356,100.30294312)
\closepath
}
}
{
\newrgbcolor{curcolor}{0.65490198 0.66274512 0.67450982}
\pscustom[linestyle=none,fillstyle=solid,fillcolor=curcolor]
{
\newpath
\moveto(248.3919685,98.38294322)
\lineto(248.3919685,98.63094321)
\curveto(247.85596853,98.63094321)(247.76796853,98.63094321)(247.76796853,98.99094319)
\lineto(247.76796853,103.93494293)
\lineto(246.6159686,103.84694293)
\lineto(246.6159686,103.59894295)
\curveto(247.17596857,103.59894295)(247.23996856,103.54294295)(247.23996856,103.15094297)
\lineto(247.23996856,98.99094319)
\curveto(247.23996856,98.63094321)(247.15196857,98.63094321)(246.6159686,98.63094321)
\lineto(246.6159686,98.38294322)
\lineto(247.50396855,98.40694322)
\closepath
}
}
{
\newrgbcolor{curcolor}{0.65490198 0.66274512 0.67450982}
\pscustom[linestyle=none,fillstyle=solid,fillcolor=curcolor]
{
\newpath
\moveto(250.61596644,98.38294322)
\lineto(250.61596644,98.63094321)
\curveto(250.07996647,98.63094321)(249.99196648,98.63094321)(249.99196648,98.99094319)
\lineto(249.99196648,103.93494293)
\lineto(248.83996654,103.84694293)
\lineto(248.83996654,103.59894295)
\curveto(249.39996651,103.59894295)(249.4639665,103.54294295)(249.4639665,103.15094297)
\lineto(249.4639665,98.99094319)
\curveto(249.4639665,98.63094321)(249.37596651,98.63094321)(248.83996654,98.63094321)
\lineto(248.83996654,98.38294322)
\lineto(249.72796649,98.40694322)
\closepath
}
}
{
\newrgbcolor{curcolor}{0.65490198 0.66274512 0.67450982}
\pscustom[linestyle=none,fillstyle=solid,fillcolor=curcolor]
{
\newpath
\moveto(257.00795244,99.94294314)
\curveto(257.00795244,100.6069431)(256.64795245,101.01494308)(256.57595246,101.08694308)
\curveto(256.25595248,101.43894306)(255.96795249,101.51094306)(255.58395251,101.60694305)
\lineto(255.58395251,103.76694294)
\curveto(256.26395247,103.73494294)(256.64795245,103.38294296)(256.76795245,102.93494298)
\curveto(256.73595245,102.94294298)(256.71995245,102.95094298)(256.63995246,102.95094298)
\curveto(256.43195247,102.95094298)(256.27195247,102.80694299)(256.27195247,102.582943)
\curveto(256.27195247,102.33494301)(256.47195246,102.21494302)(256.63995246,102.21494302)
\curveto(256.66395245,102.21494302)(257.00795244,102.22294302)(257.00795244,102.614943)
\curveto(257.00795244,103.39094296)(256.47995246,103.97494293)(255.58395251,104.02294292)
\lineto(255.58395251,104.38294291)
\lineto(255.33595252,104.38294291)
\lineto(255.33595252,104.01494292)
\curveto(254.44795257,103.93494293)(253.9119526,103.21494297)(253.9119526,102.48694301)
\curveto(253.9119526,101.93494303)(254.20795258,101.55094305)(254.35195258,101.41494306)
\curveto(254.65595256,101.10294308)(254.92795255,101.03894308)(255.33595252,100.93494309)
\lineto(255.33595252,98.55094321)
\curveto(254.63195256,98.59094321)(254.25595258,98.99894319)(254.15195259,99.49494316)
\curveto(254.18395258,99.48694316)(254.24795258,99.47894316)(254.27995258,99.47894316)
\curveto(254.49595257,99.47894316)(254.64795256,99.63094316)(254.64795256,99.84694314)
\curveto(254.64795256,100.07094313)(254.47995257,100.21494312)(254.27995258,100.21494312)
\curveto(254.23195258,100.21494312)(253.9119526,100.19894313)(253.9119526,99.80694315)
\curveto(253.9119526,99.09494318)(254.33595258,98.35894322)(255.33595252,98.29494323)
\lineto(255.33595252,97.93494324)
\lineto(255.58395251,97.93494324)
\lineto(255.58395251,98.30294323)
\curveto(256.42395247,98.37494322)(257.00795244,99.11894318)(257.00795244,99.94294314)
\closepath
\moveto(256.59995246,99.70294315)
\curveto(256.59995246,99.16694318)(256.21595248,98.63094321)(255.58395251,98.55094321)
\lineto(255.58395251,100.87094309)
\curveto(256.54395246,100.6709431)(256.59995246,99.89494314)(256.59995246,99.70294315)
\closepath
\moveto(255.33595252,101.67094305)
\curveto(254.35195258,101.87894304)(254.31995258,102.566943)(254.31995258,102.72694299)
\curveto(254.31995258,103.20694297)(254.69595256,103.70294294)(255.33595252,103.76694294)
\closepath
}
}
{
\newrgbcolor{curcolor}{0.65490198 0.66274512 0.67450982}
\pscustom[linestyle=none,fillstyle=solid,fillcolor=curcolor]
{
\newpath
\moveto(261.11993645,99.72694315)
\curveto(261.11993645,100.01494314)(261.03193645,100.37494312)(260.72793647,100.7109431)
\curveto(260.57593648,100.87894309)(260.44793649,100.95894309)(259.93593651,101.27894307)
\curveto(260.51193648,101.57494305)(260.90393646,101.99094303)(260.90393646,102.518943)
\curveto(260.90393646,103.25494296)(260.1919365,103.71094294)(259.46393654,103.71094294)
\curveto(258.66393658,103.71094294)(258.01593661,103.11894297)(258.01593661,102.37494301)
\curveto(258.01593661,102.23094302)(258.03193661,101.87094304)(258.36793659,101.49494306)
\curveto(258.45593659,101.39894306)(258.75193657,101.19894307)(258.95193656,101.06294308)
\curveto(258.48793659,100.83094309)(257.79993662,100.38294312)(257.79993662,99.59094316)
\curveto(257.79993662,98.7429432)(258.61593658,98.20694323)(259.45593654,98.20694323)
\curveto(260.35993649,98.20694323)(261.11993645,98.8709432)(261.11993645,99.72694315)
\closepath
\moveto(260.55193648,102.518943)
\curveto(260.55193648,102.06294303)(260.2399365,101.67894305)(259.75993652,101.39894306)
\lineto(258.76793657,102.03894303)
\curveto(258.39993659,102.27894302)(258.36793659,102.550943)(258.36793659,102.68694299)
\curveto(258.36793659,103.17494297)(258.88793657,103.51094295)(259.45593654,103.51094295)
\curveto(260.03993651,103.51094295)(260.55193648,103.09494297)(260.55193648,102.518943)
\closepath
\moveto(260.71993647,99.43894317)
\curveto(260.71993647,98.8469432)(260.1199365,98.43094322)(259.46393654,98.43094322)
\curveto(258.77593657,98.43094322)(258.1999366,98.92694319)(258.1999366,99.59094316)
\curveto(258.1999366,100.05494313)(258.45593659,100.56694311)(259.13593655,100.94294309)
\lineto(260.1199365,100.31894312)
\curveto(260.34393649,100.16694313)(260.71993647,99.92694314)(260.71993647,99.43894317)
\closepath
}
}
{
\newrgbcolor{curcolor}{0.65490198 0.66274512 0.67450982}
\pscustom[linestyle=none,fillstyle=solid,fillcolor=curcolor]
{
\newpath
\moveto(265.00792048,99.94294314)
\curveto(265.00792048,100.6069431)(264.6479205,101.01494308)(264.5759205,101.08694308)
\curveto(264.25592052,101.43894306)(263.96792053,101.51094306)(263.58392055,101.60694305)
\lineto(263.58392055,103.76694294)
\curveto(264.26392052,103.73494294)(264.6479205,103.38294296)(264.76792049,102.93494298)
\curveto(264.73592049,102.94294298)(264.71992049,102.95094298)(264.6399205,102.95094298)
\curveto(264.43192051,102.95094298)(264.27192052,102.80694299)(264.27192052,102.582943)
\curveto(264.27192052,102.33494301)(264.4719205,102.21494302)(264.6399205,102.21494302)
\curveto(264.66392049,102.21494302)(265.00792048,102.22294302)(265.00792048,102.614943)
\curveto(265.00792048,103.39094296)(264.4799205,103.97494293)(263.58392055,104.02294292)
\lineto(263.58392055,104.38294291)
\lineto(263.33592056,104.38294291)
\lineto(263.33592056,104.01494292)
\curveto(262.44792061,103.93494293)(261.91192064,103.21494297)(261.91192064,102.48694301)
\curveto(261.91192064,101.93494303)(262.20792062,101.55094305)(262.35192062,101.41494306)
\curveto(262.6559206,101.10294308)(262.92792059,101.03894308)(263.33592056,100.93494309)
\lineto(263.33592056,98.55094321)
\curveto(262.6319206,98.59094321)(262.25592062,98.99894319)(262.15192063,99.49494316)
\curveto(262.18392062,99.48694316)(262.24792062,99.47894316)(262.27992062,99.47894316)
\curveto(262.49592061,99.47894316)(262.6479206,99.63094316)(262.6479206,99.84694314)
\curveto(262.6479206,100.07094313)(262.47992061,100.21494312)(262.27992062,100.21494312)
\curveto(262.23192062,100.21494312)(261.91192064,100.19894313)(261.91192064,99.80694315)
\curveto(261.91192064,99.09494318)(262.33592062,98.35894322)(263.33592056,98.29494323)
\lineto(263.33592056,97.93494324)
\lineto(263.58392055,97.93494324)
\lineto(263.58392055,98.30294323)
\curveto(264.42392051,98.37494322)(265.00792048,99.11894318)(265.00792048,99.94294314)
\closepath
\moveto(264.5999205,99.70294315)
\curveto(264.5999205,99.16694318)(264.21592052,98.63094321)(263.58392055,98.55094321)
\lineto(263.58392055,100.87094309)
\curveto(264.5439205,100.6709431)(264.5999205,99.89494314)(264.5999205,99.70294315)
\closepath
\moveto(263.33592056,101.67094305)
\curveto(262.35192062,101.87894304)(262.31992062,102.566943)(262.31992062,102.72694299)
\curveto(262.31992062,103.20694297)(262.6959206,103.70294294)(263.33592056,103.76694294)
\closepath
}
}
{
\newrgbcolor{curcolor}{0 0 0}
\pscustom[linestyle=none,fillstyle=solid,fillcolor=curcolor]
{
\newpath
\moveto(106.06115207,16.60009988)
\curveto(106.06115207,17.70676652)(105.46115209,18.3867665)(105.34115209,18.50676649)
\curveto(104.80781878,19.09343314)(104.32781879,19.21343314)(103.68781881,19.37343314)
\lineto(103.68781881,22.97343303)
\curveto(104.82115211,22.9200997)(105.46115209,22.33343305)(105.66115208,21.5867664)
\curveto(105.60781875,21.60009973)(105.58115209,21.61343307)(105.44781876,21.61343307)
\curveto(105.1011521,21.61343307)(104.83448544,21.37343307)(104.83448544,21.00009975)
\curveto(104.83448544,20.58676643)(105.16781877,20.38676644)(105.44781876,20.38676644)
\curveto(105.48781876,20.38676644)(106.06115207,20.40009977)(106.06115207,21.05343308)
\curveto(106.06115207,22.34676638)(105.1811521,23.32009968)(103.68781881,23.40009968)
\lineto(103.68781881,24.00009966)
\lineto(103.27448549,24.00009966)
\lineto(103.27448549,23.38676635)
\curveto(101.79448553,23.25343302)(100.90115223,22.05343305)(100.90115223,20.84009976)
\curveto(100.90115223,19.92009979)(101.39448555,19.2800998)(101.63448554,19.05343314)
\curveto(102.14115219,18.53343316)(102.59448551,18.4267665)(103.27448549,18.25343317)
\lineto(103.27448549,14.28009995)
\curveto(102.10115219,14.34676662)(101.47448554,15.0267666)(101.30115222,15.85343324)
\curveto(101.35448555,15.84009991)(101.46115221,15.82676657)(101.51448554,15.82676657)
\curveto(101.87448553,15.82676657)(102.12781886,16.0800999)(102.12781886,16.44009989)
\curveto(102.12781886,16.81343321)(101.84781887,17.0534332)(101.51448554,17.0534332)
\curveto(101.43448554,17.0534332)(100.90115223,17.02676654)(100.90115223,16.37343323)
\curveto(100.90115223,15.18676659)(101.60781887,13.96009996)(103.27448549,13.8534333)
\lineto(103.27448549,13.25343332)
\lineto(103.68781881,13.25343332)
\lineto(103.68781881,13.86676663)
\curveto(105.08781877,13.98676663)(106.06115207,15.22676659)(106.06115207,16.60009988)
\closepath
\moveto(105.38115209,16.2000999)
\curveto(105.38115209,15.30676659)(104.74115211,14.41343328)(103.68781881,14.28009995)
\lineto(103.68781881,18.14676651)
\curveto(105.28781876,17.81343318)(105.38115209,16.52009989)(105.38115209,16.2000999)
\closepath
\moveto(103.27448549,19.4800998)
\curveto(101.63448554,19.82676645)(101.58115221,20.97343309)(101.58115221,21.24009975)
\curveto(101.58115221,22.04009972)(102.20781885,22.86676636)(103.27448549,22.97343303)
\closepath
}
}
{
\newrgbcolor{curcolor}{0 0 0}
\pscustom[linestyle=none,fillstyle=solid,fillcolor=curcolor]
{
\newpath
\moveto(116.43445986,17.3334332)
\curveto(116.43445986,17.60009985)(116.18112654,17.60009985)(115.99445988,17.60009985)
\lineto(112.27445999,17.60009985)
\lineto(112.27445999,21.33343308)
\curveto(112.27445999,21.52009974)(112.27445999,21.77343306)(112.00779333,21.77343306)
\curveto(111.74112667,21.77343306)(111.74112667,21.52009974)(111.74112667,21.33343308)
\lineto(111.74112667,17.60009985)
\lineto(108.00779345,17.60009985)
\curveto(107.82112679,17.60009985)(107.56779346,17.60009985)(107.56779346,17.3334332)
\curveto(107.56779346,17.06676654)(107.82112679,17.06676654)(108.00779345,17.06676654)
\lineto(111.74112667,17.06676654)
\lineto(111.74112667,13.33343332)
\curveto(111.74112667,13.14676666)(111.74112667,12.89343333)(112.00779333,12.89343333)
\curveto(112.27445999,12.89343333)(112.27445999,13.14676666)(112.27445999,13.33343332)
\lineto(112.27445999,17.06676654)
\lineto(115.99445988,17.06676654)
\curveto(116.18112654,17.06676654)(116.43445986,17.06676654)(116.43445986,17.3334332)
\closepath
}
}
{
\newrgbcolor{curcolor}{0 0 0}
\pscustom[linestyle=none,fillstyle=solid,fillcolor=curcolor]
{
\newpath
\moveto(123.10112361,16.60009988)
\curveto(123.10112361,17.70676652)(122.50112363,18.3867665)(122.38112363,18.50676649)
\curveto(121.84779032,19.09343314)(121.36779033,19.21343314)(120.72779035,19.37343314)
\lineto(120.72779035,22.97343303)
\curveto(121.86112365,22.9200997)(122.50112363,22.33343305)(122.70112362,21.5867664)
\curveto(122.64779029,21.60009973)(122.62112363,21.61343307)(122.4877903,21.61343307)
\curveto(122.14112364,21.61343307)(121.87445698,21.37343307)(121.87445698,21.00009975)
\curveto(121.87445698,20.58676643)(122.20779031,20.38676644)(122.4877903,20.38676644)
\curveto(122.5277903,20.38676644)(123.10112361,20.40009977)(123.10112361,21.05343308)
\curveto(123.10112361,22.34676638)(122.22112364,23.32009968)(120.72779035,23.40009968)
\lineto(120.72779035,24.00009966)
\lineto(120.31445703,24.00009966)
\lineto(120.31445703,23.38676635)
\curveto(118.83445707,23.25343302)(117.94112377,22.05343305)(117.94112377,20.84009976)
\curveto(117.94112377,19.92009979)(118.43445709,19.2800998)(118.67445708,19.05343314)
\curveto(119.18112373,18.53343316)(119.63445705,18.4267665)(120.31445703,18.25343317)
\lineto(120.31445703,14.28009995)
\curveto(119.14112373,14.34676662)(118.51445708,15.0267666)(118.34112375,15.85343324)
\curveto(118.39445709,15.84009991)(118.50112375,15.82676657)(118.55445708,15.82676657)
\curveto(118.91445707,15.82676657)(119.1677904,16.0800999)(119.1677904,16.44009989)
\curveto(119.1677904,16.81343321)(118.8877904,17.0534332)(118.55445708,17.0534332)
\curveto(118.47445708,17.0534332)(117.94112377,17.02676654)(117.94112377,16.37343323)
\curveto(117.94112377,15.18676659)(118.64779041,13.96009996)(120.31445703,13.8534333)
\lineto(120.31445703,13.25343332)
\lineto(120.72779035,13.25343332)
\lineto(120.72779035,13.86676663)
\curveto(122.12779031,13.98676663)(123.10112361,15.22676659)(123.10112361,16.60009988)
\closepath
\moveto(122.42112363,16.2000999)
\curveto(122.42112363,15.30676659)(121.78112365,14.41343328)(120.72779035,14.28009995)
\lineto(120.72779035,18.14676651)
\curveto(122.3277903,17.81343318)(122.42112363,16.52009989)(122.42112363,16.2000999)
\closepath
\moveto(120.31445703,19.4800998)
\curveto(118.67445708,19.82676645)(118.62112375,20.97343309)(118.62112375,21.24009975)
\curveto(118.62112375,22.04009972)(119.24779039,22.86676636)(120.31445703,22.97343303)
\closepath
}
}
{
\newrgbcolor{curcolor}{0 0 0}
\pscustom[linewidth=0.99999995,linecolor=curcolor]
{
\newpath
\moveto(111.99998362,47.73085984)
\lineto(111.99998362,27.38282835)
}
}
{
\newrgbcolor{curcolor}{0 0 0}
\pscustom[linestyle=none,fillstyle=solid,fillcolor=curcolor]
{
\newpath
\moveto(111.99998362,31.38282814)
\lineto(109.99998373,33.38282803)
\lineto(111.99998362,26.3828284)
\lineto(113.99998352,33.38282803)
\closepath
}
}
{
\newrgbcolor{curcolor}{0 0 0}
\pscustom[linewidth=0.53333332,linecolor=curcolor]
{
\newpath
\moveto(111.99998362,31.38282814)
\lineto(109.99998373,33.38282803)
\lineto(111.99998362,26.3828284)
\lineto(113.99998352,33.38282803)
\closepath
}
}
{
\newrgbcolor{curcolor}{0.65490198 0.66274512 0.67450982}
\pscustom[linestyle=none,fillstyle=solid,fillcolor=curcolor]
{
\newpath
\moveto(125.53204942,33.54289718)
\curveto(125.53204942,33.54289718)(125.53204942,33.58289718)(125.49204942,33.68689717)
\lineto(122.77204956,41.20689678)
\curveto(122.74004957,41.29489677)(122.70804957,41.38289677)(122.59604957,41.38289677)
\curveto(122.50804958,41.38289677)(122.43604958,41.31089677)(122.43604958,41.22289678)
\curveto(122.43604958,41.22289678)(122.43604958,41.18289678)(122.47604958,41.07889678)
\lineto(125.19604944,33.55889718)
\curveto(125.22804943,33.47089718)(125.26004943,33.38289719)(125.37204943,33.38289719)
\curveto(125.46004942,33.38289719)(125.53204942,33.45489719)(125.53204942,33.54289718)
\closepath
}
}
{
\newrgbcolor{curcolor}{0.65490198 0.66274512 0.67450982}
\pscustom[linestyle=none,fillstyle=solid,fillcolor=curcolor]
{
\newpath
\moveto(128.86803392,36.40689703)
\curveto(128.86803392,36.83089701)(128.62803394,37.070897)(128.53203394,37.16689699)
\curveto(128.26803396,37.42289698)(127.95603397,37.48689697)(127.62003399,37.55089697)
\curveto(127.17203401,37.63889697)(126.63603404,37.74289696)(126.63603404,38.20689694)
\curveto(126.63603404,38.48689692)(126.84403403,38.8148969)(127.53203399,38.8148969)
\curveto(128.41203395,38.8148969)(128.45203395,38.09489694)(128.46803395,37.84689695)
\curveto(128.47603394,37.77489696)(128.56403394,37.77489696)(128.56403394,37.77489696)
\curveto(128.66803393,37.77489696)(128.66803393,37.81489696)(128.66803393,37.96689695)
\lineto(128.66803393,38.77489691)
\curveto(128.66803393,38.9108969)(128.66803393,38.9668969)(128.58003394,38.9668969)
\curveto(128.54003394,38.9668969)(128.52403394,38.9668969)(128.42003395,38.8708969)
\curveto(128.39603395,38.8388969)(128.31603395,38.76689691)(128.28403395,38.74289691)
\curveto(127.98003397,38.9668969)(127.65203399,38.9668969)(127.53203399,38.9668969)
\curveto(126.55603405,38.9668969)(126.25203406,38.43089692)(126.25203406,37.98289695)
\curveto(126.25203406,37.70289696)(126.38003406,37.47889697)(126.59603404,37.30289698)
\curveto(126.85203403,37.09489699)(127.07603402,37.046897)(127.65203399,36.934897)
\curveto(127.82803398,36.902897)(128.48403394,36.77489701)(128.48403394,36.19889704)
\curveto(128.48403394,35.79089706)(128.20403396,35.47089708)(127.58003399,35.47089708)
\curveto(126.90803403,35.47089708)(126.62003404,35.92689706)(126.46803405,36.60689702)
\curveto(126.44403405,36.71089701)(126.43603405,36.74289701)(126.35603406,36.74289701)
\curveto(126.25203406,36.74289701)(126.25203406,36.68689702)(126.25203406,36.54289702)
\lineto(126.25203406,35.48689708)
\curveto(126.25203406,35.35089709)(126.25203406,35.29489709)(126.34003406,35.29489709)
\curveto(126.38003406,35.29489709)(126.38803405,35.30289709)(126.54003405,35.45489708)
\curveto(126.55603405,35.47089708)(126.55603405,35.48689708)(126.70003404,35.63889707)
\curveto(127.05203402,35.30289709)(127.412034,35.29489709)(127.58003399,35.29489709)
\curveto(128.50003394,35.29489709)(128.86803392,35.83089706)(128.86803392,36.40689703)
\closepath
}
}
{
\newrgbcolor{curcolor}{0.65490198 0.66274512 0.67450982}
\pscustom[linestyle=none,fillstyle=solid,fillcolor=curcolor]
{
\newpath
\moveto(135.64402632,35.38289708)
\lineto(135.64402632,35.63089707)
\curveto(135.22802634,35.63089707)(135.02802635,35.63089707)(135.02002635,35.87089706)
\lineto(135.02002635,37.39889698)
\curveto(135.02002635,38.08689694)(135.02002635,38.33489693)(134.77202637,38.62289691)
\curveto(134.66002637,38.75889691)(134.39602639,38.9188969)(133.93202641,38.9188969)
\curveto(133.26002645,38.9188969)(132.90802647,38.43889692)(132.77202647,38.13489694)
\curveto(132.66002648,38.8308969)(132.06802651,38.9188969)(131.70802653,38.9188969)
\curveto(131.12402656,38.9188969)(130.74802658,38.57489692)(130.52402659,38.07889694)
\lineto(130.52402659,38.9188969)
\lineto(129.39602665,38.8308969)
\lineto(129.39602665,38.58289692)
\curveto(129.95602662,38.58289692)(130.02002662,38.52689692)(130.02002662,38.13489694)
\lineto(130.02002662,35.99089705)
\curveto(130.02002662,35.63089707)(129.93202662,35.63089707)(129.39602665,35.63089707)
\lineto(129.39602665,35.38289708)
\lineto(130.3000266,35.40689708)
\lineto(131.19602656,35.38289708)
\lineto(131.19602656,35.63089707)
\curveto(130.66002658,35.63089707)(130.57202659,35.63089707)(130.57202659,35.99089705)
\lineto(130.57202659,37.46289697)
\curveto(130.57202659,38.29489693)(131.14002656,38.74289691)(131.65202653,38.74289691)
\curveto(132.1560265,38.74289691)(132.2440265,38.31089693)(132.2440265,37.85489695)
\lineto(132.2440265,35.99089705)
\curveto(132.2440265,35.63089707)(132.1560265,35.63089707)(131.62002653,35.63089707)
\lineto(131.62002653,35.38289708)
\lineto(132.52402649,35.40689708)
\lineto(133.42002644,35.38289708)
\lineto(133.42002644,35.63089707)
\curveto(132.88402647,35.63089707)(132.79602647,35.63089707)(132.79602647,35.99089705)
\lineto(132.79602647,37.46289697)
\curveto(132.79602647,38.29489693)(133.36402644,38.74289691)(133.87602641,38.74289691)
\curveto(134.38002639,38.74289691)(134.46802638,38.31089693)(134.46802638,37.85489695)
\lineto(134.46802638,35.99089705)
\curveto(134.46802638,35.63089707)(134.38002639,35.63089707)(133.84402642,35.63089707)
\lineto(133.84402642,35.38289708)
\lineto(134.74802637,35.40689708)
\closepath
}
}
{
\newrgbcolor{curcolor}{0.65490198 0.66274512 0.67450982}
\pscustom[linestyle=none,fillstyle=solid,fillcolor=curcolor]
{
\newpath
\moveto(139.66801213,36.09489705)
\lineto(139.66801213,36.54289702)
\lineto(139.46801214,36.54289702)
\lineto(139.46801214,36.09489705)
\curveto(139.46801214,35.63089707)(139.26801215,35.58289707)(139.18001215,35.58289707)
\curveto(138.91601217,35.58289707)(138.88401217,35.94289705)(138.88401217,35.98289705)
\lineto(138.88401217,37.58289697)
\curveto(138.88401217,37.91889695)(138.88401217,38.23089693)(138.59601218,38.52689692)
\curveto(138.2840122,38.8388969)(137.88401222,38.9668969)(137.50001224,38.9668969)
\curveto(136.84401227,38.9668969)(136.2920123,38.59089692)(136.2920123,38.06289694)
\curveto(136.2920123,37.82289696)(136.45201229,37.68689696)(136.66001228,37.68689696)
\curveto(136.88401227,37.68689696)(137.02801226,37.84689695)(137.02801226,38.05489694)
\curveto(137.02801226,38.15089694)(136.98801227,38.41489692)(136.62001229,38.42289692)
\curveto(136.83601227,38.70289691)(137.22801225,38.79089691)(137.48401224,38.79089691)
\curveto(137.87601222,38.79089691)(138.3320122,38.47889692)(138.3320122,37.76689696)
\lineto(138.3320122,37.47089697)
\curveto(137.92401222,37.44689698)(137.36401225,37.42289698)(136.86001227,37.18289699)
\curveto(136.2600123,36.910897)(136.06001232,36.49489703)(136.06001232,36.14289704)
\curveto(136.06001232,35.49489708)(136.83601227,35.29489709)(137.34001225,35.29489709)
\curveto(137.86801222,35.29489709)(138.2360122,35.61489707)(138.38801219,35.99089705)
\curveto(138.42001219,35.67089707)(138.63601218,35.33489709)(139.01201216,35.33489709)
\curveto(139.18001215,35.33489709)(139.66801213,35.44689708)(139.66801213,36.09489705)
\closepath
\moveto(138.3320122,36.50289703)
\curveto(138.3320122,35.74289707)(137.75601223,35.47089708)(137.39601224,35.47089708)
\curveto(137.00401227,35.47089708)(136.67601228,35.75089706)(136.67601228,36.15089704)
\curveto(136.67601228,36.59089702)(137.01201227,37.25489699)(138.3320122,37.30289698)
\closepath
}
}
{
\newrgbcolor{curcolor}{0.65490198 0.66274512 0.67450982}
\pscustom[linestyle=none,fillstyle=solid,fillcolor=curcolor]
{
\newpath
\moveto(141.84399714,35.38289708)
\lineto(141.84399714,35.63089707)
\curveto(141.30799717,35.63089707)(141.21999718,35.63089707)(141.21999718,35.99089705)
\lineto(141.21999718,40.93489679)
\lineto(140.06799724,40.8468968)
\lineto(140.06799724,40.59889681)
\curveto(140.62799721,40.59889681)(140.6919972,40.54289681)(140.6919972,40.15089683)
\lineto(140.6919972,35.99089705)
\curveto(140.6919972,35.63089707)(140.60399721,35.63089707)(140.06799724,35.63089707)
\lineto(140.06799724,35.38289708)
\lineto(140.95599719,35.40689708)
\closepath
}
}
{
\newrgbcolor{curcolor}{0.65490198 0.66274512 0.67450982}
\pscustom[linestyle=none,fillstyle=solid,fillcolor=curcolor]
{
\newpath
\moveto(144.06799508,35.38289708)
\lineto(144.06799508,35.63089707)
\curveto(143.53199511,35.63089707)(143.44399512,35.63089707)(143.44399512,35.99089705)
\lineto(143.44399512,40.93489679)
\lineto(142.29199518,40.8468968)
\lineto(142.29199518,40.59889681)
\curveto(142.85199515,40.59889681)(142.91599515,40.54289681)(142.91599515,40.15089683)
\lineto(142.91599515,35.99089705)
\curveto(142.91599515,35.63089707)(142.82799515,35.63089707)(142.29199518,35.63089707)
\lineto(142.29199518,35.38289708)
\lineto(143.17999513,35.40689708)
\closepath
}
}
{
\newrgbcolor{curcolor}{0.65490198 0.66274512 0.67450982}
\pscustom[linestyle=none,fillstyle=solid,fillcolor=curcolor]
{
\newpath
\moveto(150.45998108,36.942897)
\curveto(150.45998108,37.60689697)(150.0999811,38.01489695)(150.0279811,38.08689694)
\curveto(149.70798112,38.43889692)(149.41998113,38.51089692)(149.03598115,38.60689691)
\lineto(149.03598115,40.7668968)
\curveto(149.71598112,40.7348968)(150.0999811,40.38289682)(150.21998109,39.93489684)
\curveto(150.18798109,39.94289684)(150.17198109,39.95089684)(150.0919811,39.95089684)
\curveto(149.88398111,39.95089684)(149.72398112,39.80689685)(149.72398112,39.58289686)
\curveto(149.72398112,39.33489688)(149.9239811,39.21489688)(150.0919811,39.21489688)
\curveto(150.11598109,39.21489688)(150.45998108,39.22289688)(150.45998108,39.61489686)
\curveto(150.45998108,40.39089682)(149.9319811,40.97489679)(149.03598115,41.02289679)
\lineto(149.03598115,41.38289677)
\lineto(148.78798116,41.38289677)
\lineto(148.78798116,41.01489679)
\curveto(147.89998121,40.93489679)(147.36398124,40.21489683)(147.36398124,39.48689687)
\curveto(147.36398124,38.9348969)(147.65998122,38.55089692)(147.80398122,38.41489692)
\curveto(148.1079812,38.10289694)(148.37998119,38.03889694)(148.78798116,37.93489695)
\lineto(148.78798116,35.55089708)
\curveto(148.0839812,35.59089707)(147.70798122,35.99889705)(147.60398123,36.49489703)
\curveto(147.63598122,36.48689703)(147.69998122,36.47889703)(147.73198122,36.47889703)
\curveto(147.94798121,36.47889703)(148.0999812,36.63089702)(148.0999812,36.84689701)
\curveto(148.0999812,37.070897)(147.93198121,37.21489699)(147.73198122,37.21489699)
\curveto(147.68398122,37.21489699)(147.36398124,37.19889699)(147.36398124,36.80689701)
\curveto(147.36398124,36.09489705)(147.78798122,35.35889709)(148.78798116,35.29489709)
\lineto(148.78798116,34.93489711)
\lineto(149.03598115,34.93489711)
\lineto(149.03598115,35.30289709)
\curveto(149.87598111,35.37489708)(150.45998108,36.11889705)(150.45998108,36.942897)
\closepath
\moveto(150.0519811,36.70289701)
\curveto(150.0519811,36.16689704)(149.66798112,35.63089707)(149.03598115,35.55089708)
\lineto(149.03598115,37.87089695)
\curveto(149.9959811,37.67089696)(150.0519811,36.894897)(150.0519811,36.70289701)
\closepath
\moveto(148.78798116,38.67089691)
\curveto(147.80398122,38.8788969)(147.77198122,39.56689686)(147.77198122,39.72689686)
\curveto(147.77198122,40.20689683)(148.1479812,40.7028968)(148.78798116,40.7668968)
\closepath
}
}
{
\newrgbcolor{curcolor}{0.65490198 0.66274512 0.67450982}
\pscustom[linestyle=none,fillstyle=solid,fillcolor=curcolor]
{
\newpath
\moveto(154.26796511,35.38289708)
\lineto(154.26796511,35.63089707)
\lineto(154.01196512,35.63089707)
\curveto(153.29196516,35.63089707)(153.26796516,35.71889707)(153.26796516,36.01489705)
\lineto(153.26796516,40.50289682)
\curveto(153.26796516,40.6948968)(153.26796516,40.7108968)(153.08396517,40.7108968)
\curveto(152.5879652,40.19889683)(151.88396523,40.19889683)(151.62796525,40.19889683)
\lineto(151.62796525,39.95089684)
\curveto(151.78796524,39.95089684)(152.25996521,39.95089684)(152.67596519,40.15889683)
\lineto(152.67596519,36.01489705)
\curveto(152.67596519,35.72689707)(152.65196519,35.63089707)(151.93196523,35.63089707)
\lineto(151.67596524,35.63089707)
\lineto(151.67596524,35.38289708)
\curveto(151.95596523,35.40689708)(152.65196519,35.40689708)(152.97196517,35.40689708)
\curveto(153.29196516,35.40689708)(153.98796512,35.40689708)(154.26796511,35.38289708)
\closepath
}
}
{
\newrgbcolor{curcolor}{0.65490198 0.66274512 0.67450982}
\pscustom[linestyle=none,fillstyle=solid,fillcolor=curcolor]
{
\newpath
\moveto(158.57194911,36.75089701)
\curveto(158.57194911,37.40689698)(158.06794914,38.03089694)(157.23594918,38.19889694)
\curveto(157.89194915,38.41489692)(158.35594912,38.9748969)(158.35594912,39.60689686)
\curveto(158.35594912,40.26289683)(157.65194916,40.7108968)(156.8839492,40.7108968)
\curveto(156.07594924,40.7108968)(155.46794927,40.23089683)(155.46794927,39.62289686)
\curveto(155.46794927,39.35889688)(155.64394927,39.20689688)(155.87594925,39.20689688)
\curveto(156.12394924,39.20689688)(156.28394923,39.38289687)(156.28394923,39.61489686)
\curveto(156.28394923,40.01489684)(155.90794925,40.01489684)(155.78794926,40.01489684)
\curveto(156.03594924,40.40689682)(156.56394922,40.51089681)(156.8519492,40.51089681)
\curveto(157.17994918,40.51089681)(157.61994916,40.33489682)(157.61994916,39.61489686)
\curveto(157.61994916,39.51889687)(157.60394916,39.05489689)(157.39594917,38.70289691)
\curveto(157.15594919,38.31889693)(156.8839492,38.29489693)(156.68394921,38.28689693)
\curveto(156.61994921,38.27889693)(156.42794922,38.26289693)(156.37194923,38.26289693)
\curveto(156.30794923,38.25489693)(156.25194923,38.24689693)(156.25194923,38.16689694)
\curveto(156.25194923,38.07889694)(156.30794923,38.07889694)(156.44394922,38.07889694)
\lineto(156.7959492,38.07889694)
\curveto(157.45194917,38.07889694)(157.74794915,37.53489697)(157.74794915,36.75089701)
\curveto(157.74794915,35.66289707)(157.19594918,35.43089708)(156.8439492,35.43089708)
\curveto(156.49994922,35.43089708)(155.89994925,35.56689707)(155.61994927,36.03889705)
\curveto(155.89994925,35.99889705)(156.14794924,36.17489704)(156.14794924,36.47889703)
\curveto(156.14794924,36.76689701)(155.93194925,36.926897)(155.69994926,36.926897)
\curveto(155.50794927,36.926897)(155.25194929,36.81489701)(155.25194929,36.46289703)
\curveto(155.25194929,35.73489707)(155.99594925,35.20689709)(156.8679492,35.20689709)
\curveto(157.84394915,35.20689709)(158.57194911,35.93489706)(158.57194911,36.75089701)
\closepath
}
}
{
\newrgbcolor{curcolor}{0.65490198 0.66274512 0.67450982}
\pscustom[linestyle=none,fillstyle=solid,fillcolor=curcolor]
{
\newpath
\moveto(162.45993494,36.942897)
\curveto(162.45993494,37.60689697)(162.09993496,38.01489695)(162.02793496,38.08689694)
\curveto(161.70793498,38.43889692)(161.41993499,38.51089692)(161.03593501,38.60689691)
\lineto(161.03593501,40.7668968)
\curveto(161.71593498,40.7348968)(162.09993496,40.38289682)(162.21993495,39.93489684)
\curveto(162.18793495,39.94289684)(162.17193495,39.95089684)(162.09193496,39.95089684)
\curveto(161.88393497,39.95089684)(161.72393498,39.80689685)(161.72393498,39.58289686)
\curveto(161.72393498,39.33489688)(161.92393497,39.21489688)(162.09193496,39.21489688)
\curveto(162.11593496,39.21489688)(162.45993494,39.22289688)(162.45993494,39.61489686)
\curveto(162.45993494,40.39089682)(161.93193497,40.97489679)(161.03593501,41.02289679)
\lineto(161.03593501,41.38289677)
\lineto(160.78793503,41.38289677)
\lineto(160.78793503,41.01489679)
\curveto(159.89993507,40.93489679)(159.3639351,40.21489683)(159.3639351,39.48689687)
\curveto(159.3639351,38.9348969)(159.65993509,38.55089692)(159.80393508,38.41489692)
\curveto(160.10793506,38.10289694)(160.37993505,38.03889694)(160.78793503,37.93489695)
\lineto(160.78793503,35.55089708)
\curveto(160.08393506,35.59089707)(159.70793508,35.99889705)(159.60393509,36.49489703)
\curveto(159.63593509,36.48689703)(159.69993508,36.47889703)(159.73193508,36.47889703)
\curveto(159.94793507,36.47889703)(160.09993506,36.63089702)(160.09993506,36.84689701)
\curveto(160.09993506,37.070897)(159.93193507,37.21489699)(159.73193508,37.21489699)
\curveto(159.68393509,37.21489699)(159.3639351,37.19889699)(159.3639351,36.80689701)
\curveto(159.3639351,36.09489705)(159.78793508,35.35889709)(160.78793503,35.29489709)
\lineto(160.78793503,34.93489711)
\lineto(161.03593501,34.93489711)
\lineto(161.03593501,35.30289709)
\curveto(161.87593497,35.37489708)(162.45993494,36.11889705)(162.45993494,36.942897)
\closepath
\moveto(162.05193496,36.70289701)
\curveto(162.05193496,36.16689704)(161.66793498,35.63089707)(161.03593501,35.55089708)
\lineto(161.03593501,37.87089695)
\curveto(161.99593496,37.67089696)(162.05193496,36.894897)(162.05193496,36.70289701)
\closepath
\moveto(160.78793503,38.67089691)
\curveto(159.80393508,38.8788969)(159.77193508,39.56689686)(159.77193508,39.72689686)
\curveto(159.77193508,40.20689683)(160.14793506,40.7028968)(160.78793503,40.7668968)
\closepath
}
}
{
\newrgbcolor{curcolor}{0 0 0}
\pscustom[linestyle=none,fillstyle=solid,fillcolor=curcolor]
{
\newpath
\moveto(176.05333284,16.60009988)
\curveto(176.05333284,17.70676652)(175.45333286,18.3867665)(175.33333287,18.50676649)
\curveto(174.79999955,19.09343314)(174.31999956,19.21343314)(173.67999958,19.37343314)
\lineto(173.67999958,22.97343303)
\curveto(174.81333288,22.9200997)(175.45333286,22.33343305)(175.65333286,21.5867664)
\curveto(175.59999952,21.60009973)(175.57333286,21.61343307)(175.43999953,21.61343307)
\curveto(175.09333287,21.61343307)(174.82666621,21.37343307)(174.82666621,21.00009975)
\curveto(174.82666621,20.58676643)(175.15999954,20.38676644)(175.43999953,20.38676644)
\curveto(175.47999953,20.38676644)(176.05333284,20.40009977)(176.05333284,21.05343308)
\curveto(176.05333284,22.34676638)(175.17333287,23.32009968)(173.67999958,23.40009968)
\lineto(173.67999958,24.00009966)
\lineto(173.26666626,24.00009966)
\lineto(173.26666626,23.38676635)
\curveto(171.78666631,23.25343302)(170.893333,22.05343305)(170.893333,20.84009976)
\curveto(170.893333,19.92009979)(171.38666632,19.2800998)(171.62666631,19.05343314)
\curveto(172.13333296,18.53343316)(172.58666628,18.4267665)(173.26666626,18.25343317)
\lineto(173.26666626,14.28009995)
\curveto(172.09333296,14.34676662)(171.46666632,15.0267666)(171.29333299,15.85343324)
\curveto(171.34666632,15.84009991)(171.45333298,15.82676657)(171.50666631,15.82676657)
\curveto(171.8666663,15.82676657)(172.11999963,16.0800999)(172.11999963,16.44009989)
\curveto(172.11999963,16.81343321)(171.83999964,17.0534332)(171.50666631,17.0534332)
\curveto(171.42666632,17.0534332)(170.893333,17.02676654)(170.893333,16.37343323)
\curveto(170.893333,15.18676659)(171.59999964,13.96009996)(173.26666626,13.8534333)
\lineto(173.26666626,13.25343332)
\lineto(173.67999958,13.25343332)
\lineto(173.67999958,13.86676663)
\curveto(175.07999954,13.98676663)(176.05333284,15.22676659)(176.05333284,16.60009988)
\closepath
\moveto(175.37333286,16.2000999)
\curveto(175.37333286,15.30676659)(174.73333288,14.41343328)(173.67999958,14.28009995)
\lineto(173.67999958,18.14676651)
\curveto(175.27999953,17.81343318)(175.37333286,16.52009989)(175.37333286,16.2000999)
\closepath
\moveto(173.26666626,19.4800998)
\curveto(171.62666631,19.82676645)(171.57333298,20.97343309)(171.57333298,21.24009975)
\curveto(171.57333298,22.04009972)(172.19999963,22.86676636)(173.26666626,22.97343303)
\closepath
}
}
{
\newrgbcolor{curcolor}{0 0 0}
\pscustom[linestyle=none,fillstyle=solid,fillcolor=curcolor]
{
\newpath
\moveto(186.42664063,17.3334332)
\curveto(186.42664063,17.60009985)(186.17330731,17.60009985)(185.98664065,17.60009985)
\lineto(182.26664076,17.60009985)
\lineto(182.26664076,21.33343308)
\curveto(182.26664076,21.52009974)(182.26664076,21.77343306)(181.9999741,21.77343306)
\curveto(181.73330744,21.77343306)(181.73330744,21.52009974)(181.73330744,21.33343308)
\lineto(181.73330744,17.60009985)
\lineto(177.99997422,17.60009985)
\curveto(177.81330756,17.60009985)(177.55997423,17.60009985)(177.55997423,17.3334332)
\curveto(177.55997423,17.06676654)(177.81330756,17.06676654)(177.99997422,17.06676654)
\lineto(181.73330744,17.06676654)
\lineto(181.73330744,13.33343332)
\curveto(181.73330744,13.14676666)(181.73330744,12.89343333)(181.9999741,12.89343333)
\curveto(182.26664076,12.89343333)(182.26664076,13.14676666)(182.26664076,13.33343332)
\lineto(182.26664076,17.06676654)
\lineto(185.98664065,17.06676654)
\curveto(186.17330731,17.06676654)(186.42664063,17.06676654)(186.42664063,17.3334332)
\closepath
}
}
{
\newrgbcolor{curcolor}{0 0 0}
\pscustom[linestyle=none,fillstyle=solid,fillcolor=curcolor]
{
\newpath
\moveto(193.09330438,16.60009988)
\curveto(193.09330438,17.70676652)(192.4933044,18.3867665)(192.3733044,18.50676649)
\curveto(191.83997109,19.09343314)(191.3599711,19.21343314)(190.71997112,19.37343314)
\lineto(190.71997112,22.97343303)
\curveto(191.85330442,22.9200997)(192.4933044,22.33343305)(192.6933044,21.5867664)
\curveto(192.63997106,21.60009973)(192.6133044,21.61343307)(192.47997107,21.61343307)
\curveto(192.13330441,21.61343307)(191.86663775,21.37343307)(191.86663775,21.00009975)
\curveto(191.86663775,20.58676643)(192.19997108,20.38676644)(192.47997107,20.38676644)
\curveto(192.51997107,20.38676644)(193.09330438,20.40009977)(193.09330438,21.05343308)
\curveto(193.09330438,22.34676638)(192.21330441,23.32009968)(190.71997112,23.40009968)
\lineto(190.71997112,24.00009966)
\lineto(190.3066378,24.00009966)
\lineto(190.3066378,23.38676635)
\curveto(188.82663784,23.25343302)(187.93330454,22.05343305)(187.93330454,20.84009976)
\curveto(187.93330454,19.92009979)(188.42663786,19.2800998)(188.66663785,19.05343314)
\curveto(189.1733045,18.53343316)(189.62663782,18.4267665)(190.3066378,18.25343317)
\lineto(190.3066378,14.28009995)
\curveto(189.1333045,14.34676662)(188.50663785,15.0267666)(188.33330453,15.85343324)
\curveto(188.38663786,15.84009991)(188.49330452,15.82676657)(188.54663785,15.82676657)
\curveto(188.90663784,15.82676657)(189.15997117,16.0800999)(189.15997117,16.44009989)
\curveto(189.15997117,16.81343321)(188.87997118,17.0534332)(188.54663785,17.0534332)
\curveto(188.46663786,17.0534332)(187.93330454,17.02676654)(187.93330454,16.37343323)
\curveto(187.93330454,15.18676659)(188.63997118,13.96009996)(190.3066378,13.8534333)
\lineto(190.3066378,13.25343332)
\lineto(190.71997112,13.25343332)
\lineto(190.71997112,13.86676663)
\curveto(192.11997108,13.98676663)(193.09330438,15.22676659)(193.09330438,16.60009988)
\closepath
\moveto(192.4133044,16.2000999)
\curveto(192.4133044,15.30676659)(191.77330442,14.41343328)(190.71997112,14.28009995)
\lineto(190.71997112,18.14676651)
\curveto(192.31997107,17.81343318)(192.4133044,16.52009989)(192.4133044,16.2000999)
\closepath
\moveto(190.3066378,19.4800998)
\curveto(188.66663785,19.82676645)(188.61330452,20.97343309)(188.61330452,21.24009975)
\curveto(188.61330452,22.04009972)(189.23997117,22.86676636)(190.3066378,22.97343303)
\closepath
}
}
{
\newrgbcolor{curcolor}{0 0 0}
\pscustom[linewidth=0.99999995,linecolor=curcolor]
{
\newpath
\moveto(181.99216252,47.73085984)
\lineto(181.99216252,27.38282835)
}
}
{
\newrgbcolor{curcolor}{0 0 0}
\pscustom[linestyle=none,fillstyle=solid,fillcolor=curcolor]
{
\newpath
\moveto(181.99216252,31.38282814)
\lineto(179.99216262,33.38282803)
\lineto(181.99216252,26.3828284)
\lineto(183.99216241,33.38282803)
\closepath
}
}
{
\newrgbcolor{curcolor}{0 0 0}
\pscustom[linewidth=0.53333332,linecolor=curcolor]
{
\newpath
\moveto(181.99216252,31.38282814)
\lineto(179.99216262,33.38282803)
\lineto(181.99216252,26.3828284)
\lineto(183.99216241,33.38282803)
\closepath
}
}
{
\newrgbcolor{curcolor}{0.65490198 0.66274512 0.67450982}
\pscustom[linestyle=none,fillstyle=solid,fillcolor=curcolor]
{
\newpath
\moveto(195.52423019,33.54289718)
\curveto(195.52423019,33.54289718)(195.52423019,33.58289718)(195.48423019,33.68689717)
\lineto(192.76423034,41.20689678)
\curveto(192.73223034,41.29489677)(192.70023034,41.38289677)(192.58823034,41.38289677)
\curveto(192.50023035,41.38289677)(192.42823035,41.31089677)(192.42823035,41.22289678)
\curveto(192.42823035,41.22289678)(192.42823035,41.18289678)(192.46823035,41.07889678)
\lineto(195.18823021,33.55889718)
\curveto(195.22023021,33.47089718)(195.2522302,33.38289719)(195.3642302,33.38289719)
\curveto(195.45223019,33.38289719)(195.52423019,33.45489719)(195.52423019,33.54289718)
\closepath
}
}
{
\newrgbcolor{curcolor}{0.65490198 0.66274512 0.67450982}
\pscustom[linestyle=none,fillstyle=solid,fillcolor=curcolor]
{
\newpath
\moveto(198.8602147,36.40689703)
\curveto(198.8602147,36.83089701)(198.62021471,37.070897)(198.52421471,37.16689699)
\curveto(198.26021473,37.42289698)(197.94821474,37.48689697)(197.61221476,37.55089697)
\curveto(197.16421479,37.63889697)(196.62821481,37.74289696)(196.62821481,38.20689694)
\curveto(196.62821481,38.48689692)(196.8362148,38.8148969)(197.52421477,38.8148969)
\curveto(198.40421472,38.8148969)(198.44421472,38.09489694)(198.46021472,37.84689695)
\curveto(198.46821472,37.77489696)(198.55621471,37.77489696)(198.55621471,37.77489696)
\curveto(198.66021471,37.77489696)(198.66021471,37.81489696)(198.66021471,37.96689695)
\lineto(198.66021471,38.77489691)
\curveto(198.66021471,38.9108969)(198.66021471,38.9668969)(198.57221471,38.9668969)
\curveto(198.53221471,38.9668969)(198.51621471,38.9668969)(198.41221472,38.8708969)
\curveto(198.38821472,38.8388969)(198.30821472,38.76689691)(198.27621473,38.74289691)
\curveto(197.97221474,38.9668969)(197.64421476,38.9668969)(197.52421477,38.9668969)
\curveto(196.54821482,38.9668969)(196.24421483,38.43089692)(196.24421483,37.98289695)
\curveto(196.24421483,37.70289696)(196.37221483,37.47889697)(196.58821482,37.30289698)
\curveto(196.8442148,37.09489699)(197.06821479,37.046897)(197.64421476,36.934897)
\curveto(197.82021475,36.902897)(198.47621472,36.77489701)(198.47621472,36.19889704)
\curveto(198.47621472,35.79089706)(198.19621473,35.47089708)(197.57221476,35.47089708)
\curveto(196.9002148,35.47089708)(196.61221481,35.92689706)(196.46021482,36.60689702)
\curveto(196.43621482,36.71089701)(196.42821482,36.74289701)(196.34821483,36.74289701)
\curveto(196.24421483,36.74289701)(196.24421483,36.68689702)(196.24421483,36.54289702)
\lineto(196.24421483,35.48689708)
\curveto(196.24421483,35.35089709)(196.24421483,35.29489709)(196.33221483,35.29489709)
\curveto(196.37221483,35.29489709)(196.38021483,35.30289709)(196.53221482,35.45489708)
\curveto(196.54821482,35.47089708)(196.54821482,35.48689708)(196.69221481,35.63889707)
\curveto(197.04421479,35.30289709)(197.40421477,35.29489709)(197.57221476,35.29489709)
\curveto(198.49221472,35.29489709)(198.8602147,35.83089706)(198.8602147,36.40689703)
\closepath
}
}
{
\newrgbcolor{curcolor}{0.65490198 0.66274512 0.67450982}
\pscustom[linestyle=none,fillstyle=solid,fillcolor=curcolor]
{
\newpath
\moveto(205.63620709,35.38289708)
\lineto(205.63620709,35.63089707)
\curveto(205.22020711,35.63089707)(205.02020713,35.63089707)(205.01220713,35.87089706)
\lineto(205.01220713,37.39889698)
\curveto(205.01220713,38.08689694)(205.01220713,38.33489693)(204.76420714,38.62289691)
\curveto(204.65220714,38.75889691)(204.38820716,38.9188969)(203.92420718,38.9188969)
\curveto(203.25220722,38.9188969)(202.90020724,38.43889692)(202.76420724,38.13489694)
\curveto(202.65220725,38.8308969)(202.06020728,38.9188969)(201.7002073,38.9188969)
\curveto(201.11620733,38.9188969)(200.74020735,38.57489692)(200.51620736,38.07889694)
\lineto(200.51620736,38.9188969)
\lineto(199.38820742,38.8308969)
\lineto(199.38820742,38.58289692)
\curveto(199.94820739,38.58289692)(200.01220739,38.52689692)(200.01220739,38.13489694)
\lineto(200.01220739,35.99089705)
\curveto(200.01220739,35.63089707)(199.92420739,35.63089707)(199.38820742,35.63089707)
\lineto(199.38820742,35.38289708)
\lineto(200.29220737,35.40689708)
\lineto(201.18820733,35.38289708)
\lineto(201.18820733,35.63089707)
\curveto(200.65220736,35.63089707)(200.56420736,35.63089707)(200.56420736,35.99089705)
\lineto(200.56420736,37.46289697)
\curveto(200.56420736,38.29489693)(201.13220733,38.74289691)(201.6442073,38.74289691)
\curveto(202.14820728,38.74289691)(202.23620727,38.31089693)(202.23620727,37.85489695)
\lineto(202.23620727,35.99089705)
\curveto(202.23620727,35.63089707)(202.14820728,35.63089707)(201.6122073,35.63089707)
\lineto(201.6122073,35.38289708)
\lineto(202.51620726,35.40689708)
\lineto(203.41220721,35.38289708)
\lineto(203.41220721,35.63089707)
\curveto(202.87620724,35.63089707)(202.78820724,35.63089707)(202.78820724,35.99089705)
\lineto(202.78820724,37.46289697)
\curveto(202.78820724,38.29489693)(203.35620721,38.74289691)(203.86820719,38.74289691)
\curveto(204.37220716,38.74289691)(204.46020715,38.31089693)(204.46020715,37.85489695)
\lineto(204.46020715,35.99089705)
\curveto(204.46020715,35.63089707)(204.37220716,35.63089707)(203.83620719,35.63089707)
\lineto(203.83620719,35.38289708)
\lineto(204.74020714,35.40689708)
\closepath
}
}
{
\newrgbcolor{curcolor}{0.65490198 0.66274512 0.67450982}
\pscustom[linestyle=none,fillstyle=solid,fillcolor=curcolor]
{
\newpath
\moveto(209.6601929,36.09489705)
\lineto(209.6601929,36.54289702)
\lineto(209.46019291,36.54289702)
\lineto(209.46019291,36.09489705)
\curveto(209.46019291,35.63089707)(209.26019292,35.58289707)(209.17219292,35.58289707)
\curveto(208.90819294,35.58289707)(208.87619294,35.94289705)(208.87619294,35.98289705)
\lineto(208.87619294,37.58289697)
\curveto(208.87619294,37.91889695)(208.87619294,38.23089693)(208.58819295,38.52689692)
\curveto(208.27619297,38.8388969)(207.87619299,38.9668969)(207.49219301,38.9668969)
\curveto(206.83619305,38.9668969)(206.28419307,38.59089692)(206.28419307,38.06289694)
\curveto(206.28419307,37.82289696)(206.44419307,37.68689696)(206.65219306,37.68689696)
\curveto(206.87619304,37.68689696)(207.02019304,37.84689695)(207.02019304,38.05489694)
\curveto(207.02019304,38.15089694)(206.98019304,38.41489692)(206.61219306,38.42289692)
\curveto(206.82819305,38.70289691)(207.22019303,38.79089691)(207.47619301,38.79089691)
\curveto(207.86819299,38.79089691)(208.32419297,38.47889692)(208.32419297,37.76689696)
\lineto(208.32419297,37.47089697)
\curveto(207.91619299,37.44689698)(207.35619302,37.42289698)(206.85219304,37.18289699)
\curveto(206.25219308,36.910897)(206.05219309,36.49489703)(206.05219309,36.14289704)
\curveto(206.05219309,35.49489708)(206.82819305,35.29489709)(207.33219302,35.29489709)
\curveto(207.86019299,35.29489709)(208.22819297,35.61489707)(208.38019296,35.99089705)
\curveto(208.41219296,35.67089707)(208.62819295,35.33489709)(209.00419293,35.33489709)
\curveto(209.17219292,35.33489709)(209.6601929,35.44689708)(209.6601929,36.09489705)
\closepath
\moveto(208.32419297,36.50289703)
\curveto(208.32419297,35.74289707)(207.748193,35.47089708)(207.38819302,35.47089708)
\curveto(206.99619304,35.47089708)(206.66819305,35.75089706)(206.66819305,36.15089704)
\curveto(206.66819305,36.59089702)(207.00419304,37.25489699)(208.32419297,37.30289698)
\closepath
}
}
{
\newrgbcolor{curcolor}{0.65490198 0.66274512 0.67450982}
\pscustom[linestyle=none,fillstyle=solid,fillcolor=curcolor]
{
\newpath
\moveto(211.83617791,35.38289708)
\lineto(211.83617791,35.63089707)
\curveto(211.30017794,35.63089707)(211.21217795,35.63089707)(211.21217795,35.99089705)
\lineto(211.21217795,40.93489679)
\lineto(210.06017801,40.8468968)
\lineto(210.06017801,40.59889681)
\curveto(210.62017798,40.59889681)(210.68417798,40.54289681)(210.68417798,40.15089683)
\lineto(210.68417798,35.99089705)
\curveto(210.68417798,35.63089707)(210.59617798,35.63089707)(210.06017801,35.63089707)
\lineto(210.06017801,35.38289708)
\lineto(210.94817796,35.40689708)
\closepath
}
}
{
\newrgbcolor{curcolor}{0.65490198 0.66274512 0.67450982}
\pscustom[linestyle=none,fillstyle=solid,fillcolor=curcolor]
{
\newpath
\moveto(214.06017586,35.38289708)
\lineto(214.06017586,35.63089707)
\curveto(213.52417588,35.63089707)(213.43617589,35.63089707)(213.43617589,35.99089705)
\lineto(213.43617589,40.93489679)
\lineto(212.28417595,40.8468968)
\lineto(212.28417595,40.59889681)
\curveto(212.84417592,40.59889681)(212.90817592,40.54289681)(212.90817592,40.15089683)
\lineto(212.90817592,35.99089705)
\curveto(212.90817592,35.63089707)(212.82017592,35.63089707)(212.28417595,35.63089707)
\lineto(212.28417595,35.38289708)
\lineto(213.1721759,35.40689708)
\closepath
}
}
{
\newrgbcolor{curcolor}{0.65490198 0.66274512 0.67450982}
\pscustom[linestyle=none,fillstyle=solid,fillcolor=curcolor]
{
\newpath
\moveto(220.45216185,36.942897)
\curveto(220.45216185,37.60689697)(220.09216187,38.01489695)(220.02016187,38.08689694)
\curveto(219.70016189,38.43889692)(219.4121619,38.51089692)(219.02816192,38.60689691)
\lineto(219.02816192,40.7668968)
\curveto(219.70816189,40.7348968)(220.09216187,40.38289682)(220.21216186,39.93489684)
\curveto(220.18016186,39.94289684)(220.16416186,39.95089684)(220.08416187,39.95089684)
\curveto(219.87616188,39.95089684)(219.71616189,39.80689685)(219.71616189,39.58289686)
\curveto(219.71616189,39.33489688)(219.91616188,39.21489688)(220.08416187,39.21489688)
\curveto(220.10816187,39.21489688)(220.45216185,39.22289688)(220.45216185,39.61489686)
\curveto(220.45216185,40.39089682)(219.92416188,40.97489679)(219.02816192,41.02289679)
\lineto(219.02816192,41.38289677)
\lineto(218.78016194,41.38289677)
\lineto(218.78016194,41.01489679)
\curveto(217.89216198,40.93489679)(217.35616201,40.21489683)(217.35616201,39.48689687)
\curveto(217.35616201,38.9348969)(217.652162,38.55089692)(217.79616199,38.41489692)
\curveto(218.10016197,38.10289694)(218.37216196,38.03889694)(218.78016194,37.93489695)
\lineto(218.78016194,35.55089708)
\curveto(218.07616197,35.59089707)(217.70016199,35.99889705)(217.596162,36.49489703)
\curveto(217.628162,36.48689703)(217.69216199,36.47889703)(217.72416199,36.47889703)
\curveto(217.94016198,36.47889703)(218.09216197,36.63089702)(218.09216197,36.84689701)
\curveto(218.09216197,37.070897)(217.92416198,37.21489699)(217.72416199,37.21489699)
\curveto(217.67616199,37.21489699)(217.35616201,37.19889699)(217.35616201,36.80689701)
\curveto(217.35616201,36.09489705)(217.78016199,35.35889709)(218.78016194,35.29489709)
\lineto(218.78016194,34.93489711)
\lineto(219.02816192,34.93489711)
\lineto(219.02816192,35.30289709)
\curveto(219.86816188,35.37489708)(220.45216185,36.11889705)(220.45216185,36.942897)
\closepath
\moveto(220.04416187,36.70289701)
\curveto(220.04416187,36.16689704)(219.66016189,35.63089707)(219.02816192,35.55089708)
\lineto(219.02816192,37.87089695)
\curveto(219.98816187,37.67089696)(220.04416187,36.894897)(220.04416187,36.70289701)
\closepath
\moveto(218.78016194,38.67089691)
\curveto(217.79616199,38.8788969)(217.76416199,39.56689686)(217.76416199,39.72689686)
\curveto(217.76416199,40.20689683)(218.14016197,40.7028968)(218.78016194,40.7668968)
\closepath
}
}
{
\newrgbcolor{curcolor}{0.65490198 0.66274512 0.67450982}
\pscustom[linestyle=none,fillstyle=solid,fillcolor=curcolor]
{
\newpath
\moveto(224.26014588,35.38289708)
\lineto(224.26014588,35.63089707)
\lineto(224.00414589,35.63089707)
\curveto(223.28414593,35.63089707)(223.26014593,35.71889707)(223.26014593,36.01489705)
\lineto(223.26014593,40.50289682)
\curveto(223.26014593,40.6948968)(223.26014593,40.7108968)(223.07614594,40.7108968)
\curveto(222.58014597,40.19889683)(221.876146,40.19889683)(221.62014602,40.19889683)
\lineto(221.62014602,39.95089684)
\curveto(221.78014601,39.95089684)(222.25214598,39.95089684)(222.66814596,40.15889683)
\lineto(222.66814596,36.01489705)
\curveto(222.66814596,35.72689707)(222.64414596,35.63089707)(221.924146,35.63089707)
\lineto(221.66814601,35.63089707)
\lineto(221.66814601,35.38289708)
\curveto(221.948146,35.40689708)(222.64414596,35.40689708)(222.96414595,35.40689708)
\curveto(223.28414593,35.40689708)(223.98014589,35.40689708)(224.26014588,35.38289708)
\closepath
}
}
{
\newrgbcolor{curcolor}{0.65490198 0.66274512 0.67450982}
\pscustom[linestyle=none,fillstyle=solid,fillcolor=curcolor]
{
\newpath
\moveto(228.56412988,36.75089701)
\curveto(228.56412988,37.40689698)(228.06012991,38.03089694)(227.22812995,38.19889694)
\curveto(227.88412992,38.41489692)(228.34812989,38.9748969)(228.34812989,39.60689686)
\curveto(228.34812989,40.26289683)(227.64412993,40.7108968)(226.87612997,40.7108968)
\curveto(226.06813001,40.7108968)(225.46013005,40.23089683)(225.46013005,39.62289686)
\curveto(225.46013005,39.35889688)(225.63613004,39.20689688)(225.86813002,39.20689688)
\curveto(226.11613001,39.20689688)(226.27613,39.38289687)(226.27613,39.61489686)
\curveto(226.27613,40.01489684)(225.90013002,40.01489684)(225.78013003,40.01489684)
\curveto(226.02813002,40.40689682)(226.55612999,40.51089681)(226.84412997,40.51089681)
\curveto(227.17212996,40.51089681)(227.61212993,40.33489682)(227.61212993,39.61489686)
\curveto(227.61212993,39.51889687)(227.59612993,39.05489689)(227.38812994,38.70289691)
\curveto(227.14812996,38.31889693)(226.87612997,38.29489693)(226.67612998,38.28689693)
\curveto(226.61212999,38.27889693)(226.42013,38.26289693)(226.36413,38.26289693)
\curveto(226.30013,38.25489693)(226.24413,38.24689693)(226.24413,38.16689694)
\curveto(226.24413,38.07889694)(226.30013,38.07889694)(226.43612999,38.07889694)
\lineto(226.78812998,38.07889694)
\curveto(227.44412994,38.07889694)(227.74012993,37.53489697)(227.74012993,36.75089701)
\curveto(227.74012993,35.66289707)(227.18812996,35.43089708)(226.83612997,35.43089708)
\curveto(226.49212999,35.43089708)(225.89213002,35.56689707)(225.61213004,36.03889705)
\curveto(225.89213002,35.99889705)(226.14013001,36.17489704)(226.14013001,36.47889703)
\curveto(226.14013001,36.76689701)(225.92413002,36.926897)(225.69213003,36.926897)
\curveto(225.50013004,36.926897)(225.24413006,36.81489701)(225.24413006,36.46289703)
\curveto(225.24413006,35.73489707)(225.98813002,35.20689709)(226.86012997,35.20689709)
\curveto(227.83612992,35.20689709)(228.56412988,35.93489706)(228.56412988,36.75089701)
\closepath
}
}
{
\newrgbcolor{curcolor}{0.65490198 0.66274512 0.67450982}
\pscustom[linestyle=none,fillstyle=solid,fillcolor=curcolor]
{
\newpath
\moveto(232.45211571,36.942897)
\curveto(232.45211571,37.60689697)(232.09211573,38.01489695)(232.02011573,38.08689694)
\curveto(231.70011575,38.43889692)(231.41211577,38.51089692)(231.02811579,38.60689691)
\lineto(231.02811579,40.7668968)
\curveto(231.70811575,40.7348968)(232.09211573,40.38289682)(232.21211572,39.93489684)
\curveto(232.18011573,39.94289684)(232.16411573,39.95089684)(232.08411573,39.95089684)
\curveto(231.87611574,39.95089684)(231.71611575,39.80689685)(231.71611575,39.58289686)
\curveto(231.71611575,39.33489688)(231.91611574,39.21489688)(232.08411573,39.21489688)
\curveto(232.10811573,39.21489688)(232.45211571,39.22289688)(232.45211571,39.61489686)
\curveto(232.45211571,40.39089682)(231.92411574,40.97489679)(231.02811579,41.02289679)
\lineto(231.02811579,41.38289677)
\lineto(230.7801158,41.38289677)
\lineto(230.7801158,41.01489679)
\curveto(229.89211585,40.93489679)(229.35611587,40.21489683)(229.35611587,39.48689687)
\curveto(229.35611587,38.9348969)(229.65211586,38.55089692)(229.79611585,38.41489692)
\curveto(230.10011583,38.10289694)(230.37211582,38.03889694)(230.7801158,37.93489695)
\lineto(230.7801158,35.55089708)
\curveto(230.07611584,35.59089707)(229.70011586,35.99889705)(229.59611586,36.49489703)
\curveto(229.62811586,36.48689703)(229.69211586,36.47889703)(229.72411585,36.47889703)
\curveto(229.94011584,36.47889703)(230.09211584,36.63089702)(230.09211584,36.84689701)
\curveto(230.09211584,37.070897)(229.92411584,37.21489699)(229.72411585,37.21489699)
\curveto(229.67611586,37.21489699)(229.35611587,37.19889699)(229.35611587,36.80689701)
\curveto(229.35611587,36.09489705)(229.78011585,35.35889709)(230.7801158,35.29489709)
\lineto(230.7801158,34.93489711)
\lineto(231.02811579,34.93489711)
\lineto(231.02811579,35.30289709)
\curveto(231.86811574,35.37489708)(232.45211571,36.11889705)(232.45211571,36.942897)
\closepath
\moveto(232.04411573,36.70289701)
\curveto(232.04411573,36.16689704)(231.66011575,35.63089707)(231.02811579,35.55089708)
\lineto(231.02811579,37.87089695)
\curveto(231.98811574,37.67089696)(232.04411573,36.894897)(232.04411573,36.70289701)
\closepath
\moveto(230.7801158,38.67089691)
\curveto(229.79611585,38.8788969)(229.76411585,39.56689686)(229.76411585,39.72689686)
\curveto(229.76411585,40.20689683)(230.14011583,40.7028968)(230.7801158,40.7668968)
\closepath
}
}
{
\newrgbcolor{curcolor}{0 0 0}
\pscustom[linestyle=none,fillstyle=solid,fillcolor=curcolor]
{
\newpath
\moveto(246.05334246,16.60009988)
\curveto(246.05334246,17.70676652)(245.45334247,18.3867665)(245.33334248,18.50676649)
\curveto(244.80000916,19.09343314)(244.32000917,19.21343314)(243.68000919,19.37343314)
\lineto(243.68000919,22.97343303)
\curveto(244.81334249,22.9200997)(245.45334247,22.33343305)(245.65334247,21.5867664)
\curveto(245.60000914,21.60009973)(245.57334247,21.61343307)(245.44000914,21.61343307)
\curveto(245.09334248,21.61343307)(244.82667583,21.37343307)(244.82667583,21.00009975)
\curveto(244.82667583,20.58676643)(245.16000915,20.38676644)(245.44000914,20.38676644)
\curveto(245.48000914,20.38676644)(246.05334246,20.40009977)(246.05334246,21.05343308)
\curveto(246.05334246,22.34676638)(245.17334248,23.32009968)(243.68000919,23.40009968)
\lineto(243.68000919,24.00009966)
\lineto(243.26667587,24.00009966)
\lineto(243.26667587,23.38676635)
\curveto(241.78667592,23.25343302)(240.89334261,22.05343305)(240.89334261,20.84009976)
\curveto(240.89334261,19.92009979)(241.38667593,19.2800998)(241.62667592,19.05343314)
\curveto(242.13334257,18.53343316)(242.58667589,18.4267665)(243.26667587,18.25343317)
\lineto(243.26667587,14.28009995)
\curveto(242.09334257,14.34676662)(241.46667593,15.0267666)(241.2933426,15.85343324)
\curveto(241.34667593,15.84009991)(241.45334259,15.82676657)(241.50667593,15.82676657)
\curveto(241.86667591,15.82676657)(242.12000924,16.0800999)(242.12000924,16.44009989)
\curveto(242.12000924,16.81343321)(241.84000925,17.0534332)(241.50667593,17.0534332)
\curveto(241.42667593,17.0534332)(240.89334261,17.02676654)(240.89334261,16.37343323)
\curveto(240.89334261,15.18676659)(241.60000926,13.96009996)(243.26667587,13.8534333)
\lineto(243.26667587,13.25343332)
\lineto(243.68000919,13.25343332)
\lineto(243.68000919,13.86676663)
\curveto(245.08000915,13.98676663)(246.05334246,15.22676659)(246.05334246,16.60009988)
\closepath
\moveto(245.37334248,16.2000999)
\curveto(245.37334248,15.30676659)(244.7333425,14.41343328)(243.68000919,14.28009995)
\lineto(243.68000919,18.14676651)
\curveto(245.28000915,17.81343318)(245.37334248,16.52009989)(245.37334248,16.2000999)
\closepath
\moveto(243.26667587,19.4800998)
\curveto(241.62667592,19.82676645)(241.57334259,20.97343309)(241.57334259,21.24009975)
\curveto(241.57334259,22.04009972)(242.20000924,22.86676636)(243.26667587,22.97343303)
\closepath
}
}
{
\newrgbcolor{curcolor}{0 0 0}
\pscustom[linestyle=none,fillstyle=solid,fillcolor=curcolor]
{
\newpath
\moveto(256.42665025,17.3334332)
\curveto(256.42665025,17.60009985)(256.17331692,17.60009985)(255.98665026,17.60009985)
\lineto(252.26665037,17.60009985)
\lineto(252.26665037,21.33343308)
\curveto(252.26665037,21.52009974)(252.26665037,21.77343306)(251.99998371,21.77343306)
\curveto(251.73331705,21.77343306)(251.73331705,21.52009974)(251.73331705,21.33343308)
\lineto(251.73331705,17.60009985)
\lineto(247.99998383,17.60009985)
\curveto(247.81331717,17.60009985)(247.55998385,17.60009985)(247.55998385,17.3334332)
\curveto(247.55998385,17.06676654)(247.81331717,17.06676654)(247.99998383,17.06676654)
\lineto(251.73331705,17.06676654)
\lineto(251.73331705,13.33343332)
\curveto(251.73331705,13.14676666)(251.73331705,12.89343333)(251.99998371,12.89343333)
\curveto(252.26665037,12.89343333)(252.26665037,13.14676666)(252.26665037,13.33343332)
\lineto(252.26665037,17.06676654)
\lineto(255.98665026,17.06676654)
\curveto(256.17331692,17.06676654)(256.42665025,17.06676654)(256.42665025,17.3334332)
\closepath
}
}
{
\newrgbcolor{curcolor}{0 0 0}
\pscustom[linestyle=none,fillstyle=solid,fillcolor=curcolor]
{
\newpath
\moveto(263.093314,16.60009988)
\curveto(263.093314,17.70676652)(262.49331401,18.3867665)(262.37331402,18.50676649)
\curveto(261.8399807,19.09343314)(261.35998071,19.21343314)(260.71998073,19.37343314)
\lineto(260.71998073,22.97343303)
\curveto(261.85331403,22.9200997)(262.49331401,22.33343305)(262.69331401,21.5867664)
\curveto(262.63998068,21.60009973)(262.61331401,21.61343307)(262.47998068,21.61343307)
\curveto(262.13331402,21.61343307)(261.86664737,21.37343307)(261.86664737,21.00009975)
\curveto(261.86664737,20.58676643)(262.19998069,20.38676644)(262.47998068,20.38676644)
\curveto(262.51998068,20.38676644)(263.093314,20.40009977)(263.093314,21.05343308)
\curveto(263.093314,22.34676638)(262.21331402,23.32009968)(260.71998073,23.40009968)
\lineto(260.71998073,24.00009966)
\lineto(260.30664741,24.00009966)
\lineto(260.30664741,23.38676635)
\curveto(258.82664746,23.25343302)(257.93331415,22.05343305)(257.93331415,20.84009976)
\curveto(257.93331415,19.92009979)(258.42664747,19.2800998)(258.66664746,19.05343314)
\curveto(259.17331411,18.53343316)(259.62664743,18.4267665)(260.30664741,18.25343317)
\lineto(260.30664741,14.28009995)
\curveto(259.13331411,14.34676662)(258.50664747,15.0267666)(258.33331414,15.85343324)
\curveto(258.38664747,15.84009991)(258.49331413,15.82676657)(258.54664747,15.82676657)
\curveto(258.90664745,15.82676657)(259.15998078,16.0800999)(259.15998078,16.44009989)
\curveto(259.15998078,16.81343321)(258.87998079,17.0534332)(258.54664747,17.0534332)
\curveto(258.46664747,17.0534332)(257.93331415,17.02676654)(257.93331415,16.37343323)
\curveto(257.93331415,15.18676659)(258.6399808,13.96009996)(260.30664741,13.8534333)
\lineto(260.30664741,13.25343332)
\lineto(260.71998073,13.25343332)
\lineto(260.71998073,13.86676663)
\curveto(262.11998069,13.98676663)(263.093314,15.22676659)(263.093314,16.60009988)
\closepath
\moveto(262.41331402,16.2000999)
\curveto(262.41331402,15.30676659)(261.77331403,14.41343328)(260.71998073,14.28009995)
\lineto(260.71998073,18.14676651)
\curveto(262.31998069,17.81343318)(262.41331402,16.52009989)(262.41331402,16.2000999)
\closepath
\moveto(260.30664741,19.4800998)
\curveto(258.66664746,19.82676645)(258.61331413,20.97343309)(258.61331413,21.24009975)
\curveto(258.61331413,22.04009972)(259.23998078,22.86676636)(260.30664741,22.97343303)
\closepath
}
}
{
\newrgbcolor{curcolor}{0 0 0}
\pscustom[linewidth=0.99999995,linecolor=curcolor]
{
\newpath
\moveto(251.99216126,47.73085984)
\lineto(251.99216126,27.38282835)
}
}
{
\newrgbcolor{curcolor}{0 0 0}
\pscustom[linestyle=none,fillstyle=solid,fillcolor=curcolor]
{
\newpath
\moveto(251.99216126,31.38282814)
\lineto(249.99216136,33.38282803)
\lineto(251.99216126,26.3828284)
\lineto(253.99216115,33.38282803)
\closepath
}
}
{
\newrgbcolor{curcolor}{0 0 0}
\pscustom[linewidth=0.53333332,linecolor=curcolor]
{
\newpath
\moveto(251.99216126,31.38282814)
\lineto(249.99216136,33.38282803)
\lineto(251.99216126,26.3828284)
\lineto(253.99216115,33.38282803)
\closepath
}
}
{
\newrgbcolor{curcolor}{0.65490198 0.66274512 0.67450982}
\pscustom[linestyle=none,fillstyle=solid,fillcolor=curcolor]
{
\newpath
\moveto(265.5242398,33.54289718)
\curveto(265.5242398,33.54289718)(265.5242398,33.58289718)(265.4842398,33.68689717)
\lineto(262.76423995,41.20689678)
\curveto(262.73223995,41.29489677)(262.70023995,41.38289677)(262.58823996,41.38289677)
\curveto(262.50023996,41.38289677)(262.42823996,41.31089677)(262.42823996,41.22289678)
\curveto(262.42823996,41.22289678)(262.42823996,41.18289678)(262.46823996,41.07889678)
\lineto(265.18823982,33.55889718)
\curveto(265.22023982,33.47089718)(265.25223982,33.38289719)(265.36423981,33.38289719)
\curveto(265.45223981,33.38289719)(265.5242398,33.45489719)(265.5242398,33.54289718)
\closepath
}
}
{
\newrgbcolor{curcolor}{0.65490198 0.66274512 0.67450982}
\pscustom[linestyle=none,fillstyle=solid,fillcolor=curcolor]
{
\newpath
\moveto(268.86022431,36.40689703)
\curveto(268.86022431,36.83089701)(268.62022432,37.070897)(268.52422433,37.16689699)
\curveto(268.26022434,37.42289698)(267.94822436,37.48689697)(267.61222437,37.55089697)
\curveto(267.1642244,37.63889697)(266.62822443,37.74289696)(266.62822443,38.20689694)
\curveto(266.62822443,38.48689692)(266.83622441,38.8148969)(267.52422438,38.8148969)
\curveto(268.40422433,38.8148969)(268.44422433,38.09489694)(268.46022433,37.84689695)
\curveto(268.46822433,37.77489696)(268.55622432,37.77489696)(268.55622432,37.77489696)
\curveto(268.66022432,37.77489696)(268.66022432,37.81489696)(268.66022432,37.96689695)
\lineto(268.66022432,38.77489691)
\curveto(268.66022432,38.9108969)(268.66022432,38.9668969)(268.57222432,38.9668969)
\curveto(268.53222433,38.9668969)(268.51622433,38.9668969)(268.41222433,38.8708969)
\curveto(268.38822433,38.8388969)(268.30822434,38.76689691)(268.27622434,38.74289691)
\curveto(267.97222435,38.9668969)(267.64422437,38.9668969)(267.52422438,38.9668969)
\curveto(266.54822443,38.9668969)(266.24422445,38.43089692)(266.24422445,37.98289695)
\curveto(266.24422445,37.70289696)(266.37222444,37.47889697)(266.58822443,37.30289698)
\curveto(266.84422441,37.09489699)(267.0682244,37.046897)(267.64422437,36.934897)
\curveto(267.82022436,36.902897)(268.47622433,36.77489701)(268.47622433,36.19889704)
\curveto(268.47622433,35.79089706)(268.19622434,35.47089708)(267.57222438,35.47089708)
\curveto(266.90022441,35.47089708)(266.61222443,35.92689706)(266.46022443,36.60689702)
\curveto(266.43622444,36.71089701)(266.42822444,36.74289701)(266.34822444,36.74289701)
\curveto(266.24422445,36.74289701)(266.24422445,36.68689702)(266.24422445,36.54289702)
\lineto(266.24422445,35.48689708)
\curveto(266.24422445,35.35089709)(266.24422445,35.29489709)(266.33222444,35.29489709)
\curveto(266.37222444,35.29489709)(266.38022444,35.30289709)(266.53222443,35.45489708)
\curveto(266.54822443,35.47089708)(266.54822443,35.48689708)(266.69222442,35.63889707)
\curveto(267.0442244,35.30289709)(267.40422438,35.29489709)(267.57222438,35.29489709)
\curveto(268.49222433,35.29489709)(268.86022431,35.83089706)(268.86022431,36.40689703)
\closepath
}
}
{
\newrgbcolor{curcolor}{0.65490198 0.66274512 0.67450982}
\pscustom[linestyle=none,fillstyle=solid,fillcolor=curcolor]
{
\newpath
\moveto(275.6362167,35.38289708)
\lineto(275.6362167,35.63089707)
\curveto(275.22021673,35.63089707)(275.02021674,35.63089707)(275.01221674,35.87089706)
\lineto(275.01221674,37.39889698)
\curveto(275.01221674,38.08689694)(275.01221674,38.33489693)(274.76421675,38.62289691)
\curveto(274.65221676,38.75889691)(274.38821677,38.9188969)(273.92421679,38.9188969)
\curveto(273.25221683,38.9188969)(272.90021685,38.43889692)(272.76421686,38.13489694)
\curveto(272.65221686,38.8308969)(272.06021689,38.9188969)(271.70021691,38.9188969)
\curveto(271.11621694,38.9188969)(270.74021696,38.57489692)(270.51621697,38.07889694)
\lineto(270.51621697,38.9188969)
\lineto(269.38821703,38.8308969)
\lineto(269.38821703,38.58289692)
\curveto(269.948217,38.58289692)(270.012217,38.52689692)(270.012217,38.13489694)
\lineto(270.012217,35.99089705)
\curveto(270.012217,35.63089707)(269.92421701,35.63089707)(269.38821703,35.63089707)
\lineto(269.38821703,35.38289708)
\lineto(270.29221699,35.40689708)
\lineto(271.18821694,35.38289708)
\lineto(271.18821694,35.63089707)
\curveto(270.65221697,35.63089707)(270.56421697,35.63089707)(270.56421697,35.99089705)
\lineto(270.56421697,37.46289697)
\curveto(270.56421697,38.29489693)(271.13221694,38.74289691)(271.64421691,38.74289691)
\curveto(272.14821689,38.74289691)(272.23621688,38.31089693)(272.23621688,37.85489695)
\lineto(272.23621688,35.99089705)
\curveto(272.23621688,35.63089707)(272.14821689,35.63089707)(271.61221692,35.63089707)
\lineto(271.61221692,35.38289708)
\lineto(272.51621687,35.40689708)
\lineto(273.41221682,35.38289708)
\lineto(273.41221682,35.63089707)
\curveto(272.87621685,35.63089707)(272.78821685,35.63089707)(272.78821685,35.99089705)
\lineto(272.78821685,37.46289697)
\curveto(272.78821685,38.29489693)(273.35621682,38.74289691)(273.8682168,38.74289691)
\curveto(274.37221677,38.74289691)(274.46021677,38.31089693)(274.46021677,37.85489695)
\lineto(274.46021677,35.99089705)
\curveto(274.46021677,35.63089707)(274.37221677,35.63089707)(273.8362168,35.63089707)
\lineto(273.8362168,35.38289708)
\lineto(274.74021675,35.40689708)
\closepath
}
}
{
\newrgbcolor{curcolor}{0.65490198 0.66274512 0.67450982}
\pscustom[linestyle=none,fillstyle=solid,fillcolor=curcolor]
{
\newpath
\moveto(279.66020251,36.09489705)
\lineto(279.66020251,36.54289702)
\lineto(279.46020252,36.54289702)
\lineto(279.46020252,36.09489705)
\curveto(279.46020252,35.63089707)(279.26020253,35.58289707)(279.17220253,35.58289707)
\curveto(278.90820255,35.58289707)(278.87620255,35.94289705)(278.87620255,35.98289705)
\lineto(278.87620255,37.58289697)
\curveto(278.87620255,37.91889695)(278.87620255,38.23089693)(278.58820257,38.52689692)
\curveto(278.27620258,38.8388969)(277.8762026,38.9668969)(277.49220262,38.9668969)
\curveto(276.83620266,38.9668969)(276.28420269,38.59089692)(276.28420269,38.06289694)
\curveto(276.28420269,37.82289696)(276.44420268,37.68689696)(276.65220267,37.68689696)
\curveto(276.87620266,37.68689696)(277.02020265,37.84689695)(277.02020265,38.05489694)
\curveto(277.02020265,38.15089694)(276.98020265,38.41489692)(276.61220267,38.42289692)
\curveto(276.82820266,38.70289691)(277.22020264,38.79089691)(277.47620262,38.79089691)
\curveto(277.8682026,38.79089691)(278.32420258,38.47889692)(278.32420258,37.76689696)
\lineto(278.32420258,37.47089697)
\curveto(277.9162026,37.44689698)(277.35620263,37.42289698)(276.85220266,37.18289699)
\curveto(276.25220269,36.910897)(276.0522027,36.49489703)(276.0522027,36.14289704)
\curveto(276.0522027,35.49489708)(276.82820266,35.29489709)(277.33220263,35.29489709)
\curveto(277.8602026,35.29489709)(278.22820258,35.61489707)(278.38020258,35.99089705)
\curveto(278.41220257,35.67089707)(278.62820256,35.33489709)(279.00420254,35.33489709)
\curveto(279.17220253,35.33489709)(279.66020251,35.44689708)(279.66020251,36.09489705)
\closepath
\moveto(278.32420258,36.50289703)
\curveto(278.32420258,35.74289707)(277.74820261,35.47089708)(277.38820263,35.47089708)
\curveto(276.99620265,35.47089708)(276.66820267,35.75089706)(276.66820267,36.15089704)
\curveto(276.66820267,36.59089702)(277.00420265,37.25489699)(278.32420258,37.30289698)
\closepath
}
}
{
\newrgbcolor{curcolor}{0.65490198 0.66274512 0.67450982}
\pscustom[linestyle=none,fillstyle=solid,fillcolor=curcolor]
{
\newpath
\moveto(281.83618753,35.38289708)
\lineto(281.83618753,35.63089707)
\curveto(281.30018755,35.63089707)(281.21218756,35.63089707)(281.21218756,35.99089705)
\lineto(281.21218756,40.93489679)
\lineto(280.06018762,40.8468968)
\lineto(280.06018762,40.59889681)
\curveto(280.62018759,40.59889681)(280.68418759,40.54289681)(280.68418759,40.15089683)
\lineto(280.68418759,35.99089705)
\curveto(280.68418759,35.63089707)(280.59618759,35.63089707)(280.06018762,35.63089707)
\lineto(280.06018762,35.38289708)
\lineto(280.94818757,35.40689708)
\closepath
}
}
{
\newrgbcolor{curcolor}{0.65490198 0.66274512 0.67450982}
\pscustom[linestyle=none,fillstyle=solid,fillcolor=curcolor]
{
\newpath
\moveto(284.06018547,35.38289708)
\lineto(284.06018547,35.63089707)
\curveto(283.5241855,35.63089707)(283.4361855,35.63089707)(283.4361855,35.99089705)
\lineto(283.4361855,40.93489679)
\lineto(282.28418556,40.8468968)
\lineto(282.28418556,40.59889681)
\curveto(282.84418553,40.59889681)(282.90818553,40.54289681)(282.90818553,40.15089683)
\lineto(282.90818553,35.99089705)
\curveto(282.90818553,35.63089707)(282.82018553,35.63089707)(282.28418556,35.63089707)
\lineto(282.28418556,35.38289708)
\lineto(283.17218552,35.40689708)
\closepath
}
}
{
\newrgbcolor{curcolor}{0.65490198 0.66274512 0.67450982}
\pscustom[linestyle=none,fillstyle=solid,fillcolor=curcolor]
{
\newpath
\moveto(290.45217146,36.942897)
\curveto(290.45217146,37.60689697)(290.09217148,38.01489695)(290.02017148,38.08689694)
\curveto(289.7001715,38.43889692)(289.41217151,38.51089692)(289.02817153,38.60689691)
\lineto(289.02817153,40.7668968)
\curveto(289.7081715,40.7348968)(290.09217148,40.38289682)(290.21217147,39.93489684)
\curveto(290.18017147,39.94289684)(290.16417148,39.95089684)(290.08417148,39.95089684)
\curveto(289.87617149,39.95089684)(289.7161715,39.80689685)(289.7161715,39.58289686)
\curveto(289.7161715,39.33489688)(289.91617149,39.21489688)(290.08417148,39.21489688)
\curveto(290.10817148,39.21489688)(290.45217146,39.22289688)(290.45217146,39.61489686)
\curveto(290.45217146,40.39089682)(289.92417149,40.97489679)(289.02817153,41.02289679)
\lineto(289.02817153,41.38289677)
\lineto(288.78017155,41.38289677)
\lineto(288.78017155,41.01489679)
\curveto(287.89217159,40.93489679)(287.35617162,40.21489683)(287.35617162,39.48689687)
\curveto(287.35617162,38.9348969)(287.65217161,38.55089692)(287.7961716,38.41489692)
\curveto(288.10017158,38.10289694)(288.37217157,38.03889694)(288.78017155,37.93489695)
\lineto(288.78017155,35.55089708)
\curveto(288.07617158,35.59089707)(287.7001716,35.99889705)(287.59617161,36.49489703)
\curveto(287.62817161,36.48689703)(287.69217161,36.47889703)(287.7241716,36.47889703)
\curveto(287.94017159,36.47889703)(288.09217158,36.63089702)(288.09217158,36.84689701)
\curveto(288.09217158,37.070897)(287.92417159,37.21489699)(287.7241716,37.21489699)
\curveto(287.67617161,37.21489699)(287.35617162,37.19889699)(287.35617162,36.80689701)
\curveto(287.35617162,36.09489705)(287.7801716,35.35889709)(288.78017155,35.29489709)
\lineto(288.78017155,34.93489711)
\lineto(289.02817153,34.93489711)
\lineto(289.02817153,35.30289709)
\curveto(289.86817149,35.37489708)(290.45217146,36.11889705)(290.45217146,36.942897)
\closepath
\moveto(290.04417148,36.70289701)
\curveto(290.04417148,36.16689704)(289.6601715,35.63089707)(289.02817153,35.55089708)
\lineto(289.02817153,37.87089695)
\curveto(289.98817148,37.67089696)(290.04417148,36.894897)(290.04417148,36.70289701)
\closepath
\moveto(288.78017155,38.67089691)
\curveto(287.7961716,38.8788969)(287.7641716,39.56689686)(287.7641716,39.72689686)
\curveto(287.7641716,40.20689683)(288.14017158,40.7028968)(288.78017155,40.7668968)
\closepath
}
}
{
\newrgbcolor{curcolor}{0.65490198 0.66274512 0.67450982}
\pscustom[linestyle=none,fillstyle=solid,fillcolor=curcolor]
{
\newpath
\moveto(294.26015549,35.38289708)
\lineto(294.26015549,35.63089707)
\lineto(294.0041555,35.63089707)
\curveto(293.28415554,35.63089707)(293.26015554,35.71889707)(293.26015554,36.01489705)
\lineto(293.26015554,40.50289682)
\curveto(293.26015554,40.6948968)(293.26015554,40.7108968)(293.07615555,40.7108968)
\curveto(292.58015558,40.19889683)(291.87615562,40.19889683)(291.62015563,40.19889683)
\lineto(291.62015563,39.95089684)
\curveto(291.78015562,39.95089684)(292.2521556,39.95089684)(292.66815557,40.15889683)
\lineto(292.66815557,36.01489705)
\curveto(292.66815557,35.72689707)(292.64415558,35.63089707)(291.92415561,35.63089707)
\lineto(291.66815563,35.63089707)
\lineto(291.66815563,35.38289708)
\curveto(291.94815561,35.40689708)(292.64415558,35.40689708)(292.96415556,35.40689708)
\curveto(293.28415554,35.40689708)(293.98015551,35.40689708)(294.26015549,35.38289708)
\closepath
}
}
{
\newrgbcolor{curcolor}{0.65490198 0.66274512 0.67450982}
\pscustom[linestyle=none,fillstyle=solid,fillcolor=curcolor]
{
\newpath
\moveto(298.56413949,36.75089701)
\curveto(298.56413949,37.40689698)(298.06013952,38.03089694)(297.22813956,38.19889694)
\curveto(297.88413953,38.41489692)(298.34813951,38.9748969)(298.34813951,39.60689686)
\curveto(298.34813951,40.26289683)(297.64413954,40.7108968)(296.87613958,40.7108968)
\curveto(296.06813963,40.7108968)(295.46013966,40.23089683)(295.46013966,39.62289686)
\curveto(295.46013966,39.35889688)(295.63613965,39.20689688)(295.86813964,39.20689688)
\curveto(296.11613962,39.20689688)(296.27613961,39.38289687)(296.27613961,39.61489686)
\curveto(296.27613961,40.01489684)(295.90013963,40.01489684)(295.78013964,40.01489684)
\curveto(296.02813963,40.40689682)(296.5561396,40.51089681)(296.84413959,40.51089681)
\curveto(297.17213957,40.51089681)(297.61213954,40.33489682)(297.61213954,39.61489686)
\curveto(297.61213954,39.51889687)(297.59613955,39.05489689)(297.38813956,38.70289691)
\curveto(297.14813957,38.31889693)(296.87613958,38.29489693)(296.67613959,38.28689693)
\curveto(296.6121396,38.27889693)(296.42013961,38.26289693)(296.36413961,38.26289693)
\curveto(296.30013961,38.25489693)(296.24413962,38.24689693)(296.24413962,38.16689694)
\curveto(296.24413962,38.07889694)(296.30013961,38.07889694)(296.43613961,38.07889694)
\lineto(296.78813959,38.07889694)
\curveto(297.44413955,38.07889694)(297.74013954,37.53489697)(297.74013954,36.75089701)
\curveto(297.74013954,35.66289707)(297.18813957,35.43089708)(296.83613959,35.43089708)
\curveto(296.4921396,35.43089708)(295.89213964,35.56689707)(295.61213965,36.03889705)
\curveto(295.89213964,35.99889705)(296.14013962,36.17489704)(296.14013962,36.47889703)
\curveto(296.14013962,36.76689701)(295.92413963,36.926897)(295.69213965,36.926897)
\curveto(295.50013966,36.926897)(295.24413967,36.81489701)(295.24413967,36.46289703)
\curveto(295.24413967,35.73489707)(295.98813963,35.20689709)(296.86013958,35.20689709)
\curveto(297.83613953,35.20689709)(298.56413949,35.93489706)(298.56413949,36.75089701)
\closepath
}
}
{
\newrgbcolor{curcolor}{0.65490198 0.66274512 0.67450982}
\pscustom[linestyle=none,fillstyle=solid,fillcolor=curcolor]
{
\newpath
\moveto(302.45212532,36.942897)
\curveto(302.45212532,37.60689697)(302.09212534,38.01489695)(302.02012535,38.08689694)
\curveto(301.70012536,38.43889692)(301.41212538,38.51089692)(301.0281254,38.60689691)
\lineto(301.0281254,40.7668968)
\curveto(301.70812536,40.7348968)(302.09212534,40.38289682)(302.21212534,39.93489684)
\curveto(302.18012534,39.94289684)(302.16412534,39.95089684)(302.08412534,39.95089684)
\curveto(301.87612535,39.95089684)(301.71612536,39.80689685)(301.71612536,39.58289686)
\curveto(301.71612536,39.33489688)(301.91612535,39.21489688)(302.08412534,39.21489688)
\curveto(302.10812534,39.21489688)(302.45212532,39.22289688)(302.45212532,39.61489686)
\curveto(302.45212532,40.39089682)(301.92412535,40.97489679)(301.0281254,41.02289679)
\lineto(301.0281254,41.38289677)
\lineto(300.78012541,41.38289677)
\lineto(300.78012541,41.01489679)
\curveto(299.89212546,40.93489679)(299.35612549,40.21489683)(299.35612549,39.48689687)
\curveto(299.35612549,38.9348969)(299.65212547,38.55089692)(299.79612546,38.41489692)
\curveto(300.10012545,38.10289694)(300.37212543,38.03889694)(300.78012541,37.93489695)
\lineto(300.78012541,35.55089708)
\curveto(300.07612545,35.59089707)(299.70012547,35.99889705)(299.59612547,36.49489703)
\curveto(299.62812547,36.48689703)(299.69212547,36.47889703)(299.72412547,36.47889703)
\curveto(299.94012546,36.47889703)(300.09212545,36.63089702)(300.09212545,36.84689701)
\curveto(300.09212545,37.070897)(299.92412546,37.21489699)(299.72412547,37.21489699)
\curveto(299.67612547,37.21489699)(299.35612549,37.19889699)(299.35612549,36.80689701)
\curveto(299.35612549,36.09489705)(299.78012546,35.35889709)(300.78012541,35.29489709)
\lineto(300.78012541,34.93489711)
\lineto(301.0281254,34.93489711)
\lineto(301.0281254,35.30289709)
\curveto(301.86812535,35.37489708)(302.45212532,36.11889705)(302.45212532,36.942897)
\closepath
\moveto(302.04412534,36.70289701)
\curveto(302.04412534,36.16689704)(301.66012536,35.63089707)(301.0281254,35.55089708)
\lineto(301.0281254,37.87089695)
\curveto(301.98812535,37.67089696)(302.04412534,36.894897)(302.04412534,36.70289701)
\closepath
\moveto(300.78012541,38.67089691)
\curveto(299.79612546,38.8788969)(299.76412546,39.56689686)(299.76412546,39.72689686)
\curveto(299.76412546,40.20689683)(300.14012544,40.7028968)(300.78012541,40.7668968)
\closepath
}
}
{
\newrgbcolor{curcolor}{0 0 0}
\pscustom[linewidth=0.99999995,linecolor=curcolor]
{
\newpath
\moveto(121.99998236,17.38284094)
\lineto(171.99997984,17.38284094)
}
}
{
\newrgbcolor{curcolor}{0 0 0}
\pscustom[linestyle=none,fillstyle=solid,fillcolor=curcolor]
{
\newpath
\moveto(167.99998005,17.38284094)
\lineto(165.99998016,15.38284105)
\lineto(172.99997979,17.38284094)
\lineto(165.99998016,19.38284084)
\closepath
}
}
{
\newrgbcolor{curcolor}{0 0 0}
\pscustom[linewidth=0.53333332,linecolor=curcolor]
{
\newpath
\moveto(167.99998005,17.38284094)
\lineto(165.99998016,15.38284105)
\lineto(172.99997979,17.38284094)
\lineto(165.99998016,19.38284084)
\closepath
}
}
{
\newrgbcolor{curcolor}{0 0 0}
\pscustom[linewidth=0.99999995,linecolor=curcolor]
{
\newpath
\moveto(191.99998488,17.38284094)
\lineto(241.99998236,17.38284094)
}
}
{
\newrgbcolor{curcolor}{0 0 0}
\pscustom[linestyle=none,fillstyle=solid,fillcolor=curcolor]
{
\newpath
\moveto(237.99998257,17.38284094)
\lineto(235.99998268,15.38284105)
\lineto(242.99998231,17.38284094)
\lineto(235.99998268,19.38284084)
\closepath
}
}
{
\newrgbcolor{curcolor}{0 0 0}
\pscustom[linewidth=0.53333332,linecolor=curcolor]
{
\newpath
\moveto(237.99998257,17.38284094)
\lineto(235.99998268,15.38284105)
\lineto(242.99998231,17.38284094)
\lineto(235.99998268,19.38284084)
\closepath
}
}
{
\newrgbcolor{curcolor}{0 0 0}
\pscustom[linewidth=0.99999995,linecolor=curcolor]
{
\newpath
\moveto(260.9999811,17.38284094)
\lineto(310.99997858,17.38284094)
}
}
{
\newrgbcolor{curcolor}{0 0 0}
\pscustom[linestyle=none,fillstyle=solid,fillcolor=curcolor]
{
\newpath
\moveto(306.99997879,17.38284094)
\lineto(304.9999789,15.38284105)
\lineto(311.99997853,17.38284094)
\lineto(304.9999789,19.38284084)
\closepath
}
}
{
\newrgbcolor{curcolor}{0 0 0}
\pscustom[linewidth=0.53333332,linecolor=curcolor]
{
\newpath
\moveto(306.99997879,17.38284094)
\lineto(304.9999789,15.38284105)
\lineto(311.99997853,17.38284094)
\lineto(304.9999789,19.38284084)
\closepath
}
}
{
\newrgbcolor{curcolor}{0 0 0}
\pscustom[linestyle=none,fillstyle=solid,fillcolor=curcolor]
{
\newpath
\moveto(331.39322912,17.98290014)
\curveto(331.39322912,19.08956686)(330.79324013,19.76956689)(330.67324234,19.88956689)
\curveto(330.13991879,20.47623359)(329.6599276,20.59623359)(329.01993935,20.7562336)
\lineto(329.01993935,24.35623377)
\curveto(330.15325188,24.30290043)(330.79324013,23.71623374)(330.99323646,22.96956704)
\curveto(330.93990411,22.98290037)(330.91323793,22.9962337)(330.77990704,22.9962337)
\curveto(330.43324674,22.9962337)(330.16658497,22.75623369)(330.16658497,22.38290034)
\curveto(330.16658497,21.96956699)(330.49991218,21.76956698)(330.77990704,21.76956698)
\curveto(330.81990631,21.76956698)(331.39322912,21.78290031)(331.39322912,22.43623368)
\curveto(331.39322912,23.72956707)(330.51324527,24.70290045)(329.01993935,24.78290045)
\lineto(329.01993935,25.38290048)
\lineto(328.6066136,25.38290048)
\lineto(328.6066136,24.76956712)
\curveto(327.12664076,24.63623378)(326.23332383,23.43623372)(326.23332383,22.22290033)
\curveto(326.23332383,21.30290029)(326.7266481,20.66290026)(326.9666437,20.43623359)
\curveto(327.47330107,19.91623356)(327.92662608,19.80956689)(328.6066136,19.63623355)
\lineto(328.6066136,15.66290003)
\curveto(327.4333018,15.7295667)(326.80664664,16.40956673)(326.63331648,17.23623344)
\curveto(326.68664884,17.2229001)(326.79331355,17.20956677)(326.8466459,17.20956677)
\curveto(327.20663929,17.20956677)(327.45996798,17.46290011)(327.45996798,17.82290013)
\curveto(327.45996798,18.19623348)(327.17997312,18.43623349)(326.8466459,18.43623349)
\curveto(326.76664737,18.43623349)(326.23332383,18.40956682)(326.23332383,17.75623346)
\curveto(326.23332383,16.56956674)(326.93997752,15.34290002)(328.6066136,15.23623334)
\lineto(328.6066136,14.63623332)
\lineto(329.01993935,14.63623332)
\lineto(329.01993935,15.24956668)
\curveto(330.41991365,15.36956668)(331.39322912,16.60956674)(331.39322912,17.98290014)
\closepath
\moveto(330.7132416,17.58290012)
\curveto(330.7132416,16.68956675)(330.07325335,15.79623337)(329.01993935,15.66290003)
\lineto(329.01993935,19.52956688)
\curveto(330.61990998,19.19623353)(330.7132416,17.90290013)(330.7132416,17.58290012)
\closepath
\moveto(328.6066136,20.86290027)
\curveto(326.9666437,21.20956695)(326.91331135,22.35623367)(326.91331135,22.62290035)
\curveto(326.91331135,23.42290039)(327.53996651,24.24956709)(328.6066136,24.35623377)
\closepath
}
}
{
\newrgbcolor{curcolor}{0 0 0}
\pscustom[linestyle=none,fillstyle=solid,fillcolor=curcolor]
{
\newpath
\moveto(338.92639896,20.7162336)
\lineto(338.92639896,21.12956695)
\curveto(338.61973792,21.10290028)(338.23307835,21.08956695)(337.92641731,21.08956695)
\lineto(336.7664386,21.12956695)
\lineto(336.7664386,20.7162336)
\curveto(337.17976435,20.70290026)(337.39309377,20.47623359)(337.39309377,20.14290024)
\curveto(337.39309377,20.0095669)(337.37976068,19.98290023)(337.31309524,19.82290022)
\lineto(335.96645329,16.54290007)
\lineto(334.48648045,20.1295669)
\curveto(334.4331481,20.26290024)(334.40648192,20.31623358)(334.40648192,20.36956692)
\curveto(334.40648192,20.7162336)(334.8998062,20.7162336)(335.15313488,20.7162336)
\lineto(335.15313488,21.12956695)
\lineto(333.69982822,21.08956695)
\curveto(333.33983483,21.08956695)(332.80651128,21.10290028)(332.40651863,21.12956695)
\lineto(332.40651863,20.7162336)
\curveto(333.04650688,20.7162336)(333.29983556,20.7162336)(333.4864988,20.24956691)
\lineto(335.4864621,15.38290002)
\lineto(335.15313488,14.59623332)
\curveto(334.85980693,13.86289995)(334.48648045,12.94289991)(333.63316278,12.94289991)
\curveto(333.56649733,12.94289991)(333.2598363,12.94289991)(333.00650761,13.18289992)
\curveto(333.41983336,13.23623325)(333.52649807,13.5295666)(333.52649807,13.74289994)
\curveto(333.52649807,14.08956663)(333.27316939,14.30289997)(332.96650835,14.30289997)
\curveto(332.69984657,14.30289997)(332.40651863,14.12956663)(332.40651863,13.72956661)
\curveto(332.40651863,13.12956658)(332.96650835,12.64956656)(333.63316278,12.64956656)
\curveto(334.47314736,12.64956656)(335.01980399,13.40956659)(335.33979812,14.16956663)
\lineto(337.68642172,19.86290023)
\curveto(338.03308202,20.70290026)(338.71306954,20.7162336)(338.92639896,20.7162336)
\closepath
}
}
{
\newrgbcolor{curcolor}{0 0 0}
\pscustom[linestyle=none,fillstyle=solid,fillcolor=curcolor]
{
\newpath
\moveto(349.19287608,13.54289993)
\lineto(349.19287608,13.88956662)
\lineto(339.19305962,13.88956662)
\lineto(339.19305962,13.54289993)
\closepath
}
}
{
\newrgbcolor{curcolor}{0 0 0}
\pscustom[linestyle=none,fillstyle=solid,fillcolor=curcolor]
{
\newpath
\moveto(354.88607992,12.1962332)
\curveto(354.88607992,12.32956654)(354.80608138,12.32956654)(354.6727505,12.34289988)
\curveto(353.6194365,12.40956655)(353.12611222,13.00956658)(353.00611442,13.4895666)
\curveto(352.96611515,13.63623327)(352.96611515,13.66289994)(352.96611515,14.12956663)
\lineto(352.96611515,16.12956672)
\curveto(352.96611515,16.52956674)(352.96611515,17.20956677)(352.93944898,17.34290011)
\curveto(352.76611883,18.22290015)(351.91280115,18.56956683)(351.3928107,18.71623351)
\curveto(352.96611515,19.16956686)(352.96611515,20.11623357)(352.96611515,20.48956692)
\lineto(352.96611515,22.88956703)
\curveto(352.96611515,23.84956708)(352.96611515,24.14290042)(353.28610928,24.47623377)
\curveto(353.52610488,24.71623378)(353.83276592,25.0362338)(354.76608212,25.08956713)
\curveto(354.83274756,25.10290047)(354.88607992,25.1562338)(354.88607992,25.23623381)
\curveto(354.88607992,25.38290048)(354.77941521,25.38290048)(354.61941814,25.38290048)
\curveto(353.28610928,25.38290048)(352.09946439,24.70290045)(352.07279822,23.7429004)
\lineto(352.07279822,21.30290029)
\curveto(352.07279822,20.0495669)(352.07279822,19.83623356)(351.72613791,19.46290021)
\curveto(351.53947467,19.27623353)(351.17948128,18.91623351)(350.3394967,18.86290018)
\curveto(350.24616508,18.86290018)(350.15283346,18.84956685)(350.15283346,18.71623351)
\curveto(350.15283346,18.58290017)(350.23283199,18.58290017)(350.36616287,18.56956683)
\curveto(350.93948568,18.52956683)(352.07279822,18.24956682)(352.07279822,16.91623342)
\lineto(352.07279822,14.2762333)
\curveto(352.07279822,13.50289993)(352.07279822,13.04956658)(352.76611883,12.55623322)
\curveto(353.33944164,12.1562332)(354.2060924,12.04956653)(354.61941814,12.04956653)
\curveto(354.77941521,12.04956653)(354.88607992,12.04956653)(354.88607992,12.1962332)
\closepath
}
}
{
\newrgbcolor{curcolor}{0 0 0}
\pscustom[linestyle=none,fillstyle=solid,fillcolor=curcolor]
{
\newpath
\moveto(359.2593091,12.04956653)
\lineto(359.2593091,12.58289989)
\lineto(357.96599951,12.58289989)
\lineto(357.96599951,24.84956712)
\lineto(359.2593091,24.84956712)
\lineto(359.2593091,25.38290048)
\lineto(357.43267596,25.38290048)
\lineto(357.43267596,12.04956653)
\closepath
}
}
{
\newrgbcolor{curcolor}{0 0 0}
\pscustom[linestyle=none,fillstyle=solid,fillcolor=curcolor]
{
\newpath
\moveto(366.69916761,15.38290002)
\lineto(366.69916761,15.79623337)
\curveto(366.005847,15.79623337)(365.67251979,15.79623337)(365.6591867,16.19623339)
\lineto(365.6591867,18.74290017)
\curveto(365.6591867,19.88956689)(365.6591867,20.30290025)(365.24586095,20.78290027)
\curveto(365.05919771,21.00956694)(364.61920578,21.27623362)(363.84588664,21.27623362)
\curveto(362.87257118,21.27623362)(362.24591601,20.70290026)(361.87258953,19.87623356)
\lineto(361.87258953,21.27623362)
\lineto(359.99262403,21.12956695)
\lineto(359.99262403,20.7162336)
\curveto(360.92594024,20.7162336)(361.03260495,20.62290026)(361.03260495,19.9695669)
\lineto(361.03260495,16.3962334)
\curveto(361.03260495,15.79623337)(360.88594097,15.79623337)(359.99262403,15.79623337)
\lineto(359.99262403,15.38290002)
\lineto(361.49926305,15.42290002)
\lineto(362.99256897,15.38290002)
\lineto(362.99256897,15.79623337)
\curveto(362.09925204,15.79623337)(361.95258806,15.79623337)(361.95258806,16.3962334)
\lineto(361.95258806,18.84956685)
\curveto(361.95258806,20.23623358)(362.89923735,20.98290028)(363.75255502,20.98290028)
\curveto(364.59253961,20.98290028)(364.73920358,20.26290024)(364.73920358,19.50290021)
\lineto(364.73920358,16.3962334)
\curveto(364.73920358,15.79623337)(364.59253961,15.79623337)(363.69922267,15.79623337)
\lineto(363.69922267,15.38290002)
\lineto(365.20586168,15.42290002)
\closepath
}
}
{
\newrgbcolor{curcolor}{0 0 0}
\pscustom[linestyle=none,fillstyle=solid,fillcolor=curcolor]
{
\newpath
\moveto(369.09912088,12.04956653)
\lineto(369.09912088,25.38290048)
\lineto(367.27248774,25.38290048)
\lineto(367.27248774,24.84956712)
\lineto(368.56579734,24.84956712)
\lineto(368.56579734,12.58289989)
\lineto(367.27248774,12.58289989)
\lineto(367.27248774,12.04956653)
\closepath
}
}
{
\newrgbcolor{curcolor}{0 0 0}
\pscustom[linestyle=none,fillstyle=solid,fillcolor=curcolor]
{
\newpath
\moveto(376.37898593,18.71623351)
\curveto(376.37898593,18.84956685)(376.2989874,18.84956685)(376.16565651,18.86290018)
\curveto(375.5923337,18.90290018)(374.45902117,19.18290019)(374.45902117,20.51623359)
\lineto(374.45902117,23.15623371)
\curveto(374.45902117,23.92956708)(374.45902117,24.38290043)(373.76570056,24.87623379)
\curveto(373.19237775,25.26290047)(372.33906008,25.38290048)(371.91240124,25.38290048)
\curveto(371.77907036,25.38290048)(371.64573947,25.38290048)(371.64573947,25.23623381)
\curveto(371.64573947,25.10290047)(371.725738,25.10290047)(371.85906889,25.08956713)
\curveto(372.91238289,25.02290046)(373.40570717,24.42290044)(373.52570496,23.94290041)
\curveto(373.56570423,23.79623374)(373.56570423,23.76956707)(373.56570423,23.30290038)
\lineto(373.56570423,21.30290029)
\curveto(373.56570423,20.90290027)(373.56570423,20.22290024)(373.59237041,20.0895669)
\curveto(373.76570056,19.20956686)(374.61901823,18.86290018)(375.13900869,18.71623351)
\curveto(373.56570423,18.26290015)(373.56570423,17.31623344)(373.56570423,16.94290009)
\lineto(373.56570423,14.54289998)
\curveto(373.56570423,13.58289994)(373.56570423,13.28956659)(373.2457101,12.95623324)
\curveto(373.00571451,12.71623323)(372.69905347,12.39623321)(371.76573727,12.34289988)
\curveto(371.69907182,12.32956654)(371.64573947,12.27623321)(371.64573947,12.1962332)
\curveto(371.64573947,12.04956653)(371.77907036,12.04956653)(371.91240124,12.04956653)
\curveto(373.2457101,12.04956653)(374.43235499,12.72956656)(374.45902117,13.68956661)
\lineto(374.45902117,16.12956672)
\curveto(374.45902117,17.38290011)(374.45902117,17.59623345)(374.80568147,17.9695668)
\curveto(374.99234471,18.15623348)(375.3523381,18.5162335)(376.19232269,18.56956683)
\curveto(376.28565431,18.56956683)(376.37898593,18.58290017)(376.37898593,18.71623351)
\closepath
}
}
{
\newrgbcolor{curcolor}{0 0 0}
\pscustom[linestyle=none,fillstyle=solid,fillcolor=curcolor]
{
\newpath
\moveto(383.25883578,17.98290014)
\curveto(383.25883578,19.08956686)(382.65884679,19.76956689)(382.53884899,19.88956689)
\curveto(382.00552545,20.47623359)(381.52553426,20.59623359)(380.885546,20.7562336)
\lineto(380.885546,24.35623377)
\curveto(382.01885853,24.30290043)(382.65884679,23.71623374)(382.85884312,22.96956704)
\curveto(382.80551076,22.98290037)(382.77884459,22.9962337)(382.6455137,22.9962337)
\curveto(382.2988534,22.9962337)(382.03219162,22.75623369)(382.03219162,22.38290034)
\curveto(382.03219162,21.96956699)(382.36551884,21.76956698)(382.6455137,21.76956698)
\curveto(382.68551297,21.76956698)(383.25883578,21.78290031)(383.25883578,22.43623368)
\curveto(383.25883578,23.72956707)(382.37885193,24.70290045)(380.885546,24.78290045)
\lineto(380.885546,25.38290048)
\lineto(380.47222026,25.38290048)
\lineto(380.47222026,24.76956712)
\curveto(378.99224742,24.63623378)(378.09893048,23.43623372)(378.09893048,22.22290033)
\curveto(378.09893048,21.30290029)(378.59225476,20.66290026)(378.83225035,20.43623359)
\curveto(379.33890772,19.91623356)(379.79223274,19.80956689)(380.47222026,19.63623355)
\lineto(380.47222026,15.66290003)
\curveto(379.29890846,15.7295667)(378.67225329,16.40956673)(378.49892314,17.23623344)
\curveto(378.55225549,17.2229001)(378.6589202,17.20956677)(378.71225256,17.20956677)
\curveto(379.07224595,17.20956677)(379.32557463,17.46290011)(379.32557463,17.82290013)
\curveto(379.32557463,18.19623348)(379.04557977,18.43623349)(378.71225256,18.43623349)
\curveto(378.63225403,18.43623349)(378.09893048,18.40956682)(378.09893048,17.75623346)
\curveto(378.09893048,16.56956674)(378.80558418,15.34290002)(380.47222026,15.23623334)
\lineto(380.47222026,14.63623332)
\lineto(380.885546,14.63623332)
\lineto(380.885546,15.24956668)
\curveto(382.28552031,15.36956668)(383.25883578,16.60956674)(383.25883578,17.98290014)
\closepath
\moveto(382.57884826,17.58290012)
\curveto(382.57884826,16.68956675)(381.93886,15.79623337)(380.885546,15.66290003)
\lineto(380.885546,19.52956688)
\curveto(382.48551664,19.19623353)(382.57884826,17.90290013)(382.57884826,17.58290012)
\closepath
\moveto(380.47222026,20.86290027)
\curveto(378.83225035,21.20956695)(378.778918,22.35623367)(378.778918,22.62290035)
\curveto(378.778918,23.42290039)(379.40557317,24.24956709)(380.47222026,24.35623377)
\closepath
}
}
{
\newrgbcolor{curcolor}{0 0 0}
\pscustom[linewidth=0.99999995,linecolor=curcolor]
{
\newpath
\moveto(41.9999811,47.38284094)
\lineto(41.9999811,17.38284094)
\lineto(101.9999811,17.38284094)
}
}
{
\newrgbcolor{curcolor}{0 0 0}
\pscustom[linestyle=none,fillstyle=solid,fillcolor=curcolor]
{
\newpath
\moveto(97.99998131,17.38284094)
\lineto(95.99998142,15.38284105)
\lineto(102.99998105,17.38284094)
\lineto(95.99998142,19.38284084)
\closepath
}
}
{
\newrgbcolor{curcolor}{0 0 0}
\pscustom[linewidth=0.53333332,linecolor=curcolor]
{
\newpath
\moveto(97.99998131,17.38284094)
\lineto(95.99998142,15.38284105)
\lineto(102.99998105,17.38284094)
\lineto(95.99998142,19.38284084)
\closepath
}
}
{
\newrgbcolor{curcolor}{0.65490198 0.66274512 0.67450982}
\pscustom[linewidth=0.99999995,linecolor=curcolor]
{
\newpath
\moveto(40.01193071,37.38285354)
\lineto(44.01192945,41.38284094)
}
}
{
\newrgbcolor{curcolor}{0.65490198 0.66274512 0.67450982}
\pscustom[linestyle=none,fillstyle=solid,fillcolor=curcolor]
{
\newpath
\moveto(55.54399933,33.54289718)
\curveto(55.54399933,33.54289718)(55.54399933,33.58289718)(55.50399934,33.68689717)
\lineto(52.78399948,41.20689678)
\curveto(52.75199948,41.29489677)(52.71999948,41.38289677)(52.60799949,41.38289677)
\curveto(52.51999949,41.38289677)(52.4479995,41.31089677)(52.4479995,41.22289678)
\curveto(52.4479995,41.22289678)(52.4479995,41.18289678)(52.48799949,41.07889678)
\lineto(55.20799935,33.55889718)
\curveto(55.23999935,33.47089718)(55.27199935,33.38289719)(55.38399934,33.38289719)
\curveto(55.47199934,33.38289719)(55.54399933,33.45489719)(55.54399933,33.54289718)
\closepath
}
}
{
\newrgbcolor{curcolor}{0.65490198 0.66274512 0.67450982}
\pscustom[linestyle=none,fillstyle=solid,fillcolor=curcolor]
{
\newpath
\moveto(58.87998384,36.40689703)
\curveto(58.87998384,36.83089701)(58.63998385,37.070897)(58.54398386,37.16689699)
\curveto(58.27998387,37.42289698)(57.96798389,37.48689697)(57.6319839,37.55089697)
\curveto(57.18398393,37.63889697)(56.64798396,37.74289696)(56.64798396,38.20689694)
\curveto(56.64798396,38.48689692)(56.85598395,38.8148969)(57.54398391,38.8148969)
\curveto(58.42398386,38.8148969)(58.46398386,38.09489694)(58.47998386,37.84689695)
\curveto(58.48798386,37.77489696)(58.57598385,37.77489696)(58.57598385,37.77489696)
\curveto(58.67998385,37.77489696)(58.67998385,37.81489696)(58.67998385,37.96689695)
\lineto(58.67998385,38.77489691)
\curveto(58.67998385,38.9108969)(58.67998385,38.9668969)(58.59198385,38.9668969)
\curveto(58.55198386,38.9668969)(58.53598386,38.9668969)(58.43198386,38.8708969)
\curveto(58.40798386,38.8388969)(58.32798387,38.76689691)(58.29598387,38.74289691)
\curveto(57.99198389,38.9668969)(57.6639839,38.9668969)(57.54398391,38.9668969)
\curveto(56.56798396,38.9668969)(56.26398398,38.43089692)(56.26398398,37.98289695)
\curveto(56.26398398,37.70289696)(56.39198397,37.47889697)(56.60798396,37.30289698)
\curveto(56.86398394,37.09489699)(57.08798393,37.046897)(57.6639839,36.934897)
\curveto(57.83998389,36.902897)(58.49598386,36.77489701)(58.49598386,36.19889704)
\curveto(58.49598386,35.79089706)(58.21598387,35.47089708)(57.59198391,35.47089708)
\curveto(56.91998394,35.47089708)(56.63198396,35.92689706)(56.47998397,36.60689702)
\curveto(56.45598397,36.71089701)(56.44798397,36.74289701)(56.36798397,36.74289701)
\curveto(56.26398398,36.74289701)(56.26398398,36.68689702)(56.26398398,36.54289702)
\lineto(56.26398398,35.48689708)
\curveto(56.26398398,35.35089709)(56.26398398,35.29489709)(56.35198397,35.29489709)
\curveto(56.39198397,35.29489709)(56.39998397,35.30289709)(56.55198396,35.45489708)
\curveto(56.56798396,35.47089708)(56.56798396,35.48689708)(56.71198395,35.63889707)
\curveto(57.06398393,35.30289709)(57.42398392,35.29489709)(57.59198391,35.29489709)
\curveto(58.51198386,35.29489709)(58.87998384,35.83089706)(58.87998384,36.40689703)
\closepath
}
}
{
\newrgbcolor{curcolor}{0.65490198 0.66274512 0.67450982}
\pscustom[linestyle=none,fillstyle=solid,fillcolor=curcolor]
{
\newpath
\moveto(65.65597624,35.38289708)
\lineto(65.65597624,35.63089707)
\curveto(65.23997626,35.63089707)(65.03997627,35.63089707)(65.03197627,35.87089706)
\lineto(65.03197627,37.39889698)
\curveto(65.03197627,38.08689694)(65.03197627,38.33489693)(64.78397628,38.62289691)
\curveto(64.67197629,38.75889691)(64.4079763,38.9188969)(63.94397633,38.9188969)
\curveto(63.27197636,38.9188969)(62.91997638,38.43889692)(62.78397639,38.13489694)
\curveto(62.67197639,38.8308969)(62.07997642,38.9188969)(61.71997644,38.9188969)
\curveto(61.13597647,38.9188969)(60.75997649,38.57489692)(60.53597651,38.07889694)
\lineto(60.53597651,38.9188969)
\lineto(59.40797656,38.8308969)
\lineto(59.40797656,38.58289692)
\curveto(59.96797653,38.58289692)(60.03197653,38.52689692)(60.03197653,38.13489694)
\lineto(60.03197653,35.99089705)
\curveto(60.03197653,35.63089707)(59.94397654,35.63089707)(59.40797656,35.63089707)
\lineto(59.40797656,35.38289708)
\lineto(60.31197652,35.40689708)
\lineto(61.20797647,35.38289708)
\lineto(61.20797647,35.63089707)
\curveto(60.6719765,35.63089707)(60.5839765,35.63089707)(60.5839765,35.99089705)
\lineto(60.5839765,37.46289697)
\curveto(60.5839765,38.29489693)(61.15197647,38.74289691)(61.66397645,38.74289691)
\curveto(62.16797642,38.74289691)(62.25597641,38.31089693)(62.25597641,37.85489695)
\lineto(62.25597641,35.99089705)
\curveto(62.25597641,35.63089707)(62.16797642,35.63089707)(61.63197645,35.63089707)
\lineto(61.63197645,35.38289708)
\lineto(62.5359764,35.40689708)
\lineto(63.43197635,35.38289708)
\lineto(63.43197635,35.63089707)
\curveto(62.89597638,35.63089707)(62.80797639,35.63089707)(62.80797639,35.99089705)
\lineto(62.80797639,37.46289697)
\curveto(62.80797639,38.29489693)(63.37597636,38.74289691)(63.88797633,38.74289691)
\curveto(64.3919763,38.74289691)(64.4799763,38.31089693)(64.4799763,37.85489695)
\lineto(64.4799763,35.99089705)
\curveto(64.4799763,35.63089707)(64.3919763,35.63089707)(63.85597633,35.63089707)
\lineto(63.85597633,35.38289708)
\lineto(64.75997628,35.40689708)
\closepath
}
}
{
\newrgbcolor{curcolor}{0.65490198 0.66274512 0.67450982}
\pscustom[linestyle=none,fillstyle=solid,fillcolor=curcolor]
{
\newpath
\moveto(69.67996204,36.09489705)
\lineto(69.67996204,36.54289702)
\lineto(69.47996205,36.54289702)
\lineto(69.47996205,36.09489705)
\curveto(69.47996205,35.63089707)(69.27996206,35.58289707)(69.19196207,35.58289707)
\curveto(68.92796208,35.58289707)(68.89596208,35.94289705)(68.89596208,35.98289705)
\lineto(68.89596208,37.58289697)
\curveto(68.89596208,37.91889695)(68.89596208,38.23089693)(68.6079621,38.52689692)
\curveto(68.29596211,38.8388969)(67.89596213,38.9668969)(67.51196215,38.9668969)
\curveto(66.85596219,38.9668969)(66.30396222,38.59089692)(66.30396222,38.06289694)
\curveto(66.30396222,37.82289696)(66.46396221,37.68689696)(66.6719622,37.68689696)
\curveto(66.89596219,37.68689696)(67.03996218,37.84689695)(67.03996218,38.05489694)
\curveto(67.03996218,38.15089694)(66.99996218,38.41489692)(66.6319622,38.42289692)
\curveto(66.84796219,38.70289691)(67.23996217,38.79089691)(67.49596215,38.79089691)
\curveto(67.88796213,38.79089691)(68.34396211,38.47889692)(68.34396211,37.76689696)
\lineto(68.34396211,37.47089697)
\curveto(67.93596213,37.44689698)(67.37596216,37.42289698)(66.87196219,37.18289699)
\curveto(66.27196222,36.910897)(66.07196223,36.49489703)(66.07196223,36.14289704)
\curveto(66.07196223,35.49489708)(66.84796219,35.29489709)(67.35196216,35.29489709)
\curveto(67.87996213,35.29489709)(68.24796212,35.61489707)(68.39996211,35.99089705)
\curveto(68.43196211,35.67089707)(68.64796209,35.33489709)(69.02396207,35.33489709)
\curveto(69.19196207,35.33489709)(69.67996204,35.44689708)(69.67996204,36.09489705)
\closepath
\moveto(68.34396211,36.50289703)
\curveto(68.34396211,35.74289707)(67.76796214,35.47089708)(67.40796216,35.47089708)
\curveto(67.01596218,35.47089708)(66.6879622,35.75089706)(66.6879622,36.15089704)
\curveto(66.6879622,36.59089702)(67.02396218,37.25489699)(68.34396211,37.30289698)
\closepath
}
}
{
\newrgbcolor{curcolor}{0.65490198 0.66274512 0.67450982}
\pscustom[linestyle=none,fillstyle=solid,fillcolor=curcolor]
{
\newpath
\moveto(71.85594706,35.38289708)
\lineto(71.85594706,35.63089707)
\curveto(71.31994709,35.63089707)(71.23194709,35.63089707)(71.23194709,35.99089705)
\lineto(71.23194709,40.93489679)
\lineto(70.07994715,40.8468968)
\lineto(70.07994715,40.59889681)
\curveto(70.63994712,40.59889681)(70.70394712,40.54289681)(70.70394712,40.15089683)
\lineto(70.70394712,35.99089705)
\curveto(70.70394712,35.63089707)(70.61594712,35.63089707)(70.07994715,35.63089707)
\lineto(70.07994715,35.38289708)
\lineto(70.9679471,35.40689708)
\closepath
}
}
{
\newrgbcolor{curcolor}{0.65490198 0.66274512 0.67450982}
\pscustom[linestyle=none,fillstyle=solid,fillcolor=curcolor]
{
\newpath
\moveto(74.079945,35.38289708)
\lineto(74.079945,35.63089707)
\curveto(73.54394503,35.63089707)(73.45594503,35.63089707)(73.45594503,35.99089705)
\lineto(73.45594503,40.93489679)
\lineto(72.30394509,40.8468968)
\lineto(72.30394509,40.59889681)
\curveto(72.86394506,40.59889681)(72.92794506,40.54289681)(72.92794506,40.15089683)
\lineto(72.92794506,35.99089705)
\curveto(72.92794506,35.63089707)(72.83994506,35.63089707)(72.30394509,35.63089707)
\lineto(72.30394509,35.38289708)
\lineto(73.19194505,35.40689708)
\closepath
}
}
{
\newrgbcolor{curcolor}{0.65490198 0.66274512 0.67450982}
\pscustom[linestyle=none,fillstyle=solid,fillcolor=curcolor]
{
\newpath
\moveto(80.47193099,36.942897)
\curveto(80.47193099,37.60689697)(80.11193101,38.01489695)(80.03993101,38.08689694)
\curveto(79.71993103,38.43889692)(79.43193105,38.51089692)(79.04793107,38.60689691)
\lineto(79.04793107,40.7668968)
\curveto(79.72793103,40.7348968)(80.11193101,40.38289682)(80.231931,39.93489684)
\curveto(80.19993101,39.94289684)(80.18393101,39.95089684)(80.10393101,39.95089684)
\curveto(79.89593102,39.95089684)(79.73593103,39.80689685)(79.73593103,39.58289686)
\curveto(79.73593103,39.33489688)(79.93593102,39.21489688)(80.10393101,39.21489688)
\curveto(80.12793101,39.21489688)(80.47193099,39.22289688)(80.47193099,39.61489686)
\curveto(80.47193099,40.39089682)(79.94393102,40.97489679)(79.04793107,41.02289679)
\lineto(79.04793107,41.38289677)
\lineto(78.79993108,41.38289677)
\lineto(78.79993108,41.01489679)
\curveto(77.91193113,40.93489679)(77.37593115,40.21489683)(77.37593115,39.48689687)
\curveto(77.37593115,38.9348969)(77.67193114,38.55089692)(77.81593113,38.41489692)
\curveto(78.11993111,38.10289694)(78.3919311,38.03889694)(78.79993108,37.93489695)
\lineto(78.79993108,35.55089708)
\curveto(78.09593112,35.59089707)(77.71993114,35.99889705)(77.61593114,36.49489703)
\curveto(77.64793114,36.48689703)(77.71193114,36.47889703)(77.74393113,36.47889703)
\curveto(77.95993112,36.47889703)(78.11193112,36.63089702)(78.11193112,36.84689701)
\curveto(78.11193112,37.070897)(77.94393112,37.21489699)(77.74393113,37.21489699)
\curveto(77.69593114,37.21489699)(77.37593115,37.19889699)(77.37593115,36.80689701)
\curveto(77.37593115,36.09489705)(77.79993113,35.35889709)(78.79993108,35.29489709)
\lineto(78.79993108,34.93489711)
\lineto(79.04793107,34.93489711)
\lineto(79.04793107,35.30289709)
\curveto(79.88793102,35.37489708)(80.47193099,36.11889705)(80.47193099,36.942897)
\closepath
\moveto(80.06393101,36.70289701)
\curveto(80.06393101,36.16689704)(79.67993103,35.63089707)(79.04793107,35.55089708)
\lineto(79.04793107,37.87089695)
\curveto(80.00793102,37.67089696)(80.06393101,36.894897)(80.06393101,36.70289701)
\closepath
\moveto(78.79993108,38.67089691)
\curveto(77.81593113,38.8788969)(77.78393113,39.56689686)(77.78393113,39.72689686)
\curveto(77.78393113,40.20689683)(78.15993111,40.7028968)(78.79993108,40.7668968)
\closepath
}
}
{
\newrgbcolor{curcolor}{0.65490198 0.66274512 0.67450982}
\pscustom[linestyle=none,fillstyle=solid,fillcolor=curcolor]
{
\newpath
\moveto(84.27991502,35.38289708)
\lineto(84.27991502,35.63089707)
\lineto(84.02391503,35.63089707)
\curveto(83.30391507,35.63089707)(83.27991507,35.71889707)(83.27991507,36.01489705)
\lineto(83.27991507,40.50289682)
\curveto(83.27991507,40.6948968)(83.27991507,40.7108968)(83.09591508,40.7108968)
\curveto(82.59991511,40.19889683)(81.89591515,40.19889683)(81.63991516,40.19889683)
\lineto(81.63991516,39.95089684)
\curveto(81.79991515,39.95089684)(82.27191513,39.95089684)(82.68791511,40.15889683)
\lineto(82.68791511,36.01489705)
\curveto(82.68791511,35.72689707)(82.66391511,35.63089707)(81.94391514,35.63089707)
\lineto(81.68791516,35.63089707)
\lineto(81.68791516,35.38289708)
\curveto(81.96791514,35.40689708)(82.66391511,35.40689708)(82.98391509,35.40689708)
\curveto(83.30391507,35.40689708)(83.99991504,35.40689708)(84.27991502,35.38289708)
\closepath
}
}
{
\newrgbcolor{curcolor}{0.65490198 0.66274512 0.67450982}
\pscustom[linestyle=none,fillstyle=solid,fillcolor=curcolor]
{
\newpath
\moveto(88.58389903,36.75089701)
\curveto(88.58389903,37.40689698)(88.07989905,38.03089694)(87.2478991,38.19889694)
\curveto(87.90389906,38.41489692)(88.36789904,38.9748969)(88.36789904,39.60689686)
\curveto(88.36789904,40.26289683)(87.66389907,40.7108968)(86.89589911,40.7108968)
\curveto(86.08789916,40.7108968)(85.47989919,40.23089683)(85.47989919,39.62289686)
\curveto(85.47989919,39.35889688)(85.65589918,39.20689688)(85.88789917,39.20689688)
\curveto(86.13589915,39.20689688)(86.29589915,39.38289687)(86.29589915,39.61489686)
\curveto(86.29589915,40.01489684)(85.91989917,40.01489684)(85.79989917,40.01489684)
\curveto(86.04789916,40.40689682)(86.57589913,40.51089681)(86.86389912,40.51089681)
\curveto(87.1918991,40.51089681)(87.63189908,40.33489682)(87.63189908,39.61489686)
\curveto(87.63189908,39.51889687)(87.61589908,39.05489689)(87.40789909,38.70289691)
\curveto(87.1678991,38.31889693)(86.89589911,38.29489693)(86.69589913,38.28689693)
\curveto(86.63189913,38.27889693)(86.43989914,38.26289693)(86.38389914,38.26289693)
\curveto(86.31989914,38.25489693)(86.26389915,38.24689693)(86.26389915,38.16689694)
\curveto(86.26389915,38.07889694)(86.31989914,38.07889694)(86.45589914,38.07889694)
\lineto(86.80789912,38.07889694)
\curveto(87.46389908,38.07889694)(87.75989907,37.53489697)(87.75989907,36.75089701)
\curveto(87.75989907,35.66289707)(87.2078991,35.43089708)(86.85589912,35.43089708)
\curveto(86.51189913,35.43089708)(85.91189917,35.56689707)(85.63189918,36.03889705)
\curveto(85.91189917,35.99889705)(86.15989915,36.17489704)(86.15989915,36.47889703)
\curveto(86.15989915,36.76689701)(85.94389916,36.926897)(85.71189918,36.926897)
\curveto(85.51989919,36.926897)(85.2638992,36.81489701)(85.2638992,36.46289703)
\curveto(85.2638992,35.73489707)(86.00789916,35.20689709)(86.87989912,35.20689709)
\curveto(87.85589906,35.20689709)(88.58389903,35.93489706)(88.58389903,36.75089701)
\closepath
}
}
{
\newrgbcolor{curcolor}{0.65490198 0.66274512 0.67450982}
\pscustom[linestyle=none,fillstyle=solid,fillcolor=curcolor]
{
\newpath
\moveto(92.47188485,36.942897)
\curveto(92.47188485,37.60689697)(92.11188487,38.01489695)(92.03988488,38.08689694)
\curveto(91.71988489,38.43889692)(91.43188491,38.51089692)(91.04788493,38.60689691)
\lineto(91.04788493,40.7668968)
\curveto(91.72788489,40.7348968)(92.11188487,40.38289682)(92.23188487,39.93489684)
\curveto(92.19988487,39.94289684)(92.18388487,39.95089684)(92.10388487,39.95089684)
\curveto(91.89588488,39.95089684)(91.73588489,39.80689685)(91.73588489,39.58289686)
\curveto(91.73588489,39.33489688)(91.93588488,39.21489688)(92.10388487,39.21489688)
\curveto(92.12788487,39.21489688)(92.47188485,39.22289688)(92.47188485,39.61489686)
\curveto(92.47188485,40.39089682)(91.94388488,40.97489679)(91.04788493,41.02289679)
\lineto(91.04788493,41.38289677)
\lineto(90.79988494,41.38289677)
\lineto(90.79988494,41.01489679)
\curveto(89.91188499,40.93489679)(89.37588502,40.21489683)(89.37588502,39.48689687)
\curveto(89.37588502,38.9348969)(89.671885,38.55089692)(89.81588499,38.41489692)
\curveto(90.11988498,38.10289694)(90.39188496,38.03889694)(90.79988494,37.93489695)
\lineto(90.79988494,35.55089708)
\curveto(90.09588498,35.59089707)(89.719885,35.99889705)(89.615885,36.49489703)
\curveto(89.647885,36.48689703)(89.711885,36.47889703)(89.743885,36.47889703)
\curveto(89.95988499,36.47889703)(90.11188498,36.63089702)(90.11188498,36.84689701)
\curveto(90.11188498,37.070897)(89.94388499,37.21489699)(89.743885,37.21489699)
\curveto(89.695885,37.21489699)(89.37588502,37.19889699)(89.37588502,36.80689701)
\curveto(89.37588502,36.09489705)(89.79988499,35.35889709)(90.79988494,35.29489709)
\lineto(90.79988494,34.93489711)
\lineto(91.04788493,34.93489711)
\lineto(91.04788493,35.30289709)
\curveto(91.88788488,35.37489708)(92.47188485,36.11889705)(92.47188485,36.942897)
\closepath
\moveto(92.06388488,36.70289701)
\curveto(92.06388488,36.16689704)(91.6798849,35.63089707)(91.04788493,35.55089708)
\lineto(91.04788493,37.87089695)
\curveto(92.00788488,37.67089696)(92.06388488,36.894897)(92.06388488,36.70289701)
\closepath
\moveto(90.79988494,38.67089691)
\curveto(89.81588499,38.8788969)(89.783885,39.56689686)(89.783885,39.72689686)
\curveto(89.783885,40.20689683)(90.15988498,40.7028968)(90.79988494,40.7668968)
\closepath
}
}
\end{pspicture}

  \caption{A direct form FIR filter with worst-case growth along the adder chain}
  \label{fig:fir_direct}
\end{figure}

% Zoom in on adder chain and talk about worst-case bit growth and how this is
% usually a type-level function (different from dependent types!). For known
% coefficients, we can do better here than worst-case growth.

For this trivial example, the input is an 8-bit word, the coefficients are all
5-bit words, and the adder chain produces a 16-bit output. For worst-case bit
growth, the result of the multiplication of an $n$-bit word and an $m$-bit word
is represented as an $[n+m]$-bit word, and the addition of an $n$-bit and an
$m$-bit word is a $[Max(n,m)+1]$-bit word. This sort of growth can be captured
by VHDL designs using generics and \texttt{for generate} statements. (Note that
there is no type inference, however, so the word length of every intermediate
signal must be explicitly defined.). As an introduction to Idris' syntax we
present this worst-case bit growth for arithmetic functions in Listing
\ref{lst:worst_arith}, where the type \texttt{Unsigned n} represents an unsigned
integer of $n$ bits.

\begin{codefig}[h]
  \caption{Worst-case bit growth for binary arithmetic functions}
\begin{lstlisting}[language=idris]
mul : Unsigned n -> Unsigned m -> Unsigned (n+m)
mul (U a) (U b) = U (a * b)

add : Unsigned n -> Unsigned m
   -> Unsigned (max n m + 1)
add (U a) (U b) = U (a + b)
\end{lstlisting}
\label{lst:worst_arith}
\end{codefig}

Each function has a type (after \texttt{:}), and an implementation (after \texttt{=}).
For clarity, the type of the multiplication function, \texttt{mul}, should be
read as ``a function accepting arguments of type \texttt{Unsigned n} and
\texttt{Unsigned m}, and returns a value of type \texttt{Unsigned (n+m)}''.

In the case of the FIR filter presented in Figure \ref{fig:fir_direct}, a circuit can be described with word lengths better than the worst-case for two reasons:

\begin{enumerate}
\item Each arithmetic operation is considered in isolation but repeated
  additions will accumulate any quantisation effects when the true range of a
  number does not align with power of 2 limits. For example, $y$ in Figure
  \ref{fig:fir_direct} will only inhabit values within the range
  $[\![0,2^{14}-1]\!]$, despite its 16-bit annotation.
\item Often the coefficients, $w_n$, will be constant. In this case the bit
  growth due to multiplication should vary with the numerical value of each
  constant coefficient.
\end{enumerate}

Improvement 2) is particularly relevant here, as its solution clearly demands a
language with dependent types --- a term-level value (a coefficient) must be
used to compute a type (the output word length).

% Let's try! Instead of tracking bits, we track the exact range that a signal
% can inhabit. This can be reduced back to an n-bit number using the Rep type
% class. Look at how we do arithmetic with a constant now... this is an example
% of dependent types! Value is lifted up to the type level and we do
% calculations with it.

Now consider how Idris can be applied to these challenges.

\subsection{An Idris Implementation}

Both improvements are facilitated by types that track the integer range each
signal can inhabit, rather than immediately rounding to the number of bits
required (i.e. $\lceil log_2(range) \rceil$). Our implementation of this is the
\texttt{Bounded} type, where a number of type \texttt{Bounded n} is in the
closed interval $[\![0,n]\!]$ (i.e. any value between 0 and $n$, inclusive).
Such a type can implement the \texttt{Rep} type class in order to maintain a
known binary representation, required for circuit synthesis.

Listing \ref{lst:better_arith} shows two arithmetic functions on
\texttt{Bounded} that help ensure minimum bit growth for the FIR filter example
--- \texttt{mulConst} to multiply a \texttt{Bounded} with a constant, and
\texttt{add} to add two \texttt{Bounded} arguments.

\begin{codefig}[h]
  \caption{Minimum bit growth for \texttt{Bounded} arithmetic functions}
\begin{lstlisting}[language=idris]
mulConst : Bounded n -> (m: Nat) -> Bounded (n*m)
mulConst (B x n) m = B (x*m) (n*m)

add : Bounded n -> Bounded m -> Bounded (n+m)
add (B x n) (B y m) = B (x+y) (n+m)
\end{lstlisting}
\label{lst:better_arith}
\end{codefig}

Paying particular attention to \texttt{mulConst}'s use of dependent types (the
\emph{type} of the output depends on the \emph{value} of an argument), a full
FIR filter circuit can be constructed using these arithmetic functions. At its
core, an FIR is a dot product of $n$ coefficients ($w$) and the last $n$ samples
of a discrete time signal ($x$).

\begin{equation}
  y_{[k]} = \sum_{k=0}^{n} w_k \cdot x_{[n-k]}
\label{eqn:fir}
\end{equation}

To construct the type for the dot product's output, consider the worst-case
magnitude of each term in Equation \ref{eqn:fir}. As the range of $x_{[n]}$ is
constant for all $n$, it can be taken out as a factor:

\begin{equation}
  |y| = |x|\sum_{k=0}^{n} |w_k|
\label{eqn:fir_mag}
\end{equation}

% Now we can continue and build this into an entire FIR structure. Look at types
% first. See how much of the implementation I can include without being scary...

From this, we can deduce that a valid type for the dot product function is
\texttt{Bounded (m * sum ws)}, given a collection of $n$ coefficients,
\texttt{Vect n Nat} called \texttt{ws}, and a collection of $n$ delayed samples
of $x$, \texttt{Vect n (Bounded m)}. Note that \texttt{sum} is an ordinary
function and it is a consequence of dependent types that we can use it to
construct a type for the dot product output. Listing \ref{lst:dot_prod} shows
the implementation of this combinatorial dot product and the FIR model that
``lifts'' this dot product into a \texttt{Stream}, which models synchronous
logic.

\begin{codefig}[h]
  \caption{Dependently typed FIR implementation}
\begin{lstlisting}[language=idris]
dotProd : (ws : Vect n Nat)
       -> Vect n (Bounded m)
       -> Bounded (m * sum ws)
dotProd {n=Z}        _         _        = zeros
dotProd {n=S l} {m} (w :: ws) (x :: xs) =
  let y = add (mulConst x w) (dotProd ws xs)
  in  rewrite dotProdDistrib m w l ws in y

fir : (ws : Vect n Nat)
   -> Stream (Bounded m)
   -> Stream (Bounded (m * sum ws))
fir {n} ws x = liftA (dotProd ws) (window n zeros x)
\end{lstlisting}
\label{lst:dot_prod}
\end{codefig}

Notice that line 7 includes a ``rewrite'' rule that seems extraneous to the
model of the circuit. This acts as a small proof for the Idris compiler,
demonstrating that the implementation does agree with the type we described. The
type of the dot product is defined as in Equation \ref{eqn:fir_mag}, but the
recursive implementation actually describes something with the following type:

\begin{equation}
|y| = |x\cdot w_i|+|x|\sum_{k=i+1}^{n} |w_k|
\end{equation}

The rewrite rule is simply reminding the compiler of the distributive property
of multiplication in order to have the function type check properly.

To summarise this effort, dependent types have been used to implement an FIR
filter in Idris that models \emph{minimal} bit growth (hence minimal resources)
based on the constant values of the provided coefficients. Because the word
lengths/ranges are tracked in the types used in this implementation, the
compiler ensures that our circuit implements the minimal growth at each
arithmetic stage as defined by our specification in Equation \ref{eqn:fir_mag}.
If an implementation does not meet this specification for any single combination
of parameters, this will be caught and manifest as a compiler error.

% OK, why is this better then?

\subsection{Comparisons to existing HDLs}

% VS VHDL & verilog, there is now an implementation that has minimal bit growth.
% Depending on EDA tools, maybe this does get optimised away to match out
% implementation, but think about the design process. We would use this filter
% as one part of a larger DSP chain. After this filter, it's likely we'd want
% resize to keep the wordlengths managable. We either truncate, 100% losing
% precision, or we'd write the weights down and do the exact calculation we've
% just told the Idris compiler to do to find out how many MSBs will be
% uninhabited. Doh. This is quite a clear advantage. We can also use this
% implementation to reason about circuit resource cost and how this varies with
% parameter changes without going through Vivado!

It is worth taking a moment here to compare this dependently typed, minimum bit
growth FIR filter to the implementations possible in other HDLs. It is clear
from this example that modern functional languages are much more expressive than
traditional HDLs, such as Verilog and VHDL. A typical VHDL FIR filter can be
parameterised in terms of the number of coefficients, the word length of the
coefficients, and the input word length. However, the bit growth is likely to be
worse than even the scenario presented in Figure \ref{fig:fir_direct}. Because
of the lack of type inference or type-level generate statements in VHDL, a
common approach is to simply resize all arithmetic stages to match the full
precision output --- heavily relying on synthesis tools to remove unused nets.

Relying on synthesis tools to prune unused nets is often a valid choice,
especially with simple example circuits. However, consider this filter example
in the context of a real design. It is likely that the filter will be just one
part of a larger chain of DSP circuits. At several points along the data path,
the full precision signals will be shortened to avoid excessive resource
usage. In this case, the designer employ multiple strategies:

\begin{itemize}
\item Truncation/rounding of the LSBs --- introducing some quantisation error
\item Removing any uninhabited MSBs --- retaining full precision, but the
  designer needs to manually evaluate Equation \ref{eqn:fir_mag} to identify any
  uninhabited MSBs for their particular set of coefficients.
\end{itemize}

The second option is appealing as it can reduce word lengths without loss in
precision, but it requires extra effort (for each coefficient set!) just to
emulate a guarantee that is provided statically in our Idris implementation.
Beyond this, breaking the reliance on synthesis tools allows the designer to
reason about resource usages from the source descriptions alone --- including
how resource usage mathematically relates to the choice of any input parameters.

% VS other Haskell HDLs: we would need to make word lengths a term-level thing,
% and we'd get no checking from the compiler. Closest is Lava. Can nearly get
% there. Haskell doesn't really have dependent types but we can use singletons.
% We'd be able to implement the same sort of arithmetic stage as seen in Fig
% XX, but when we come to writing the full FIR filter, we need collections of
% singletons and then this gets super messy. Clash has a similar problem with
% collections of singletons because we don't have general recursion -- just a
% fold. So, we need to hold collections of type-level weights. Eugh. Could do
% some risky type level stuff, but that's blergh.
% If we do keep this bit growth at a type level in Idris, then we 
%\begin{enumerate}
%\item compiler verification that we have implemented what we set out to implement (we've got the right word-lengths at each stage), or...
%\item conversely, we can let our implementation be type-directed. At each stage,
%  we infer new wordlengths (or similar) purely by what the types say should
%  happen.
%\item nice environment to reason about circuit resource costs and how they are
%  linked to the parameters... without running through Vivado.
%\end{enumerate}

There are also clear benefits above other modern functional HDLs. The closest
comparison is with the family of HDLs embedded in Haskell, called
Lava\cite{gill_09}. In Lava, a similar circuit can be described using dynamically
sized lists of bits to represent each word. It is then the execution of a
(software) Haskell program that generates the circuit, since statically sized
structures are required for most hardware descriptions. In this case, the output
circuit \emph{might} be equivalent to the Idris implementation, but there are no
guarantees about word lengths checked by the compiler --- this is what we have
addressed with dependent types. Without these compiler checks, there is a large
burden on the developer to provide good evidence of testing. There have been
efforts to introduce a subset of basic dependent types to Haskell, using
singletons, but this is not enough to replicate our Idris
implementation\cite{lindley_13}. The coefficient set is a collection of
potentially different singletons, which then requires heterogeneous collections;
another painful Haskell experience.

\section{Going further:\\Lossless Pruning in CIC Interpolators/Decimators}

Moving away from simple FIR examples now, it is exciting to see a range of other
standard DSP components that place similar (generally unmet) demands on existing
HDLs. One other noteworthy word length example cannot even rely on synthesis
tools to mask the rough approximations made with traditional HDLs --- Cascaded
Integrator-Comb (CIC) decimators/integrators.

CIC filters are often employed as a very low resource solution for digital
decimation or interpolation. CIC decimators consist of a chain of $N$ integrator
stages, followed by a $\frac{1}{R}$-rate down-sampler, followed by $N$ comb
stages with a differential delay of $M$. Figure \ref{fig:cic} shows an example
CIC decimator with $R=8$, $N=3$ and $M=1$.

\begin{figure}[htb]
  \centering
  %\def\svgwidth{\columnwidth}
  \resizebox{\columnwidth}{!}{
    \input{img/cic_dec.pdf_tex}
  }
  \caption{A CIC decimator without pruning. ($R=8$, $N=3$ and $M=1$)}
  \label{fig:cic}
\end{figure}

A register pruning technique is discussed in Hogenauer's original paper on CIC
filters\cite{hogenauer_81}. Equations are derived that describe the mean error
and variance introduced by truncation at each stage. It is suggested that, given
a desired output word length, a legitimate design choice is to prune the word
lengths of all previous stages as much as possible without accumulating an error
greater than that introduced by the final rounding/truncation. This choice
results in the following equation for the number of LSBs to discard at the
$j^{th}$ stage.

\begin{equation}
  B_j = {\Bigg \lfloor} -\log_2F_j + \log_2\sigma_{T_{2N+1}} +\frac{1}{2}\log_2\frac{6}{N} {\Bigg \rfloor}
  \label{eqn:cic_bj}
\end{equation}

where $F_j$ is the variance error gain for the $j^{th}$ stage,
$\sigma_{T_{2N+1}}$ is the total variance at the output due to truncation, and
$N$ is the number of stages. Note that the first two terms have complex
definitions of their own, including cases, sums, exponentials, and binomial
coefficients\cite{hogenauer_81}.

Because the pruned bits at each stage do contain information (we are discarding
these within our own design constraints for error variance), synthesis tools
cannot perform an equivalent optimisation given an un-pruned description. This
section continues by demonstrating how dependent types are used to concisely
describe a fully parameterised, pruned CIC decimator. The conciseness is, in
part, thanks to Idris having a single language that can be used at the
term-level and the type-level -- so all language constructs can be used to
implement Equation \ref{eqn:cic_bj}, and then this function can be used to
direct the type of each stage in a CIC implementation.

\subsection{An Idris Implementation}

Listing \ref{lst:cic_bj} shows the top-level implementation of Equation
\ref{eqn:cic_bj}, after the application of several logarithmic identities to
avoid rounding errors with an integer implementation of $\log_2$.

\begin{codefig}[h]
  \caption{Implementation of Equation \ref{eqn:cic_bj} bit pruning calculation}
\begin{lstlisting}[language=idris]
bj : (r : Nat)   -> (n : Nat)    -> (m : Nat)
  -> (bin : Nat) -> (bout : Nat) -> (j : Nat)
  -> Nat
bj r n m bin bout j =
  flog2Approx $ sigmaT2n1 r n m bin bout
              * (sqrt $ 6.0 / (cast n))
              / (fj r n m j)
\end{lstlisting}
\label{lst:cic_bj}
\end{codefig}

As well as the stage parameter $j$, all CIC parameters are present: the rate
change ($R$), the number of stages ($N$), the differential comb delay ($M$), and
input/output word lengths. Another function \texttt{prunej} inverts
\texttt{bj} to give the number of bits to retain, rather than discard. This
function can now be lifted up to the type definition of a CIC decimator circuit
to help the compiler confirm that each CIC stage is pruned to meet the error
variance design constraint.

\begin{codefig}[h]
  \caption{Type definitions for CIC decimator}
\begin{lstlisting}[language=idris]
-- Internal recursive step
cicDecimateRec :
     (r : Nat)   -> (n : Nat)    -> (m : Nat)
  -> (bin : Nat) -> (bout : Nat) -> (j : Nat)
  -> Stream (Unsigned (prunej r n m bin bout 0))
  -> Stream (Unsigned (prunej r n m bin bout (S j)))
cicDecimateRec r n m bin bout j xs = ?omitted1

-- Top-level CIC
cicDecimate : (r : Nat) -> (n : Nat) -> (m : Nat)
           -> Stream (Unsigned bin)
           -> Stream (Unsigned bout)
cicDecimate {bin} {bout} r n m xs = ?omitted2
\end{lstlisting}
\label{lst:cic_types}
\end{codefig}

Only the type definitions are shown above, but the full CIC decimator
implementation is available at \cite{ramsay_20_gh}.

While this is another example focusing on computing type-level word lengths,
these techniques can be directly applied to many other type-level constructs ---
including circuit topology. It is then only a small leap to apply these
techniques to write type-safe descriptions of many other DSP circuits. For
example, capturing the structural complexity in Multiple Constant Multiplier
blocks, such as those described by the RAG-n algorithm\cite{dempster_95}.

\subsection{Comparisons to existing HDLs}

This gives us the same advantages over VHDL as the FIR example in section
\ref{sec:fir}, but with the additional distinction that the limitations of
traditional HDLs cannot be partly mitigated by optimisations built into
synthesis tools.

In comparison to HDLs embedded in Haskell, such as Lava, a similar structure may
be technically realisable with type-level functions and singletons. However,
Haskell does not offer the luxury of one rich language that applies to both the
term and type levels. Because of this, the type-level function required to
represent Equation \ref{eqn:cic_bj} (including conditionals, sums, exponentials
and binomial coefficients) would prove to be a painful excursion.

\section{Conclusions}

TODO!

There are common DSP circuits that clearly benefit from them

\section*{Acknowledgement}

The authors would like to thank Xilinx for supplying hardware and EDA tools for
this project. We acknowledge funding for Craig Ramsay under EPSRC grant no.
EP/N509760/1.

\bibliographystyle{IEEEtran}
\bibliography{references} 

\end{document}
